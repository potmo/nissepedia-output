\documentclass[twoside,9pt]{book}
\usepackage{multicol}
\usepackage{ifthen}
\usepackage{fancyhdr}
\usepackage{enumerate}
\usepackage{lipsum}

%%
% For nicely typeset tabular material
\usepackage{booktabs}

%\usepackage[swedish]{babel}
\usepackage[T2A,T1]{fontenc}
\usepackage[utf8]{inputenc}
\usepackage[russian,greek,german,finnish,danish,swedish]{babel}
\usepackage{lmodern}
\usepackage{textcomp}
% japanese
\usepackage{CJKutf8}
%% phonetic chars
\usepackage{tipa}

% makes it possible to use $special$
%\usepackage{unixode}
\usepackage{newunicodechar}

\usepackage{color}

\usepackage{enumitem}

%\usepackage{blindtext, showframe}
\usepackage{geometry}

%adding underlines
\usepackage{ulem}

% vector fonts
\usepackage{ae}

%squeezing fonts
\usepackage{microtype}

%\geometry{a5paper, margin=1in}
\usepackage[paperheight=9in,paperwidth=6in,top=1in,bottom=1in,right=1in,left=1in]{geometry}

% Images
\usepackage{epsfig}

%% special clearpage for figures
\usepackage{floatpag}

\title{Nissepedia - Ett urval}
\author{Kulturföreningen Nissepedia}

%fix non breaking spaces
\DeclareUnicodeCharacter{00A0}{ }

%ignore overfull hbox
\hfuzz=20pt

%more errors
\errorcontextlines 10000

%euro sign
\usepackage{marvosym}
\DeclareUnicodeCharacter{20AC}{\EUR{}}

% Inserts a blank page
\newcommand{\blankpage}{\newpage\hbox{}\newpage}

\newcommand{\quotetext}[1]{``#1''} 
\newcommand{\charaa}{\aa}
\newcommand{\charAA}{\AA}
\newcommand{\charae}{\"a}
\newcommand{\charAE}{\"A}
\newcommand{\charoo}{\"o}
\newcommand{\charOO}{\"O}

%\newcommand{\bashidoA}{\begin{CJK}{UTF8}{min}武\end{CJK}}
%\newcommand{\bashidoB}{\begin{CJK}{UTF8}{min}士\end{CJK}}
%\newcommand{\bashidoC}{\begin{CJK}{UTF8}{min}道\end{CJK}}



% add unicode chars
\newunicodechar{å}{\aa}
\newunicodechar{Å}{\AA}
\newunicodechar{ä}{\"a}
\newunicodechar{Ä}{\"A}
\newunicodechar{ö}{\"o}
\newunicodechar{Ö}{\"O}
\newunicodechar{ø}{\o}
\newunicodechar{Ø}{\O}

\newunicodechar{ʃ}{\textipa{S}}
\newunicodechar{ŋ}{\textipa{N}}
\newunicodechar{ŋ}{\textipa{N}}

\newunicodechar{ɠ}{\textipa{!g}}
\newunicodechar{Ō}{\={O}}
\newunicodechar{ō}{\={o}}
%\newunicodechar{武}{\begin{CJK}{UTF8}{min}武\end{CJK}}
%\newunicodechar{士}{\unichar{22763}}
%\newunicodechar{道}{\unichar{36947}}
\newunicodechar{Ē}{\={E}}
\newunicodechar{ē}{\={e}}

\newunicodechar{―}{-}

% dash instead of bullets in items
%\def\labelitemi{\circ}

\renewcommand{\ref}[1]{\textsuperscript{\pageref{#1}}}
\newcommand{\indexref}[1]{\textit{\pageref{#1}}}

% empty \sectionmark
\renewcommand{\sectionmark}[1]{}
% create a command to create marks and collapse them if they are identical

\setlength{\headheight}{15pt}

\newcommand{\mymarks}{%
\ifthenelse{\equal{\leftmark}{\rightmark}}
{\rightmark} % if equal
{\rightmark\enspace--\enspace\leftmark}} % if not equal 
\fancyhead[LE,RO]{\mymarks} 
\fancyhead[LO,RE]{\thepage}
% define a wrapper on the description environment to add the marks
% this part is quick and dirty; it would be better to define one's own
% environment to do this, but the basic idea is the same: you need
% to add the \markboth command to each dictionary entry
\def\ditem[#1]{\item[#1]\markboth{#1}{#1}}
\def\mydots{ \leavevmode\xleaders\hbox to 0.25em{\hfil. \hfil}\hfill\kern5pt}
\setmainfont{Adobe Garamond Pro}

\begin{document}



\pagestyle{empty}

\setenumerate{itemindent=0em,leftmargin=1em,noitemsep,topsep=0pt}
\setitemize{itemindent=0em,leftmargin=1em,noitemsep,topsep=0pt}
\setdescription{itemindent=2.1em,leftmargin=0.1in,style=unboxed}


\tolerance=1
%\emergencystretch=\maxdimen
%\hyphenpenalty=10000
%\hbadness=10000


\floatpagestyle{empty}
\begin{figure}
	\centering
	\includegraphics[width=0.5\textwidth]{frontpage.eps}
\end{figure}
\newpage



\begin{center}
\begin{minipage}{0.75\linewidth}
    \centering
    
    \vspace{3cm}

    {\uppercase{\Huge Nissepedia\par}}
    \vspace{0.5cm}
    {\Large i urval\par}
    \vspace{13cm}

    {\small Videnskabens Förlag 2014\par}
    
\end{minipage}
\end{center}
\newpage



\begin{minipage}{0.75\linewidth}
    
Som tack för sin extraordinära filantropiska insats, i kunskapens, hedonismens och de bortglömda perspektivens tjänst, föräras nedanstående personer med titeln Honorarie-Doktor i Nissepedias akademi

\begin{center}
\\
Erik Karlsson\\
Peter Östlund\\
Björn Jonas\\
Magnus Larnhed\\
Karl-Linus Börjesson\\
Niklas Gawell\\
Andreas Holmén\\
Stig Bergman\\
Erik Gunnarsson\\
Erik Koivisto\\
Katarina Gregersdotter\\
Simon Nilsson\\
Joel Brändström\\
Sara Skur\\
Peter Burström\\
John Eneroth\\
Anton Andersson\\
Jonas Larson\\
Patrik Lindmark\\
Lennart Johansson\\
Jens Christensen\\
My Larsson\\
Joakim Hedlund\\
Thomas Ericson\\
\end{center}

\\
Motivationen lyder:\\
\\
Då det stundom blåst kallt runt oss som bejakat det annorlunda, har detta fåtal trotsat massans obeskrivliga förtryck. Med hjälp av deras ekonomiska bidrag har det grävts en djup och varm bivack som skydd, i vilken bakteriekulturen som är Nissepedia har tillåtits frodas, muteras och mångfaldigas. Dessa personer har agerat apostlar. Deras öppna sinnen har tillåtit dem att se och förstå Nissepedia. \\
Denna apostlarnas öppenhet och kärlek till det sanna och riktiga har sedan drivit dem till att sprida ordet om Nissepedias lov, och pådyvlat allmänheten det bisarra.
Låt det för alltid vara känt att var person på denna lista, vid vilken tidpunkt som helst, kan vända sig till en verksam skribent på Nissepedia, och på frågan \quotetext{har du bärs?}, kommer svaret att vara \quotetext{ja}, tills det att jorden blir till damm under våra fötter, och himlen rämnar över våra huvuden.
    
\end{minipage}

\newpage





\begin{center}
\begin{minipage}{0.75\linewidth}
I en medialiserad värld där beskyllningar kastas hit och dit och desinformation besvaras med kontradesinformation verkar det ha uppstått ett sug efter en informationskanal som verkar för mänsklighetens bästa utan någon dold agenda. Det började med att Nisse behövde en källa för att bevisa att tunntarmen kommer innan magsäcken i tarmsystemet. Något som visade sig svårt. Hellre än att erkänna sig besegrad valde Nisse att registrera en hemsida och slänga upp en enkel mall där vem som helst kunde gå in och skriva informativa artiklar om vad som helst. Upplägget var inte helt olikt ett visst konkurrerande onlineuppslagsverk men på Nissepedia fanns inga jobbiga administratörer som ifrågasatte sanningshalten i det som skrevs. En klar för- eller nackdel beroende på hur man ser det. Det visade sig vara en stor fördel på så sätt att några stycken tyckte det var kul att gå in och skriva artiklar. Ganska många artiklar. Smala ämnen såsom fördelar med att bo i ett gryt i skogen, sydamerikanska alkemister och tyska toalettstolar fick äntligen en naturlig plats att berättas och beskrivas på. Antalet aktiva användare var färre än tio och jargongen ganska intern till en början eftersom ingen trodde att besökarna kom från mer avlägset håll än den yttre bekantskapskretsen. När Fagersta-Postens chefredaktör hörde av sig och insisterade på att artikeln om hans tidning borde skrivas om blev det dock uppenbart att en del av informationen faktiskt nådde längre ut än vad vi trodde. \\

Sidans uppbyggnad verkar vara väldigt uppskattad hos Google då artiklarna ofta rankas högt i träfflistorna, något vi i sig tycker är ganska roligt. Med makt följer ansvar och vi bestämde oss därför tidigare i år för att ge ut de bästa artiklarna i bokform så att även människor utan Internet kan få tillgång till all matnyttig information som samlats på sidan.\\

Innehållet i denna bok är resultatet av fem års obetald arbetstid av ett okänt antal människor. Färre än tio användare står för majoriteten av innehållet men vi har ingen aning om vem som skrivit allt eller vilka som står bakom alla användare. I det urval vi publicerar här har vi plockat bort de sämsta artiklarna och rättat stavfel och liknande som fortfarande ligger kvar på hemsidan. Allt som skrivits på Nissepedia är verkligen inte bra och maskulina perspektiv är klart överrepresenterade. Så långt vi känner till är en överväldigande majoritet av användarna vita snubbar, vilket inte ursäktar något men är en delförklaring. Sexism och rasism förekommer på Nissepedia precis som på alla andra forum som erbjuder anonymitet. Det är inget vi varit intresserade av att ta med i boken men ibland har det varit svårt att avgöra vad som är satir och vad som bara är dålig smak. Det här är inte en ursäkt men ett försök att förklara vad du håller i din hand och med detta sagt tycker vi att boken har många intressanta perspektiv och fakta att skänka läsaren oavsett vem du är.\\

\vspace{20mm}
Nissepedias Redaktion

\end{minipage}
\end{center}
\newpage


\begin{center}
\begin{minipage}{0.75\linewidth}
För ett tag sen fick jag ett mail med en förfrågan om jag ville skriva ett förord till bokutgåvan av onlinelexikonet Nissepedia. \\
Upphovsmännen beskrev sitt verk med orden: \quotetext{En del är ren skit men det mesta har ändå någon slags verkshöjd.} Eftersom detta på ett mycket koncist sätt sammanfattar hela min egen skaparkarriär så kände jag omedelbart en stark sympati med publikationen och tackade därför gladeligen ja. Att jag dessutom skrattade som tomten i Kalle Ankas julafton när jag läste artiklarna gjorde inte saken sämre.\\

Här kommer således min förtext:\\
För att bli festens självklara medelpunkt och personen runt vilken prominensen flockas så bör man, förutom det basala förrådet av tveksamma ordvitsar, även kunna briljera med en viss faktabaserad trivia. \\

Vilken plebej som helst kan googla fram folkmängden I Ulan Bator, fråga en släkting om frukostvanorna på 50-talet eller slå upp betydelsen av eskatologi i Svenska Akademiens Ordlista.
För att däremot få reda på vad ASEA-grönt, Deportees-trevlig och en likgömmarmössa är, samt även få en djuplodande analys av sjungande trummisar samt begreppet “ta för sig” så krävs något alldeles speciellt.\\

Detta alldeles speciella håller du just nu i din hand i form av en bok. (En bok är ungefär som internet, fast gjord av träd.) 
En bok har många fördelar. Den laddar inte ur när man håller på att knäcka en mardrömsbana på Candy Crush, det ringer inga telefonförsäljare till den och den tappar inte täckningen när man håller på att bli mördad på landsbygden. Med en bok i handen kommer du dessutom att framstå som en mycket mer sofistikerad person än du förmodligen är, då en bok alltid övertrumfar en mobiltelefon vid informationshämtning i intellektuella sammanhang.\\

Jag ber därför att få gratulera till ett särdeles lyckat inköp som inte bara kommer att göra dig till en bättre och mer allmänbildad människa utan även bidra till spridningen av den kulturskatt som dväljes innanför dessa pärmar. (Har du stulit boken så är det till på köpet ett tecken på att du besitter den handlingskraft och entreprenörsanda som krävs för att ta sig fram i dagens kalla och hårda samhälle, varför du i så fall är att gratulera ytterligare.)

Nåväl… För att till fullo göra Nissepedias genialitet rättvisa nödgas jag låna ett kraftfullt citat från den glasögonprydde och välkammade programledaren I TV-Shop på 80-talet: \quotetext{I’m impressed!}\\

\vspace{20mm}
Gustave Lund\\

\end{minipage}
\end{center}
\newpage





\begin{center}
\begin{minipage}{0.75\linewidth}
Nissepedia är ett uppslagsverk. Ett uppslagsverk är ”en facklitterär, genre vars innehåll består av en sammanfattning av mänskligt vetande.” Denna information har jag hittat i ett annat uppslagsverk, inte helt olikt Nissepedia till sin utformning. Men ordet uppslagsverk finns inte i Nissepedias databas, ej heller i denna bok, eller Uppslagsbok, som det även kallas, framförallt i bokform. Alla vet väl vad ett uppslagsverk är. Och vad en bok är för den delen. Nissepedia är inte mycket för den sortens självklarheter.\\

En kompis berättade om ett gräl med sin bror. Efter mycket argsint ältande, rent av hyttande, utbrast brodern till sist ”Fakta biter inte på mig!”. Och så är det ju lite med kunskap. Folk kan ju hävda vad som helst egentligen, och tjurigast vinner. Om man accepterar att det är en tävling. Och just det sura äpplet är väl bara att tugga i sig.\\

Människor definierar sig utifrån kunskap och kunskap är makt. Och den som vet mest är bäst. Eller snarare, den som vet bäst är bäst i flest ögon. Oftast är ju den en gubbe i min ålder, ungefär mellan 90 år och födseln. Statistiskt och sålunda vetenskapligt är jag antagligen närmre döden än födseln. Men samtidigt, vad vet jag eller någon annan om det?\\

Min avsikt är nu inte att ställa existentiella frågor här, det slutar alltid med värsta deppet om hur allt är meningslöst i det långa loppet, och att vi skola alla dö. Men igen, det vet ju alla. Säg något jag inte vet, liksom. Och även om jag och du och ni och vi och dem snart ska dö så lever ju alla i nuet. Det må låta som ett tips ur valfri självhjälpsbok, men det är även ett fysiskt och filosofiskt faktum. Inte för att jag vet så mycket om filosofi.\\

När jag tänker efter så lever vi inte alls i nuet. Säkert finns det någon utilitaristisk eller logisk princip som säger att eftersom nuet är så kort jämfört med allt som kom innan och det som kommer efter så är det rent felaktigt att påstå att vi lever i nuet. Det är ju inte ens nu nu, det var ju nyss. Innan du sagt ordet nu färdigt är det ju nyss, i det förflutna.\\

Och så här kan en ju förstås hålla på och älta fram och tillbaka, medans skäggvårtor och klimakteriet sakta tar över oss och vi multnar. Fast då blir vi galna. Vi måste ju göra något, vi måste tänka på annat. Men vad? Ja, det är här uppslagsverket skulle kunna komma in i bilden. Inte de där vanliga uppslagsverken dock, det är ju som att gå i skolan och det lade en ju ned för länge sedan av precis den anledningen: bläddra i böcker för att en måste kunna aptråkiga saker om ekvationer, etruskernas vattenledningssystem och sparvar.\\

Varken etrusker eller sparvar tas upp av Nissepedia, däremot stockholmare (med en något annan definition än andra uppslagsverks) och crustpippi.\\

Om en nu inte är hemskt, hemskt ledsen eller misantropisk så går livet rätt mycket ut på att interagera med andra och att ha roligt. Det är vägen och målet. Och Nissepedia kan vara dig till stor hjälp i båda dessa ärenden. Det är kunskap i dess mest praktiskt användbara form. Nissepedia är såväl allmänbildande som rolig som intressant.\\
\end{minipage}

\begin{minipage}{0.75\linewidth}
Och sann, i filosofisk mening åtminstone. Och som jag illustrerade ovan och ni förhoppningsvis inte hunnit glömma så tar det filosofiska upp en rätt stor del av vår vakna tid (och när vi drömmer, herreminje, det området är så stort att jag inte ens kan börja skriva om det, inte ens Nissepedia tar upp drömmar, fatta!).\\

Sann i den mening att all sanning är subjektiv. Betänk dessutom att religioner är de som påstår sig veta, medan vetenskapen är de som tror saker. Det betyder alltså att saker inte alltid är vad de verkar vara och att man kan fan inte lita på någon. Och se, nu börjar jag hålla på med det där existentiella deppet igen. Måste göra något annat. Som att slå upp något viktigt i boken vi har framför oss.\\

\vspace{20mm}
Mattias Alkberg\\

\end{minipage}
\end{center}
\newpage



\begin{center}
\begin{minipage}{0.75\linewidth}
\lipsum[3-6]
\\
\\
\\
\\
Någon annan\\
\textit{Något annat}

\end{minipage}
\end{center}
\newpage



\begin{center}
\begin{minipage}{0.75\linewidth}
\lipsum[3-6]
\\
\\
\\
\\
Någon annan\\
\textit{Något annat}

\end{minipage}
\end{center}
\newpage



\begin{center}
\begin{minipage}{0.75\linewidth}
\lipsum[3-6]
\\
\\
\\
\\
Någon annan\\
\textit{Något annat}

\end{minipage}
\end{center}
\newpage


\pagestyle{fancy}
\setcounter{page}{1}

\newpage

\begin{multicols}{2}





%%%%%%%%%%%%%%
\null
\\
\null
\\
\Huge
A
\normalsize
\\
\null
\\
\null
%%%%%%%%%%%%%%

\begin{description}[itemindent=2.1em,leftmargin=0.1in,style=unboxed]
\ditem[A-traktor]\label{a-traktor}
 Innan man skaffar dieselbil med lastgaller\ref{dieselbil med lastgaller} kan man skaffa en a-traktor.

Alternativt skaffar man senare i livet en A-traktorreggad Scania 112 för snöröjningens skull.

Det är inte bara jävligt coolt med a-traktor, man får numera köra den redan som 15-åring och man kan då skjutsa sina kompisar till grannbyn när det vankas åkerdisco\ref{aakerdisco}. Skatten är även förmånlig.

Man skulle lätt kunna tro att ett så här briljant fordon vore vanligast bland mer utvecklade folkslag, men statistiken visar att Västergötland är det A-traktortätaste länet med hela 1950 stycken inregistrerade.

Det finns några eviga frågor som gäller A-traktorer:

\begin{itemize}
\item Dubbla växellådor eller AGA-spärr?
\item Flak eller lucka?
\item Lägga pengar på stereon eller Jokkmokkstrim?
\end{itemize}

Den som finner svar på dessa frågor och är utrustad med svets och ett tålamod i klass med Dalai Llama kan framåt vårkanten rulla ut på vägarna.

\ditem[Abdera]\label{abdera}
 är en ort i Grekland som i dagsläget har ungefär 4000 invånare. Redan under antiken\ref{de gamla grekerna} beskrevs luften där som så dålig att man blev dum av att inandas den. Detta gav senare upphov till adjektivet \textit{abderitisk}, som är den etymologiska källan till det idag flitigt använda \quotetext{idiot}. Man förstår att det inte bor så många kvar där.

\ditem[Abu Garcia]\label{abu garcia}
 är ett företag som sysslar med försäljning av fiskeartiklar. Tidigare var det två företag, Abu och Garcia. Abu var svenskt och är en förkortning för Aktiebolaget Urmakarna. I begynnelsen pysslade man enbart med urmakeri, men arbetarnas gedigna kunskaper i finmekanik visade sig på 1930-talet även passa utmärkt till att knyta fast fiskelina på en bit klarlackad bambu. Garcia var amerikanskt och är inte förkortning för någonting. Varje fiskeredskapsaffär\ref{fiskeredskapsaffaer} av rang erbjuder produkter från Abu Garcia.

\uline{Andra användningar}

Abu Garcia! används ofta i Västerbotten\ref{vaesterbotten} som \textit{iAy caramba!} används i Mexiko, alltså som en interjektion.

\ditem[AC/DC-gitarr]\label{acdc-gitarr}
 En AC/DC-gitarr är en gitarr med vinröd kropp (ibland svart) och svart huvud som är extremt rå. Till formen påminner kroppen lite om eldflammor, vilket bara det är extremt rått. Dessutom är det ingen vanlig gitarr utan en elgitarr; också rått som fan. Och så figurerar den som mordvapen på omslaget till AC/DCs liveskiva\ref{tolva} \textit{If You Want Blood You've Got It}, så case closed. AC/DC-gitarren saluförs av den amerikanska tillverkaren Gibson, som även bygger oråa instrument såsom mandoliner och basgitarrer. Vill man höra äkta AC/DC-gitarr kan man med fördel lyssna på \textit{Manglar som ägg} eller \textit{Bonfire\ref{bonfire}}.

\uline{Personer som spelar på AC/DC-gitarrer}

\begin{itemize}
\item Angus Young
\item Mob 47-Åke
\item Tony Iommi
\item Pete Townshend
\item Nisse Hellberg
\end{itemize}

\ditem[Adak]\label{adak}
 är en ort i Malå\ref{malaa} kommun där det bor ungefär 200 personer. En gång i tiden fanns den tillsynes outömliga Adakgruvan med tillhörande by inte så långt från orten. Ett militärt stenkast därifrån låg den lika majestätiska gruvan Rudtjebäcken, även den med tillhörande bebyggelse. På sjuttiotalet, då Boliden AB bestämde sig för att fokusera sin produktion i just Boliden, fylldes bägge gruvorna igen och bägge byarna förvandlades till spökstäder. Idag finns där bara tall, gran och en minnessten som berättar om hur kapitalismen och staten svek glesbygden.

Idag kretsar produktionen kring kultur och institutionen Sagabiografen som varje sommar har filmfestival. På vintern kan man åka på drive-in bio medelst skoter.

\ditem[Adde Malmberg]\label{adde malmberg}
 är Sveriges\ref{sverige} rolighetsminister. Tillsammans med \quotetext{Babben} Larsson kan han få vem som helst att skratta på sig.

\ditem[Aforismer]\label{aforismer}
 Många äro de munläderförsedda som ägnat tid åt att författa aforismer - Oscar Wilde\ref{oscar wilde}, Nietzsche, Mark Twain och Platon är endast en handfull av dem. En aforism är en slagkraftig sentens som fångar upp och förmedlar en sanning om människan, varat eller något annat tidlöst och centralt.

\uline{Exempel på aforismer}

\begin{itemize}
\item Av sina fiender lär sig den vise mycket. \textit{- Aristoteles}
\item Nöden är geniets drivfjäder. \textit{- Honoré de Balzac}
\item Moralister är människor som kliar sig själva där det kliar på andra. \textit{- Samuel Beckett}
\item Mitt i den mörkaste vintern, lärde jag mig äntligen, att det inom mig, finns en oförgänglig sommar. \textit{- Albert Camus}
\item Det är som min far brukade säga; \quotetext{Vi har kanske inte så mycket att leva av, men om pojken vill ha ett smörpapper så ska han fan i mig ha ett smörpapper!} \textit{- Mark Frygell}
\item Den främsta dygden är att lägga band på sig och hålla tungan rätt i mun.\textit{ - Geoffrey Chaucer}
\item De flesta människor lever i ruinerna av sina vanor. \textit{- Jean Cocteau}
\item Hä ä bar å åk! \textit{- Ingemar Stenmark}
\end{itemize}

\ditem[Akademiker]\label{akademiker}
 Att vara akademiker är att förstå det högtravande och pretentiösa språk som används på universitet och högre lärosäten. Det handlar om att till exempel säga \quotetext{esoterisk} istället för \quotetext{intern} eller \quotetext{kontext} istället för \quotetext{sammanhang}.

\ditem[Aktersegla]\label{aktersegla}
 Lämna någon eller något bakom sig. Ett klassiskt exempel är när Dia Psalma akterseglade Birdnest records och började ge ut sina skivor på egen hand.

\ditem[Albert II]\label{albert ii}
 född 19 april 1942 i Florida, USA, död 14 juni 1949 i rymden, var den första apan i rymden. Albert II föddes i ett av NASA:s laboratorier där han spenderade större delen av sin levnad med att klättra i träd och delta i experiment. Han döptes efter apan Albert som var den första apan att nästan besöka rymden men som olyckligtvis dog innan han hann fram. Albert II överlevde färden upp i rymden men klarade inte tillbakaresan. Inga officiella statyer eller frimärken har någonsin skapats till Albert II:s minne.

\ditem[Albin]\label{albin}
 är ett namn som nyblivna föräldrar ger till sitt barn med förhoppningen att det ska göra att barnet automatiskt blir lillgammalt\ref{lillgammal}. Albin betyder inte just något särskilt.

\ditem[Aleksandr Karelin]\label{aleksandr karelin}
 Aleksandr Aleksandrovitj Karelin, även kallad \quotetext{det ryska experimentet} och \quotetext{lyftkranen från Sibirien}, född 19 september 1967 i Novosibirsk, Sovjet, är ingen kille man bråkar med. Till yrket brottare.

\uline{Föda och beteenden}

För några år sedan visade Sportspegeln ett inslag från Karelins hem i Ryssland i vilket det tydligt framgick vilken fin människa Karelin är. Där visades också hur Karelin stod i en snödriva och grillade korv på en klotgrill medan han drack en klunk vodka då och då för att hålla värmen, en typisk ägmästarsituation\ref{aegmaestare}.

\ditem[Ales Stenar]\label{ales stenar}
 är en fornlämning från Sveriges\ref{sverige} första lärosäte, långt innan Uppsala Universitet och folkskolereformen. Här fick man lära sig hur man gjorde sitt eget tråg\ref{traag}, hur man gjorde för att plundra, varför det är viktigt att slänga gamlingar utför ättestupan och andra viktiga saker för att klara sig förr i tiden. Själva stenarna tros ha använts till stenkrig på rasterna, en lek som ansågs mycket pedagogisk enligt dåtidens läroplan.

\ditem[Alice Tegnér]\label{alice tegnér}
Tegnér är ledare för den ljusskygga organisation som i folkmun har kommit att kallas Alice Tegnér-sällskapet. Det har länge spekulerats i organisationens utbredning. Envisa urbana legender om vilda inträdesriter samt infiltration av samhällsuppehållande institutioner förekommer, men mycket är nog bara snack. Enligt avhoppare ska Tegnér kommunicera med organisationens medlemmar genom infernalisk musik, som organisationen sedan studerar i bokform likväl som genom framförandet av sånger som förts ned i leden från ledare till medlem. Gällande detta musikaliska inslag finns vissa påtagliga bevis, där boken \quotetext{Nu ska vi sjunga,} som kan studeras bland annat på Karolina Rediviva och Kungliga Biblioteket är det mest framträdande.

\ditem[Alkisschäfer]\label{alkisschaefer}
 En alkisschäfer är en hund som inte är riktigt herrelös men heller inte har någon tydlig ägare. Vanligtvis saknar den någon av kroppens lemmar, typ ett ben eller ett öga. De flesta har lätt att vistas bland människor men vissa kan va lite folkilskna av sig och då måste man riva i. Alkisschäfern är vanligtvis kopplad, men det är inte alltid säkert att det är någon som håller i kopplet. Den är rätt skitig men glad. Typiska namn på en alkisschäfer är Pudas, Kari, Sam och Torax. Som vanligt är det inte hunden det är fel på utan ägaren. Namnet till trots kan en hund från nästan vilken ras som helt bli en alkisschäfer.

\ditem[Alkohol]\label{alkohol}
 är ett vanligt lösningsmedel. Det löser problem.
Utan alkohol kan man inte ha roligt, och har man roligt utan alkohol vore det roligare med alkohol. Inte minst enligt Boris Jeltsin.

\ditem[Alkoläskfylla]\label{alkolaeskfylla}
 Sittandes på en stol vid ett köksbord med chipsskål. Någon spelar en Gessle-låt på gitarr. Vem är hon med flätorna? Reda ut vem som kom först, Niclas Strömstedt eller Mikael Rickfors. Ingen vet. Resa sig för att gå ut på balkongen, till synes för att röka men egentligen för att prutta lite. \textit{Vafalls!} Dörrposten gungar. \textit{-Alkoläskfylla.}

\ditem[Allergi]\label{allergi}
 Identitetsmarkör hos stadsbarn. Tidigare kallat högfärdshosta.
I praktiken även ett fiktivt pseudomedicinskt tillstånd, ofta använt när en person till exempel är för lat för att promenera från en stadsdel till en annan. Då lämpar sig denna ursäkt utmärkt, ofta hänvisas då lämpligen till så kallad \quotetext{kattallergi}. Tillståndet används även som ursäkt vid omplaceringar av impulsinköpta husdjur, efter att man insett att exempelvis hundar måste rastas regelbundet och att hundvalpen förr eller senare blir stor och börjar äta skor. Man skyller gärna ogenerat på att exempelvis sambo eller barn är påverkat för att själv undfly allt ansvar. Medicinskt finns inga empiriska eller konklusiva studier som visar på existensen av detta tillstånd. Således rör sig detta med största sannolikhet om en påhittad åkomma i likhet med exempelvis magsjuka och migrän, de senare även de med ett brett användningsområde gällande undflyende av flera olika typer av ansvar.

\ditem[Allting]\label{allting}
 är på låtsas.

\ditem[Allväderstövlar]\label{allvaederstoovlar}
 Till skillnad från den av många älskade gummistöveln, de lite fräckare MC-stöveln och fuck-me-stöveln samt mer mystikomspunna varianter som såna där jättehöga stövlar som fiskegubbar har när de av nån anledning står mitt i en å och fiskar är allväderstöveln en stövel som skapats genom ett gediget allround-tänkande. Här har man satt funktionen först, men för den delen inte glömt att tänka på design, för att ta fram en stövel som är det självklara alternativet vecka efter vecka, året runt. Allväderstöveln värmer och skyddar mot väta, och tillåter således bäraren att kunna njuta av att delta i polarexpeditioner eller hoppa i vattenpölar utan att plågas av elementen. Dess onda tvilling är snowjoggern\ref{snowjoggers}.
Allväderstöveln förknippas ofta med Christopher Tolkien, JRRs\ref{j.r.r tolkien}
ohängde son.

\ditem[Alternativa namn på bakverk]\label{alternativa namn paa bakverk}
 Alla som hänger med i vad som diskuteras ute i Svea Rikes fikarum, mellan hyllraderna på din lokala Konsumbutik\ref{konsumbutik} eller på caféet vid torget kan omöjligen ha undgått diskussionen om att chokladbollen faktiskt, FAKTISKT!, heter negerboll. Alla som säger nåt annat är bara demokratihatare eller törra jävlar som borde slappna av och inte bli så provocerade bara för att någon lägger sig till med att säga N-ordet. Sen att man inte får säga vad man vill i det här jävla landet, det är en annan, om än tätt sammanflätad, fråga. Det har baskemig alltid hetat negerboll och det är alla etniska svenskar mitt i livets förbannade RÄTT att säga det. Nåväl, kan man tycka, men varför begränsa sig vid rasistiska tillmälen i samband med trefikat? Ta steg in i tvåtusentalet och välj ett lite mer intersektionellt spår när du beställer fikabröd.

Ska du ta in en bit prinsesstårta för att fira att Jimmie Åkesson gått ner i vikt? Prinsesstårta!? Prenumererar du på BANG eller? En bit Feministäckeltårta tack! Kanske något matigare, en källarfranska? FEL! Josef Fritzl-franska heter det såklart! Slappna av för fan. Dags för mer sötebröd, kanske en Napoleonbakelse? App app app, Adolf Hitlerbakelse ska det va, eller vill du vara politiskt \quotetext{korrekt}? Tönt. Ett wienerbröd med lite extra mormorshosta, kanske? Sieg Heilbröd med äckligt gammalt var och pojksperma!
Ska du prompt bete dig som ett kräk så ska det fanimig märkas ordentligt.

\ditem[Alvparty]\label{alvparty}
 Ett alvparty är ett party med alvtema. Ofta är gästerna utklädda till alver - man äter mat inspirerad av J.R.R Tolkiens\ref{j.r.r tolkien} böcker, och man har stövlar. Vissa gillar att bära en stav och att fästa latex på öronen så att de ser ut att vara spetsiga. Att memorera någon fras på \quotetext{alviska} ur någon rollspelsbok brukar vara uppskattat. Har man riktig tur får man ligga med någon, och inget säger att det nödvändigtvis måste vara med festens obligatoriska överviktiga gothare.

\ditem[Ambigram]\label{ambigram}
 Ett ambigram är ett ord eller fras skrivet med en symmetri som gör att det går att läsa även upp och ner eller spegelvänt. De tidigast kända ambigramen skapades i England av Peter Newell, illustratör åt Mark Twain och Lewis Carroll. Det är mycket svårare att göra ambigram än besläktade fenomen såsom palindromer och anagram. Det är även mycket psykedelisk.

\ditem[Amebix]\label{amebix}
 är inte världens bästa band. De är dock enhälligt framröstade till titeln: Världens ballaste band.

\ditem[Ana uvar i mossen]\label{ana uvar i mossen}
 \quotetext{Att ana ugglor i mossen} betyder att något lurt är på gång. \quotetext{Att ana uvar\ref{uv} i mossen} är att jordens jävla undergång är på väg, men omständigheterna kring den är lite vaga. Lite som Susan Sontag menar när hon säger att \quotetext{It isn't Apocalypse now! any more. It's apocalypse from now on.}

\ditem[Andörjan]\label{andoorjan}
 Västerbottnisk benämning på skidspår.
Används oftast när en något snabbare skidåkare kommer ikapp en något långsammare dito och tycker att den sistnämnda borde ge plats för den snabbare. Kommandot som den snabbare skidåkaren utbrister i är -He\ref{he} dej bortu Andörjan!! Fritt översatt: -Var vänlig och flytta dig ur spåret så att jag kan åka förbi! I södra Sverige används termen \quotetext{Ur spår!}.

\ditem[Angel of death]\label{angel of death}
 är en låt av det amerikanska thrash metalbandet Slayer. Låten skrevs i samarbete med Vägverket och syftar till att uppmärksamma trafikanter om vikten av att alltid ta hänsyn till den döda vinkeln vid omkörningar. I ett tidigt skede av låtskrivarprocessen var Dr. Alban påtänkt som gästartist, men denne skrev istället den egna låten \textit{10 små moppepojkar }där Kerry King\ref{kerry king} bidrar med gitarrpålägg.

\ditem[Anglosax]\label{anglosax}
 Modernt fiskeredskap som uppstod ur ett juridiskt kryphål i EU:s fiskedeklaration om förbud mot angling och gäddsaxar.

\ditem[Ankeborgslagstiftning]\label{ankeborgslagstiftning}
 Man äger allt som hamnar på ens gård.

\ditem[Ankfot]\label{ankfot}
 är en benämning på det stadium av yttre dekadens då ens strumpa har halkat ner på foten och nu sticker ut en bit utanför tå-raden, så att foten påminner om en ands fot. Detta fenomen är vanligt bland ensamstående medelålders män och deras tonåriga gelikar, men förekommer också i andra samhällsgrupper, om än med betydligt mer sparsam frekvens. Ankfot har blivit vanligare i senare tid i takt med att kvalitén i strumpresåren stadigt blivit sämre, medan antalet par strumpor i storpackserbjudanden blivit fler. Detta i full harmoni med det nyliberala\ref{nyliberalism} konsumtionssamhälle som tvingats fram på mer ordnade samhällsformers bekostnad.

\uline{Åtgärder}

Den som drabbas av ankfot har, lite förenklat, två huvudsakliga åtgärder att välja mellan. Det ena är dra upp strumpan så att den sitter som den ska och det andra är att helt sonika dra av strumpan och kliva omkring barfota. I ett längre perspektiv kan strumpor med bättre kvalité (t.ex. Biltemas) vara nödvändigt att inhandla, men detta är i första hand möjligt för den förvärvsarbetande medelklassen, varför krav på statliga åtgärder framförts från vänsterkanten av det politiska spektrumet. Ett annat förslag, som framförts av handarbetande miljöpartister\ref{miljoopartiet}, är att höja momsen på vanliga strumpor och reducera momsen för strumpor som är försedda med dragsko\ref{dragsko}.

\uline{Ankfot i danskt mode}

I Danmark\ref{danmark} har ankfot samma funktion som rökrocken har i resten av västvärlden. När dansken är hemma och njuter av ledighet klär hen av sig sina kläder och går omkring i tubsockor med en cigg i mungipan, petar lite i brasan och mår bra i största allmänhet.

\ditem[Annie Lööf]\label{annie loooof}
 Margaret Thatcher\ref{margaret thatcher} och Ayn Rands kärleksbarn. Obekräftade källor säger också att hon har en affär med Joseph Kony?

\ditem[Annika]\label{annika}
 är granne till Pippi Långstrump och syster till Tommy. Liksom honom är hon i jämförelse med den betydligt mer framfusiga och företagsamme Pippi lite av en mes.

\ditem[Ansiktsmålning]\label{ansiktsmaalning}
 är en konstform som främst brukas av tre grupper människor; små gulliga barn, djävulsdyrkande norrmän och fotbollssupportrar. På grund av de stora kulturella skillnaderna mellan grupperna har falanger uppstått i varje läger som hävdar att just de var först med att komma på fenomenet. Barnen hänvisar till de vetenskapliga rön som säger att alla är barn i början, något som främst upprör norrmännen eftersom de var mobbade under uppväxten och därför vill förtränga den. Fotbollsnördarna är ingen idé att fråga närmare för då utbrister de bara i ett rungande \quotetext{Vi æ røøøøøde! Vi æ hviiiide!!!}. Diskussionen fortsätter dock och en vacker dag kan vi förhoppningsvis uppdatera denna artikel med ett säkert besked.

\ditem[Anslagstavlor på lanthandlar]\label{anslagstavlor paa lanthandlar}
 På lanthandlar är det vanligt att det finns en anslagstavla som i bygden har samma funktion som agoran, det centrala torget, hade i det antika Grekland. Här postar du dina clip art-prydda turkosa A4-papper med ditt mobilnummer på så att alla som vill köpa din 5-hästars Evinrude, sett din bortsprungna katt eller behöver hjälp med trädfällning kan nå dig på mobilen. Finns det sommarstugor i området dit Stockholmare kommer när det är säsong kan du göra reklam för björkved på kubik för hutlösa summor. En sak som utmärker stockholmare är nämligen att de inte vet vad vedpriset ligger på, så här kan du tjäna en hacka.

\ditem[Anti-speciesism]\label{anti-speciesism}
 Att ogilla när folk säger \quotetext{Ditt jävla svin!}, \quotetext{Fan jag måste gå och räva!} eller \quotetext{Du är ju lika oattraktiv som en flodhäst}. Speciesism är strängt förbjudet på Folkkök\ref{folkkook} i Umeå\ref{umeaa}, och detta uttrycks \textit{med emfas} på en handskriven och vackert illustrerad lapp, så där får man hålla tand för tunga!

\ditem[Antikvärde]\label{antikvaerde}
 är den mystiska kraft som håller landets loppmarknader igång.

Några handfasta exempel:
\begin{itemize}
\item En Pelle Karlsson-lp\ref{pelle karlsson} som i praktiken har ett värde under noll ökar genom antikvärde till över nypris.
\item Chartersouvenirer från 70-talets Mallorca får ett värde i klass med Hope-diamanten.
\item Kopparbyttor
\item Shabbychic
\item Trälådor som det förvarats margarin eller socker i för ett halvt sekel sen.
\end{itemize}

\ditem[Antilop]\label{antilop}
 (\textit{Taurotragus oryx}) är ett enigmatiskt och mytomspunnet djur som vi vet mycket lite om men som ofta förekommer i folksägner och sagor. Den sägs inge en närmast spirituell känsla där den svävar på sina långa vingar över isbrytaren som enträget stävar fram genom Nordvästpassagen. Bristen på kunskap om djuret och de många myter som omgärdar det gör det svårt att med säkerhet uttala sig om dess naturliga beteendemönster. En stark hypotes är dock att den lever på slem och annat snusk och att den under den mörkaste årstiden lägger ett litet, litet ägg i en liten, liten korg som den försiktigt tar i näbben innan den beger sig iväg ut över de vita vidderna.

\ditem[Antiutilitarism]\label{antiutilitarism}
 är en filosofisk åskådning som är utbredd i ljudteknikerskrået\ref{ljudtekniker} och bland fascister med Dressmann-brallor. Du är på en spelning och just när introt på High on Fires \textit{Rumors or War} övergår i ett maffigt riff och sången kommer in, det vill säga det fetaste ögonblicket på hela spelningen bortsett slutet, materialiserar sig ofelbart en roddare eller ljudtekniker i en för ändamålet ägnad dörr på sidan av scenen. Han springer sedan hukandes gatlopp mellan bandmedlemmarna fram till någon liten mick som ingen ser eller hör och rättar till den eller vad han nu gör. Detta gör naturligtvis att den smocka av tung rock som bandet hade planerat att leverera helt uteblir och att ett ögonblick av förvirring istället uppstår. Det här är förstås inte första gången som den lite gråhårige mannen med Köttgrottorna-tshirt och Leatherman tagit udden av en spelning, och han beklagar i sitt inre att detta än en gång skett. Men om han måste välja mellan att förstöra spelningen för band och publik eller att den tredje hängpukan hörs lite dåligt, ja då väljer han fyrtiotusen miljarder gånger av hundra det förra.

Denna ofta förekommande situation har gett upphov till begreppet antiutilitarism, som benämner prioriteringar där allas sämsta väger tyngre än en situation som inte är absolut felfri. Inom politiken har Sverigedemokraterna gjort ideologi av detta folkliga fenomen. De har uppmärksammat en sociologisk motsvarighet till ett felvinklat mickställ – att det sitter en rumänsk tant med en pappmugg på torget i Bandhagen – och prioriterar då att arbeta för att skapa en raspolariserad polisstat. Står det mellan att acceptera att en halvkeff ståuppare står utanför donken och drar vitsar om en eller att rusa omkring med ett järnrör och sedan ansluta sig till kampen för att förvärra läget i Ungern – ja då är det naturligtvis det sistnämnda som gäller.

\ditem[Anton Abele]\label{anton abele}
 är en moderat\ref{moderat} riksdagspolitiker (sveriges yngsta!) vars hjärtefråga i valtid var att han var emot våld\ref{vaald}. Denna valkampanj sköttes på internet och när den avslutades hade Abele bytt hjärtefråga. Nu är han emot näthat. I hans ansikte sitter också Sveriges största haka\ref{haka (vanlig)}. Om dessa två stycken fakta har med varandra att göra låter vi vara osagt.

När Anton flyttade hemifrån höll det på att resultera i katastrof. Man hade helt enkelt förträngt hakan vid lägenhetsanskaffandet och flytten till hans första lägenhet, en etta, blev ytterst traumatisk. Som den äkta moderat\ref{moderat} han är så lyckades han dock köpa sig förtur i bostadskön och fick förstahandskontrakt på en 1,5a på Kungsholmen på rekordtid.

\ditem[Anton Lavey]\label{anton lavey}
 40-talistgenerationens Alexander Bard. Han hittade på en egen religion där han var bäst, nämligen \quotetext{den moderna} satanismen, som i princip är en plankning av vilket nyliberalt partiprogram som helst + cape.

\ditem[Apor vi minns]\label{apor vi minns}
 med glädje och ett visst vemod.

\uline{Herr Nilsson}

En något diffus liten apa iförd kofta. Nilsson hölls fången av en rödhårig riskkapitalist och tvingades bevittna ungdomlig baluns och dekadens. Koftan skvallrar om att Nilsson trots allt hade högre mål i livet men det är oklart var Nilsson tog vägen då hans matmor flyttade sin väska med guldpengar till nåt skatteparadis.

\uline{Apan Ola}

Apan Ola blev tv-kändis när han iförd blöja levde city-life i Stockholm. Han åkte sedan som många andra ungdomar på semester i Thailand. I likhet med dessa blev han sedan kvar, dock inte på Bangkok Hilton efter att ha försökt bära en väska med \quotetext{örter} genom tullen. Olas öde blev istället Bangkok Zoo. När detta uppdagats steg Olas popularitet till nära Noshörningen Nelson\ref{noshoorningen nelson}. Apan Ola har besjungits av Arsedestroyer.

\uline{Albert II}

Albert\ref{albert ii} var den första apan i rymden, men vad hade han för det?

\uline{Nicke Nyfiken}

Många barns första möte med en tecknad apa. Väldigt intetsägande personlighet och har förmodligen haft det jävligt tungt sedan strålkastarljuset slocknade. Nicke Nyfiken har aldrig besjungits av Arsedestroyer.

\ditem[Arbetarblåsa]\label{arbetarblaasa}
 En riktig arbetare pissar var tredje timme oavsett behov.

\ditem[Arbetarklassrock]\label{arbetarklassrock}
 är ett begrepp som betonar kvalité och äkthet hos rockmusik. Just därför utger sig många rockband för att spela arbetarklassrock medan de i själva verket inte alls spelar arbetarklassrock. Riktig hederlig arbetarklassrock kan låta lite hur som helst och återfinns över hela rockens spektrum; i Oi!, metal, gubbrock\ref{gubbrock}, sydstatsrock och så vidare. Gemensamt för många arbetarklassrockband är att de har minst en låt om hur mycket det suger att fira jul med släkten. Ett exempel på en sådan låt är The Warriors - \textit{Holiday Songs}

\uline{Exempel på arbetarklassrock}

\begin{itemize}
\item AC/DC
\item Thin Lizzy
\item The Last Resort
\item Bruce Springsteen
\item The Replacements
\item Creedence Clearwater Revival
\item Grateful Dead\ref{deadhead}
\item Rose Tattoo
\item Taste
\item Köttgrottorna
\item Motörhead
\item Lynyrd Skynyrd
\item Hurriganes
\item KSMB
\item Manowar
\item Snoddas
\item Slade
\item Jerry Williams
\item Troublemakers
\item Kicki Danielsson\ref{kicki danielsson}
\item Wo Fat
\end{itemize}

\ditem[Arbetslinjen]\label{arbetslinjen}
 är en tavla av den svenske skäggförsedde konstnären, sexköparen, examensfuskaren, låtsasforskaren och moderate\ref{moderat} politikern Sven-Otto \quotetext{Svotto}\ref{svotto} Littorin\ref{urin} (1966-). Tavlan har också av dess skapare försetts med ett slags tagline, som följer: \quotetext{Arbetslinjen: Det som skiljer det inrutade, kontrollerade från det fria, från egenmakten.} Tavlan föreställer ett Piet Mondrian-inspirerat men avsevärt mycket fulare rutmönster som med en tydlig diagonal linje (arbetslinjen) skiljs av från en vacker, solstrålande sommarhimmel. Det rutiga fältet i tavlans undre del representerar alltså det inrutade och kontrollerade som årtionden av arbetsplatskamp, fackliga aktiviteter och vänsterpolitik skapat, medan den övre delen representerar nyliberal politik och en avreglerad arbetsmarknad. Det är oklart om Littorin också vill peka mot den himmel från vilken hans idoler inom NATO fäller bomber över fattiga araber.

\ditem[Arbetsplatskamp]\label{arbetsplatskamp}
 Betydde förr att ge förmän och chefer kompanistryk. Betyder sen Saltsjöbadsavtalet\ref{saltsjoobadsavtalet} att sätta upp en \quotetext{Livsfarlig ledning}-skylt på chefens dörr.

\ditem[Arggissa]\label{arggissa}
 Ett fenomen som seglat upp på tapeten de senaste åren är fenomenet att \quotetext{killgissa.} Detta innebär att en man tack vare sitt patriarkalkulturellt uppbackade snubbpatos tillåter sig att i alla lägen uttala sig tvärsäkert om allt mellan mullvadshål och vad rymden egentligen består av.

Ett besläktat fenomen är att arggissa, vilket är en könlös företeelse. Arggissning är när en person ombeds att uttala sig om något den hatar. Låt säga att en person deltar i en dokumentärfilm om det lokala punkhuset. I intervjun arggissar personen att punkhuset aldrig fått en spänn av kommunen, eftersom personen har en negativ förförståelse av kommunal kulturverksamhet. Vad som är problematiskt med arggissningar är att de ibland inte stämmer överens med verkligheten. Till exempel kan det visa sig att kommunen i det hypotetiska exemplet ger ett bidrag på nästan 250 000 kronor om året till det lokala punkhuset. I såna fall är det lätt hänt att arggissaren känner sig som ett klantarsle\ref{praktarsle (negativ)}.

\ditem[Arkivskita]\label{arkivskita}
 Att arkivskita är att gå ner i arkivet på universitetsbiblioteket i Umeå och träcka på den fräschaste toaletten på campus. Det är alltid värt en omväg att gå till detta marint tematiserade avträde och vilskita\ref{vilskita}. Bara några meter från hemlighuset ligger dessutom barnboksavdelningen så ny förströelse finns alltid att tillgå. Om man lite snyggt vill meddela kollektivet att man måste besöka denna anrättning kan en säga \quotetext{Jag ska gå och arkivera en grej}.

\ditem[Armlängd]\label{armlaengd}
 Svårdefinierad måttenhet som är högst subjektiv då den alltid är lika lång som mätarens utsträckta arm. Således är en armlängd för dvärgen i Willow lika giltig som en armlängd för Boris Karloff. Kloakdjursvurmaren\ref{kloakdjur} tillika vetenskapsmannen Carl von Linné\ref{carl von linné} jobbade länge för att standardisera armlängden, där en armlängd för en man skulle vara lika lång som hans egen arm och en armlängd för kvinnor skulle vara lika lång som drottning Marie Antoinettes arm. Detta förslag gick aldrig igenom eftersom bakåtsträvaren Ludvig XVI vägrade låta Linné utföra den noggranna, avklädda fysiska undersökning han menade att uppdraget krävde. I Sverige motsvarar dock en armlängd 1,5 alnar på en icke-vanskapt fullvuxen människa - det vill säga, en människa som inte har t-rexarmar\ref{t-rexarmar}.

\uline{Armlängd inom sociologin}

Att hålla en person på en armlängds avstånd är väldigt stigmatiserande\ref{stigma}, och detta beror oftast på kön, klass eller etnicitet. På senare tid beror det på alla \textit{samtidigt}.

\ditem[Aron Jonason]\label{aron jonason}
 (1838 - 1914) var göteborgare\ref{gooteborg} nåt så jävulskt, och även Oscar II:s hovfotograf. Under en av upplagorna till brugd-racet\ref{brugd} Orust Runt där Jonason jobbade med att fota kungen i fördelaktiga vinklar råkade han uppfinna den moderna ordvitsen. På den här tiden var fotografiet beroende av magnesiumblixt för att få verkligheten att fastna på film, och dess sken blir mycket intensivt. Kungen störde sig på detta och utbrast \quotetext{Det var hemskt vad Jonason blixtrar mycket!} och Jonason replikerade \quotetext{Ja, blixtrar den ene så åskar den andre.} Detta blev också hans sista ord innan han slukades av en brugd, men då Jonason trots allt var människa och inte plankton klättrade han ut relativt oskadd. Hans riktiga sista ord var i själva verket \quotetext{Lite kaliumcyanid har ingen dött av!}

\ditem[Arselhaka]\label{arselhaka}
 Genetisk betingelse som är få förunnat men mångas dröm.

\ditem[Artur Hazelius]\label{artur hazelius}
 Artur Immanuel Hazelius (1833 - 1901) är en av de mest produktiva skojarna i Svea Rikes brokiga historia, men mycket av hans repertoar av jävelskap\ref{jaevelskap} har fallit i glömska på grund av svenskens oförmåga att ta till sig sin historia. Bland de som fortfarande har Hazelius färskt i minnet, och av dessa har samtliga lärt om honom postumt, vet man att berätta att Hazelius sällan eller aldrig sågs utan sin käpp, vilken ofta spelade stor roll i hans streck och skojarfasoner.

\uline{Barndom}

Redan i barndomen uppvisade Hazelius anlag för skojeri och båg. Redan som femåring ska han ha stulit en gardinstång från fattigstugan på Gärdet i Stockholm\ref{stockholm} och använt denna för att knacka på fönster på andra våningen runt om på Östermalm och på så vis skrämt slag på intet ont anande damer och herrar. Enligt ett brev från Hazelius till sin vän Anders Abraham Grafström (1790-1870) ska denna gardinstång ha legat honom så varm om hjärtat att han lät göra en promenadkäpp av den - och denna käpp var han aldrig utan i resten av sitt liv. Det berättas att han som tioåring ska ha använt käppen för att slå ner skatbon som han sedan placerade i skorstenar runt om i Gamla stan så att byggnader blev rökfyllda, ett spratt som i dagspressen omtalades som \quotetext{eld-och-svavel-eftermiddagen,} en referens till den bibliska berättelsen om Sodom och Gomorra vilket i sin tur medförde associationer till det faktum att Gamla stan på denna tid var centrum för prostitution och syndfullt leverne.

\uline{Ungdomsår}

Under tonåren blev Hazelius utskickad i riket av sin fader, officiellt för att besöka släktingar och lära sig om landets historia, seder och bruk, inofficiellt för att inte betunga sina arma föräldrar med mer socialt uppseendeväckande tilltag än han redan gjort. Hazelius tog detta tillfälle i akt och satte igång en spiral av skojeri och spratt. Redan på tåget till Borlänge ska han ha använt käppen sin för att slå av hatten på folk som stod på perronger utmed resans väg när tåget startade efter att ha släppt på och av passagerare. Enligt lokala tidningar ska dessa spratt ha åtföljts av tillrop i stil med \quotetext{Länge leve den danske kongen} och \quotetext{Klang i Storkyrkans klockor!} inifrån kupén. Trovärdigheten bakom dessa rapporterade citeringar bör ses på med viss skepsis, men antyder den infantila brist på koherens som Hazelius upptåg i ungdomsåren forfarande karaktäriserades av. Väl i Borlänge, sägs det, ska Hazelius med hjälp av sin käpp ha avlägsnat tuppen från Stora Tuna Kyrkas klocktorn och istället lagt dit en skjuten och avfjädrad fasan som han stulit från en av släktens vänner, Jägmästare Efrahim Andersson. Densamme utsattes för ännu ett spratt när Hazelius maskerad och iklädd militäruniform beordrade Andersson att stå vakt vid gärdsgårdsgrinden då armén enligt den förklädde Hazelius misstänkte en närstående anstormning av dansken\ref{danmark} och var underbemannad tills dess att man fick understöd från Stockholm. Andersson, som ömmade för sin byggd, ska ha intagit försvarsposition vid gärdsgårdsgrinden och stannat på sin post långt in på småtimmarna då han bara med ytterst stora mödor av socknens präst kunde övertalas att acceptera att han blivit utsatt för ett spratt och att den väntande anstormningen var ett påhitt, samt att prästen var just en präst och inte en dansk medlöpare.

\uline{I vuxen ålder}

Det är dock i vuxen ålder som Artur Hazelius skojerier når sin fulla komplexitet, för vilket han trots allt gått till historien. Hazelius blev nu lektor vid Stockholm Högre Lärareseminarium där Hazelius allt som oftast for med osanning och lade krokben för sina elever och kollegor medelst sin älskade käpp. Han författade här sin \textit{\quotetext{Swenska folkdräkter och lokala seder}}, ett hopkok av påhitt och lögner som man fortfarande in på nittonhundratalet använde som grundbok i etnologi och folkhistoria vid Uppsala och Lunds Universitet. Vidare ska han ha slagit sig ihop med en annan ökänd skojare, nämligen Gustav \quotetext{Frippe} Fredriksson, för att på Kungliga Dramatiska Teatern sätta upp en pjäs som annonserades som en odyssé genom svensk landsbygdshistoria. Vid premiären visade det sig dock att Hazelius och Fredriksson fyllt Dramaten med hönsfåglar och uvar\ref{uv} som satte skräck i Stockholmsfolket. Åter igen kunde Hazelius få sig ett gott skratt på sina medmänniskors bekostnad.

\uline{Ålderns höst och Hazelius död}

Hazelius skojerier fortgick fram till 60årsåldern, då han ska ha svindlat prinsessan Margareta i ett av Sveriges\ref{sverige} första telefonsamtal. Hazelius utgav sig då för att vara Emir av kalifatet och tillstod att han hade vägarna förbi Sveriges huvudstad på sin resa till S:t Petersburg. Han förslog att prinsessan skulle möta upp med pompa och ståt vid Skeppsholmens angöringsplats, men då prinsessan och hennes ekipage anlände fanns där bara Hazelius som ska ha dunkat till henne i ryggen med sin käpp, vilket påstås ha orsakat njurproblem som i sin tur föranledde prinsessans gulaktiga hy. Till slut fick man nog av Hazelius i Stockholm och Sveriges alla andra hörn som besökts av denne skojare och skurk. Han infångades, fråntogs käppen och avlivades genom att han slängdes i en djup brunn. Hazelius kropp fördes till Skansen, som Hazelius för övrigt grundade (liksom Nordiska Museet ett stenkast bort), och begravdes där under en hög med skräp\ref{skraep} i björngrottan.

\ditem[Asbest]\label{asbest}
 är en stad i den ryska oblasten Sverdlovsk, belägen strax intill Uralbergen. Med en befolkning på ungefär 70.000 människor motsvarar det en medelstor svensk stad eller årsförbrukningen arbetare i Apples kinesiska fabriker. Staden har också gett namn till en mineral som är bra till i princip allt: billig, beständig, brandsäker och estetiskt tilltalande. Med ett hus klätt i eternitplattor försäkrar du dig om en släktgård som kommer stå stadigt längre än akropolis, vilket förmodligen inte gäller för den ryska staden.

\ditem[ASEA-grönt]\label{asea-groont}
 Har du någon gång varit inne på kontoret hos en slipsnisse har du säkert noterat att stolstyget, skrivbordsunderlägget, kassaskåpet och skrivmaskinen går i samma pacifiserande nyans. Denna stavas NCS 6020 G10Y.

\ditem[Astronomi]\label{astronomi}
 (ej att förväxla med pseudo-vetenskapen astrologi) är en av världens äldsta vetenskaper och har under årtusenden utvecklats till en extremt komplex sådan, då det finns en bisarr mängd stjärnor och luriga rymdfenomen att hålla reda på.

I vardagssammanhang är astronomi mest användbart när man ska få folk att komma ner på jorden under högtravande diskussioner. Detta görs enklast genom att alltid ha med sig en karta över vintergatan som man kan ta fram när en väns resonemang börjar spåra ur och bestämt påpeka för henom att \quotetext{vi är såhär små, alltså såhär små...}. Det brukar fungera, ibland.

Ett annat vardagssammanhang astronomin lämpar sig i, är när man vill dricka öl och lyssna på Pink Floyds skiva \textit{Wish you were here}. Att sitta och pimpla bira själv och youtubea rymden med lite Floyd i bakgrunden går inte av för hackor.

\ditem[Atlantica]\label{atlantica}
 är namnet på en flera volymer lång rapport från ett kontroversiellt forskningsarbete av Olof Rudbeck publicerat under slutet av 1600-talet. Rudbeck argumenterar i verket för att Sverige\ref{sverige} är det kulturellt högtstående Atlantis som enligt många folksägner förstördes under förhistorisk tid. Bland annat påstår Rudbeck att den grekiska mytologins Herkules i själva verket var svensk och att namnet kom från svenskans \textit{Här + kulle}. Även om \textit{Atlantica} då den publicerades mottogs med entusiasm, speciellt inom Sverige, och fick bifall från svenska likväl som utländska skiftkunniga (vilka den alltid ödmjuke Rudbeck lät samla i ett band med titeln \textit{Testemonia}) tvivlar vissa forskare idag på Rudbecks slutsatser. Denna pessimistiska attityd hos humanister och samhällsvetare visar än en gång på behovet av en handlingskraftig och bestämd minister som Jan Björklund\ref{jan bjoorklund}, som kan vrida tillbaka utvecklingen till 1600-talet och återskapa det mer visionära klimat som rådde vid landets läroverk på Rudbecks tid.

\ditem[Att psykedelisera sin vardag]\label{att psykedelisera sin vardag}
 är ett häfte utgivet av Prof. Etiennes far Zarathustra Etienne under 60-talets allra gladaste hippiedagar\ref{hippie}. Skriften syftade dels till att öppna människors sinnen för \quotetext{softa prylar} och dels till att låta Etienne D.Ä. titulera sig författare så han kunde åka på uppläsningsturnéer med fri sprit och groupies. Häftet är slutsålt sedan länge och hela exemplar utan vin- och meskalinfläckar betingar idag höga summor på Ebay och Flashback. I Danmark\ref{danmark} fick texten sådant genomslag att anhängarna till Etienne D.Ä:s tankar länge hade status som minoritetsfolk. Statusen slopades 1996 när gruppen snarare ansågs utgöra majoritet.

\uline{Etienne D.Ä:s 10 hetaste tips till en mer psykedelisk vardag}

\begin{itemize}
\item \textbf{Gå passgång}. Även om det så bara är från soffan till klo\ref{klo} så låt inte konventionerna hindra dig från att spacea till din resa.
\item \textbf{Tilltala allt och alla med Herr Hermelin}. Tidigare generationer kan inte styra dina tankar. Vill du att allt ska heta Herr Hermelin så heter det så.
\item \textbf{Använd tejp istället för kniv och gaffel}. Be en kamrat vira lite gaffa runt dina händer med den häftande sidan utåt och frukosten blir sig aldrig lik.
\item \textbf{Sov i vatten}. Återgå till rötterna då vi alla var mikroorganismer som bara flöt runt och varken kunde tänka eller agera. OBS! Kom ihåg flytvästen\ref{flytvaest}.
\item \textbf{Ha fransar av matrialet mocka}. Alla ytor på din kropp är potentiella fästpunkter, liksom alla ytor på ditt eventuella fordon.
\item \textbf{Klä ut dig till ett djur}\ref{klae ut sig till ett djur}. Djuren tänker inte som oss andra. Dom gör som dom vill. Kura ihop dig bakom soffan med tigersmink i ansiktet och tänk att utrymmet är ett gryt\ref{foordelar med att bo i gryt}.
\item \textbf{Klä ut dig till Ray Jones IV}\ref{ray jones iv}. Han är lugn och samtidigt är han förmodligen kung. Res i dig själv genom att vara någon annan på samma gång.
\item \textbf{Använd stora mängder knark}. Gå till jobbet påtänd\ref{stenad}. Du har aldrig sett något liknande och det har inte dina arbetskamrater heller, vilket gör att revolutionen sprider sig direkt.
\item \textbf{Bli ett deadhead}\ref{deadhead}. Varje deadhead är ett mikrokosmos i det makro som stavas Greatful dead. Du är en del av ett annat universum men som lever här på jorden.
\item \textbf{Bli helt orimligt besatt av Kosta Boda-konstnären Monica Backströms glaskonst}. Din levnadsyta kan aldrig bli för full av lampetter, vaser, skålar och paraplystativ av glas - formade som svampar.
\item \textbf{Fantisera}. I fantasin bestämmer du allting. Vad som helst kan hända men vill du inte kan du bestämma att det inte ska göra det. Fuckin' groovy, man.
\item \textbf{Sitt }\ref{sitta}\textbf{ i saccosäck}. Säcken formar sig efter din kropp fastän den egentligen är en stol. Ungefär som att du skulle hoppa runt ett hörn.
\item \textbf{Tillverka ett par glasögon av två kalejdoskop}. Denna sorts glasögon kallas ofta för danska standardglasögon och är för vardagliga ändamål fullkomligt värdelösa. Men du är inte ute efter det vardagliga och med dessa glasögon har du alltid ett ständigt föränderligt mish-mash framför dig.
\item \textbf{Rulla upp en kanelbulle och använd den som skärp}. Mest för att det är ett ovanligt att göra ett skärp av kanelbullemateriel. Men också för att kanelbullen sällan används som skärp.
\item \textbf{Spela kristen folk-psych}. Det finns gott om religiöst folk som inte vill annat än att resa land och rike runt och sjunga om flummiga saker. De har sällan eller aldrig några invändningar när du föreslår att folk-psych kan vara den mest lämpliga formen för att föra ut ert (deras) budskap. Och att älskog är gudomlig njutning.
\end{itemize}

Som synes utgörs Etiennes 10 tips egentligen av 15 punkter men eftersom Jesus\ref{jesus} bara hade 10 budord ville han också ha det. Den som inte är psykedelisk nog att förstå denna förklaring är inte heller redo för frälsningen.

\ditem[Aurora]\label{aurora}
 är ett varumärkesskyddat namn på punksaft.

\ditem[Australien]\label{australien}
 är ett land i Australien. Ingen annan världsdel ville ha det så det fick bli en egen. Det utmärks främst för sin stora grad av konstighet och var för brittiska imperiet ungefär vad Danmark\ref{danmark} är för Norden.

\uline{Historia}

Ursprungligen befolkades Australien av ett storband som hette \textit{Aboriginerna} där alla medlemmar spelade psykedeliska dronetoner på didgeridoo, sittandes i en klippskreva i Ayers Rock. Ett vanligt rep\ref{repet} tog ungefär en vecka att genomföra och en konsert upp emot en månad. Dom var nöjda med sin tillvaro men plötsligt kom några engelska skepp seglandes på jakt efter en lämplig plats att dumpa alla jobbiga skottar\ref{skottar} som bara satt hemma och lirade drone på sina säckpipor. Dronemusik var nämligen jävligt ute i England. Så där satt man mest hela dagarna, tutande i sina lurar och njutande av långa släpande dronetoner, tills någon kom på känguruboxningen. Då började alla med det istället.

\uline{Ekonomi}

Australiens ekonomi är relativt god eftersom man har en skitstor ö med nästan alla typer av naturresurser och det inte finns något annat land på ön som kan kriga om det. I städerna jobbar de flesta människor med att köra trucks fulla med får och på landsbygden jobbar majoriteten med att skaffa sig en redig dagsfylla\ref{dagsfylla}. Den vanligaste valutan\ref{valuta} i Australien är AC/DC-plektrum, ett från Malcolm Young är värt 10 och ett från Angus 20. Wombats accepteras också på de flesta ställen och dom är värda 100. Exporten består främst av Vegemite, ett pålägg som görs av ett jästextrakt som uppstår som en slaggprodukt vid tillverkning av öl.

\uline{Geografi, klimat och miljö}

Alla typer av klimatzoner finns i Australien men det mesta är öken. För att det inte ska se så tråkigt ut ställer man ut en massa får där, det är det man har sina trucks till. Man har också små populationer av en massa konstiga djur som är till för djurprogrammen i TV. Landet är indelat i sex delstater utan att någon riktigt vet varför.

\uline{Fauna}

Australiens fauna karakteriseras av tre saker; farlighet, konstighet och svaghet. På Australien bor nämligen typ 90\% av alla världens giftormar, skorpioner, spindlar, havsanemoner och en massa djur som också finns på andra ställen men finns på Australien i en giftig variant. Här finns även världens enda giftiga däggdjur, nämligen näbbdjuret. Denna krabat representeras också den konstiga delen av faunan tillsammans med euchidnan, kängurun, koalan, tasmanska djävlar, emuer och kockaburrafåglar. Dessa djur utmärker sig med att de bara är konstiga, men precis som alla andra djur försöker de bara överleva och fortplanta sig, så låt dem va! Denna på gift och konstighet byggda fauna är dock lika stabilt som ett korthus i orkan. Kommer en annan art in i den australiensiska faunan så rämnar hela skiten. I resten av världen ofarliga djur som kaniner och oxgrodor har i Australien ätit upp grödor så att de hamnat på gränsen till svält. Det här beror på att de inte har några naturliga fiender på ön, och det här kan tyckas lite märkligt då alla av de jävulskt giftiga djuren borde kunna ta kål på en gullig kanin. Men icke.

\uline{Kultur}

Musiken i Australien består mest av \textit{AC/DC} men \textit{Midnight Oil} finns också. På idrottssidan har man uppfunnit en massa egna sporter för att man ligger så långt bort och inte orkar lära sig reglerna till dom som resten av världen tävlar i, inget land orkar ändå åka så långt bort för en landskamp. Landhockey till exempel håller nog ingen annan nation på med seriöst. Film finns inget att berätta om för det har dom inte kommit på där än. Detta är också anledningen till varför skådespelare som Russell Crowe, Nicole Kidman och allas vår Mel Gibson tvingats söka sig till andra delar av världen för att få ihop till lite bröd och mjölk att ställa på bordet.

\uline{Kokkonst}

I Australien grillar de inte räkor (shrimp) utan den lite större sorten som på engelska kallas prawns. De dricker inte heller ölet Foster's, utan dricker uteslutande Victoria Bitter.

\ditem[Avbolagisering]\label{avbolagisering}
 Återbördande av folkets stulna resurser.

\ditem[Axe]\label{axe}
 är en kroppsdeodorant speciellt framtagen för män. Den finns i utförandena \textit{Unlimited}, \textit{Touch}, \textit{Africa}, \textit{Pulse} och \textit{Phoenix}. Vilken som är den bästa Axe-sorten är inte tydligt och en gång för alla klarlagt. Alla måste göra sig sin egen uppfattning. Axe leverans i stryktålig aluminiumburk och sprutas på och i kroppsdelar som luktar illa.

\uline{Försäljningsstrategi}

Axes målgrupp är i första hand hockeyspelare\ref{hockey} och pojkar som går på högstadiet. Vad dessa har gemensamt är att de aldrig får ligga med frivilliga. Därför låter man från företagets håll bedyra att den som använder Axe genast blir ett glödhett sexmonster som ingen kvinna kan emotstå. Än är det inte vetenskapligt bevisat att detta är sant, så ord står mot ord.

\uline{Trivia}

Axe betyder yxa på engelska, men i England heter deodoranten Lynx, vilket betyder lodjur\ref{hur man ritar ett snyggt lodjurshuvud}.

\ditem[Axess tv]\label{axess tv}
 Tv-kanalernas Rusta. Borgerlig propaganda så odiskret att den får DDR-tv att framstå som saklig och objektiv.
Gratis bara för att ingen skulle betala för eländet.

\ditem[Axtorpet]\label{axtorpet}
 är vad Umeå-stadsdelen Berghems mest centrumliknande del kallas i folkmun, men är egentligen namnet på den spel-, porrtidnings-\ref{poorr} och tobaksaffär (billigaste stocken i stan!) som ligger där. Brevid spel-, porrtidnings- och tobaksaffären ligger en kinaresturang med en kapacitet för uppskattningsvis fyra pilsnergubbar åt gången, men på sommaren ställer man ut ett vitt platsbord så då får det plats sju gubbar (bordet står mot väggen). Bredvid kinesen ligger Vänsterpartiets lokal och ett äldreboende. Lite längre bort har vi två loppmarknader varav den ena säljer nazi-memorabilia och den andra hutlöst dyra (troligtvis stulna) cyklar.

\ditem[Ayn Rand]\label{ayn rand}
 Galen kärring som tyckte att folk skulle få göra som de ville förutom att betala skatt eller bry sig om andra människor på något som helst sätt. Hon ville t.ex. röka nåt så kopiöst och gjorde därför det, fick lungcancer och dog tack vare att hon stod utan allmän sjukvård. Poetisk rättvisa\ref{poetisk raettvisa}.

Anton Lavey\ref{anton lavey} sällade sig, tillsammans med bland andra Margaret Thatcher\ref{margaret thatcher}, till hennes mest namnkunniga beundrare. Frågan vem av de två som är ondast är en svårt nöt att knäcka.



%%%%%%% A %%%%%%%
\newpage
\null
\\
\null
\\
\Huge
B
\normalsize
\\
\null
\\
\null
%%%%%%% A %%%%%%%




\ditem[Backa med släp]\label{backa med slaep}
 Att backa med släp är en modern initiationsrit in i mandomen. Tricket är att släphelvetet svänger åt motsatt håll som bilen\ref{bil} gör när man backar\ref{backa om}, som är motsatt håll när man kör framåt, och tvärtom, beroende på perspektiv\ref{perspektiv}.

\ditem[Backa om]\label{backa om}
 är att ta en till portion av något man just ätit. Vanligtvis backar man om när man är hemma hos någon annan för att signalera att det smakade gott. Det är väldigt ovanligt att man backar om på lyxrestaurang för då blir det dubbelt så dyrt, och så mycket pengar är det inte många som har. Äter man däremot på rekorderlig lunchrestaurang där gästerna har byxor med mycket verktyg; är det närmast praxis att alltid backa om.

Är man hemma hos mormor eller farmor måste man backa om minst en gång, helst två. Att backa om har ingenting med bilar att göra, även om det förstås vore väldigt coolt.

\ditem[Backpatch]\label{backpatch}
 En backpatch är ett oftast rektangelformat tygstycke med tryck eller broderi som för tankarna till ett visst band eller skiva. Den ses vanligen fäst vid en jeansväst inuti vilken man allt som oftast hittar en hårdrockare\ref{haardrock}. Vissa, mycket få, backpatches uttrycker en åsikt, uppmaning (fuck the world) eller bärarens gängtillhörighet (Mölndal Hogs MC). De bästa backpatchesen innehåller orden \quotetext{Manowar\ref{manowar}}, \quotetext{Saxon}, \quotetext{AC/DC\ref{australien}} eller liknande och till yttermera visso en fantasieggande bild av en barbar med svärd och lättklädda ungmöer som klänger sig fast kring dennes bepansrade fötter. Kanske flyger han fram på en Harley Davidson med eldhjul? Endast fantasin sätter gränser.

\uline{Tillverka en egen backpatch}

Köp ett kilo bintje\ref{bintje} på ett välsorterat varuhus och hala ner kommunflaggan utanför kommunhuset. Skär försiktigt ut följande bokstäver ur potatisen: A, N, T, H, R, ett till A och slutligen ett X. Måla potatisen och tryck bokstäverna i ett hörn på flaggan, som du sedan klipper ut och syr fast på din jeansväst.

\ditem[Bacon]\label{bacon}
 är rökt och rimmat sidfläsk som vanligen steks i tunna skivor. Det förekommer i frukostsammanhang, framför allt i England och USA (då ofta tillsammans med stekta ägg eller äggröra), på pizza, kebab, tacos och i hamburgare.

Börjar man stekningen i sval panna får man mjukstekt bacon, men lägger man i skivorna först sedan pannan blivit ordentligt het, får man knaperstekt bacon. Om brandlarmet går vid stekning vet man att man lyckats.

\uline{140 g}

Det har länge dryftats om varför bacon säljs i förpackingar om just 140 gram.
En tänkbar förklaring är att det är nära ett tredjedels (svenskt) skålpund = 141,67 gram.
Det kan också vara för att 140g blir 7 skivor. 20 g styck. Det faktum att sju är ett primtal gör det ytterst svårt att dela förpackningens innehåll rättvist. Det är alltså ett sätt att tvinga konsumenten att köpa fler förpackningar än ett för att kunna dela rättvist om man inte är en eller sju personer i hushållet.
5 oz är ungefär 140 g. Bacon är något vi förknippar med den engelsktalande världen, inte minst USA. Därför är det troligt att vi även adopterat amerikanska mått vad det gäller förpackningen.

\ditem[Bade]\label{bade}
 Att bade är att bada genom att en yngre man slingrar omkring naken på en äldre mans rygg samtidigt som denne simmar som vanligt.

Exempel: \quotetext{Nä du Emil, nu går vi och bader!}

\ditem[Bailando]\label{bailando}
 är en belgisk ringsignal som satte halva Europa ur spel 1996 och ett litet tag därefter - nämligen den del av Europa som hellre grillar körv än botaniserar i musik. Det är mycket möjligt att den kommer att bli sjukt poppis bland hipsters inom en snar framtid så vill man investera har man chansen nu.

\ditem[Bajsalåkta]\label{bajsalaakta}
 är Sveriges\ref{sverige} högst belägna mulltoa på 1226 m ö h. Den är belägen på Låktatjokka fjällstation nära Björkliden, Norrbotten. Ett besök på denna är ofta en schysst omväxling från att bajsa bland stenrösen, en vanlig praktik i det stiglösa land som omger Låktatjokka. Dock ska det sägas att det även här drar lite om skinkorna.

\ditem[Bakficka]\label{bakficka}
 En bakficka är en ficka på baksidan av ett par byxor. Byxor med bakficka är vanligtvis av typen jeans eller chinos. Vad som gör bakfickan speciell mot andra fickor är dess förträfflighet som förvaringsutrymme under längre perioder. Medan andra fickor vanligtvis förvarar ett föremål under några timmar eller dagar kan bakfickan enkelt lagra saker i flera månader utan att det stör dess bärare.

Typiska föremål lämpade att förvaras i bakfickan för att alltid finnas till hands kan vara: viktiga papper\ref{viktiga papper}, fiskelina, en pennstump, plånbok\ref{haesthandlarplaanbok}, snusdosa och isskrapa.

För den som vantrivs i byxor rekommenderas i stället magväska för att enkelt kunna bära med sig ovan listade nödvändigheter.

\ditem[Bakisångest]\label{bakisaangest}
 Ahh, bakisångest. En grå slöja som läggs över ens ansikte dagen efter man haft en spritfylla\ref{spritfylla}. Allt man ser går i ett skimmer av menlöshet och det känns som att hjärtat halkar neråt i bröstkorgen, som en loska på en betongvägg. Man kan försöka trösta sig med att det bara är kemiskt (om man inte gjort något askefft), men det hjälper sällan. Ofta sitter bakisångesten i någon dag eller två efter att den fysiska bakfyllan upphört. Det finns inget knep för att undfly bakfylleångest. 

\ditem[Balticgruppen]\label{balticgruppen}
 är en landburen gren av de somaliska sjörövare som satt adenviken i skräck de senaste åren. Till skillnad från dessa somaliska släktingar har Balticgruppen Västerbotten\ref{vaesterbotten} och Umeå som huvudområde och istället för sjöröveri ägnar man sig åt entreprenad. Annars är likheterna slående: Likt de desperata unga män från Afrikas horn vilka ser pirateri som en väg ur svälten har man tagit 111.000 Umebor som gisslan och kapar med denna förhandlingsfördel till sig byggkontrakt och gåvor i form av rökelse och myrra framburna av Umeå universitets ledning. Balticgruppens tentakler når ända in i den politiska maktens toppskikt, det vill säga kommunalrådet Lennart Holmlunds\ref{lennart holmlund} bastu, i vilken lösensummorna kamratligt diskuteras medan khat-bladen går runt bland de inblandade och får dem att slutligen omfamna varandra under galna gapskratt. Stråtrövarbandets ledare är en enigmatisk figur som enligt sägnen går under namnet Krister \quotetext{Svartskägg} Olsson, som sägs rida fram på en vit häst med brinnande ögon. Och helvetet följer honom.

\ditem[Balutägg]\label{balutaegg}
 är en österländsk delikatess. Den görs av ett inseminerat, men inte färdigruvat ägg. Kort och gott ett fågelfoster. Traditionellt används ank- eller hönsägg, men vi på Nissepedia\ref{nissepedia} ser ingen anledning att inte göra balut av uv-\ref{uv}, svan-\ref{svan} och strutsägg.

\ditem[Banan]\label{banan}
 är en sorts frukt\ref{frukt}. Dess gula skal har kommit att bli en internationell symbol för slapstick-humor och har som sådan hyllats i kända pop-konstverk, skivomslag och tatueringar. 
Bananen hittade ursprungligen in på den internationella handelsmarknaden som en restprodukt av den geniala uppfinningen banankartongen.

\ditem[Bananas]\label{bananas}
 När man går bananas betyder det att man tappar sin skit och ballar/freakar ur. Man går bananas\ref{banan} när man blir extremt glad så man inte kan hejda sig utan i ett tillstånd av total eufori tappar koncepten och struntar i sociala normer. Ett lysande exempel på en man som gått bananas var Tom Cruise när han gästade Oprah Winfrey och hoppade upp och ner i hennes soffa för att han var så exalterad.

Inte bara människor kan gå bananas. Även händelser kan vara helt bananas. Exempel på en sån händelse kan vara en fest där alla har jättekul och dansar väldigt mycket. Om det kan man säga: \textit{\quotetext{Det var heeelt bananas i fredags!}}

Då man struntar i sociala normer när man går bananas kan det ibland kopplas till sinnesförvirring och vansinne. Vissa individer kan befinna sig i ett konstant bananastillstånd, varpå det upphör vara något roligt eller positivt och blir ensidigt obehagligt. När det inträffar blir handlingen att kalla någon bananas något nedsättande och ett sätt att ifrågasätta en persons mentala status. Om t.ex. forskare X i sin avhandling påstår att månen är gjord av ost och att fåglar är förklädda råttor kan forskare Y lätt avfärda dennes tes genom att till kollega Z säga något i stil med: \textit{\quotetext{Ähhh den där liraren är helt bananas. Jag har hört att han driver en kombinationsaffär. Jo, det är sant! Hälften skivnasare, hälften fiskmånglare! Inget han säger kan vara sant.}}

\ditem[Bar överkropp]\label{bar ooverkropp}
 alt. bar över, är att vara klädd i omvänd Kalle Anka\ref{kalle anka}. Det betyder att man har byxor men ingen tröja och är något som är förbehållet män, då det heter \quotetext{topless} för kvinnor. Det här är populärt främst bland män i femtioårsåldern som inte riktigt fattat att deras kropp inte åldrats särskilt väl - här är premissen ju fulare desto troligare. Det är också ganska vanligt bland yngre män som är berusade\ref{baersfylla} och bara känner att det vore lite \quotetext{skönt} att ta av sig tröjan. Det är nästan aldrig ok att ha bar överkropp bland folk. Är du osäker kan du gärna konsultera listan nedan.

\uline{Platser och tillfällen där det är socialt accepterat att ha bar överkropp}

\begin{itemize}
\item Inomhus i sitt eget hem
\item På sin egen tomt
\item På stranden eller kring poolen
\item När man rider på en häst, om man är Rysslands president, Vladimir Putin.
\item När ingen ser
\item När man lirar gura i High on Fire.
\item När man grovarbetar utomhus och är fackansluten.
\end{itemize}

\ditem[Barbados]\label{barbados}
 Den mixade doft av klor och urin som uppstår när man sitter flera timmar och super i en bubbelpool.

\ditem[Barn]\label{barn}
 Vad barn egentligen är satte länge griller i huvudet\ref{huvud} på glädjevetenskapen\ref{glaedjevetenskaper}, men nu är det allmänt känt att barn är som vuxna fast mindre och yngre. Barn är alltså små, små personer. De skiljer sig från dvärgar i det att de ofta växer till sig. Många barn har svårt att ta ansvar och brukar därför normalt inte förses med samma rättigheter och skyldigheter som andra. Än så länge får barn till exempel inte förvärvsarbeta, något som frihetsälskande moderater\ref{moderat} dock har för avsikt att ändra på å det snaraste. I linje med Fas 3 åläggs de dock inte sällan enklare sysslor, så som att måla toalettrullar och klistra fast fjädrar på påskris.

\uline{Kännetecken}

Barn är ofta småvuxna och har normalt nedsatt förmåga vad gäller bordsskick och de enklaste motoriska rörelser. Många barn uppvisar också stora svårigheter med att uttrycka sig i tal såväl som skrift. Liksom kvinnor saknar barn ofta skäggväxt. På plussidan kan sägas att de har en betydligt lägre tyngdpunkt än andra människor, vilket är bra när det blåser kraftigt.

\uline{Barn i mytologin}

Jesusbarnet känner många till. Jesus\ref{jesus} föddes uti ett stall och påstås ha legat i ett tråg\ref{traag} avsett för utfodring av åsnor. Också i Buddhismens kosmologi återfinns barnet, närmare bestämt i berättelsen om Buddhas födelse. Före sin upplysning ska Buddha, som egentligen inte alls är ett namn utan en beteckning på en upplyst människa, ha kallats Siddharta Guatama. Siddhartas mor ska enligt sägnen ha stått upp då hon födde sin son. Sonen föll således till marken, varpå han ska ha rest sig upp, tagit ett steg mot norr, ett mot söder, ett mot väst och ett mot öst och sedan sagt \quotetext{detta är sista gången jag föds,}vilket kan förstås som en kommentar föranledd av hans irritation över att ha fallit från moderns sköte ner på marken så att han blev vimmelkantig.

\uline{Barn i kulturen}

Popbandet Hanson består uteslutande av barn.

\ditem[Barnagans förträffliga pedagogik]\label{barnagans foortraeffliga pedagogik}
 är en bok av Prof. Etienne som utgavs av bokförlaget Rabén \& Sjögren 1998. Redan då boken enbart existerade i manusformat blev den omtalad då det kommit till allmänhetens (kulturelitens) kännedom att boken alls inte skulle ha med barnaga att göra utan istället var en lätt omarbetad översättning av Kenneth Grahames klassiker \textit{The Wind in the Willows} (Sv. Det susar i säven). Detta visade sig dock inte stämma. Boken är precis som den utger sig för att vara en handbok i barnaga och har ett långt appendix i vilket författaren argumenterar för ett utbrett bruk av denna traditionstyngda uppfostringsmetod.

\ditem[Barndom]\label{barndom}
 är en tid som den västerländske människan bör förtränga med alla till buds stående medel.

Det finns en mängd olika strategier för detta, såsom:

\begin{itemize}
\item Att sluta umgås med sina föräldrar
\item Att sluta träffa personer man lärde känna före levnadsår 14
\item Att flytta långt från uppväxtsorten
\item Sprit
\end{itemize}

\ditem[Barnuppfostran]\label{barnuppfostran}
 är en aktivitet där den vuxna gemenskapen undviker att ge ett barn uppmärksamhet förutom när den anses göra något fel. Det kan till exempel handla om att skälla högt med en sträng röst för att barnet ägnat sig åt osedligheter som att: stå upp när den äter chips, äta en bullbit som hamnat på golvet, upprepa namnet Sean Banan ett femtiotal gånger, försöka öppna ryggsäcken själv men råka gå på fel fack och så vidare och så vidare. Tanken med denna aktivitet är att skydda barnet från det alltid gäckande utanförskapet, en löst definierad samhällelig stämpel som gör det otroligt ansträngande och i många fall omöjligt för det senare vuxna barnet att bli en högpresterande producent med någon som helst möjlighet att garantera sig själv, sina nära och eventuella barn en dräglig materiell standard.

\ditem[Bartolomaios]\label{bartolomaios}
 är liksom Phillipos en av de mest anonyma av Jesu\ref{jesus} apostlar och förtjänar således en artikel på Nissepedia\ref{nissepedia}, denna outtömliga källa av svåråtkomlig kunskap. Efter att Jesus blev haffad av romarna ska Bartolomaios ha dragit till Indien för att missionera och, får man anta, vila ut sig lite efter ett par händelserika år. Att han valde att åka på semester till just Indien säger något om vilken anspråkslös och tillbakalutad person Bartolomaios var. Tyvärr blev han skinnflådd på vägen hem från Indien och detta blir inte mindre makabert av att han är skyddshelgon för bokbindare, garvare, handskmakare, läderarbetare och skomakare.

\ditem[Basist i refused]\label{basist i refused}
 är det tredje vanligaste jobbet i Umeå kommun, efter städare på NUS och lektor vid universitetet.

\ditem[Batyskaf]\label{batyskaf}
 (från grekiskans \begin{otherlanguage*}{greek}βαθύς\end{otherlanguage*} bathys, \quotetext{djup}, och \begin{otherlanguage*}{greek}σκάφη\end{otherlanguage*} skaphē, \quotetext{båt}) är en undervattensfarkost som uppfanns 1939 av Auguste Piccard och är en föregångare till moderna ubåtar. Den saknar maskineri för att kunna färdas åt sidorna och klarar bara av att åka ner och upp en gång innan den måste tillbaka till hamn för service, så den är inte så särdeles praktisk. Det som är bra med batyskafen är att den reglerar sin dyknivå med ballast i form av järnkulor och fotogen istället för komprimerad luft som i moderna ubåtar. Det gör att man kan dyka mycket djupare än med en ubåt, och 1960 gjorde en batyskaf det första av hittills två bemannade besök på jordens djupaste punkt, Marianergraven, 10911 meter under havsytan. Det fanns inte så mycket att se där så ingen åkte dit igen förens den där självgoda regisören som gjort Terminator och Titanic minsann fick för sig att han skulle ner. Vi kan bara beklaga att han inte stannade kvar.

\ditem[Beat]\label{beat}
 Lika delar sodomi, frijazz och opiater.

\ditem[Belgien]\label{belgien}
 är ett land i Europa som är känt för att vara världens sämsta land, något som gång på gång styrkts i empiriska studier.

\uline{Belgiens historia}

Belgien är Europas direkta motsvarighet till Afghanistan. Landet var i grunden inte relevant utan skapades enbart som en buffertzon mellan dåvarande stormakterna Frankrike och Nederländerna. Det är många som idag ångrar beslutet.

\uline{Belgiens politiska läge}

Belgien beskrivs dagligen av framstående statsvetare som ett politiskt moras. För er som inte vet vad det ordet betyder så är det synonymt med sump- eller träskmark. Grejen med politiken i Belgien är att det egentligen består av två länder, nämligen Flandern där de flesta pratar förflackad holländska och Vallonien där alla pratar franska. Det finns också en liten minoritet av tysktalande i östra Belgien, men de har inget att säga till om. Inte nog med att dessa länder skiljer sig i språk, de skiljer sig också i ekonomiska tillgångar, där Vallonien är fattigt och Flandern är rikt. När då dessa två länder ska försöka enas och tillsammans skapa en federal regering för att förvalta landet så går det rakt åt helvete varje gång. Belgarna (alltså flamländarna och vallonerna) bråkar något så jävulskt om allt mellan himmel och jord och kommer aldrig fram till något. Det kan vara så att det mesta beror på att de helt enkelt pratar olika språk, men man får hoppas att de i alla fall har tolk. I slutet av dessa skrikmatcher går alla hem och är skitsura inför nästa möte och så där håller det på. Det är inte helt omöjligt att vi i framtiden kommer att förpassa landet Belgien till historiens annaler, precis som Leopolds Kongo, och istället prata om Flandern och Vallonien.

EUs \quotetext{huvudstad} är dock densamma som Belgiens, nämligen Bryssel. Här möts de flesta av EUs institutioner för att fatta beslut, utom en av dem som träffas i Strasbourg.

\uline{Belgisk kultur}

Belgarna uppfann pommes frites, trots att fransmännen idogt hävdar att det var dom. Men Frankrike har bidragit med så mycket annat till världshistorien att de gott borde kunna unna belgarna detta. Tack vare belgarnas koloniala erövringar så blev de också hejare på att tillverka choklad. På det mer finkulturella planet så har Belgien levererat väldigt många kvalitativa tecknade serier, där Hergés Tintin kanske är den mest kända. Bland de kulturella avarterna bör det nämns att barnamordsfrekvensen är helt oproportionerligt hög.

\uline{Belgien som idrottsnation}

De är kassa på de flesta sporter, men landsvägscykling är de ena riktiga hejare på. Eddy Merckx, historiens genom tiderna bästa cyklist, är ju som bekant belgare.

\uline{Infrastruktur}

I Belgien finns det inte mindre än 7 kärnkraftverk. Varav fyra särskilt avsedda för att tillgodose landets energibehov för konsumering av internetporr.

\ditem[Belgisk jättekanin]\label{belgisk jaettekanin}
 är världens största kaninras och kallas därför också, kort och gott, Belgisk\ref{belgien} jätte. Arten avlades fram under mellankrigstiden av den Belgiska armén genom korsning av Bayersk vädur och den franska rasen Fauve de Bourgogne. Detta blev känt av den brittiska underrättelsetjänsten som under trettiotalet lyckades smuggla ut en individ\ref{individ} av rasen från en anläggning i Antwerpen. Dessvärre dog detta exemplar av stress\ref{stress} under resan över engelska kanalen. Kroppen obducerades för att eventuella ledtrådar till kaninens militära användning skulle kunna sökas, trots att dess indresserade beteendemönster gått förlorat, men till föga resultat. Efter andra världskrigets slut infångade Belgiska och Nederländska bönder\ref{boonder} ett antal exemplar som hade släppts ut inför anstående Nazitysk belägring och minst två exemplar hamnade så småningom i MI6s laboratorium, men inte heller dessa gav något svar på rasens gåta. Däremot blev ättlingar till de infångade individerna populära husdjur i Belgien\ref{belgien} såväl som i mer civiliserade länder runtom i Europa.

\ditem[Belgisk öl]\label{belgisk ool}
 Bortsett från blask såsom Stella Artois så är Belgisk öl god och ofta rysligt stark. Det är så på grund av Belgares vana att utöva sin kultur, alltså folkmord och pedofili. Detta skapar ångest och som alla vet botar man ångest med en bärsfylla\ref{baersfylla}. Man kan även grubbla över varför så många Belgiska ölsorter tillverkas i så kallade \quotetext{kloster}.

\ditem[Belka]\label{belka}
 var den andra hunden i omloppsbana runt jorden.

\ditem[Bellman]\label{bellman}
 Carl Mikael Bellman (1740-1795) var en svensk skald och föreståndare för det kungliga lotteriet. Han var bosatt i Stockholm\ref{stockholm} och ska ha umgåtts en del med en viss tysk och en ryss, och utöver det även med en viss frk Ulla Vinblad. Enligt utsago ska Bellman vid ett tillfälle ha diskuterat den överlägsna hastigheten hos riket Sveriges\ref{sverige} tåg med sina tyska och ryska vänner och då ha överdrivit något.

\ditem[Benny Bus]\label{benny bus}
 är en träskpunkare\ref{traeskpunkare} från Avesta i dalarnas län. Han tycker om att dricka mäsk och vara arbetslös. Vi på Nissepedia önskar Benny all lycka.

\ditem[Bensträckare]\label{benstraeckare}
 Slang för att ta en paus. Bensträckaren går ofta ut på att man som tjänsteman går till fikarummet, häller upp en cacao creme\ref{cacao creme}, sen går tillbaka till kontoret och spelar MS Röj i fyrtiofem minuter eller så.

\ditem[Berghem HC]\label{berghem hc}
 är ett fotbollslag hemmavarande på Berghem, Umeå. Det består av ett okänt antal mer eller mindre talangfulla, men framför allt spelsugna, män och kvinnor i åldersspannet 20-30 år, men laget är också öppet för förslag vad gäller spelare i andra åldrar.

\uline{Hemmaplan}

Lagets hemmaplan är Berghemsskolans grusplan, i folkmun även kallad Berghemsvallen. Lagets klubbstuga heter Rött\ref{roott} och dess sekretariat tillika kansli heter Hallonvägen 2.

\uline{Vunna titlar}

I Berghem HC vinner alla, alltid. Vill man vara petig har laget dock gripit följande åtrovärda pokaler:

\begin{itemize}
\item Silvermedalj i Frihetliga fotbollscupen 2013
\end{itemize}

\uline{Lagets filosofi}

Lagets filosofi är att \quotetext{passar du så passar du,} ett finurligt motto och ordlek där ordet passas båda betydelser nyttjas. Laget utövar strikt nolltolerans mot sexism, rasism och småborgerligt beteende. Omotiverat gap och skrik beivras.

\uline{Sticking it to the man}

Laget är förföljt och förtryckt av bylingen som inte låter laget klippa upp lås som använts för att låsa ihop målen på Berghemsskolans grusplan. Laget gör detta ändå, naturligtvis.

\uline{Support och tifo}

Laget supportas helhjärtat av det legendariska Oi!-bandet Skinned Alive, MC-outlawsen i Rainbow Riders och av lagmedlemmar spridda i diaspora.

\uline{Andra betydelser}

Skinned Alive har en låt som heter Berghem HC på sjuan\ref{sjua} \textit{Axtorpet\ref{axtorpet} Bootboys}.

\ditem[Bergslagen]\label{bergslagen}
 är ett område i Sverige\ref{sverige} tidigare helt finskspråkigt.

\ditem[Berguv]\label{berguv}
 ( latin: \textit{Bubo bubo} ) är enligt många den förnämaste av fåglar och är lika mytomspunnen som den är älskad och fruktad. Berguven är också den största uven\ref{uv} och kan bli upp till två meter i diameter.

\uline{Föda}

Berguven kan äta allt som andra djur kan äta, och mer därtill. Ofta har den setts flyga in på indiska restauranger och flyga iväg med en Tandoori eller Tikka Massala, men oftast äter den sådant som den kan hitta i naturen: Pinnar, grus och små djur står många gånger på menyn för denna fantastiska uv.

\uline{Rede}

Berguvens traditionella rede är som en stor plattform uppe på något högt och otillgängligt berg. Där ruvar den på sina ägg\ref{aegg} och ser ner över mänskligheten som hukar under dess penetrerade blick.

\uline{Häckning}

Berguven häckar en gång vart tredje år och får då tre ungar som den låter stanna i boet i tre månader. Sen ska de ut, så är det bara!

\uline{Berguven i populärkulturen}

Rockgruppen Sleep ägnade hela andra skivan, \textit{Holy Mountain}, åt berguvar och berguvsrelaterade teman. Ett exempel är textraderna: \quotetext{Ride the Eurasian Eagle-owl toward the crimson eye/ Flap your wings under Mars' red sky} från låten \textit{Bubonaut}.

\uline{Berguven i sportens värld}

En berguv kallad Bubi förhalade ett EM-kval i fotboll mellan Finland och Belgien\ref{belgien} med hela sex\ref{sexa} minuter. Detta genom att bara landa och sprida skräck, såsom uvar gör.

\uline{Berguven inom friluftsutrustningsbranschen}

Berguven var en tillverkare av ryggsäckar och tält, vars främsta egenskap är att de är ganska fula men håller i ungefär en aeon.

\ditem[Berätta]\label{beraetta}
 Att \quotetext{berätta} är ett annat ord för att skriva deckare. Deckarförfattare har ett medfött behov av att \quotetext{berätta}. De måste \quotetext{berätta,} annars stängs deras \quotetext{berättelser} in. Då kan de inte längre leva, utan dör. När de har börjat skriva (dvs berätta) tar texten över och författarna \textit{kan} inte längre styra den, säger de i alla eventuella intervjuer. I sina memoarer berättar de om deras fascination av \quotetext{ordet} som utvecklades redan i tidig ålder. Medan andra barn var ute och lekte läste författaren/berättaren i faderns bibliotek och \quotetext{orden} var hans eller hennes vänner, inte de andra idiotiska rackarungarna som ägnade sig åt att åka kälke.

\ditem[Beskinnad]\label{beskinnad}
 Beskinningen är ett extra hårt skinn.
Exempel saker som kan vara beskinnade.

\begin{itemize}
\item Druvor
\item Korvar
\item Näsor
\end{itemize}

\ditem[Besvikelse]\label{besvikelse}
 uppstår när något man hoppats på inte inträffar. Om man är en person som är attraherad av tjejer, kan det till exempel vara så att man sitter på en uteservering i en sydsvensk stad, med ett gäng balla tjejer och en trevlig killkompis. Under kvällens gång pratar man lite extra med en av tjejerna och kommer fram till att: \textit{\quotetext{det här verkar vara en tjej som inte skulle banga på att röka lite mary jane och dela ett sexpack med mig vid ett senare skede}}. Senare under kvällen visar det sig att hon inte alls bangar på något sådant, med undantaget att det inte är dig hon vill göra det med, utan någon annan.

I många fall, särskilt i samhällen präglade av en luthersk anda, leder ofta besvikelser till skam\ref{skam}, då man känner sig som en idiot som någonsin vågat hoppas på något. I särskilt svårartade fall, eller vid en lång serie av mindre besvikelser, kan även bitterhet\ref{bitterhet} uppstå.

\ditem[Bibeln]\label{bibeln}
 är en späckad faktabok om fårskötsel. Det haglar tips om hur herden bäst tar hand om sina lamm till vardags, men även hur han bör handla i specifika situationer såsom om en buske börjar brinna, man faller ner i en brunn eller om det regnar jättemycket. Boken är skriven av gud\ref{gud} och har sålts i fler exemplar än till och med \textit{Fass}.

\ditem[Bil]\label{bil}
 En bil är ett transportfordon som vanligtvis rullar på fyra\ref{fyra} hjul, drivs av en bensinmotor och har plats för en förare och fyra\ref{fyra} medpassagerare. Bilen uppfanns för ungefär 100 år sedan och dess popularitet växer fortfarande stadigt. Det finns väldigt många olika modeller av bilar och den snyggaste är Volvo 240.
Bil bör inte förväxlas med buss\ref{buss}, ett annat motorfordon som körs på hjul.

\ditem[Bilbatteri]\label{bilbatteri}
 är ett fiskeredskap som används av yrkesfiskare för att komma åt större mängder av havets läckerheter. Bilbatteriet slängs helt sonika i vattnet och efter någon vecka är det bara att ro ut och håva in allt som flutit upp till ytan. För större vattendrag med stor genomströmning, såsom älvar, kan det vara värt att slänga i en kvicksilvertermometer också. För att få fiska med bilbatteri krävs tillstånd från länsstyrelsen, men det är idag mest en formalitet som lever kvar från sovjettiden när man inte var riktigt säker på vad som fanns i importerade bilbatterier från östblocket.

\ditem[Bildekal]\label{bildekal}
 en är ett klassiskt verktyg för enkelriktad masskommunikation. Fram till internet\ref{world wide web} intåg var bildekalen det överlägset billigaste alternativet för att nå ut till tusentals människor. Avsändaren dikterar helt sonika sitt budskap på en plastremsa med klisterförsedd baksida och applicerar remsan på bilens bakre kofångare. Alla bakomliggande trafikanter kommer därefter att läsa, och med säkerhet också begrunda, den korta prosan. Sedan internet blev en del av var människans hushåll är det många som numera väljer att vädra sina åsikter där istället. Den sista stora bildekalsvågen i Sverige skedde 1994 efter folkomröstningen om EU och löd \quotetext{Skyll inte på mig - jag röstade NEJ}. Man kan fortfarande se en och annan \quotetext{Sänk dieselskatten!} och \quotetext{Skjut alla vargar - För djurens skull}, men det är inte alls lika vanligt. Föregångare till bildekalen var den skitiga bakrutan och före det standaret.

\ditem[Bilprovningen]\label{bilprovningen}
 är en institution som syftar till att skydda bilägande medborgare från sitt fordon.
Alla måste besiktiga vare sig det behövs eller ej, vanliga pantade knegare har inte vett att byta vindrutetorkare eller medföra varningstriangel, därför måste en betrodd kontrollant se åt så detta blir gjort.
Det finaste man kan få på bilprovningen är ett blankt protokoll\ref{blankt protokoll} och en mugg ljummet kaffe. Det sämsta är körförbud.

\ditem[Bilsupa]\label{bilsupa}
 Få saker är väl så fina som att sitta ned i en bekväm soffa med en bira i näven och AC/DC\ref{bonfire} på stereon. Det skulle väl i så fall vara att susa fram på en slingrande landsväg med en kvarting\ref{aaka vikingaskepp} innanför västen och Lemmy dånande ur stereon. Att bilsupa kombinerar dessa två aktiviteter på det mest angenäma sätt.
Bilsuparen tar plats i baksätet på en svensk-\ref{volvo 240-serien}, tysk- eller amerikansksnickrad bil där benutrymmet är väl tilltaget. Någon tvingas sitta shot gun där gemenskapen inte är lika stor, men detta kompenseras i och med närheten till stereon. För att bilsupningen ska fungera krävs också att en chaufför engageras för att ratta fordonet och lagen förbjuder strängeligen att denna person låter sig berusas. Detta gör att kamrater betrodda med körkort sällan stället upp så bilsupning är idag ett allt för ovanligt fenomen.

\ditem[Bintje]\label{bintje}
 (latinskt namn \textit{poop-orbis})är en potatissort som togs fram 1905 av den holländske potatiskonnässören Kornelis Lieuwes de Vries. Sorten är döpt efter de Vries kompis Bintje Jansma, vilket gör den till en av två sorters potatis som är döpta efter mansnamn (det andra är King Edward)\ref{king edward}. Bintje smakar nåt rent förjävligt om man inte dränker i ketchup. Dess mycket hårda konsistens gör dock sorten mycket lämplig till potatistryck\ref{potatistryck}, vilket också är dess främsta användningsområde. Bintje är skattefritt eftersom det är så äckligt.

\ditem[Biologer]\label{biologer}
 jobbar med vetenskapen biologi, det vill säga läran om allt levande. De kan det mesta om djur och växter och vet precis var olika utrotningshotade djur\ref{utrotningshotade djur} och växter lever. Kunskap är som många vet synonymt med makt och i biologernas fall är detta tydligare än mycket. Hur då undrar ni? Tänk er att en ny motorväg ska byggas och när alla beslut fattats, budget gjorts i ordning och alla maskiner just börjat hålla på. En biolog träder då fram och skriker: \quotetext{STOOOPP!}. Det visar sig att på just den platsen häckar den sista Brunfläckiga Skedstorken\ref{skedstork} i Europa. Bara att dra om vägen, ingen vill väl rubba den biologiska mångfalden? Tänkte väl inte det.

\ditem[Bitterhet]\label{bitterhet}
 är en värdelös känsla. Om man, för att ta ett slumpmässigt exempel, har suttit på en uteservering i en sydsvensk stad och kuttrat lite med en soft tjej, sen blivit skamsen\ref{skam} och besviken\ref{besvikelse} när hon drar med din polare, är det inte omöjligt att bitterhet uppstår. Vissa blir (helt felaktigt) bittra på kvinnosläktet vid såna händelser, men rent teoretiskt är det inte omöjligt att man i det här fallet blir bitter på sin polare istället och ba: \textit{\quotetext{Helvete! Han är både snygg, trevlig och spelar i ett band som är fräckare än mitt och har en massa balla skivor jag vill ha... satan! Jag ska ta de där jävla skivorna han bad mig ta med till distrot i Tjeckien och slänga i en flod... as!!! Åhhh den jäveln!!!!}}. Den sortens bitterhet är väldigt barnslig, då ens polare knappast kan hjälpa att den är snygg, trevlig och spelar i ett band som är mer populärt än ditt. I många fall går bitterhet över efter ett tag. Men en del personer, särskilt skitgubbar, har lätt för att göra bitterhet till sitt mest framträdande karaktärsdrag. Sådana bittra skitgubbar är rätt roliga på håll, men egentligen djupt tragiska, då bitterhet är en känsla som förhindrar en från att leva i nuet.

\ditem[Bjuddosa]\label{bjuddosa}
 En bjuddosa är en dosa med redan använda snusar i. Den används mest i nödlägen eller av människor som inte kan kontrollera snåltarmen\ref{snaaltarmen}.

\ditem[Bjudsprit]\label{bjudsprit}
 är en flaska som står längst in i barskåpet och åker fram när det kommer gäster. Gemensamt för alla bjudspritsorter är att de smakar vidrigt, t.ex. risbrännvin eller polsk whiskey. Smaken påminner gärna om lysfotogen.

\ditem[Bjudtermos]\label{bjudtermos}
 En bjudtermos är en kaffetermos med kopp i båda ändarna. Perfekt om man gillar sällskap.

\ditem[Björn (djur)]\label{bjoorn (djur)}
 är ett däggdjur som finns över nästan hela världen i två olika varianter: vanlig björn och cirkusbjörn. Ibland är dom bruna och finns i skogen, ibland vita på tundran, ibland uppstoppade bredvid eldstaden. Nästan all tecknad film gestaltar björnar som snälla och gulliga men det är helt fel. Går du fram till en björn kommer den garanterat att nita dig. Föreställ dig att gå runt på alla fyra\ref{fyra} i skogen en hel sommar och bara leva på kottar och bär så skulle nog inte du heller vara så glad. Eller att hela dagarna cykla runt på enhjuling jonglerandes med tre träskbabianer, ganska överskattat i längden.

När vintern kommer drar cirkusbjörnarna på sig grillorna och lirar hockey inför extatiska ryssar. Dom andra björnarna tycker vintern suger så dom sveper en dunk glykol och gräver ner sig i snön i ett halvår istället.

\ditem[Blandfylla]\label{blandfylla}
 En blandfylla uppnås genom att inmundiga rusdrycker utan att ta hänsyn till dess karaktär. Den som vill uppnå en blandfylla sveper starköl, torr cider, Riesling, brännvin, mellanöl, likör, Fernet Branca, rakvatten och rosévin utan hejd. Blandfyllan är ett tveeggat svärd, antingen spårar man ur och spenderar dagen efter med att spy tills man får näsblod, eller så händer inget spektakulärt alls. Var aktsamma!

\ditem[Blandsvulst]\label{blandsvulst}
 en är en svulst vars innehåll består av naglar och hår från ett foster som aldrig utvecklades till fullo. Blandsvulsten uppträder således på barn som skulle ha blivit tvillingar men som istället, i livmodern, absorberat det andra barnet. Tvillingbarn som absorberar sina tilltänkta syskon absorberar dess styrka och blir dubbelt så mäktiga som vanliga människor.

\ditem[Blankt protokoll]\label{blankt protokoll}
 Ett blankt protokoll är det finaste ett fordon kan föräras hos bilprovningen\ref{bilprovningen}. Dito anses dock ej fint som betyg från realskolan.

\ditem[Blaze Baylika]\label{blaze baylika}
 är ett kvinnonamn som ges särskilt partyglada och samtidigt råa flickor. För vissa slår det runt helt, men det kan ju hända alla människor oavsett vad man heter.

\uline{Kända personer med detta namn:}

\begin{itemize}
\item Agneta Blaze Baylika Paulo Coelho Sjödin
\item Vivienne Patricia Blaze Baylica \quotetext{Patti} Scialfa
\item Carola Blaze Baylika Maria Varg Vikernes Häggkvist
\item Pippilotta Viktualia Rullgardina Krusmynta Blaze Baylika \quotetext{Pippi} Efraimsdotter Långstrump
\end{itemize}

\ditem[Blekinge]\label{blekinge}
 är landets mest tätbefolkade landskap, sägs det. Vem bor då i Blekinge, kan man med all rätt fråga sig? Jo, majoriteten av Sveriges\ref{sverige} nynazister, that's who! Av ett märkligt sammanträffande\ref{maerkliga sammantraeffanden} kommer också Sverigedemokraternas \textit{übersturmführer} Jimmie Åkesson från detta germanska paradis. Många av bygdens söner och döttrar har sedan publicerandet av Olof Rudbecks \textit{Atlantica\ref{atlantica}} gått i polemik med denna Uppsala-son och hävdat att någon av öarna i landskapets vackra skärgård är den ariska rasens svunna Atlantis och alls inte Uppsalabygden som Rudbeck fräckt nog lät påskina. Som landskapsvapen har man ett ståtligt släktträd som heraldiskt representerar det Blekingska folkets genetiska utveckling. På grund av den edeniska atmosfär som råder har man problem med slaviska båtflyktingar som med alla till buds stående medel försöker ta sig till Blekinge och hotar att anhopa sig i kåkstäder utanför Karlskrona. Ett av de mest uppmärksammade exemplen härpå är den Sovjetiska U-båt som turligt nog gick på grund utanför landskapets kust och alltså inte hann lossa sin last av bleksiktiga karelska ryssar och andra svaga människor. Båten kunde sedemera avvisas å det bestämdaste av det lokala regementet där för övrigt merparten av Blekinges alla nynazister arbetar.

\ditem[Blomkålsöra]\label{blomkaalsoora}
 är ett muterat öra\ref{oora} som är vanligt hos brottare. Utövare av grekisk-romersk brottning får ofta huvudet nedtryckt i mattan, vilket gör att brosket i örat förstörs. Detta gör i sin tur att örat blir missbildat och lite påminner om blomkål, och \textit{voila!} - blomkålsöra!
Sveriges mest kända blomkålsöra sitter på sidan av Pelle Svenssons huvud.

\ditem[Blottare]\label{blottare}
En blottare är i nio fall av tio en man, som i tio fall av tio drabbats av galopperande storhetsvansinne. Likt Hitler blitzade sig genom 30- och 40-talets Europa, hemsöker blottaren cykelstråk och skogsstigar med sin aggressiva, påtvingande form av freikörperkultur. Forskning vill mena att blottare är människor med sexualpsykologiska besvär. Och det stämmer säkert. Det stämmer också att blottning är ett obehagligt sexuellt maktutövande. Men snarare än att härleda beteendet till en dålig relation patienten och dess mamma emellan, bör man söka den utlösande faktorn i blottarens bokhylla. Inte sällan dignar den av Nietzsche, Schopenhauer och Hegel på originalspråk. Där flockas också biografier över Bismarck och brittisk reselitteratur från 1895. På skivspelaren alterneras det mellan Nibelungens ring och Therion.

Varje brun trenchcoat som öppnas är (precis som Knausgårds författarskap) att betrakta som exempel på den tyska mustighetens frontalangrepp på allt vad hederlighet och måttfullhet innebär.

\ditem[Blåval]\label{blaaval}
 är katastrofer som inträffar med jämna mellanrum på grund av människors historielöshet. Allt går åt helvete då. Det senaste svenska blåvalet inträffade 19 september 2010.

\ditem[Blåvitt]\label{blaavitt}
 Det fanns en tid då livet var mer än att konsumera glättiga saker. En tid då konsumtion bara var ett nödvändigt ont för att hålla sig vid liv. En limpa var en limpa. Toapapper var toapapper. Utan några jävla lamm som skuttade runt på en äng i TV-reklamer för att få folk att gå på muggen oftare.

\ditem[Bockskäggsmetal]\label{bockskaeggsmetal}
 är en musikstil som är rätt trist.

\ditem[Boden]\label{boden}
 betyder botten på tyska och det är en rätt målande beskrivning av denna mänsklighetens bakgård.

Om svensk film behöver ett stycke miserabel betong väljer man Boden som inspelningsplats (sant!).

Man hade en gång ett sjukhus, 10 000 bassar och en travbana, nu är blott travbanan kvar och bara tills någon kommer på att man kan flytta den till Luleå. Det största som hänt i Boden på modern tid är att Lenin bytte tåg där 1917. Sen åkte han till Ryssland och gjorde historia. Som alla kommuner på dekis har Boden självklart ett äventyrsbad\ref{aeventyrsbad}, därtill en vilda västernstad där man kan dricka sig redlös och hamna i slagsmål, gärna efter travet.

\uline{Bodens Centrum}

Utanför Bodens bibliotek kan man träffa infödingar som köper tjack och gillar att åka moped. Längre bort i byn ligger Kafe Kaf som har helt okej mackor. Ännu längre bort ligger Rönnbäcks gatukök vilket saluför sin berömda flottbomb, vi talar alltså 2x250g, en kartong strippe samt all gegga du vill. Det är ett monster!

\uline{Bodens landsbygd}

Ända fram till den nesliga kommunsammanslagningen 1968 ingick den nordöstra delen av kommunen i Råneå landskommun och den västliga delen var självstyrande, Edefors landskommun. När det nya storboden skapades gick det raskt utför med dessa trevliga småorter.

\ditem[Boetius de Dacia]\label{boetius de dacia}
 var en kristen filosof som levde under senare hälften av 1200-talet. Boetius är en latinsk version av det nordiska namnet Bo och de Dacia betyder \quotetext{från Danmark\ref{danmark}}. Informellt, nordiska kristna filosofer emellan, kallades han Dannebosse.

Boetius dog i skam, efter att ha blivit utsparkad ur Rom. Hans exil berodde på att han utöver att vara kristen var naturvetare. Således sökte han tvångsmässigt logiska lösningar på allt möjligt. Eftersom logik och religion sällan går hand i hand uttryckte Boetius en del tankar som var, minst sagt, impopulära hos storfräsare\ref{storfraesare} som påven. Åsikten att döda inte någonsin kan leva igen fungerade dåligt i 1200-talets Rom.

Sedan Boetius de Dacias död har logiken varit tabu i konungariket Danmark. Skolbarn med nyfikna, utforskande sinnen som funnit logiska samband i sin vardag har i århundraden fått sig en dansk skalle av ansvarstagande vuxna, ackompanjerat av det danska ordstävet: \quotetext{\textit{Nej du raske lille dreng, som tænke sig sandheden se - hold du din kæft eller gå ud i eksil som han Dannebosse!}}

\ditem[Bohuslän]\label{bohuslaen}
 finns inte längre, upphörde att existera 1998. Det finns ett landskap kvar som heter Bohuslän men det är en annan femma\ref{femma}.

\ditem[Bonad]\label{bonad}
 En bonad är ett stycke broderad väv som hängs upp i hemmet, ofta ovanför dörrar, för att inge de boende och eventuella besökare en känsla av frid och lågintensiv glädje, eller insikten att i universums perspektiv vara blott ett futtigt sandkorn. Den typiska bonaden är rektangulär och kantad med ett blommigt mönster, vari en klok uppmaning, sanning eller betraktelse står att läsa i snirkligt utförda bokstäver. Likt en rustik fax\ref{fax} sänder alltså bonaden ut ett meddelande från brodör till mottagare.

\uline{Typiska klokheter som kommunicerats via bonader}

\begin{itemize}
\item Livet är för kort för att vara dammfritt
\item Gör ej det idag, som kan skjutas upp till morgondagen
\item Små små ord av kärlek / sagda varje dag / ger åt livet lycka / åt hemmet sitt behag
\item Man ska inte sörja det man saknar, utan glädjas åt det man har
\item Försumma inte det lilla du förmår för det stora du inte mäktar
\item Livet är inte de dagar som gått, utan de dagar man minns
\end{itemize}

\ditem[Bonfire]\label{bonfire}
 är en samlingsbox av det australiensiska\ref{australien} rockbandet AC/DC. Över fem\ref{femma} skivor får man ta del av bandets musik både live och i studio, utgivet och tidigare outgivet material. Boxen är en hyllning till bandets avlidne sångare Bon Scott, om vilken Malcolm Young en gång sa: \quotetext{when he\'s a fucking big-shot, he wants his solo album to be called \textit{Bonfire}}. Förutom musiken innehåller boxen också en poster, nyckelring, backstagepass och en bok full med häftiga bilder på Bon och bandet. Eftersom boxar är ganska dyra grejer är det inte alla som har råd att köpa såna, särskilt inte AC/DCs fans som alla är vanliga hederliga knegare. För att komma runt detta problem startade AC/DC den digitala musiktjänsten \textit{Spotify} där man kan lyssna på hela Bonfire gratis.

\ditem[Bonnseg]\label{bonnseg}
 (Adjektiv) beskriver någon som till det visuella mest ser senig ut men som i själva verket är senig och exceptionellt stark. Ordet kan ha uppkommit pga att bönder\ref{boonder} förr i tiden inte hade resurser nog att äta sig tjocka trots att de kroppsarbetade hela dagarna. Personer av rang som är eller har varit bonnsega:

\begin{itemize}
\item Kåra-Henrik, hovslagare.
\item Lisbeth Salander, mördare och datasnille.
\item Nestor Machno, en av arkitekterna bakom den anarkistiska rådsrepublik som uppstod kring 1920 i östra Ukraina genom handlingens propaganda, företrädelsevis till häst\ref{haest}.
\end{itemize}

Bonnseghet finns även i en internationell form, då kallad indianmuskler\ref{indianmuskler}

\ditem[Boris Jeltsin]\label{boris jeltsin}
 är namnet på en extremt prisvärd vodka man kan köpa i Tyskland.

\ditem[Bortamatch]\label{bortamatch}
 Man vaknar och funderar på var man är. Man letar ihop sina kläder, men glömmer minst ett\ref{etta} (1) plagg. Man möts av okända människor i hallen och slås av en oerhörd känsla av skam. Man letar sig ut ur huset och möts av en aldrig så irriterande sol. Man frågar en förbipasserande vad klockan är och vilken dag det är. Man använder sina sista vakna hjärnceller till att hitta hem. Man möts av ens kombos som kommer med gliringar och jobbiga frågor. -\textit{Bortamatch}

\ditem[Bosporen]\label{bosporen}
 är ett sund i Turkiet som delar Istanbul i två delar. Utan Bosporen skulle det inte finnas någon sjöförbindelse mellan Svarta havet och Medelhavet, så ni kan tro det är en viktig farled. Då den bara är 700 meter bred på vissa ställen så blir det riktigt trångt när alla båtar ska kajka fram och tillbaka. Antagligen har ett flertal olika grupper genom historien krigat om vem som kontrollerar Bosporen men det var för jobbigt att leta fram fakta om. Men förr eller senare brukar det ju bli krig om det mesta.

\ditem[Botte]\label{botte}
 är ett namn som ges till exceptionellt trevliga människor som har ett brett socialt nätverk. Det är inte så många som heter Botte heller så det är jättelätt att presentera sig. Säger man att man är Bottes pajk så är allt lugnt. Botte är en försvenskning av engelskans \textit{bottle} (butelj).


\ditem[Brakare]\label{brakare}
 En brakare är ett annat ord för en sedel\ref{valuta} med valören 1000 kronor och synonymt med tunka, lakan, lök, släng, tuss, och ett K.

\ditem[Brakflopp]\label{brakflopp}
 Någonting som går åt skogen.

\uline{Kända Brakfloppar}

\begin{itemize}
\item Carolas inhopp som sångerska i Marduk
\item Birgit Friggebo försöker \quotetext{få ner} stämningen under det infekterade PR-mötet i Rinkeby 1992, genom att mana till allsång av \quotetext{We Shall Overcome}.
\item Karl XII invaderar Ryssland
\item Microsofts lansering av Digerdöden 1340
\item Prof. Etienne\ref{prof. etienne} återplanterar ormar på Irland i ett misslyckat försök att ta över titeln som landets nationalhelgon.
\item Minitel
\end{itemize}

\ditem[Brandklipparen]\label{brandklipparen}
 var Karl XIIs häst\ref{haest}. Den ligger begravd med minnessten och allt utanför Ängsö slott där den dog efter att ha sprungit rätt in i en stenmur. På stenen står: \quotetext{Konung Carl XII:s siste häst stört anno 1740}. Klippare (är ett onomatopoeiskt ord och) betyder ungefär klappra, precis som engelskans \quotetext{clipper} och latinets \quotetext{clipperus}. Var hästen dessutom rödbrun till färgen var namnet givet. Han härstammade från Småland och fick inga kända avkommor.

En hyllningsdikt till Brandklipparen, skaldad av Carl Snoilsky:

\textit{Brandklipparn frågade ej: varthän?}
\textit{Och ännu mindre: varför?}
\textit{Men travade friskt bland snön i Ukrän}
\textit{Som på sin äng vid Kungsör.}

\textit{En häst, en häst att rädda vår kung,}
\textit{Ur hängbår krossad och stjälpt!}
\textit{Brandklipparn kom över rökhöljd ljung}
\textit{På honom vart kungen hjälpt.}

\textit{En sakta gnäggning han från sig gav,}
\textit{Det låg en klagan däri:}
\textit{Jag mäktar ej mera - herre, sitt av!}
\textit{Jag tror, att det är förbi.}

\textit{Kungen han dödde, han trillade av,}
\textit{Men havremoppen travade på.}
\textit{Av majestätet blev det bara skit kvar,}
\textit{Men pollen var glad ändå!}

\ditem[Brevlåda]\label{brevlaada}
 En brevlåda är en behållare, ofta  av plåt, som används för att förmedla budskap, korta och långa, över hela världen. Sändaren skriver sitt meddelande på ett papper, eller annat lämpligt material med en plan yta, och anger även en position för vart i världen den vill att meddelandet ska föras. En person i blå kläder tömmer varje dag behållaren och för sedan meddelandet till den angivna punkten och placerar den i närmsta postlåda\ref{postlaada}. Snabbt, enkelt, bekvämt!

\ditem[Brian Epstein]\label{brian epstein}
 (1934-1967) var en brittisk skivhandlare och manager. Han upptäcktes av Pelle Karlsson\ref{pelle karlsson} och deras samarbete ledde bland annat till att Pelle blev kontrakterad av Hemmets Härold\ref{hemmets haerold}. Vis av de erfarenheter han fått av Pelle gick Epstein sedan vidare till att sköta affärerna åt The Beatles och Garry and the Pacemakers.

\ditem[Bricka]\label{bricka}
 En bricka är ett slags plan skiva av trä, metall eller plast varpå saker ställs och transporteras. Brickan kan ha en kant som förhindrar att godset som transporteras på den faller av och därmed också att genanta situationer uppstår. De bästa och mest omtyckta brickorna är prydda med mönster, så som kurbits, eller bilder, som kan föreställa söta djur, barn\ref{barn} som pussas eller en vacker vy av äng och sjö. De sämsta brickorna är tillverkade på träslöjden och skänks bort som julklapp. Dessa brickor saknar ofta både bilder och mönster. Istället har skaparen av okänd anledning bränt in \quotetext{AIK} eller \quotetext{häst är bäst} i brickan med glödpenna.

\uline{Olika sorters brickor}

\begin{itemize}
\item Ostbricka
\item Mäklarbricka\ref{maeklarbricka}
\end{itemize}

\ditem[Brinner för att sälja]\label{brinner foor att saelja}
 Den som brinner för att sälja brinner också någon annanstans, vad det lider.

\ditem[Brismonstret]\label{brismonstret}
 är den ideella organisationen Barnens\ref{barn} Rätt I Samhällets maskot. Ingen vet varför BRIS valde ett monster som maskot, men en möjlig förklaring är att den kontrakterade reklambyrån bjuder sina kunder på spacecake när det är dags att välja bland deras förslag. Brismonstret har inget namn, dock ser det lite ut att heta något i stil med Reidar den Röde.

\ditem[Britts mode]\label{britts mode}
 är en anrik kvinnoklädsaffär\ref{kvinnoklaeder} i byn Rimbo\ref{rimbo} i Norrtälje\ref{norrtaelje} kommun. Britts mode tillhandahåller byxdresser, kjolar och blusar, samt klänningar till damer i den övre medelåldern som härstammar från glesbygden snarare än den mer mode-intensiva storstaden. I området finns också en sjundedagsadventistisk skola\ref{sjundedagsadventistisk skola} och förmodligen utgör dess kvinnliga gymnasieungdomar också en potentiell kundkrets, eftersom dessa förkastar samtida mode och, om man ska vara helt ärlig, samtiden generellt. Många Rimbo-kvinnor handlar dock ändå sina kläder från etablerade klädkedjor eller i någon av Britts bittra konkurrenters klädbodar, av en anledning som inte enkelt låter sig förklaras.

\ditem[Brobo]\label{brobo}
 En leva som brobo innebär att man bor med sin bro'. Detta levnadsskick föregår ofta vad som ofta kallas foebo.

\ditem[Brokiga Blad]\label{brokiga blad}
 var signaturen luffaren och konstnären Herman Sixtus Andersson (f.1873 i Norrköping - d. 1954 i Norrköping) ibland använde vid signeringen av sina tavlor. Anderssons mål med sitt konstnärskap var, förutom överlevnad, att måla en miljon tavlor så att alla i Sverige skulle ha varsin. Enligt Hönsalottas luffarmuseum hann han dock bara måla mellan 200 och 300.000 tavlor, samtliga med landskapsmotiv. För att hinna med att måla alla brukade Andersson måla ungefär 20 stycken åt gången. På så sätt kunde han grönmåla alla först för att få gräs och när den sista var klar hade den första torkat så att han kunde börja måla blå himmel. När djuren skulle in i bilden använde han schabloner och många tavlor kom därför att se ganska lika ut. Vinkeln på djuren kunde dock skilja sig åt för ju fullare han var, ju snedare lär dom ha blivit. Tavlorna kostade vanligtvis 2.50 styck. Istället för färg använde Andersson sockerlösningen Dextrin, strösocker, vatten och potatismjöl. Det kokades ihop till en sörja som han sedan blandade kulörpulver i för att få rätt färg. Strösockret hade han i då det gjorde att färgen torkade snabbare. Under sitt kringflackande liv lär Andersson ha mött både Anders Zorn och prins Eugen, men det är oklart om något utbyte i inspiration och teknik skedde.

\ditem[Brugd]\label{brugd}
 en är världens näst största fisk.

\uline{Taxonomi}

Brugden döptes av Carl von Linné\ref{carl von linné} till \textit{Cetorhinus maximus} efter det latinska översättningen av den gamla KSMB-dängan \textit{Glappkäft}. Den är ensam medlem i familjen \textit{Cetorhinus} och är jätteledsen över det här. Ofta simmar de runt och gråter, men i havet kan ingen se dess tårar.

\uline{Föda}

Brugden är enligt alla definitioner en haj, men om hajsläktet liknas vid olika subkulturer där vithajen är hårdrockare, hammarhajen skinhead och tigerhajen punkare så är brugden hippie. När dess artfränder simmar omkring och äter andra fiskar och ställer till med jävelskap\ref{jaevelskap} så simmar brugden omkring och gapar och äter plankton. Det kan också vara så att brugden fascineras asmycket över havet och helt enkelt simmar runt och gapar över dess storslagenhet, råkar på köpet svälja en massa plankton och har genom detta ödets nyck inte slagits ut av det naturliga urvalet. Där har den moderna vetenskapen, återigen, misslyckats.

\uline{Fortplantning}

När brugdar ska \quotetext{få till det} så simmar en av dem uppochner var på den andra liksom dockar sina könsorgan med den andras. Det är helt omöjligt för den andra att veta om det var skönt eller inte då brugden ändå gapar hela tiden. Att brugdar inte har utvecklat något språk är också en bidragande orsak. När brugdarna kopulerat färdigt skiljs de åt lite tafatt sådär och ses inte förän äggen kläckts. Brugdhanar är asdåliga farsor.

\uline{Brugden inom sociologin}

Socialfilosofen Anthony Giddens har liknat moderniteten vid en brugd, en kraft som ingen kan kontrollera utan bara går på utan en tanke på vad som komma skall. 

\uline{Brugden inom sporten}

I Mexiko - tuppfäktning, på Sveriges\ref{sverige} västkust - brugdrace. Tävlingen går ut på att två brugdar fångas in och spänns framför varsin luftmadrass. Vidare placeras en bohuslänning\ref{bohuslaen} som inte bryr sig om denne lever eller dör på madrassen och racet är igång. Det största brugdracet är Orust Runt där även hinder i form av flytminor placerats ut. Trots att djurrättsaktivister försöker stoppa tävlingen varje år har den hållit på sedan den uttalade brugdrace-entusiasten Oskar II:s tid.

\ditem[Brugdguldet]\label{brugdguldet}
 delas årligen ut till en person eller grupp som åstadkommit en stark prestation relaterad till näringsintag, eller någon annan extraordinär prestation som för tankarna till brugden\ref{brugd}, denna världens näst största fisk.

\ditem[Bruksgök]\label{bruksgook}
 En bruksgök är en tam eller halvtam gök som hålls för vissa bruksändamål. Vanliga sådana ändamål är:

\begin{itemize}
\item Tidtagning
\item Rensning av avlopp
\item Underhållning
\end{itemize}

\uline{Pris och prestanda}

En bra bruksgök med god stamtavla pensioneras vanligtvis efter ca sju år, men i vissa fall har bruksgökar varit arbetsaktiva så länge som tio år. Priset på en vaccinerad bruksgök med veterinärbesiktning låg i Augusti 2011 på mellan 100 och 150 kr, men sjönk under slutet av 2011 och början av 2012 som en effekt av oroligheten på de europeiska finansmarknaderna. Inte ens denna näringsgren går opåverkat ur oroligheterna på världens börser.

\uline{Skötsel}

Används bruksgöken för att rensa avlopp bör den spolas av ordentligt efter varje brukstillfälle. Annars räcker det med en gång i månaden. Kasta åt den lite cornflakes en gång om dagen så blir det bra. Vatten kan också vara nödvändigt, men den behöver inget fint bordsvatten, utan vanligt krandito fungerar utmärkt.

\uline{Vanliga namn på bruksgökar}

Fartyg döps ofta efter kvinnor, brukgökar ofta efter arenarockband och isländska skalder som:

\begin{itemize}
\item Egil Skalagrimsson
\item U2
\end{itemize}

\uline{Synonym}

Bruksgök är även en arkaisk fras, synonym med fenomenet vi idag kallar produktionsknull\ref{produktionsknull}.

\ditem[Bruksortskäng]\label{bruksortskaeng}
 är en speciell sorts kängpunk, geografiskt specifik för mellansverige\ref{bergslagen}. Än mer specifikt kan genrens kulturella center preciseras till vad som inom akademiska kretsar kallas bruksortskängtriangeln. Bruksortskängtriangelns noder är Hedemora i norr, Örebro i sydväst och Eskilstuna i sydost. Inom denna triangel har all relevant bruksortskäng någonsin producerats. \textit{Meanwhile}, \textit{Asocial}, \textit{No Security}, \textit{Svart Parad}, \textit{Crude SS}, \textit{Uncurbed} m.fl. har alla haft sin bas i den mellansvenska myllan.

Vad som skiljer bruksortskängen från vanlig jävla käng är svårdefinierat. Rent generellt handlar det givetvis om d-takt och risig produktion - business as usual så att säga. Kanske är det den stora finska diasporan inom triangeln som bidrar med ett mystifierande \textit{je ne sais quoi} till evergreens som \textit{Offer för ett mord} av \textit{Svart Parad}? Eller bör man söka sig bortom kultur och gå till materia och ekonomiska betingelser, som Marx hade påbjudit? I såna fall finner vi anledningen till bruksortkängens överlägsenhet, i att denna triangel är den del av Sverige som blivit mest brutaliserad av modernitetens framfart (näst de stackars Baggböleborna, men det resulterade inte i nån bra käng). Lika snabbt som löften om guld och gröna skogar kom, lika kvickt ryktes de undan av rikskapitalister som flyttade sina affärer till utvecklingsländer på 80-talet. Den besvikelse och desperation som präglar bruksortskängtriangeln, har onekligen färgat musiken som kommer därifrån.

\ditem[Brumma]\label{brumma}
 är slang för skita och härstammar från den oförglömliga badrumsbombscenen i \textit{Dödligt vapen 2} där Mel Gibson frågar Danny Glover, som sitter på toa, vad han håller på med. Danny svarar då, lite nonchalant, \quotetext{jag brummar}.

\ditem[Brunka]\label{brunka}
 (verb, obestämd form singular; bestämd form brunk) är fenomenet att masturbera efter avklarad avförings-session, gärna av ren och skär uttråkning. Några få individer har berättelser om hur de utfört dessa handlingar samtidigt, men det har ännu inte styrkts av SIFO.

\ditem[Bruvd]\label{bruvd}
 en är ett fasansfullt kreatur som är en korsning mellan en brugd\ref{brugd} och en uv\ref{uv}. Brugdens gigantiska gap och kroppstorlek har kombinerats med uvens förmåga att flyga och även deras onda sinnelag. Ser du en bruvd är det kört. Du kan försöka springa, men enda skillnaden är att du dör trött.

\ditem[Brännvin]\label{braennvin}
 är ett urtida bedövningsmedel. Det är även ett mycket bra lösningsmedel, löser fett utan problem. Det är också bra på att lösa problem, dock i mindre utsträckning än att skapa dem. C2H5OH är det som bör förtäras, andra varianter kan ge vissa oönskade bieffekter, som t.ex. CH3OH som kan leda till en formalinindränkt död. Introducerades i Sverige\ref{sverige} av Eva Ekeblad\ref{eva ekeblad}.

\ditem[Bröka]\label{brooka}
 (verb, obestämd form singular; bestämd form \textit{brök}) är en multipel handling där utövaren njuter ett rökdon samtidigt som den träckar. Fenomenet har blivit mindre vanligt på senare tid i och med utedassens dalande popularitet och rökförbud på krogen.

\ditem[Bröstarkt]\label{broostarkt}
 är ett ord som beskriver något som innehåller mycket bröd. En macka kan till exempel upplevas som bröstark om man har för lite smör, ost och skinka på den. En pizza kan också vara bröstark om man har för lite tomatsås, ost och skinka på den.

\ditem[Bukfylla]\label{bukfylla}
 är mat som ger valuta för pengarna utan särskilt mycket glädje. Bukfylla är gärna brödstark\ref{broostarkt} och inköpt i stora kvantiteter. Bukfylla skall inte blandas ihop med Bärsfylla\ref{baersfylla}.

\ditem[Bull]\label{bull}
 \quotetext{Fan va bull!} kan man höra Malmöiter utbrista. Det betyder att något är tråkigt och/eller dåligt. Hos andra folkslag är \quotetext{bull} en förkortning av engelskans \quotetext{bullshit}, som betyder ungefär \quotetext{snickelidinkelsnack}, men den betydelsen är naturligtvis högst perifer.

\ditem[Burre]\label{burre}
 är en karaktär i tidningen Bamse av Rune Andreasson. Burre är en humanoid hund med randig tröja och lila byxor med dragsko\ref{dragsko}. Han är lite av en bråkstake och strulputte\ref{strulputte}. Han är lätt missanpassad och kolerisk och har fattiga föräldrar som antyds ha alkoholproblem. Samtidigt som Burre har en aggressiv framtoning och ibland uppvisar ett oacceptabelt beteende kan man ana en komplex djup-psykologi hos honom. Det närmsta en vän han har är Nalle-Maja, med vilken han har ett slags motsägelsefullt och distansierat förhållande. Han har också uppvisat ett genuint medmänskligt beteende vid flera tillfällen. Denna komplicerade och ambivalenta natur är kanske det som skapar Burres magnetiska kraft och han fortsätter att fascinera nya generationer av läsare. Bland de som tagit Burre till sitt hjärta återfinns Thorsten Flinck och Jessika Gedin.

\ditem[Buss]\label{buss}
 En boll av snus, älskad bland bandyspelare.

\uline{Andra användningar}

Ett slags korvformad bil för kommunal transport av bandyspelare.

\ditem[Buss 85]\label{buss 85}
 var den sista av Länstrafikens bussar\ref{buss}, och all annan allmän kommunikation, som trafikerade Kärrgruvan. Turen utgick från Lilla Heden, svängde in på Linnévägen och sedan vidare mot Norberg, Fagersta och resten av världen\ref{bergslagen}. Numer finns ingen tidtabell ut ur Kärrgruvan. Kommunikationen är bruten och allt återgår så sakteliga till vad det var för 200 år sedan.

\ditem[Busskruven]\label{busskruven}
 är ett sätt för bussresenärer\ref{buss} att signalera till övriga passagerare att hon/han ska stiga av vid nästa hållplats. Den går ut på att man plockar ihop sina tillhörigheter, sätter eventuell väska i knät, sträcker ut kroppen och tar ett stort andetag, tittar sig omkring samt antar ett lätt nervöst uttryck i ansiktet. Den genomsnittliga busskruvningen sker ungefär 40 sekunder före avstigning, eller utifall resenären sitter inklämd mot fönstret ca 2-3 minuter innan. Busskruven står för ungefär 85\% av all icke-verbal kommunikation i Sverige\ref{sverige}.

\ditem[Bärsfylla]\label{baersfylla}
 En riktig bärsfylla går till på följande sätt: Efter att ha kommit hem från Systembolaget på fredag\ref{fredag} eftermiddag (om du pluggar gick du hem tidigt för att du kan, om du jobbar på fabrik låter du din vän stämpla ut dig när hen går hem) knäpper du en öl. Den första är med fördel en mindre, lätt variant, typ en Andersson. Du kollar på någon rolig serie och fnissar, alternativt sitter på verandan och finljuger\ref{finljuga} till tonerna av CCR. Middagsdags nalkas och du knäcker nummer två; krokodilen, Benny Bus\ref{benny bus} favoritöl. Krokodilen avnjuts samtidigt som du fixar en flottig paj\ref{paj} och lyssnar på Deep Purple på helgvolym\ref{helgvolym} i köket. Orgelsolot i \textit{Speed king} har aldrig låtit bättre. Efter att ha svullat din paj och druckit minst en illeröl (5.2\ref{tvaaa} i folkmun) kommer ditt kompisgäng över. Ni dricker, röker cigg och pratar, högljutt. Med fördel spelas \textit{Anti-Cimex} - \textit{Scandinavian Jawbreaker} på maxvolym. Efter att ha suttit och låtit alkoholen stiga er till huvudet\ref{huvud} stressar ni ut på lokal.

På lokal dansar ni till skämmig musik, köper minst två stora stark, skriker högljutt och kanske ber någon att dra åt helvete. I den här fasen når bärsfyllan sin peak. Någon som är bärsfull delar under klimax många symptom med en person som är spritfull\ref{spritfylla}, men den bärsfulle beter sig någorlunda rationellt, vilket den spritfulle sällan gör.

När du har tagit taxi hem med dina polare (efter ett snabbt och oerhört kostsamt stopp för folköl på Statoil), följer en extremt lam, men seglivad efterfest. Du däckar klockan sju och dagen efter har din avföring karaktär av jäst, malt och humle.

\ditem[Bönder]\label{boonder}
 kläcks ur de stora vita ägg\ref{aegg} som brukar ligga på olika svenska fält under slutet av sommaren. Alla hanar kläcks iklädda jeansoveraller\ref{kanadensisk frack} men alfahannarna får med sig en EPA som de kan köra direkt. Honor finns men är ytters sällsynta


%%%%%%%%%%%%%%
\newpage
\null
\\
\null
\\
\Huge
C
\normalsize
\\
\null
\\
\null
%%%%%%%%%%%%%%



\ditem[Cacao creme]\label{cacao creme}
 är den finaste drycken man kan få ut ur en kaffeautomat av märket Wittenborg. Kaffet i en sådan automat är alltid äckligt, tevattnet ljummet och wiener mélangen skrattretande. Cacao cremen är vad som räddar tiefikat, lunchen och trefikat samt alla eventuella bensträckare\ref{benstraeckare} däremellan.

Cacao creme består av mjölk, chokladpulver och vatten, tillsatt i den ordningen. Om man är ett riktigt proffs drar man undan sitt dryckeskärl från automaten under de sista två sekunderna av upphällning då det under de sekunderna är vattnet som tillsätts. Vattnet undviker man för att ge en fylligare, mustigare upplevelse. Detta knep tas ofta till av de riktigt sura gubbarna på jobbet som egentligen vill dricka kaffe men blir för risiga i kistan av det och därför skiter i allt och bara vill hinka i sig chokladsörja hela dagarna.

\ditem[Calskrove]\label{calskrove}
 Ett mellanmål i Skellefteå. Carskroven är en inbakt pizza, men till skillnad från sin vanligare kusin Calzonen, innehåller den inte ost och skinka utan en komplett hamburgertallrik med strips och allt. Den är även nära besläktat med svanskrove och transkrove och serveras ofta som en vardaglig variant av dessa mer högtidliga rätter.

\ditem[Calzona]\label{calzona}
 (Verb, infinitiv) är att klippa upp en dubbelvikt pizza, en s.k. Calzone, i ena änden och sedan trä den över en annan persons huvud, likt en ostfylld dumstrut. Detta är en populär bestraffning mot personer som inte är från Västmanland och flörtar med lokalbefolkningen på Green Man i Fagersta.

\ditem[Carl von Linné]\label{carl von linné}
Blom- och djurkillen, (före adlandet 1757 Carl Linnæus, Carolus Linnæus), föddes i Råshult, Småland.

\uline{Tidig barndom}

Som barn beskrevs Carl som oerhört vetgirig. Tidigt började han dissekera smådjur med sin första pennkniv, som han kallade \textit{Gram}, efter det svärd Sigurd använde för att dräpa draken Fafner i Völsungasagan. I historieskrivning om Linné har detta väckt debatt. Var Linné en psykopat, i klass med John Wayne Gacy? En akademiker som tog Linné i försvar, var Prof. Etienne\ref{prof. etienne}, särskilt i hans essäsamling \textit{\quotetext{Morsan! Farsan! Digga kniven! - en stridsskrift för barns rätt till att plåga djur utan att bli stigmatiserade av samhället}}. I övrigt var Linné ett väldigt glatt barn, som ofta sprang runt på ängarna kring Råshult och lekte med tygdrakar och frisbees av bark.

\uline{Skolgång}

Linnés fallenhet för anatomi blev tydlig under hans skolgång, där han fick högsta betyg i alla naturämnen. Även i svenska och historia excellerade den unge Linné. Hans enda akilleshäl var gymnastiska övningar, något han vid elva års ålder skrev en dikt om:

\textit{Hjärnan vara min starkaste muskel}
\textit{Min lekamen dock, likna mest ett ruckel}
\textit{Spik och plank, huller om buller}
\textit{Ett aber av kött, som jag var dag fördömer}

Den självhatande lille gossen vantrivdes också i Småland, där han kände att det klassiska räksalladsfrossande småländska råskinnet hyllades, samtidigt som hans mer introverta nördstuk inte till fullo uppskattades av omgivningen. Således bestämde han sig för att direkt efter att han gått ut grundskolan flytta till en plats som genom historien alltid varit befolkat av introverta nördar med antisociala psykopattendenser - Uppsala\ref{uppsala}.

\uline{Uppsala och systematik}

I Uppsala\ref{uppsala} stormtrivdes Linné. Han hade ett helt gäng kompisar som ofta spanade på brudar, handlade damasker på stan och drack ganska mycket bärs och absint blandat i soppskål\ref{spritfylla}. En kväll, efter en redig bärsfylla\ref{baersfylla}, stod Linné och sprutspydde bakom knuten på Gustavianum. Mitt i sin kaskad kom en tanke till Linné, om hur alla blommor är lika, men ändå lite olika. Utifrån den tanken kom hans tvångsmässiga behov att under resten av sitt liv systematisera alla blommor i sin omgivning.

\uline{Rocken, torken \& resorna}

Genom att börja inordna allt i system blev Linné en rockstar i paritet med Paul Stanley, eftersom ingen tänkt på att allt kunde höra samman innan (precis som ingen kommit att tänka på riffet till \textit{The Deuce} innan Paul Stanley). Han blev riktigt fet och grisig, ständigt bärandes på ett vattenskinn fyllt med baconfett. Hans mentor, biskopen och lurendrejaren Olof Rudbeckius d.y. såg till att skriva in Linné på dåtidens svar på Betty Ford-kliniken, hotell lappland i Gällivare. Väl där kom Linné till klarhet om en massa grejer. Han insåg att inte bara blommor kunde systematiseras, utan även samerna som masserade hans frontallob tills han somnade in i sin renskinnsfäll. Efter det tyckte Linné att det var rätt smutt att checka in sig själv på olika behandlingshem, främst i Sverige\ref{sverige}, men han gjorde också en avstickare till Nederländerna för att spana in några nya spännande plantor. Efter varje incheckning skrev Linné en bok, för att hålla sin ekonomi flytande. Han spelade mycket på hästar, men la ner efter att Brandklipparen\ref{brandklipparen} lubbat in i en mur och dött. Efter Brandklipparens frånfälle ska han ha sagt att hästsporten inte var sig lik och hans humör mulnade märkbart.

\uline{Missbruk och död}

Trots all sin rehabilitering fortsatte Linné att missbruka flott och öl i stora doser. Han bar inte längre eleganta damasker, utan grå mjukisbyxor och filttofflor. Han lekte inte längre med tygdrakar, utan stegrande mängder ister och alkohol. Vid sin död hade han blivit adlad, men också ett vrak. Hans begravningståg följdes av tusentals människor, längs med Uppsalas gator. Idag vilar hans groteskt feta lik i Uppsala domkyrka.

\ditem[Carlshöjdare]\label{carlshoojdare}
 En Carlshöjdare är en extra prominent person\ref{storfraesare} som bor på stadsdelen Carlshem i Umeå\ref{umeaa}. Carlshöjdaren kan ofta ses vagga fram och tillbaka längs med Kieselstråket, där hen titt som tätt stannar för att lyssna till förbipasserande gubbar och gummors klagomål. Om det står i Carlshöjdarens makt hjälper hen gärna till, men aldrig utan att kräva en gentjänst. De flesta Carlshöjdarna är barn till kirurger, psykologer, radiopratare, datatekniker och journalister. Därmed kan livet för den som inte böjer sig efter Carlshöjdarens vilja te sig riktigt surt. Små gliringar utdelade i Filosofiska rummet i P1, en laptop-reparation som drar ut i evigheter, en kontaktannons insatt i Blänkaren som kungör att man söker en våldsamt flatulent partner, att man tvingas sitta i väntrummet hos psykologen i 40 min innan man får komma in, med bara postmodern poesi att läsa, eller att man efter ett rutinbesök på vårdcentral vaknar upp tre dagar senare i busskuren på Marmorvägen, med ett grishjärta inopererat i sin bröstkorg.

\uline{Förmåner}

Det hedersdrivna kotteriet på Carlshem åtnjuter en serie privilegier som den vanlige Carlshemsbon bara kan drömma om.

\begin{itemize}
\item Carlshöjdaren har obegränsad tillgång till den exklusiva replokalen intill Carlskyrkan.
\item Carlshöjdaren har 10 \% rabatt på pizzeria Atlanta och får alltid en gratis Trocadero på köpet.
\item Carlshöjdaren får klättra upp på Carlshemsskolans tak och ingen törs säga åt den att komma ner omedelbart.
\item Carlshöjdaren har förtur på alla balla grejer som kommer in på Gimonäs återvinningsstation, typ NES, sprayburkar och tuffa skateboards.
\end{itemize}

\ditem[Carpe diem]\label{carpe diem}
Ett allt för ofta missförstått begrepp. Sant är att det betyder \quotetext{Fånga dagen} på latin, och att det ofta uttrycks i tal och broderade dukar av folk som bor i radhus- eller villaområden. Vad som är fel i hur folk tolkar orden, är att de ofta ses som ett uttryck för belåtenhet över att ha möjlighet att fånga dagen.

Men i sitt sammanhang, de isolerade bostadsområdena, mitt i ett samhälle som till allt större del anpassas efter arbetslinjens krav på konstant underkastelse och alienerande produktion, kan Carpe diem inte ses som något annat än ett rop på hjälp – ett uttryck för total desperation. Carpe diem, bli inte som vi! skriker de kroppsligt levande men själsligt döda, som en uppmaning till nästkommande generationer.
"{I språkets periferi}

\ditem[CCM]\label{ccm}
 är en förkortning och står för Contemporary Christian Music, alltså Samtida Kristen Musik. CCM avgränsar mot traditionell kristen musik så som psalmer och körstycken. Till skillnad från de senare används i CCM vanliga rock/pop-instrument så som elgitarr\ref{gitarr}, elbas och trummor. Inom CCM finns så många underkategorier att den som vill veta mer om denna spännande musikstil måste utforska den själv, men några olika stilar är kristen rap\ref{hip-hop}, dansmusik, folkrock, kristen ful-sludge, punkrock, och till och med \quotetext{heavy metal.}

\ditem[Centerpartiet]\label{centerpartiet}
 var först ett parti för vanliga bönder\ref{boonder}, sen för nyliberala kulaker\ref{kulaker}. 

\uline{Slogans}

\begin{itemize}
\item \quotetext{För egocentriker}
\item \quotetext{Förslava bävrarna\ref{ivriga smaa baevrar}}
\item \quotetext{Det är vi som suger ut Sverige\ref{sverige}}
\item \quotetext{Blod på min jord ger gåslever på mitt bord}
\item \quotetext{Libro e moschetto — fascista perfetto}
\end{itemize}

\ditem[Champis]\label{champis}
 Fattigmans-Pommac.

\ditem[Chapeau de paysan]\label{chapeau de paysan}
 (bondhatt) är ett brunt läderartat plagg skapat för att skydda hårt arbetande bönder från solen. Bärs med fördel som en hatt.

Under medeltiden användes plagget för att förnedra uppnosiga studenter på universitet. Studenten tvingades av en magister att ta på sig bondmössan för att påminna denne om dess rang i klassrummet. Detta förekom vid alla universitet under medeltiden.

Thomas av Aquino var en mästare på att tvinga på studenter mössan för att få dem att lugna sig. Han uppfann även seden att tvinga studenten att härma olika djur på en bondgård när den väl var iförd mössan och slangtermen att \quotetext{chappa} någon, som en förkortning av det betydligt krångligare \quotetext{chapeau de paysana} någon.

Idag används inte längre chappning som en bestraffningsmetod, då det på dagens universitet är moderiktigt att se ut som en lurk.

\ditem[Charles Manson]\label{charles manson}
 Challe är den sjätte medlemmen av surf-rockbandet Beach Boys.

\ditem[Charlotte]\label{charlotte}
 är ett kvinnonamn från Frankrike och betyder kaskelott.

\ditem[Cheeseburga]\label{cheeseburga}
 Att cheeseburga någon är att försiktigt dela på en cheeseburgare så att man får två delar, en i vardera handen, och sedan klappa ihop dessa på var sida om någons huvud. Detta är en effektiv anfallsmetod som står till hands för den gatuköksbesökare som finner sig indragen i en dispyt och inte ser någon annan utväg än att gripa till ytterligheter.

\ditem[Chef]\label{chef}
 En chef är någon som parasiterar på andra. Chefer finns nästan över allt så det är svår att skydda sig, men en bra början är att studera Marx och CH Hermansson. En chef kan aldrig bli ens vän.

\ditem[Chemtrails]\label{chemtrails}
Människor som tror på chemtrails hävdar att de spår som syns i himlen efter flygplan består av kemikalier som håller människor förslavade och passiva. I de breda folklagren i allmänhet, och bland den högutbildade medelklassen i synnerhet, anses det väldigt ofint att tro på chemtrails, då det antas vara en tro förbehållen för heltokiga konspirationsteoretiker.

Men vad många inte ser är att chemtrails på ett sätt är en del av en konspiration. Regeringar världen över kan onekligen utnyttja att dårfinkar med låg social status hakar på konspirationsteorin. Det skapar en ovilja hos människor att associera sig med konspirationsteoretiker, eftersom man inte vill framstå som trasig eller psykiskt labil. Istället skrattar man åt chemtrailsivrarna, och förlöjligar alla som tror att något annat än den rådande verkligheten med tillhörande samhällssystem är den enda rimliga och möjliga.

Sen är det naturligtvis inte så att vi hålls passiva av kemikalier som läcker ur flygplan. Men den verkliga konspirationen går ut på att det tycks vara omöjligt att acceptera en annan samhällsvision än den som redan är införlivad.

\ditem[Chipslåda]\label{chipslaada}
 är en maträtt bestående av chips, ost och majonäs. Man går tillväga enligt följande:

\begin{enumerate}
\item Strö chips i ett lager i en form.
\item Häll på majonäs.
\item Repetera tills du fyllt formen.
\item Riv över så mycket ost att inget annat syns.
\item Skjuts in i ugnen!
\item Ta en mobiltelefon, slå in 11.
\item Ät.
\end{enumerate}

\ditem[Christer Sandelin]\label{christer sandelin}
 är en svensk artist som seglade upp på pop-himlen med banden \textit{Freestyle} och \textit{Style} under 80-talet. Han medverkade också med \textit{Freestyles} andra medlemmar i filmen \textit{G som i gemenskap} (1983). Efter detta har Sandelin arbetat som soloartist och släppte så sent som 2004 albumet \textit{I stereo} tillsammans med artistkollegan Tommy Ekman. Ordvitsen i skivans titel, som syftar på att albumet skapats av två artister, gjorde dock att Sandelin efter skivsläppet såg sig nödd att gå under jorden för att undvika repressalier. Svenska\ref{sverige} myndigheter utfärdade redan 2005 en begäran att Sandelin eftersöks både nationellt och internationellt för att ställas till svars för det han gjort. Den som har information om var Sandelin kan befinna sig eller vilka människor han har kontakt med bör höra av sig till polis, säkerhetspolis eller kustbevakning och kan kompenseras ekonomiskt.

\ditem[Christianiacykel]\label{christianiacykel}
 
En christianiacykel är en damcykel som kraschats in i en kundvagn med sådan kraft att de två tingen absorberat varandra och bildat en ny helhet. Eftersom det är en dansk\ref{danmark} uppfinning saknar fordonet växlar (det var för svårt att konstruera) och hjulen är av helgjutet gummi (det var för svårt att göra ett hål inne i slangen utan att det blev punktering på utsidan). Bromsen består av en träpinne som hänger i ett snöre på styret och sticks in mellan ekrarna. Christianiacyklar används främst till att göra plundringståg och frakta alkisschäfrar\ref{alkisschaefer}.

\ditem[Cigg]\label{cigg}
 är slang för cigaretter. Cigaretter är malen och torkad tobak uppblandat med allsköns skräp\ref{skraep} som placerats i en pappershylsa och ofta med ett gult filter längst ned. Änden med filtret placeras i munnen\ref{mun} och den andra tänds på med hjälp av en tändare eller tändstickor. Sedan drar rökaren in röken i munnen (om denne är 14 år) eller i lungorna. Det som då händer är att personen blir cool på ett genuint sätt. Det kommer att vara coolt att röka så länge som det är farligt.

\uline{Andra benämningar}

Tagg, puff, cancerpinne, lungtorpeder utan stötdämpare, rök, smoke, ciggaretts, giftpinne.

\uline{Tuffa Cigg}

\begin{itemize}
\item John Silver
\item Glenn
\item Commerce
\item Chesterfield utan filter
\end{itemize}

\uline{Töntiga Cigg}

\begin{itemize}
\item Blend
\item Prince
\end{itemize}

\ditem[Cirkusdirektör]\label{cirkusdirektoor}
 är ett klassiskt yrke med hög status bland entreprenörer. Jobbet går ut på att skriva livstidskontrakt med arma krakar på att dessa, mot undermålig betalning, visar upp sig i märkliga situationer inför publik. De arma krakarna får bo i gamla hästtransporter och för att Cirkusmyndigheten inte ska fatta misstanke tar man ofta med några riktiga djur för att verka trovärdig. När en föreställning är slut dödas de svagaste djuren och tillreds till punkgryta\ref{punkgryta} som resten av ensemblen äter. För att markera sin särställning slipper cirkusdirektören bära byxor som ger upphov till cirkuspung\ref{cirkuspung}, vilket står i kontraktet att alla andra måste ha. Danmark\ref{danmark} har hittills inte haft en enda statsminister som inte tidigare varit cirkusdirektör.

\ditem[Cirkuspung]\label{cirkuspung}
 är ett uråldrigt fenomen som uppstår när män bär sina byxor lite för högt uppdragna så att de skär in i skrevet. Genitalierna får därmed ingen naturligt plats att vistas på och hamnar lite över allt på ett osymmetriskt sätt. I det victorianska England var cirkuspung högsta mode, troligtvis på grund av att det endast var de högre stånden som hade råd att klä sig i något annat än taskigt åtsittande jutesäckar. Namnet kommer sig av att det inom cirkuskulturen alltid funnits en förkärlek till spandex, ett åtsittande material som cirkuspungar frodas i. Vill man undvika cirkuspung bör man välja byxor sydda av Avesta jet-tex, ett material som har \quotetext{suverän passform}.

\ditem[Civilpolis]\label{civilpolis}
 Att avslöja en civilpolis är mycket lätt. Leta efter:

\begin{itemize}
\item Öronsnäcka, mest uppseendeväckande om de går i grupp. Kanske lyssnar de alla på Scooter?
\item Ful keps/rakat huvud/polisfrisyr
\item Snabba skor
\item Osympatiskt utseende
\item Och framförallt, Jeansen som Gud glömde. Lite baggy, lite slitningar, några fräcka detaljer och en snutröv\ref{snutroov} som försöker rymmas däri.
\end{itemize}

\ditem[Colin Nutley]\label{colin nutley}
 Den enda mannen som ger nazisterna rätt i påståendet:"\textit{invandrarna kostar så mycket pengar}" för Colin Nutley har sugit många miljoner ur svenska filminstitutet.

I korthet går all hans film ut på att hans hustru ligger med nån bonnig typ och så blir det burleska förvecklingar och lite tårar. Eftersom alla filmer följer samma dramaturgi och innehåller samma skådespelare (Nutleys fru) är det bortkastad tid att se mer än en. Den upptagne gör allra bäst i att nöja sig med att läsa texten på baksidan av en VHS-kassett.

\ditem[Conny]\label{conny}
 är en dansk kortform av det grekiska Konstantin. Namnet betyder skäggig gubbe. Conny har namnsdag 21 maj.

\ditem[Corporate social responsibility]\label{corporate social responsibility}
 Ännu dummare än det låter. Namnet är på engelska för att förvirra de som drabbas.
Ett exempel på Corporate social responsibility är när BP:s styrelseordförande Carl-Henrik Svahnberg säger \quotetext{I care about the small people} efter att en oljeplattform försökt sig på en Eskimåvändning och mer eller mindre tömt hela jordens innandöme på olja i mexikanska golfen.

\ditem[CQD]\label{cqd}
 Ute på havet finns det som bekant ett oändligt antal faror. Stormar, grund, sjörövare, Bermudatriangeln, skörbjugg, spökskepp, myteri, sjömonster, för att bara nämna några. Med uppfinnandet av den trådlösa telegrafen kunde sjöfarare plötsligt kommunicera mellan båtar och därmed undsätta varandra vid en nödsituation. Alla var jätteglada att få hjälp och slippa bli bitna av späckhuggare och jättebläckfiskar. Från början använde telegrafisterna det, i morsekod, något krångliga HELP för att begära assistans. Det var svårt att skicka och missuppfattades ofta, varför man istället övergick till snabbkommandot CQD (come quick danger). Detta fungerade alldeles utmärkt och under de här åren var det nästan ingen som fick träben. Ända till 1906 när det hölls internationell radiotelegrafkonferens i Berlin och någon föreslog att man skulle ändra till det mer spirituella\ref{hippie} SOS (save our souls). Ända sedan dess har sjökaptener och radiotelegrafister tvingats kämpa med att memorera den nya signalen. Ingen vet hur många liv som gått till spillo på grund av att mottagaren ännu inte lärt sig denna nya särskeförkortning\ref{saerske}.

\ditem[Crass]\label{crass}
 En samling akademiska britter som hade ett kollektiv.

\ditem[Crustare]\label{crustare}
 En crustare är någon som identifierar sig med subgenren crust. Crusten härstammar från kängpunken, men har ofta lite mer metalinfluenser, burkigare inspelningskvalité och texter om politiska grejer.

\uline{Stil}

Stilmässigt karakteriseras crustaren av så kallad crustnacke (dreadlocks, typ), en väst prydd av många tygmärken med bands namn på (ex. Doom, Dystopia, Nausea, Hellbastard, Amebix, Misantropic osv), en keps med tygmärke på, en magväska (som de kallar crustväska för att det ska vara lite ballare), trasiga byxor, en loppbiten schäfer\ref{alkisschaefer} och friluftsskor.

\uline{Ideologi}

Crustaren utövar ofta dumpsterdiving\ref{sopletare} (äter sopor), liftar frekvent med lastbilschaffisar och super. Vad som skiljer en crustare från en träskpunkare\ref{traeskpunkare} är allt som oftast politiska idéer. Crustaren äter sopor som konsumtionskritik, inte för att det är gratis. Crustaren bor på gatan för att uppmärksamma bostadsbristen, inte för att den blev utkastad från morsan för att den var för äcklig. Crustaren liftar inte för att det är gratis och det kanske finns baconchips på golvet i lastbilshytten, utan för att den inte vill bidra till den miljöförstörande tåg- eller flygtrafiken.

Trots att crustaren sällan har några andra ägodelar än det den har på/i kroppen, lyckas den likförbannat alltid ha något störigt fanzine med sig, i vilket alla små individuella bushandlingar\ref{jaevelskap} crustaren utför bekräftas som extremt revolutionära och potentiellt samhällsomkullrunkande. På så sätt har crustscenen påfallande mycket gemensamt med vegan straight edge-scenen, men det talas det inte så ofta om.

\ditem[Crustknytning]\label{crustknytning}
 , även känt som \quotetext{fladdermöss}, är ett sätt att knyta skorna som framförallt brukas av träskpunkare\ref{traeskpunkare} och andra slan\ref{slan}. Istället för att använda den klassiska rosetten, åttan\ref{aatta} eller råbandsknopen föredrar träskpunkaren att dra ut skosnörena ur de övre hålen och sedan knyta en rejäl dubbelknut i varje ände. Snörningen är nu helt omöjlig att dra åt men snörena åker inte ur, vilket är just den effekt som eftersöks eftersom det är jobbigt att knyta skorna. Plösen hänger nu slappt som tungan på en golden retriever och det fullkomligt rasar in grus i kängorna. Men det ger träsktrollet blanka fan i eftersom det är \textit{crust as fuck existence} och hen, om sanningen ska fram, aldrig lärt sig knyta en riktig knut eftersom större delen av skoltiden spenderades i rökrutan.

\ditem[Crustpippi]\label{crustpippi}
 Alla crustband av någorlunda rang har en egen crustpippi. Det är siluetten av en fågel och återfinns någonstans på omslaget. Kanske ska pippin symbolisera crustarnas\ref{crustare} längtan efter frihet eller så är det som vanligt bara en homage till \textit{Discharge}, som populariserade pippikonceptet på sin skiva \textit{Never again} med en spetsad fredsduva. De flesta crustpippis är annars kråkor eftersom kråkorna lever ett crustigare liv. Stadsduvor är ju egentligen ännu crustigare men det är för svårt att skilja dom från fredsduvorna på ett svartvitt omslag täckt av vapen, nitar och misär.

\ditem[Currykondom]\label{currykondom}
 En currykondom är en kondom full med curry. Som preventivmedel är den värdelös.

\ditem[Cykelhjälm]\label{cykelhjaelm}
 är en anordning för att skydda kraniet från företrädesvis trubbigt våld. Ungefär som en suspensoar, fast för huvudet.
Barn är ofta ivriga användare av cykelhjälm, då de i sitt ständiga testande av gränser även stundtals försöker utmana gravitationen. När dessutom samhället har en nedlåtande syn på minderårigas bruk av folköl så är det inte lätt för de små liven att sänka sin tyngdpunkt på samma sätt som är brukligt för medelålders människor med pondus. Således får de dras med en frigolitbit på huvudet, och drömma om den dag de är stora nog att känna fartvinden blåsa i håret när de rullar ner för en kurvig grusväg på en rostig Crescent med trasig fotbroms.

\uline{Inverkan på sexualiteten}

Att skaffa sig en tillfällig sexuell kontakt bärandes cykelhjälm står när det gäller svårighetsgrad i paritet med att försöka lösa Goldbachs hypotes under en redig spritfylla\ref{spritfylla}.

\uline{Cykelhjälm på vuxna}

Hos vuxna är cykelhjälm ofta en säker signal på att bäraren har någon form av handikapp som inverkar på förståndet. Om bäraren av cykelhjälmen stirrar på en fast punkt i gatan tre meter bort, eller lallar lite för sig själv kan man nästan vara helt säker på att så är fallet. Observera att om bäraren är en äldre dam kan det vara en lågstadielärarinna som föregår med gott exempel på vägen till sin sadomasochistiska swingerträff. Man bör alltså ta sig en funderare eller två innan man i onödan ringer länsman.

\uline{Cykelhjälm i storstaden}

I Stockholm\ref{stockholm}, där man ju gillar att vara lite normbrytande, är fullvuxna, arbetsföra män med cykelhjälm en icke ovanlig syn. Fan vet varför, men kanske lever de i villfarelsen att de på något vis är unika och att världen på något vis skulle bli en fattigare plats om deras främre pannlob sipprade ner i rännstenen på Rådmansgatan.
Vi andra kan lugnt trampa på, väl medvetna om den Lovecraftianska futtigheten i vår existens.

%%%%%%%%%%%%%%
\newpage
\null
\\
\null
\\
\Huge
D
\normalsize
\\
\null
\\
\null
%%%%%%%%%%%%%%

\ditem[Dackefejden]\label{dackefejden}
 Vi är inte helt säkra på vad Dackefejden gick ut på men vi ska försöka förklara så gott vi kan: Fejden utkämpades under 1500-talet och tog formen av ett uppror som startades av missnöjda småländska bönder ledda av en man som hette Nisse, precis som Nissepedias\ref{nissepedia} mytologiske grundare. I efternamn hette han Dacke. Upproret riktades mot Gustav Vasa. Å ena sidan arga bönder\ref{boonder}, alltså, å andra sidan kungen. Ser man kronan som den tidens länsstyrelse kan Dackefejden med lite fantasi ses som Sveriges\ref{sverige} första process mot länsstyrelsen\ref{processa mot laensstyrelsen}. Dessvärre slutade fejden med att bönderna förlorade, varpå allmän utrensning och massakrer följde. Ur ruinerna av den småländska glesbygden steg föregångaren till Centerpartiet\ref{centerpartiet} upp som en fågeln Fenix och resten, det vill säga alla högextrema och smyg-fascistiska fraktioner som knoppats av från detta parti, är historia.

\ditem[Dagens Nyheter]\label{dagens nyheter}
 eller DN som den förkortas, är Sveriges\ref{sverige} största, men inte blåaste, morgontidning. Missförstå inte, den är blå, men ibland skriver typ Kajsa Ekis Ekman nåt som man tokhåller med om. Sen skriver Peter Wolodarski nåt om att marknaden ska vara fri och man ba suckar och läser på flingpaketet istället.

\ditem[Dagsedel]\label{dagsedel}
 En dagsedel är en välförtjänt örfil som man får när man missköter sig i sällskapslivet. Kanske har man kommenterat utseendet hos världparets barn då man är på besök nästgårds och sitter till bords. Kanske har man utpekat skavanker hos sitt sällskap under en kväll på lokal, eller så har man gått och förargat en bonde genom att sova i dennes potatisåker. Hursomhelst har någon blivit vred och vill genom att ge dig en dagsedel uppmana dig att genast upphöra med detta beteende.

\ditem[Dagsfylla]\label{dagsfylla}
 är när man blir full på dagen. Det är en både väldigt rolig och intressant upplevelse. Att gå på torget, lullig och fin, mitt på en torsdag är en upplevelse alla borde prova på. Man känner sig helt frånkopplad den ordinarie verkligheten, samtidig som den aldrig känts mer påträngande. En dagsfylla uppnås med fördel på ett vernissage\ref{vernissage}, då det dels är gratis och man dessutom får grunda i magen med OLW-hjärtan.

\ditem[Dagvill]\label{dagvill}
 Ett tillstånd som går ut på att en individ glömmer bort vilken dag det är. Den moderna vetenskapen har lyckats finna två anledningar till detta tillstånd. Den ena beror på arbetslöshet och/eller ledighet och gör att vilken dag det är blir oviktigt för att man ändå inte har några tider att passa. Den andra beror på att man helt enkelt har för mycket att göra och minnet sviker omedvetet för att dämpa ångest över allt som ska hinnas med, det som läkare kallar stress.

\ditem[Dank]\label{dank}
 beskriver hur läget är. Det kan betyda både segt och lugnt. Det tidigare betonar danks negativa kvalitéer, det vill säga ett läge där inget händer, vilket man tycker är jobbigt. Den senare betydelsen innebär att man tycker det är rätt skönt att det inte händer nåt. Hur man använder uttrycket beror på vilken inställning man har till livet. Softa personer man gärna blir vän med tycker det är nice att ha det dankt.

Dank är också en sorts kula, avsedd för hasardspel. Den mäktigaste danken är en järndank som är stor, tung och ständigt i dominant ställning på nötta gatuhörn världen över.

\ditem[Danmark]\label{danmark}
 eller Legoland\ref{lego} som det ibland kallas, är en fånig landmassa av kalksand där alla är rödbrusiga, heter Preben eller Margarete och röstar på nazisterna eller högerpartiet Venstre.

\uline{Hur Danmark blev till}

I bibeln\ref{bibeln} står att läsa om hur herren skapte Norden. På den första dagen skapade han Svea Rike, den andra Norge, sen Finland och sist Island. På den femte dagen tog Gud sig en redig blecka och vaknade på den sjätte dagen i Skåne och var riktigt risig i kistan. Med ena foten i Kattegatt och den andra i Östersjön satte sig vår herre och ut kom det som vi idag känner som Danmark.

\uline{Dansk kultur}

Dansk lättöl\ref{dansk laettool}, sirapskokt potatis, röd korv\ref{roood pooolse}, legoporr, Kim Larsen och brunt bröd med rostbiff och majonäs.

\uline{Danmarks historia}

Som bekant konstaterade redan Hamlet\ref{hamlet} att \textit{\quotetext{There is something rotten in the state of Denmark}} och inte har det blivit bättre precis. De enda årtalen att hålla reda på är 1984 och 1985 när Metallica spelade in \textit{Ride the lightning} och \textit{Master of puppets} i Köpenhamn.

\uline{Folkdräkt}

Propellerkeps\ref{propellerkeps} och därutöver kalle anka\ref{kalle anka}. Följer du dessa enkla tumregler kan du fly från arga barnbokskonsumenter från vårt hemland, genom den outhärdliga danska landstungan, till friheten i Andorra.

\uline{Danmarks befolkning}

Danmark befolkas framförallt av danskar. De flesta av dessa arbetar med att arrangera Roskildefestivalen, men andra populära sysselsättningar inkluderar även att hissa Dannebrogar\ref{dannebrogen} och cykla christianiacykel\ref{christianiacykel}. Ett stort problem med danskar är att dom typ inte fattar någonting. Det genererar till exempel problem när en dansk ska ringa utomlands. Man ba: -\textit{Hallå?}, och dansken på en gång ba: -\textit{Jæ gødæ, må dü spiese din kartoffel?}. Man fattar ingenting och ba: -\textit{Ursäkta, vad snackar du om?}, men dansken ba skiter i frågan och babblar på: -\textit{Jæ dæt ær gøtt. Jæg sætt hær pæ mïn chrïstïænïæcykel ok fjærsade lit grisæporr}. Man fattar ännu mindre och ba: -\textit{Va fan snackar du om?}, men han ger sig inte utan bara kör på: -\textit{Jæ, då koem Cærsten træskænde Anders And\ref{kalle anka} ok fræge om han må lån mïn Danebrøge. Mæn de er mïn Danebrøge! Mïn! Mïn! MÏN!}. Sen lägger han på.

\uline{Dansk vetenskap}

I Danmark omöjliggörs all form av naturvetenskap tack vare att deras räknesystem är helt omöjligt att lära ut, och därmed finns inte matematik \textit{as we know it}. Samhällsvetenskaperna lyser också med sin frånvaro. Detta beror på var deras politiska sympatier ligger - se ovan. Inga upptäckter har gjorts av danskar sedan Tycho Brahes tid. Länge trodde man att Sören Kirkegaard var något på spåren, men det visade sig senare att så inte var fallet - han hade bara ångest, och så var det med det. 

\uline{Sex och samlevnad i Danmark}

I Danmark är det väldigt populärt med sån därade pörr\ref{poorr}, en av få saker som lockar turister till landet. I övrigt föredrar danskarna att ligga med varandras djur, till en billig penning (en femhunka för ett hästskjut är standard). Även så kallade \quotetext{gruppeknall} är ett populärt sätt att umgås intimt.

\uline{Sport i Danmark}

Rundpingis, prutta högst\ref{prutta hoogljutt}, dricka 20 bärs\ref{ha baers} och sen försöka slå ner domaren.

\ditem[Dannebrogen]\label{dannebrogen}
 är Danmarks\ref{danmark} flagga och den äldsta nu officiellt använda flaggan i världen. Den sägs enligt legenden ha fallit från skyarna till de danska trupperna under ett slag i Estland på 1200-talet. Flaggan föreställer två korslagda smuggelcigg\ref{cigg} mot bakgrund av en flottig röd bordsduk.

\ditem[Dansk advent]\label{dansk advent}
 I Danmark\ref{danmark}, liksom i Sverige\ref{sverige}, firas sedan århundraden advent i väntan på julen, men i vårt lilla grannland i sydväst har man valt att fira på ett för oss annorlunda och ofta obegripligt sätt. Som alla andra dagar samlas man och dricker Tuborg och lyssnar på AC/DC, men till advent placerar man fyra cigaretter i tomflaskorna i den utsmyckade ölbacken. Sedan rullar man helt enkelt fram gamla farmor och tänder på när hon släpper väder. På så vis tänds \quotetext{ljusen,} som symboliserar danskens andaktsfulla vaka inför Jesu födelsedag\ref{dansk jul}. Därefter kastar man sig utan omsvep på varandra i ett naket rallarslagsmål.

På den danska statstelevisionen sänds en tillställning som är en mer grandios version av adventsfirandet i det danska folkhemmet. Denna årliga ceremoni inbegriper en hel del orgiastiska förnöjelser som alla är totalförbjudna i Sverige. De kulminerar i och med att Lars Krogh\ref{lars krogh}, bakfull och stenad\ref{stenad} som en feministisk bloggare i Teheran, släpar sig fram till en enorm fyrbåk och lätt framåtlutad öppnar sitt anus, varpå Brøndbyernes IFs gamla fotbollsmålis Peter Schmeichel antänder den legendariske skivbolagsmannens gaser med sin pinup-zippotändare. Detta är startsignalen för flera dagar långa upplopp mellan ungdomar och poliskår och under hela tiden står varje soptunna i varje bebyggt hörn av landet i fyr, som en påminnelse om att ljuset återvänder och att det är så \quotetext{jævlig dejlig} med grisfylla\ref{grisfull}.

\ditem[Dansk forskning]\label{dansk forskning}
 Sedan 1951 går alla forskningsanslag i Danmark\ref{danmark} enkom till studier rörande avståndet mellan mannens penis och anus. Alltsedan Mogens Palleprotogen Hans Jörgen Jacobsen råkade visa sin ändalykt i \quotetext{Der må være en sengekant}\ref{sengekantsfilm} så har detta relativa avstånd också varit centrum för kulturdebatten i Danmark. Forskningen är rörande överens om att ovan nämnda pikanta variabel är ständigt sjunkande. En sorts jakt mot mellangårdens krympande lebensraum. Vid nollpunkten kommer Danmark och övriga världen, som vi känner dem, att haverera. 

\ditem[Dansk jul]\label{dansk jul}
 En äkta dansk jul firas tillsammans med sina nära och kära. Alltså de tältgrannar från Roskildefestivalen som man har kvar telefonnumret till. Den lilla gruppen samlas 3-4 dagar innan dopparedagen för att hinna bli varma i kläderna och öva upp levern inför det verkliga firandet. Barnen\ref{barn} stängs in i källaren för att ingen ska komma till skada, och om någon av dem ifrågasätter detta berättar man den gamla folksagan om Boetius de Dacia\ref{boetius de dacia}. Barnen blir då livrädda och allt är frid och fröjd. Juldagsmorgonen börjar med att gänget sätter sig på sina christianiacyklar\ref{christianiacykel} och trampar iväg till Tivoli\ref{tivoli} för att sätta eld på julgranen där. De som är för bakfulla stannar hemma och släpper ut barnen ur källaren. Därefter följer den högtidliga julmiddagen där traditionella läckerheter som skagenröra\ref{skagen}, balutägg\ref{balutaegg}, uvsvane\ref{uvsvane}, rød pølse\ref{roood pooolse}, gamle Ole\ref{gamle ole} och fläsksvålar\ref{flaesksvaalar} dukas fram. Alla har svårt att vänta med klappöppningen till kvällen så för att dämpa nyfikenheten ger alla varandra en inslagen dunk snaps. När man druckit sig mätta samlas man runt brasan av läxor som ungarna gjort upp och läser grukar\ref{gruk}. Vid det här laget är alla så fulla att man glömt bort att det är julafton och det kommer därför som en total överraskning när Lars Krogh\ref{lars krogh}, iklädd nerspytt lösskägg, sparkar in dörren och börjar kasta paket på allt som rör sig. Alla har önskat sig sengekantsfilm\ref{sengekantsfilm}, vilket man också får och därmed är kvällen officiellt till ända. Ungarna schasas tillbaka till källaren och de vuxna som ännu är vakna hälsar Jesusbarnet\ref{jesus} välkommen till världen med ett entusiastiskt gruppeknall.

\ditem[Dansk kanot]\label{dansk kanot}
 Danmark\ref{danmark} är som bekant ett örike men landmassor utspridda värre än camparna på Roskildefestivalen. Något vår kung Karl X Gustav exempelvis fick erfara bittert när han släpade ut kavaleriet på öresunds bräckliga isar för att än en gång ge Preben på tassen och ta förråden av Tuborg\ref{tuborg} och Gamle Ole\ref{gamle ole} i krigsbyte. Simningens ädla konst har aldrig varit allmän kunskap hos befolkningen utan har mer betraktats som magi eftersom vatten är förknippat med tabubelagda ämnen som tvagning och alkoholfritt. Tappra försök med att pumpa christianiacykelns\ref{christianiacykel} däck med helium och fylla dannebrogen\ref{dannebrogen} med varmluft över en brinnande soptunna har allt som oftast slutat i katastrof när dansken önskat resa mellan två öar. Ett pionjärprojekt med att gå på havsbottnen mellan Fyn\ref{fyn} och Samsø och andas i en trädgårdsslang var lyckat till en början men avbröts när tanten som ägde slangen ville ha tillbaka den.

År 1970 bestämde sig den pensionerade pølsenauten Mogens Sandfær för att lösa detta problem en gång för alla. Hans mål var att skapa en farkost som kunde härma egenskaperna hos de sälkadaver han sett färdas nästan i nivå med vattenytan. Med hjälp av ett par vadarstövlar och en lång pinne lyckades han samla in tillräckligt med döda sälar som han knöt ihop med ett hamparep och fick på så sätt en konstruktion stark nog att bära honom. Den första prototypen var svår att manövrera men när han kompletterade uppfinningen med att använda en spade till att skotta bort vattnet i fören märkte han att konstruktionen faktiskt började backa. Det danska patentverket, Pax!\ref{paxa}, beviljade patent och Sandfær har allt sedan dess varit en förmögen man med monopol på inhemsk sjöfart.

\ditem[Dansk kortlek]\label{dansk kortlek}
 En dansk kortlek skiljer sig från en traditionell anglosaxisk\ref{anglosax} kortlek i det att den har numren utskrivna i alla fyra hörnen. Det klassiska, och betydligt snyggare är att man enbart har numren utskrivna uppe i vänstra hörnet och nere i högra. Visst, det blir lite krångligare att läsa men kortspel handlar ju till stor det om stil; att med eftertryck studera sin perfekta solfjäder och ödmjukt gratulera motståndaren till vinsten i en vänskaplig omgång Chicago. Stil är dock som bekant ett begrepp som inte existerar i Danmark\ref{danmark}, där man sopat in siffrorna på alla tillgängliga hörn för att så fort som möjligt kunna utropa sig till vinnare och hälla öl över motståndarna i ett parti Finns i sjön eller Svälta räv\ref{svaelta raev}.

\ditem[Dansk kostcirkel]\label{dansk kostcirkel}
 Den danska kostcirkeln skiljer sig från den internationellt erkända kostcirkeln genom att den saknar sex av dennas sju komponenter, men samtidigt har två helt egna kategorier, nämligen Tuborg\ref{tuborg} och lettisk smuggelcigg. Den näringskategori som de danska och internationella cirklarna har gemensamt är naturligtvis rotfrukterna, en kategori som i Danmark\ref{danmark} helt domineras av kartoffel.

\ditem[Dansk lättöl]\label{dansk laettool}
 En Dansk lättöl tillverkas genom att slå lite vatten i en starköl. Det gör dansken när han behöver köra bil men prompt ska dricka starköl.

\ditem[Dansk midsommar]\label{dansk midsommar}
 Trots att Danmark\ref{danmark} är platt som Jan Björklund\ref{jan bjoorklund} i en partiledardebatt är den ändå helt omöjligt att hitta en plan yta i trädgården för att duka upp till långbord. I mitten av grönytan ligger vanligtvis komposten för att enkelt kunna nås när Preben är sugen på sega maskar men inte orkar gå till Haribo-affären. Någonstans framför garaget ligger den kamouflerade björngropen som lille Cærsten byggt så där vill man inte vara. På uteplatsen ligger pantberget och eftersom hälften av flaskorna är krossade blir det för jobbigt att städa bort. Framför rabatten är inte heller någon idé för där sitter Bærsa-Lottas alkisschäfer\ref{alkisschaefer} fastkedjad vid det omkullblåsta vindkraftverket.

Så där ser det ut på i princip alla danska gårdar så när midsommarafton infinner sig lastar man istället christianiacykeln\ref{christianiacykel} med allt som behövs och trampar ut på motorvägen där marken är platt och stadig. Pigena dukar upp bordet i ytterfilen där bilarna inte kör lika fort medan drængarna fäller en telefonstolpe att bygga midsommarstång av. Barnen plockar blommor som de dekorerar molotov cocktailsen\ref{molotov cocktail} med som ska användas på natten när det är dags att städa bort kalaset.

Som traditionen bjuder inleder bestefar firandet med ett osammanhängande tal som avslutas med att han spelar luftgitarr\ref{luftgitarr} och skriker en låt av AC/DC. När någon tröttnat på att bara sitta still och supa klär den av sig Kalle Anka\ref{kalle anka} och dansar en sväng \textit{Små grodorna} på bordet. De andra brukar snabbt haka på och det tar inte lång tid innan dansen urartat i en ormgrop nere i diket. På väg hem från festen samlar de ogifta sju sorters pantburkar som de stoppar i kudden eftersom det mesta av dunet gick åt när rektorn skulle tjäras och fjädras på skolavslutningen.

\textit{Dansk midsommar}

\ditem[Dansk onsdag]\label{dansk onsdag}
 Som vanligt träffas man, lyssnar på AC/DC och dricker bärs\ref{baersfylla}. \textit{Bron} står på på TV\ref{television} men ingen kollar så noga för den överröstas ändå av den lite nedgrisade australiensiska\ref{australien} badboyen som snurrar på skivspelaren. Efter en back eller så ska Preben, umgängeskretsens lustigkurre, prompt ha en skämttatuering, som vanligt. Mot 21:34-tiden blir det sengekantsfilm\ref{sengekantsfilm} och högljudda och samstämmiga samtal om \quotetext{böglandet} Sverige\ref{sverige} som Lars von Trier brukar göra när svenska myndigheter stoppar hans planer på att slakta åsnor när han gör film i Trollhättan eller gnälla på att man inte får köra bil hem från krogen, trots att man är för full att ta sig hem för egen maskin.

\ditem[Dansk påsk]\label{dansk paask}
 Inga konstigheter här. Man samlas, lyssnar på AC/DC och dricker sig full som vanligt. Kanske kollar man på TV.

\uline{Traditioner}

Traditionsenligt knäcker dansken försiktigt ett ägg i båda ändarna, blåser ut innanmätet på garageuppfarten och fyller ägget med något som inte ska vara därinne. Sedan tappar man bort det för att man är lite på lyran och skäller ut någon för att ha snott det. På långfredagen mäter man varandra och den som är längst vinner en grenkabel att ha i förrådet när man använder borrmaskinen.

\ditem[Dansk sax]\label{dansk sax}
 Att ta en kniv i varje hand och med hjälp av dessa klippa av något. Eftersom det saknas någon ledig hand som kan hålla i det som ska klippas itu brukar det sällan bli särskilt rakt. Det är dessutom mer regel än undantag att det stänker en del blod när man klipper med dansk sax. Men eftersom allt i Danmark\ref{danmark} är batikfärgat så gör det inget.

\ditem[Dansk semester]\label{dansk semester}
 Sjukskriva sig och hänvisa till plötslig magsjuka.

\ditem[Dansk skalle]\label{dansk skalle}
 En dansk skalle är en teknik vid handgemäng som går ut på att med ett kraftigt ryck med kroppen slunga sin panna i riktning mot sin motståndares huvud eller torso..

\ditem[Dansk skrock]\label{dansk skrock}
 Även haschrökande\ref{hasch} nakennihilistister drabbas ibland av föreställningar som saknar logisk mening men ändå framstår för individen som av naturen givna. En vardaglig handling som att snatta mat till middagen kan hos en skrockfull dansk upplevas ge närmast övernaturliga konsekvenser om den inte utförs på rätt sätt. Om fenomenet ska ses som en folklig form av religion eller bara är resultatet av utbredd tillgång till kemiska droger bråkar den danska forskarvärlden fortfarande om (LSD framställs ju som bekant i laboratorier så forskarvärlden har mycket annat att ta i tu med först). Skrock är dock vanligare hos äldre, vilket talar för att det hänger ihop med långt gången delirium. Om du mot förmodan skulle befinna dig i Danmark\ref{danmark} kan det vara bra att känna till att nedanstående fenomen inte är något att oroa sig för. Det är inge dåre som gör detta, det är bara en vanlig skrockfull dansk:

\begin{itemize}
\item Lägg inte nycklarna på bordet. Då kan det hända att ägaren tar tillbaka dom.
\item Om du ser en svart katt gå över vägen får du sju års bakfylla om du inte kastar katten över din högra axel.
\item Om du hör en låt med Kim Larsen måste du dricka fem bärs.
\item Knarka alltid tre gånger innan du går in genom kyrkporten.
\item Om du krossar en spegel måste du krossa en till så det inte blir ojämt.
\end{itemize}

\ditem[Dansk tubkikare]\label{dansk tubkikare}
 Kika ner i Tuborgbuteljen\ref{tuborg} \textit{and Bob's your uncle}, som engelsmännen säger.

\ditem[Danska hedersbetygelser]\label{danska hedersbetygelser}
 Som alla andra nationer som vill framstå som civiliserade har Danmark\ref{danmark} hederstitlar som kan tilldelas särskilt framstående medborgare. Detta för att ge en stärkt känsla av den nationella gemenskap ingen dansk brydde sig om förens man började skylla alla problem på invandrarna. Utmärkelserna delas ut av De Kungelige Videnskabernas Sammensværjning som sammanträder varje lördag på Længelands nudistcamping. Ledamötena utses av Folketinget\ref{folketinget}. Problemet är som så ofta annars i Danmark att det sällan är någon som lyssnar på vad överheten\ref{driva med ooverheten} har att säga så många är dubbade utan att veta om det. Torben Ulrich gick till exempel runt i flera år på Strøget utan att veta om att han belönats med hedersbetygelsen \textit{Sængrøgere} och det tillhörande Arne Jacobsen-stipendiet på 10.000 kapsyler. Titlarna är personliga och kan inte överlåtas på någon annan eller ärvas av avkommor.

\uline{Danmarks finaste hederstitlar}

\begin{itemize}
\item \textit{Sængrøgere av guds nåde} - För mer än 40 år i frihetens tjänst (högst oklart vad som faktiskt menas med detta).
\item \textit{Pilsnerdræng}/\textit{Fæbodjænte} - För nit och särskild redlighet när Roskildefestivalens bajamajor ska tömmas i havet.
\item \textit{Olsen brother}/\textit{sister} - Tilldelas den som lyckats dricka mer än 5 000 bärs under ett kalenderår.
\item \textit{Jesus av Læsarett} - Tilldelas den som lyckats dricka mer än 10 000 bärs under en kalendermånad.
\item \textit{Brillerøv} - Föräras den som gått till bibblan och läst en hel bok utan bilder.
\item \textit{Videnskabens bedste ven} - Nedsträcks till den dansk som kommit på något smart som gör att det känns socialt accepterat att glida omkring planlöst med en starköl och en halvbrunnen John Silver i ena näven och skjuta en barnvagn i den andra, utan att behöva ta ansvar.
\item \textit{Tubsokker fremstillet af guld} - För att du kan sjunga texten till AC/DC-dängan \quotetext{Sink the Pink} samtidigt som du kör trampbåt själv, full.
\item \textit{Legokong in absurdum} - Tillskänkes endast basister i D.A.D som gör något fantastiskt för landets femte stad - Esbjerg.
\item \textit{Bestefar} - Eftersom endast en liten minoritet av danskarna har en känd fader belönas varje år av okänd anledning en framstående fritidsledare med ett trepack tubsockor.
\item \textit{Simborgarmärket} - Alltsedan 1600-talets häxförföljelser åtnjuter simkunniga danskar stor respekt och vördnad.
\item \textit{Hederslegionella} - Tilldelas var femte år en framstående utlänning. Mestadels har titeln tillfallit holländare som ådragit sig Syfilis\ref{syfilis} och Gonorré under Roskildefestivalens dasstömmning.
\item \textit{Rasmus Klump-priset\ref{rasmus klump}} - Ges den som på ett tydligt sätt uppvisar de egentskaper som serien förespråkar. Alltså piprökning och att gömma saker i munnen.
\item \textit{Legoglands Mæsterdræng} - Tilldelas män som sköter sig när de besöker Legoland\ref{lego}.
\item \textit{Professor/professora Flatologus} - Utdelas årligen av flatologiska sällskapet vid Köpenhamns universitet till den dansk som visat stor initiativkraft inom praktisk flatologi\ref{flatologi}.
\end{itemize}

\ditem[Danskt penicillin]\label{danskt penicillin}
 är en potent cocktail bestående av smørrebrød, brännvin och prostituerade. Precis som det penicillin som används i resten av världen uppfanns det danska penicillinet av en slump. Till skillnad från vanligt penicillin inträffade dock inte slumpen i ett laboratorium, utan i ett skjul utanför Palads København, där en stor orgie tog plats för att fira premiären av \textit{Olsen-banden på spanden} (1969). Niels Bohr, som deltog i firandet, skrev i sin dagbok morgonen därpå: \textit{Det vaer meg en ufaddebar kraft sem stremmede rättopp lekamen min... jeg kende meg likt Anders And om han pulet sine andfingrer oppi ett blixthul} (danska för eluttag)\textit{... ufaddbaert kraeftful!}

\ditem[De förlösande thinneråren]\label{de foorloosande thinneraaren}
 Den andra delen i Prof. Etienne självbiografiska verk avhandlar författarens tonårsperiod där han börjar utforska sin egen kropp och på allvar forma sin identitet.
 
\uline{Synopsis}

Författaren känner sig desillusionerad i skolan och längtar bort till något annat. Han börjar skolka och ligger mest hemma i sängen och räknar sina kroppsöppningar. Han experimenterar med hur dessa kan stimuleras av kemiska substanser och tror sig en dag av en händelse lösa livets gåta. Ödeshög kan inte längre erbjuda allt det författaren vill uppnå så han bestämmer sig för att ge sig ut på luffen. I ett dike träffar han på några medlemmar i könsrockbandet Rövsvett, som däckat där kvällen innan. Han tar anställning som bandets chaufför (det anses fördelaktigt att han ännu inte är straffmyndig) och får ibland gästsjunga på scenen. Efter en konsert i Köpenhamn blir han frånåkt och flyttar in i ett rivningshus tillsammans med en före detta läkare som påstår sig vara den som introducerade meskalinet i Norden. Tillsammans grundar de en sexkult och det mesta blir dimmigt under några år.

\ditem[De gamla grekerna]\label{de gamla grekerna}
 När man talar om De gamla grekerna avser man sällan Stavros, Papadopoulos och Amos som sitter och lirar backgammon utanför den lokala kiosken. Snarare avses ett vilt gäng gubbar som härjade runt medelhavet några hundra år före Kristus\ref{jesus}. Platon, Sokrates och Aristoteles är de man mest snackar om när det handlar om gamla greker. Platon var jätteelitistisk och snackade hela tiden om att allt var fake i den fysiska världen och att allt var mycket mer true i vad han kallade \quotetext{den äkta verkligheten}, dit ingen fick komma utom han och kanske Sokrates. Sokrates är mest känd för att han klagade på ungdomar innan Elvis hade juckat sitt första juck och för att han tog livet av sig genom att svepa en giftbägare, vilket man måste medge är extremt rått. Aristoteles ville katalogisera allt i hela världen. Han uppfann autismen. Aristoteles benämndes under medeltiden sällan med namn utan oftast kort och gott som Filosofen, alltid med stort F. Detta, får man anta, skulle ha gjort honom mycket, mycket nöjd och antagligen helt outhärdlig att umgås med. Detta skulle dock ha straffat sig på den tiden då de gamla grekerna även uppfann hybris och att man dör om man får sådan.

Det finns många fler saker som de gamla grekerna var först med.

\begin{itemize}
\item Poeten Sapfo uppfann hela subkulturen \quotetext{svår poetlesbian} på sin ö, Lesbos.
\item Archimedes uppfann att det bubblar om man fiser i badet. Han var även den första människan som drömde om att använda robotklor och värmestrålar som vapen.
\item Diogenes var den första hippiecrustaren. Trots att han kom från en förmögen bakgrund (farsan var bankir, vettu) valde han att bo i en gammal tunna som han squattade. Han uppfann Kynismen (idag cynismen), som senare skulle komma att utvecklas till den flummigt sköna idétraditionen Stoicismen som typ går ut på att man ska ta det lugnt. Under sin levnad rackade han ner på etablissemanget (på den tiden Alexander den store) och Platon. Hur han dog är omdiskuterat, men enligt samtidiga källor ska han ha dött antingen av att ha käkat rutten bläckfisk (antagligen dumpstrad), ha blivit biten av en äcklig hund eller helt enkelt hållit andan i protest tills han dog. Crust as fuck existence.
\item Demokritos uppfann sitt namn till trots inte demokratin. Det var det en annan gammal grek som gjorde. Demokritos uppfann tanken om att allt består av små små bitar tills det tillslut inte går längre. Då har man den minsta biten. Atomen, alltså. På senare tid har det visat sig att det vi först kallade atomen inte var odelbar (som ordets ursprungliga innebörd, odelbar, antyder). Men det gör inget, då redan de gamla grekerna hade fel.
\item De gamla grekerna uppfann även misogynin, då man tidigt hävdade att brudar var missbildade lägre stående varianter av män som borde låsas in. Därom tvistade sällan de lärde hos de gamla grekerna. Tankarna om att brudar sög var ofta vad som enade filosofer som Platon och Aristoteles som annars ofta tyckte att det den andra tänkte var helt bananas\ref{bananas}. Man bör notera att detta, precis som idag, inte hindrade grekgubbarna från att ligga med brudar. Det tyckte de var nice.
\item Författaren Homeros skrev en bok som heter Odysséen som är den första kända källan där korv omnämns.
\item Aischylos uppfann både tragiken och slapstickhumorn, båda i och med sin död. Aischylos dog genom att en \quotetext{örn} (antagligen en uv\ref{uv}) släppte en sköldpadda i huvudet på honom från hundratals meters höjd. Eftersom han var en väldigt älskvärd grek resulterade hans död i tragikens uppkomst. Innan sin död skrev han tråkiga pjäser om vikten av försoning.
\item Epikuros kom på idén med att undvika att göra jobbiga och svåra saker och istället hänga med polarna och fixa sig en sjysst dagsfylla\ref{dagsfylla}.
\end{itemize}

\ditem[Deadhead]\label{deadhead}
 kallas den person som är ett extra stort fan av den amerikanska rockgruppen Greatful Dead. Deadheads finns i tusental över hela världen och reser tillsammans runt för att se Greatful Dead framträda live. Många åker traktor eller cyklar till konsertera, eftersom det känns mest \quotetext{psychedelic}. Men är det jättelångt händer det att deadheadsen tar tåg.

Det finns ingen formell antagningsprocess så vem som helst kan bli ett deadhead när helst den vill. Men om man kallar sig det utan att kunna jättemånga låttexter utantill och ha sett Greatful Dead live jättemånga gånger lär man få svårt att få tillträde till den inre kretsen. Det finns nämligen flera hårdföra subgrupper inom deadheadsen. Wharf Rats är deadheads som hjälper varandra att hålla sig drogfria under Greatful Dead-konserter, det är nämligen inte alltid så lätt. Deras raka motsats kallas Wookiees, som är deadheads som lever ungefär som träskpunkare\ref{traeskpunkare}, vilket är väldigt lätt. En rapport från FN-organet WHO visar att det om några år beräknas finnas fler deadheads än blonda personer i världen.

\uline{Kända deadheads}

Följande personer har själva eller av media deklarerats vara fanatiska deadheads:

\begin{itemize}
\item Tony Blair
\item Bill Clinton
\item Walter Cronkite
\item Whoopi Goldberg
\item Greg Ginn
\end{itemize}

\ditem[Delfin]\label{delfin}
 Överskattat djur som ingår i djurfamiljen valar. Har en blågrå färg och töntigt utseende. Det finns dokumenterade fall där delfiner har räddat människor från hajar, vilket känns extremt taskigt mot hajarna, som precis som alla andra djur måste få äta sig mätta. Delfiner har också ett konstigt läte och begynnande flint. Deras sädesvätska är också basen för vegansvullet tofu\ref{tofu}.

Det är inte ovanligt att nyhippies, som fått för sig att det skulle vara en underbar och perspektivvidgande upplevelse att bada naken i månskenet med delfiner, blir brutalt gruppvåldtagna av de små gynnarna. Att delfiner enligt samma grupp av människor är den näst smartaste livsformen på vår planet gör bara övergreppen otäckare.

Delfiner bör inte förväxlas med den mycket snyggare släktingen Helfinen.

\ditem[Delfinapa]\label{delfinapa}
 En delfinapa är en apa med en delfins\ref{delfin} hjärna. Den är stark, smidig, snabb och farligt intelligent. Liksom den globala virus-pandemi som epidemologer länge fruktat ska bryta ut i vår globaliserade värld, men som ännu inte gjort det, har vi hittills, tack och lov, varit förskonad från delfinapan. Den förblir än så länge en evolutionär farhåga, men bör tas på stort allvar. Delfinapan skulle nämligen snabbt bli ett säkerhetsproblem på grund av sin kroppskapacitet och sitt instrumentella tänkande.

\ditem[Den ambitiösa studenten]\label{den ambitioosa studenten}
 tänker sig en framtida akademisk karriär och ägnar sig, till skillnad från hans eller hennes klasskompisar, åt att studera flitigt istället för att ränna omkring på nationer och dylikt. Den ambitiösa studenten läser Adorno, Foucault och Butler och imponerar på klassen och läraren med att säga sådant som \quotetext{men Foucault skriver att...} och så (trots att mannen i fråga varit död i snart trettio år). Den ambitiösa studenten har inget till övers för sexism och rasism. Identitetspolitik är den ambitiösa studentens shit. Den ambitiösa studenten lyssnar på samisk jazz och kvinnliga singer/songwriters och sitter uppkrupen som en katt\ref{krypa ihop i soffan som en katt} i kuddhögen på folkkök\ref{folkkook} och värmer händerna på en kopp chai-té. Att förakta facebook\ref{facebook} är självklart för henom, men hen har ändå en facebooksida för att kunna få nyheter om Ship to Gaza och djurens rätt. Den ambitiösa studenten är så trött på alla gubbar inom akademin och vill vara med och förändra. Den ambitiösa studenten borde sättas på ett tåg till Sibirien, men då skulle hen bara se det som ett spännande tillfälle att lära sig mer om vår stora granne i öst eller se det som ett sätt att hitta material till exjobbet.

\ditem[Den arga groggen]\label{den arga groggen}
 Känd som \quotetext{busgrogg} eller \quotetext{rallartoddy}, består av starköl som groggvirke blandad med brännvin\ref{braennvin} av gott märke. En lär bli på väldigt dåligt humör av denna.

\ditem[Den lilla boken]\label{den lilla boken}
 I internationell herrfotboll\ref{fotboll} på toppnivå händer det vid misslyckade målchanser där en fotbollsstorfräsare\ref{storfraesare} tyckt sig vara fri men inte fått en passning precis där han vill ha den att denne för samman sina underarmar och med handflatorna formar som en liten pixibok som han visar för spelarna och publiken. Genom att göra så vill spelaren signalera att han är missnöjd och att det är allas fel förutom hans att han inte fick göra mål. Kopplingen mellan situationen och den lilla boken är oklar men litteraturvetare och filologer över hela världen arbetar outtröttligt på att förklara den.

\ditem[Den nordiska smuggeltriangeln]\label{den nordiska smuggeltriangeln}
 är namnet på det ekonomiska ekorrhjul som håller de svarta marknaderna i Norge, Sverige och Danmark snurrande.

\uline{Kort historielektion}

Som alla vet har allt roligt alltid varit förbjudet i Norge. Förutom en halvtimme Fleksnes på lördagskvällarna finns det inget som inbjuder till skratt eller njutning. I grannlandet Danmark\ref{danmark}, å andra sidan, har det alltid varit lördag\ref{loordag} hela veckan och alla har varit fulla och tittat på sengekantsfilm\ref{sengekantsfilm} så länge att man glömt bort vad ordet tråkigt ens betyder. Dessa förhållanden gav givetvis upphov till en aldrig sinande avundsjuka. Norrmännen ville så klart också supa och gruppeknalla, och i Danmark blängde man missunsamt på lusekoftorna av varm fårull (eftersom hela Danmark partajas sönder till en leråker en gång om året är grisar numera de enda boskapsdjur man klarar av att hushålla. Och eftersom dansk ekonomi till stor del bygger på export av pörrfilm\ref{poorr} med djur i är får och getter hett eftertraktade kritter). Så där höll man på och glodde missunsamt på varandra fram till 2009 när någon förståsigpåare\ref{foorstaasigpaaare} med napoleonkomplex i Bryssel fick för sig att förbjuda riktiga glödlampor.

\uline{Nutid}

Sverige har alltid hållit sig utanför det dansk-norska ställningskriget och istället fokuserat sin kraft på att försöka återta Finland\ref{finland}. I och med glödlampsförbudet kom dock denna historiska oförrätt att hamna i skymundan (no pun intended) och Sverige letade istället desperat efter ett sätt att lösa den stundande krisen med svinjobbiga lågenergilampor som tar tio minuter att tända och genererar huvudvärk. Om man ska säga något positivt om Norge så är det att dom inte är med i EU så där kunde man lugnt fortsätta att läsa sina biblar i skenet av glödande grafit (i Danmark sket man fullständigt i denna fråga eftersom allt ljus där, utom solens, kommer från brinnande soptunnor). Ur denna kris växte dock en givande handelsmarknad fram där de tre länderna på ett solidariskt sätt täckte varandras behov. Norge fick dansk fulsprit att spy upp på Karl Johan, Sverige fick norska glödlampor att sätta i lampan ovanför kökssoffan\ref{kookssoffan}, och Danmark fick svenska får till sina porrfilmer.

\ditem[Den nya sångaren]\label{den nya saangaren}
 heter Brian Johnson och spelar i AC/DC (tidigare Geordie). Han tog över efter Bon Scott 1980 då den senare gått och dött av att spy sig själv i munnen\ref{mun}. Den nya sångaren kunde först höras på albumet \textit{Back in Black} (1980), men är inte lika bra som Bon Scott. Den nya sångaren bör ses på med 90\% skepsis och 10\% överseende.

\ditem[Den oambitiösa studenten]\label{den oambitioosa studenten}
 är till skillnad från sin kusin, den ambitiösa studenten\ref{den ambitioosa studenten}, inte motiverad. Studenten kan inte be bothered. Hen har just flugit ut ur sitt medelklasshem och vill egentligen hellre syssla med sin musik än att gå och lyssna på nån gubbe eller kärring som står och pratar om Egyptens koptiska minoritet eller dackefejden\ref{dackefejden} eller vad som nu står på schemat. Men ännu har DJ-karriären inte riktigt tagit fart. När studenten ändå tar sig till dessa tråkiga föreläsningar orkar han eller hon inte lyssna på alla tråkiga saker som lärs ut på den självvalda kurs han eller hon går utan använder sin iPhone4 till att kommentera Kalle\ref{kalle anka} och Stinas facebookuppdateringar. När tentan kommer skickar den oambitiösa studenten ett mejl till läraren och förklarar att hen har haft så mycket att göra, så trassligt på relationsfronten och så oerhört svårt att få tag på kurslitteraturen att hen undrar om hen inte kan få göra tentan en annan gång. Studenten skriver otaliga brev och sitter i telefonsamtal med studentkåren för att på något vis tvinga institutionen där hen pluggar att ge henom ett betyg även om hen kanske \textit{tekniskt} sett inte har gjort \textit{alla} uppgifter som står i kursbeskrivningen. Det är ju trots allt inte studentens fel att föreläsningarna inte är roligare och att han eller hon har så svårt att komma ihåg information om så oviktiga och tråkiga saker! Om studenten inte har lärt sig något är det väl lärarens fel, för det är ju ändå lärarens jobb att lära ut saker, eller hur? Vidare kan det väl inte rimligen vara studentens ansvar att själv skaffa sig böcker som redan är utlånade på universitetsbiblioteket? Studenten funderar stundtals på att åka till Berlin och arbeta i krogbranschen, skaffa sig erfarenhet och kanske lite kontakter inför den framtida musik-karriären. Studenten känner sig färdig med Sverige\ref{sverige}, som inte längre har något att erbjuda. Här är alla så \textit{square}.

\ditem[Den tyska mustigheten]\label{den tyska mustigheten}
 har på oräkneliga vis påverkat Europas kulturhistoria och är dessutom föremål för mycket skratt och hån utanför Tysklands\ref{tyskland} gränser, framförallt i Skandinavien. Den tyska mustigheten har lett till två världskrig och består i filosofiskt grubbleri av sådana som Kant, Leibniz, Nietzsche, Marx och Hegel\ref{friedrich hegel}, sådana ansiktsfrisyrer som dessa uppvisa, byggnadsverk så som Branderburger Tor i Berlin, lederhosen, apfelstrudel, Freikörperkultur\ref{freikoorperkultur}, Wagner, marschmusik, Caspar David Friedrich och mycket mer. Hos den enskilde individen kännetecknas den tyska mustigheten av ett slags koleriskt storhetsvansinne\ref{storhetsvansinne}, märkliga klädval och användandet av oväntade accessoarer, samt skägg.

\ditem[Den vedervärdige mannen från Säffle]\label{den vedervaerdige mannen fraan saeffle}
 Svensk kriminalroman\ref{kriminalroman} som handlar om samma saker som alla andra kriminalromaner.

\ditem[Dennis]\label{dennis}
 är ett namn som främst ges till busfrön och fetton. Ibland ges det även till korv.

\ditem[Deportees-trevlig]\label{deportees-trevlig}
 är ett adjektiv hämtat från Umeås medelklass. Enligt skägg-etnologen Billy Ehn hördes det första gången utanför DUO när en man på väg därifrån berättade om en middagsbjudning där \quotetext{allt gått som planerat och sällskapet skrattat vid flera tillfällen}. Samma man sägs sedan ha fortsatt berätta om ett öl från det lokala microbryggeriet. Och även om det låter rimligt, till och med troligt, så nöjer vi oss med att konstatera att det är Deportees-trevligt med en lokal Pale Ale.
Den som vill höra ordet användas i sin naturliga miljö kan besöka DUO, ICA gourmet eller Stora Fjäderägg i samband med fågelmärkning. Använd i en mening:

\begin{itemize}
\item På midsommar var vi ute vid Marcus stuga, det var så himla Deportees-trevligt!
\item Killarna på DUO är så himla Deportees-trevliga.
\item När gästerna gått hem kunde värdparet plocka bort glasunderläggen (coasters) och avsluta en Deportees-trevlig kväll.
\end{itemize}

Etymologi: Från svenskans trevlig, men innebörden ligger närmare svenskans hövlig eller artig.
Synonymer: Coldplay-trevligt.

\ditem[Der er et yndigt land]\label{der er et yndigt land}
 \textit{Der er et yndigt land} är Danmarks\ref{danmark} nationalsång. Den första texten skrevs redan 1819 men har reviderats flera gånger när man glömt bort hur den gick. Spåkforskare tror att titeln från början var \quotetext{Der er ett \uline{syndigt} land} och att ett S föll bort när någon slarvade, men det är svårt att veta säkert eftersom alla danska arkiv eldas upp på valborgsmässoafton varje år. Från början var det 12 verser men det fattar ju vem som helst att ingen dansk orkar sjunga så långt. Här är den vanligast förekommande texten:

\textit{Der er et yndigt land,}
\textit{det står med brede bøge}
\textit{nær salten østerstrand}

\textit{Det bugter sig i bygsans dal,}
\textit{den reser sig før Danmark}
\textit{og det er Frejas sal}

\textit{Den sad i fordums tid}
\textit{med højerhænden kæmper,}
\textit{en rødlætt pølse gå i strid}

\textit{Så drog de frem til fjenders mén,}
\textit{nu hvile deres bene}
\textit{bag højens bautasten}

\textit{Det land endnu er skønt,}
\textit{ollon blå sig søen bælter,}
\textit{og løvet smager så grønt}

\textit{Og ædle kvinder, skønne mø'r}
\textit{og mænd og raske svende}
\textit{runke på de danskes øer}

\textit{Hil kong Christian og fædreland!}
\textit{Hil hver en dannebroger\ref{dannebrogen},}
\textit{Alle er vi Anders And!}

\textit{Vort gamle Danmark skal bestå,}
\textit{så længe Gasolin spejler}
\textit{sin Lille du i bølgen blå}

Som läsaren märker består texten mest av mer eller mindre tydliga referenser till onani och droger, vilka båda är kärnämnen i dansk grundskola.

\ditem[Det gamla Silvio Berlusconi-knepet]\label{det gamla silvio berlusconi-knepet}
 är ett retoriskt grepp som används av slipade makthavare för att ge sken av en politisk nystart. På julafton iklär sig makthavaren (knepet används bara i länder där makthaverskor inte existerar på denna nivå) ett par vita y-front och lägger sin i ett åsnetråg\ref{traag} och sprattlar och jollrar glatt likt ett spädbarn. Proceduren pågår tills alla blivit glada. Detta är en symbolhandling för att visa att makthavaren lagt det gamla bakom sig och är redo att börja om på nytt. Jozef Ratzinger använde framgångsrikt detta knep när illvilliga medier började anklaga katolska\ref{katolik} kyrkan för att ha osunda åsikter. I Sverige är det främst renässansmannen Fadde Darwich som använt knepet för att markera en ny inriktning i livet, till exempel när han skulle bli ståuppkomiker 

\ditem[Det gamla Thore Skogman-knepet]\label{det gamla thore skogman-knepet}
 är en metod för att dölja fisar om man vistas i grupp. Metoden uppfanns av Thore Skogman när denne var magsjuk och hade blivit inbokad till ett direktsänt framträdande i \textit{Hylands hörna}. Thore märkte under intervjun i sändingen att han hade en fis i kroppen som var mycket angelägen att komma ut. För att inte göras till åtlöje inför hela svenska folket\ref{gooras till aatlooje infoor hela svenska folket} bröt Thore utan förvarning ut i sin älskade hitlåt \textit{Surströmmingspolka} och passade i refrängen på att lätta på gasen. Hyland märkte inget utan trodde bara att Thore glömt borsta tänderna och fått en släng av fetor ex ore\ref{fetor ex ore}.

\ditem[Det omedvetna]\label{det omedvetna}
 är en term inom psykologins historia, som idag är arkaisk men som har haft ett visst betydande för det kontinentala tänkandet om människans inre värld. I en liten bok med titeln \textit{Die Mench und der unbewusstheit} (1922) framlade den Österrikiske psykoanalytikern Hermann Taschenmesser en teori om människans psykologiska dimension i vilken han i det stora hela understödde Freuds teorier om överjaget, jaget och det undermedvetna, men lade till det omedvetna, som enligt Taschenmesser förklarade människors ignorans och generella dumhet. Överjagets kontrollerande funktion och det undermedvetnas funktion som lagerrum för undantryckta minnen tycktes inte bevisa att människor enligt Taschenmesser inte i tillräkligt stor utsträckning köpte och läste hans första bok, den självbiografiska \textit{Herr Taschenmesser - Ein schrecklich netter Mensch} (1916). Därför krävdes ännu en nivå i teorin om människans inre värld som kunde förklara detta för Taschenmesser obegripliga faktum. Det omedvetna förklaras som en kall och mörk plats i vårt inre där dumhet och oförstånd är förlagt. Det är enligt Taschenmesser ofta det omedvetna som ligger bakom sådant som att många människor inte köper och läser en fantastisk bok om den så ligger mitt framför dem och bjuds till försäljning för ett mycket generöst nedsatt pris.

\ditem[Det politiska i C.C.Rs texter]\label{det politiska i c.c.rs texter}
 är en bok av författaren, nervvraket, den ensamme trebarnsfadern och geniet\ref{storhetsvansinne} John Andersson. Det är en djupläsning av Creedence Clearwater Revivals texter, med fokus på det samhällskritiska. Man tänker lätt att detta genuina \quotetext{feelgood}-band hellre skrev om att ha kul, men allvaret stod ibland för dörren och då var inte The Fog\ref{the fog} sen att uppmärksamma detta.

\uline{Fortunate Son}

En klassanalys av vietnamkriget. Foggan tyckte att de som vurmade mest för att starta kriget var de som inte skulle åka dit och kriga, det vill säga överklassen. Han tyckte att flaggviftande rikemansungar (Fortunate sons) med silversked i skitan gått kunde hålla käften, eller hänga på sig en M-16 och åka dit och \quotetext{göra skillnad}.

\uline{Ramble Tamble}

Denna låt är bara samhällskritisk i den utsträckning som bluesmusik är det generellt. Inga pengar, ingen mat, lån att betala, räkningar att betala, då tar man sitt pick och pack och \textit{Ramble Tamblar} därifrån, en praktik som är vanlig i samhällen med en låg dekommodifieringsgrad. En intressant detalj är att Foggan redan 1971 förutspådde Ronald Reagans tillträde som president med textraden \quotetext{actors in the white house}. Om det amerikanska folket bara hade lyssnat.

\uline{Wrote a song for everyone}

Här möter vi The Fog\ref{the fog} i hans mest cyniska, men kanske också sannaste, stund. Textraderna som det bör läggas vikt vid lyder: \quotetext{Saw the people standin' thousand years in chains. Someone says it's diff'rent now, look, it's just the same}. Vilka kedjor som åsyftas är något oklart. Antingen är det svarta i U.S.A, slaveriet och kopplingarna till rasismen i dagens amerikanska samhälle, eller de kedjor som de härskande klasserna slagit på arbetare genom alla tider.

\uline{Om boken}

\textit{Det politiska i C.C.Rs texter} är fyra sidor lång och kan laddas ner som pdf från Malå\ref{malaa} Sågverks\ref{saagverk} hemsida.

\ditem[Det stora fosterländska kriget]\label{det stora fosterlaendska kriget}
 är den ryska benämningen på vad vi andra kallar det andra världskriget. Under detta krig räddade Sovjetunionen Europa från den tyska mustigheten\ref{den tyska mustigheten} och efter dess slut lät ryssarna bygga vad man i öst kallar \quotetext{den anti-fascistiska skyddsvallen} och i väst \quotetext{Berlinmuren.}

\ditem[Det stora vadet 2013]\label{det stora vadet 2013}
 handlade om huruvida Marko varit i Finland\ref{finland} eller inte. Vad som började som en skämtsam diskussion mellan två vänner kom under kvällen att urarta i en skyttegravsträta där de tvistande parterna hela tiden höjde sina insatser. Den ena parten hävdade bestämt att Marko varit där, vilket enkelt kunde bevisas genom fotodokumentation från den finlandsresa de tre inblandade gjort tillsammans. Motparten å sin sida menade att något sådant fotobevis\ref{fotografering} inte existerade, eftersom Marko aldrig var med. Frågan kan för en utomstående tyckas enkel; det är väl bara att fråga Marko. Ett rimligt antagande i vanliga fall, men när insatsen blivit så hög som i detta fall måste alla yttre parter synas noga i sömmarna innan de kan tillåtas avlägga vittnesmål. Någon kommer trots allt bli ruinerad när korten ligger på bordet. Sista ordet är ännu inte sagt i frågan men följande står på spel:

\begin{itemize}
\item 1 platta bärs
\item 10 timmar hemhjälp
\item En prydlig slips
\item En ståtlig hatt
\item En FIAT Panda till ett marknadsvärde av max 3.000 kronor
\item Två fiskedrag av märket Rapala
\item En guidad tur (i naturen)
\item Ett besök i valfri stad
\item 1 liter spolarvätska (koncentrerad)
\item 5 liter bensin i reservdunk
\item Ronnys gröna tennisshorts
\item 
\item En demokassett med No Fun At All ELLER första sjuan\ref{foorsta sjuan} med Atomångest
\end{itemize}

\ditem[Det susar i Säfve]\label{det susar i saefve}
 \textit{Det susar i Säfve} är en älskad barnbok av den brittiske författaren Kenneth Greheme (Org.titel \textit{The Wind in William Carlos Williams}). Boken handlar om gasbildningar och olika sorters baksug i den norrbottniske\ref{norrbotten} författaren och debattören Torbjörn Säfves matsmältningssystem. Boken har givit upphov till en hel massa adapteringar i form av teateruppsättningar, ballettföreställningar, stop-motionfilmer och till och med ett dataspel, i vilket spelaren försöker släppa på trycket i tarmsystemet genom att leda ut gaser genom magmunnen\ref{oovre magmunnen} och andra kaviteter, så att Säfve inte plågas av knip utan kan koncentrera sig på sitt författarskap och sin anarko-stalinism.

\ditem[Diagram]\label{diagram}
 är ett slags bild som gör att det blir relativt svårt att förstå något relativt tråkigt, till exempel antalet personer i ett medelstort svenskt län som är över femtio bast och kör Merzedes-Benz. \textit{Dia} betyder \quotetext{över} eller \quotetext{ut} och har språkhistoriskt att göra med geometriska figurer som ritas med hjälp av linjer. \textit{Gram} är ett vanligt viktmått när man köper hasch\ref{hasch}.

\ditem[Dialektik]\label{dialektik}
 är ett logiskt förfarande som används inom retorik, filosofi, naturvetenskap och politisk teori. Framförallt ligger det till grund för Marx historiematerielistiska teori om historiens utveckling, klasskampen och kapitalets tillkommande. Enkelt uttryckt består dialektiken i en progression som är resultatet av bildandet av en syntes av en tes och en antites. Följande exempel illustrerar en sådan dialektisk progression.

Två polare, vi kan kalla dem Karl och Friedrich, diskuterar vad de ska äta till kvällsfika. Karl föreslår varma mackor (tes), medan Friedrich prompt vill ha glass (antites). Denna motsättning får som resultat att Karl och Friedrich i slutändan äter varsin klementin (syntes).

\ditem[Dieselbil med lastgaller]\label{dieselbil med lastgaller}
 köres företrädesvis av ekonomiskt sinnade människor.
LB-reggad Merca är inte bara snyggt, skatten blir mycket billigare och man kan lägga pengar på annat, till exempel röd diesel från Finland. Storswänsken\ref{storswaensk} påstår sig förlora väldigt mycket pengar på de glesbygdsbor som kör LB-reggat på finndiesel. Säkert en promille av vad man förlorar på att tina bort snön från gatorna med elström istället för att skotta bort den med spade, som mer utvecklade folkslag gör.

\ditem[Dimma]\label{dimma}
 Ett meterologiskt fenomen. Det är typ vatten som inte är regn eller sånt man dricker i glas utan bara en rätt cool ånga som ofta infinner sig i London och de skotska högländerna. Dimma är dock mest känt för att Dag Vag skrev sin enda bra låt om dimma. Den heter just \textit{Dimma} och går att åtnjuta på skivan \textit{Scenbuddism} från 1979.

\ditem[Din avkomma och du]\label{din avkomma och du}
 är Prof. Etiennes mästerverk om barnuppfostran och har knappt sålts i tretio exemplar världen över. Den bör ses som ett direkt komplement till \textit{Barnagans förträffliga pedagogik\ref{barnagans foortraeffliga pedagogik}}. Vill du ha hela baletten får du köpa den direkt via förlaget, men här kommer ett urval av den gode professorns handfasta tips och råd.

\begin{itemize}
\item \textbf{Du är större än ditt barn}. Använd detta till din fördel så ofta som du kan, vare sig det gäller brottning eller i argumentationer med din fru om vem som ska ha mest att äta till middag
\item \textbf{Barn är helt jävla dumma i huvet}. Barn vet ingenting om någonting och det är din och bara din uppgift att lära dom hur det ligger till. Fritänkande är en enkelbiljett till rännstenen.
\item \textbf{Var en så liten del av ditt barns liv som möjligt}. På så vis kommer barnet uppskatta ditt sällskap mer, skäm inte bort barnet med närvaro och kravlös kärlek. Trams.
\item \textbf{Sanningen hör man av barn och fyllon}. Se till att vara helt jävla grisfull\ref{grisfull} så ofta det går, för att förenkla kommunikationen med ditt barn.
\end{itemize}

\ditem[Diskett]\label{diskett}
 En diskett är ett elegant litet föremål avsett för lagring av information. Disketten förs varsamt in i en datamaskin, varpå informationen (må det vara en bild- eller en textfil) kan föras över på den lilla disketten som sedan kan petas ut och lagras i ett diskett-etui, skickas per brev till avsedd mottagare eller helt sonika säljas till högstbjudande. Disketten finns i många olika utföranden. Den i särklass mest populära är 3.1/2"-disketten, som bland annat saluförs i paket om tio på välsorterade datavaruhus. Konsumenten kan få sina disketter i önskad färg och etiketter finns att tillgå för den ekonomiskt välbärgade\ref{storfraesare}, så att användaren kan nedteckna på disketten vad den innehåller. På så vis undviks förvirring i diskett-arkiv som med tiden växt sig så stora att de annars blir svårhanterliga.

\ditem[Djur]\label{djur}
 Det har nyligen kommit till artikelförfattarens kännedom att naturen ska vara full av olika sorters djur så som talgoxar\ref{talgoxe} och brokigt färgade fiskar, vilket djupt har chockat honom. Djur ska tydligen vara varelser som saknar förmåga att tala. Vissa är stora, medan andra tvärtom är mycket små.

\ditem[Djurens nobelpris]\label{djurens nobelpris}
 är en utmärkelse som instiftats för att uppmärksamma särskilt smarta individer i djurens värld. Den första att få priset var apan Koko, för sin uppfinning att man kan använda en pinne som redskap. Eftersom det inte finns jättemånga smarta djur är det inte säkert att priset delas ut varje år. Ibland kan juryn avstå från att dela ut det och ibland kan till synes konstiga saker premieras. En förklaring till det är att djuren som sitter i juryn ibland är av samma ras som pristagaren. Priset är på hundra djurdollar och kan användas på världens alla zoo. Trots att djurens nobelpris funnits i mer än hundra år är det färre än tio tjejdjur som fått det.

\uline{Vinnare av djurens nobelpris}

\begin{itemize}
\item 1977 Den snefotade ultrapelikanen\ref{snefotad ultrapelikan} Klas för sin upptäckt att gamla chips smakar precis lika bra som fisk.
\item 1955 Uven\ref{uv} Flaxis för sin lansering av teorin om att det bara är ett eko man hör när man hoar, och inte signaler från ett parallelt uvuniversum.
\item 1954 Rodriguesflyghunden för sin strävan att ena djur och människor.
\item 1942 Apan Albert II\ref{albert ii} för sin upptäckt av rymden.
\item 1918 Paradisfågeln Gunborg\ref{gunborg} för sin uppfinning att bygga vackra bon av gammalt skräp\ref{skraep}.
\item 1917 Hästen Natasha för att ha ridit med Lenin på ryggen in i Vinterpalatset.
\item 1907 Brugden\ref{brugd} Sune för sin upptäckt att havet är djupt.
\item 1902 Hunden Rex för sin uppfinning att tigga vid matbordet.
\item 1901 Apan Koko för sin uppfinning pinnen.
\end{imtemize}

\ditem[Dojo]\label{dojo}
 Varje villaägare av rang som är vid sina sinnens fulla bruk låter inreda en dojo i källaren. I dojon bör det finnas en matta av något mjukt materiel som tillåter en rulla på golvet när man tränar utfall. Det bör också finnas ett ställ med katanas, bambustavar och olika fantasirikt utformade vapen som beställts från hobbex eller någon liknande välsorterad postorderkatalog. Till yttermera visso kan det finnas anledning att skaffa en ninjaformad punchbag som man kan träna kaststjärnekastning på.

\ditem[Doktorand]\label{doktorand}
 är världens smartaste anka.

\ditem[Dokumentärhora]\label{dokumentaerhora}
 Att dokumentärhora är att kompromissa med sin personliga integritet för att få sitta i dokumentärer och hypa folk/band/företeelser i utbyte mot exponering i dokumentären, lite stålar och i bästa fall en näve cashewnötter i sminken innan filmning. Ett praktexempel på denna företeelse är Scott Ian, gitarrist i bandet Anthrax, som vid åtskilliga tillfällen synts i billiga VH1-studios, hypandes Metallica, Lemmy eller vem fan som helst, utan förbehåll. Gott så, Lemmy förtjänar oftast förbehållslösa hyllningar. Men uppdraget innebär också att dokumentärhoraren förväntas hylla bandet den pratar om som att det revolutionerat genren det verkar i. Och hur mycket Scott Ian än må gilla Flotsam \& Jetsam eller Armored Saint, är påståendet att de revolutionerat sin respektive genre rent struntprat och historieförfalskning. 

Att sätta sig till doms över vad någon göra med sin kropp, som att sälja den i sexuellt syfte, är inte någonting vi på Nissepedia ägnar oss åt. Att prostitution existerar är ett resultat av ett patriarkalt kapitalistiskt system, således kan inte enskilda individer belastas för vad de hittar på med sina kroppar. Däremot kommer historiens dom att falla hårt över de som förvanskar och förringar historiska händelser mot betalning i form av nötter, en handfull dollars och tio minuters uppmärksamhet i VH 1. 

Leif GW Persson har föreslagits som dokumentärhora, vilket dock är helt felaktigt. Han är proffstyckare och anställs således för att tycka lite vad som helst om vad som helst.

\ditem[Dopesmoker]\label{dopesmoker}
 är en låt av den amerikanska rockorkestern Sleep och var 2009 års sommarplåga. Låten är otroligt hyllad i knarkarkretsar.
Vill man vara en lustigkurre kan man ringa till nåt önskeprogram i radio och önska den.

\ditem[Dr. Alban]\label{dr. alban}
 Svensk tandläkare som utselöts ur Svenska tandläkareföreningen efter att i en duett tillsammans med Kicki Danielsson\ref{kicki danielsson} ha uppmanat till omåttlig konsumtion av söt frukt.

\ditem[Dragbasun]\label{dragbasun}
 är ett musikinstrument inom familjen bleckblåsingar som uppfanns för att ersätta den betydligt enformigare fanfartrumpeten. Vad som gör dragbasunen unik jämfört med andra basuner är att man drar i den istället för att trycka på knappar. Ju mer man drar, desto bättre låter det. Kända svenskar som kan traktera dragbasun är exempelvis Janne \quotetext{Loffe} Carlsson\ref{skita i det blaa skaapet} och Jan Guillou.

\ditem[Dragsko]\label{dragsko}
 är den lilla hålgången längst upp på mysbyxor, till exempel, där det finns ett snöre som alltid åker in och som man inte kan få ut igen, om man nu har några. Dragsko nämns ofta tillsammans med uttrycket \quotetext{en byxa} - typ som \quotetext{en bekväm byxa med dragsko.} Dragskon är inte speciellt populär för tillfället, förutom i skid- och friluftskläder där det är mäkta populärt.

\ditem[Dragspel]\label{dragspel}
 har i decennium efter decennium varit diskriminerat av fiolspelmän runt om i Sverige. De har ansett att dragspelet inte kunde begära få sin plats inom allmogekulturen, men ingenting kan vara mer felaktigt. Dragspelet har tillsammans med fiolen och i viss mån klarinetten i vissa landskap - t.ex. Västmanland - varit ett av instrumenten som funnits med vid dansbanor och logar och som trakterats av drängar i kammaren, då de fått någon ledig stund över.

Nu är dragspelet helt accepterat som allmogeinstrument och det är därför i hög grad representerat vid svenska spelmansstämmor. Dragspelsstämman i Hjo näst sista veckan i augusti kan därför ses som en hyllning till dragspelet som instrument och uttolkare av dagens och gårdagens melodier.

\ditem[Dragspelsmuskeln]\label{dragspelsmuskeln}
 är en muskel i människokroppen utan riktigt bestämd placering. Troligen sitter den någonstans i överarmen eller på torson. Det är vanligt att idrottare skyller dåliga tävlingsresultat på sträckt dragspelsmuskel.

\ditem[Drakonisk lag]\label{drakonisk lag}
 En drakonisk lag är en lag som är omänskligt sträng. Ordet drakonisk härstammar från latinets \textit{draconis} vilket betyder ungefär \quotetext{gasig ödla}. Alla lagar som reglerar alkoholintag i olika situationer betecknas som drakoniska, och likaså lagar som inskränker den av Gud\ref{gud} givna rätten att köra bil snabbt.

\ditem[Dressmann]\label{dressmann}
 är en norsk herrkläd-kedja. \quotetext{Dressmann} är ett anagram av \quotetext{Der SS-Mann}. Man får dra sina slutsatser själv.

\ditem[Driva med överheten]\label{driva med ooverheten}
 Klassiskt grepp för att klä sin grådaskiga vardag i en lite ljusare nyans. Man kan till exempel påpeka att påvens hatt är för jävla fånig eller att ostron ser ut som snor. Eller göra som Idi Amin Dada vid ett statsbesök i England 1971 där han bad att få besöka Skottland, Irland och Wales för att träffa \quotetext{revolutionärer som kämpar mot ert imperialistiska förtryck}.

\ditem[Dubbelsovla]\label{dubbelsovla}
 Att dubbelsovla (även \quotetext{tvesovla}) är att ha två sorters pålägg på mackan, t.ex. både ost och skinka. Det här ses som slösaktigt och väldigt ofint. Vad som inbegrips i sovel, det vill säga, om t.ex. smöret räknas som sovel och således gör en vanlig ostmacka\ref{ostmacka} till en dubbelsovlad styggelse, är en diskussion som kan behöva tas i vissa delar av konungariket Sverige\ref{sverige}.

\uline{Empiriska studier i ämnet}

I vissa skånska släkter kallas en dubbelsovlad smörgås för \quotetext{Perbengsare} efter mågen\ref{maag} Per Bengtsson. Denne (då blivande) måg\ref{maag} dubbelsovlade under en frukost när han uppvaktade en flicka i släkten, något som inte uppskattades av svärföräldrarna. \quotetext{Karin ente ska du gefta dig me han, så slösaktig som han är!} lär gubben ha sagt. Därefter lär päronen ha insett vilken storfräsare\ref{storfraesare} han var och sedan välkomnat honom i släkten.

\ditem[Duvgubbar]\label{duvgubbar}
 är en benämning på människor som ägnar alldeles för stor del av sin fritid till att föda upp duvor och tävla med dessa makalösa fjäderfän. Tydliga kännetecken är att gubben i fråga har ett flertal duvslag och ungefär ett två tum tjockt lager duvskit på tomten. Det är då förstås inga vanliga stadsduvor, eller \quotetext{flygande råttor} som duvgubbarna själva kallar dem utan det finns två sorter. Den ena sorten är brevduvor som används i sporten duvracing, där uppfödarna släpper ut ett gäng duvor var och ser vem som hittar hem snabbast. De här duvorna orienterar sig genom rent jäva trolleri och hur långt från boet man än släpper ut en duva så hittar de hem. Det närmsta ett svar som den moderna vetenskapen lyckats producera är att duvorna är \quotetext{präglade av hemmet}, vad nu det ska betyda. Den andra sortens duva är tumlarduvor som används i tävlingar där deltagarna får sin duva att flyga så högt som möjligt. 

Duvrace är även big business, i alla fall i skitlandet Belgien\ref{belgien}, och den som plockar hem titeln Belgian Master får också en solkig attachéväska fylld med den ohemula prissumman 100 000 euro. Vad som är än mer upprörande är att duvor som vunnit många tävlingar penisoneras, som travhästar, till avel och rekordet för vad en sån här duva bringat hem på en auktion är den om möjligt än mer ohemula summan 2,7 miljoner kronor.

Som ni förstår gör detta att duvgubbarna är beredda att ta till drastiska åtgärder för att skydda sina älsklingar och det är alls inte ovanligt att de förgiftar utrotningshotade rovfåglar för att undvika att en duva slutar som duvfärs i nåt falkbo. Det här leder i sin till att duvgubbar löper fyrtiotusen miljarder\ref{fyrtiotusen miljarder} gånger större risk än gemene man att bli indragna i långa processer med länsstyrelsen\ref{processa mot laensstyrelsen}.

\uline{Duvgubbar i populärkulturen}

Duvgubbar figurerar i TV-serien \textit{The Wire} likväl som i förrädaren Elia Kazans film \textit{Storstadshamn} från 1954. I Jim Jarmuschs film \textit{ Ghost Dog } spelar Forrest Withaker en yrkesmördande duvgubbe som lystrar till samurajernas hederskodex\ref{samurajernas hederskodex}.

\ditem[Dvärgpungsovare]\label{dvaergpungsovare}
 är ett litet pungdjur som ser ut ungefär som en vanlig mus. Naturligtvis återfinns det som världens alla andra konstigaste djur i Australien\ref{australien}. Arten är sjukt utrotningshotad och återfinns idag bara inom ett område på ungefär tio kvadratkilometer i en del av New South Wales som ofta är täckt av snö. Där har någon driftig australiensare självklart slagit upp en stor jävla skidanläggning med flera pister.

\ditem[Dyckert]\label{dyckert}
 är en typ av spik som i princip saknar huvud\ref{huvud}. Det är ibland pyttelite bredare upptill på den men knappt så det syns. Dyckert används främst inomhus när man ska spika lister och liknande och inte vill att spiken ska synas.
Inom byggsvängen kallas ibland tjänstemännen elakt för dyckertar av byggarbetarna.

\ditem[Däcka]\label{daecka}
 Att däcka är att somna på ett plötsligt och dramatiskt vis, och på så vis inrätta sig så att man ligger horisontellt mot golvet. Uttrycket kommer sig av att just detta är vanligt bland personer som arbetar inom sjöfarten, på grund av det mer eller mindre obligatoriska, mycket generöst tilltagna intaget av rusdrycker. Däcket är ju som bekant ett av golven, nämligen det översta, på en skuta, och därav uttrycket.

\ditem[Dävert]\label{daevert}
 En dävert är ett klassiskt redskap som i mer än 100 år hjälpt fartyg i nöd att sjösätta sina livbåtar. Ofta består däverten av en vertikal stolpe med krökt topp i en stor båge, eftersom denna form anses vara den bästa att hänga upp en livbåt i. För större livbåtar behövs normalt två dävertar som håller varsin ände, för tillsammans är man som bekant starkare. Världens mest kända dävertar är förmodligen Axel Welins kvadrantdävertar, som producerades i Kolsva, Västmanland, och installerades på Titanic.

Dävertar är enligt uppgifter på Internet också ett populärt motiv för vrakfotografer. Enligt andra, helt obekräftade, uppgifter har dävertar också gett upphov till hobbyfenomenet dävertspotting\ref{daevertspotting}.

\ditem[Dävertspotting]\label{daevertspotting}
 är en roman av den store svenske författaren Prof. Etienne. Romanen utspelar sig i Sveriges\ref{sverige} Edingburgh, Södertälje, och handlar om en grupp unga män som ägnar sig åt att besöka skepp för att studera dess dävertar\ref{daevert}, samt att injicera stora mängder heroin medan de är i farten. Romanen utsågs snabbt till en av Sveriges genom tiderna mest mediokra verk vilket gjorde att den blev halvkänd och lästes av ett lagom stort antal personer, av vilka många tyckte att den var okej.

%%%%%%%%%%%%%%
\newpage
\null
\\
\null
\\
\Huge
E
\normalsize
\\
\null
\\
\null
%%%%%%%%%%%%%%

\ditem[E=mc2]\label{e=mc2}
 Slanguttryck i Köpenhamn med omnejd för att ta syntetiska droger och köra två motorcyklar samtidigt. Hojarna måste vara vända åt varsitt håll för att en ensam person enkelt ska kunna komma åt båda gashandtagen så färden går mest i cirklar.

\ditem[Ebbe]\label{ebbe}
 är ett palindrom, vilket bara det är lite suspekt. Dessutom träffar man sällan personer med detta namn, vilket kan få en att i all rätt begrunda vad alla som heter Ebbe gör hela dagarna och var de befinner sig.

\ditem[Edmund]\label{edmund}
 känns spontant som ett lite festligt namn som är lätt att associera med den brittiska bildade borgarklassen: \quotetext{Edmund, pray do you take sugar with your tea?} \quotetext{Oh, I couldn't impose Mrs. Doppelworth!} Kända bärare av namnet är den nya zeeländske alpinisten och filantropen Edmund Hillary\ref{edmund hillary}.

\ditem[Edmund Hillary]\label{edmund hillary}
 Sir Edmund Hillary var en Nya Zeeländsk\ref{nya zeeland} alpinist och en av de första att bestiga planeten jordens högsta berg, Mount Everest\ref{mount everest}. Bestigningen gjordes tillsammans med sherpan Tenzing Norgay och när dessa två äventyrare nått toppen uppstod ett dilemma. Vem ska fota på toppen? Lotten föll på Hillary då nepalesen Norgay aldrig sett en kamera förut och, som Hillary träffsäkert lär ha uttryckt det: \quotetext{Toppen av Mount Everest är knappast en bra plats att förklara hur en kamera fungerar}. Således borde Norgay vara den förste mannen att beträda toppen, men detta har debatterats.

\ditem[Edvin]\label{edvin}
 är ett så sällsynt namn att om man mot förmodan träffar någon som lystrar till namnet Edvin bör man genast fotografera sig med sin mobilkamera brevid den personen. Man kan då hålla upp ett ironiskt handtecken som man hämtat från någon hip-hopskiva och bara har ett hum om vad det kan betyda.

\ditem[Egendom]\label{egendom}
 är stöld.

\ditem[En bärs, en bärs, min järndanksamling för en bärs]\label{en baers, en baers, min jaerndanksamling foor en baers}
 Del tre\ref{trea} i Prof. Etiennes\ref{prof. etienne} svit självbiografiska romaner.
 
\uline{Synopsis}

Alla goda ting har ett slut, något Prof. Etienne blir medveten om när ledaren i den sexkult han gått med i blir arresterad på Bälinge torg efter att beväpnad med en Luger ha delat ut antisemitisk propaganda. Vid arresteringen av ledaren upplöses sexkulten med omedelbar verkan. Vilsen och förvirrad irrar Prof. Etienne nu runt i Bälinge, enbart utrustad med en påse järndankar\ref{dank}, den enda gångbara valutan i sexkulten, samt ett par haremsbyxor han bytt till sig av Claes Malmberg i den ormgrop där Gottfrid Svartholm Warg blev till. Som så många andra sökande exkultister i Bälinge fann sig Prof. Etienne inom kort levande som busstationsalkoholist. Under en redig bläcka med några skräniga kärringar får professorn slut på bärs\ref{ha baers} och ropar ut i högan sky de bevingade orden \quotetext{En bärs, en bärs, min järndanksamling för en bärs}. En av damerna i sällskapet fick då nog och kastade en Arboga 10.2\% och ett slitet exemplar av Sven Hedins första reseskildring \textit{\quotetext{Genom Persien, Mesopotamien och Kaukasien}} på vår hjälte. Det var efter att ha inmundigat den goda brygden och sträckläst den vise Hedins ord som Prof. Etienne beslöt sig för att ta tag i sitt liv och själv bli upptäcktsresande!

\ditem[Ekfors Kraft]\label{ekfors kraft}
 var ett elbolag som ägdes av Mikael Styrman. Styrman hade en lika ondskefull som genial affärsidé som han satte till verket 2007. Ekfors Kraft fick i uppdrag av Överkalix och Haparandas kommuner att tillhandahålla gatubelysning, något som Ekfors Kraft levererade utan bekymmer till en början. Här kommer Styrmans plan in: när kommunerna väl blivit beroende av hans bolag så såg han helt enkelt till att höja priset till ungefär fyrtiotusen miljarder\ref{fyrtiotusen miljarder} kronor/kwh, så att det lönar sig liksom. Tjänstemännen och politikerna fick såklart spel, men Styrman vek sig inte en tum. För er läsare som inte varit i Norrbotten\ref{norrbotten} på vintern kan vi på Nissepedia\ref{nissepedia} informera att det är jävligt mörkt där då. Därför är gatubelysning en av lokalbefolkningen mycket uppskattad modern bekvämlighet, så kommunerna var inte direkt i någon position att vänta ut Styrman och det hela gick till rättegång. Kommunerna vann och Ekfors Kraft sattes, som den kvicktänkte redan listat ut via den inledande meningens tempus, i konkurs. Styrman var missnöjd över detta.

\ditem[Elva]\label{elva}
 Längdskidsterm för beskrivning av kraftigt rinnande snor.

\ditem[Émile Durkheim]\label{émile durkheim}
 Émile \quotetext{Durken} Durkheim (1858-1917) var en fransos som uppfann durkslaget. Han lanserade även en del sociologiska teorier, vilka ledde till att han postumt förärades titeln: \quotetext{Världens minst kända kända sociolog}.

\ditem[England]\label{england}
 är ett land som ligger på den största av de brittiska öarna och är en del av det förenade kungariket. Det har varit centrum i världens största imperium om man inte räknar det amerikanska nykoloniala imperiet. Men, let's face it, England är inte längre det där storslagna landet det ibland talas om. Maten får en att spy bara av att tänka på den, husen är dåligt byggda, de kör på fel sida, super som svin, förtrycker sin arbetarklass, som för övrigt är väldigt argsint och de har världens mest invecklade och därtill dåliga parlamentariska system. Det finns inte ett levande träd så långt ögat når, inga djur, bara gräs och grus. Städerna ser ut som skit och man ska vara glad om man tar sig därifrån oskadd.

\uline{Ägarförhållanden}

England tillhör, enligt säkra källor, Colin McFaull i Cock Sparrer.

\ditem[Enkelbeckasin]\label{enkelbeckasin}
 (Fornsv. \textit{enkel} ung. \quotetext{okomplicerad} Forndan. \textit{bekasin} ung. \quotetext{bensin}) är en vadarfågel i familjen snäppor. Enkelbeckasinen är brunspräcklig, stor som ett mjölkpaket ungefär och har en lång, rak näbb som den använder för att äta med. Näbben ser ut ungefär som en blyertspenna.

\ditem[Enkido]\label{enkido}
 är Gilgameshs polare i Gilgamesheposet, som skrevs nån gång för sisådär 3000 år sedan i Mesopotamien. Enkido och Gilgamesh drar i eposet omkring och rullar hatt\ref{rulla hatt}, hittar på jävelskap\ref{jaevelskap} och retas med jättar och oxar, men sen mular tragiskt nog Enkido. Gilgamesh klär då ut sig till ett djur\ref{klae ut sig till ett djur}. Detta är början på en gripande skildring av sorg och saknad efter en död vän.

\ditem[Entreprenör]\label{entreprenoor}
 Person som säljer sommarkatter på våren och kräftbete\ref{kraeftbete} på hösten.

\ditem[Epikurism]\label{epikurism}
 är en filosofisk åskådning namngiven efter Epikuros, en grekisk man som levde på 300-talet före kristi\ref{jesus} födelse. I korthet går läran ut på att människans främsta uppgift är att eftersträva ett ganska sjysst, lugnt och tillbakalutat liv med mycket njutning i glada vänners lag. Människan ska undvika sådant som orsakar själslig och kroppslig smärta i henne själv och i andra. Är man törstig ska man dricka. Är man hungrig ska man käka. Är man taggad på att lyssna på The Who's \textit{Live at Leeds} på helgvolym\ref{helgvolym} ska man, får man anta, göra det. Som alla förstår är epikursmen statsreligion i kungadömet Danmark\ref{danmark}. 

Detta låter ju ganska bra kan vissa tycka, och framförallt har epikurismen efterlämnat sig ett lika underanvänt som tokbra adjektiv, nämligen \textit{epikurisk}. Ordet sätts här i språklig kontext för läsarens intellektuella vinning: \textit{"TB. Det är den där epikuriska mannen med gröna småbyxor som bor i Umeå }\ref{umeaa}\textit{.}"

\ditem[Erik Hamrén]\label{erik hamrén}
 är en svensk man som gjort det uppseendeväckande valet att alltid gå klädd i väst. Därför har han blivit utvald att träna och leda Sveriges\ref{sverige} landslag i fotboll\ref{fotboll}, vilket väl går sådär, om man ska vara helt ärlig. Många fotbollsentusiaster reagerade med resignation då SIFO tydligt visade att Hamrén lagt endast 39\% av sin arbetstid på att träna fotbollslaget och hela 120\% på sin egenproducerade kammarspel, \textit{How the väst was won}, där han också spelar huvudroll.

\ditem[Erik Homburger Erikson]\label{erik homburger erikson}
 var en psykopat tillika psykoanalytiker. I stort sett stal han Sigmund Freuds tankar och påstod glatt att det var hans egna. Vetenskapliga bevis var tydligen inte ett krav på såna som höll på med vetenskap förr i tiden. Eftersom inte så många ägde papper och penna räkte det för de som ägde sådana att skriva jättemycket om något de hittat på och vips så fick man titeln vetenskapsman! Av någon outgrundlig anledning så studeras hans verk fortfarande, detta trots att det finns nya, minst lika dumma tankar om människan och hennes natur.

\ditem[Erna]\label{erna}
 är ett kvinnonamn som kom till Sverige redan med visigoterna. Dess betydelse är omtvistad men den vanligaste tolkningen är \quotetext{den som inte spottar i glaset}. Den maskulina formen av Erna är Arne, som man får om man skriver namnet baklänges. Den keltiska versionen av namnet är Enya.

\ditem[Ernst Billgren]\label{ernst billgren}
 är en av Sveriges\ref{sverige} mest välkända och uppskattade konstnärer. Det finns enligt forskningen två anledningar därtill:

\begin{enumerate}
\item Han målar vagt humanoida ankor. Detta gör att många har lätt att känna igen sig              i hans bildvärld.
\item Han trasslar inte till det med nåt slags djupare tanke med sin konst.
\end{enumerate}

\ditem[Ernst Haeckel]\label{ernst haeckel}
 (16 februari 1834 – 9 August 1919) var en tysk\ref{tyskland} som 20 september 1914 blev den som myntade begreppet \quotetext{första världskriget}. Haeckel var dock mycket mer än en mustig\ref{den tyska mustigheten} tysk som satt i en gigantist läderfotölj med en näve under den skäggbeprydda hakan och funderade ut domedagsmättade termer. Han var också en världsledande biolog med smak för att psykedelisera sin vardag\ref{att psykedelisera sin vardag}, genom att rita och färglägga spejsade bilder av maneter och allsköns frutti de mare. Troligen led han också av ett fullt utblommat storhetsvansinne\ref{storhetsvansinne}, men det kan man ju inte hålla emot honom eftersom han uppenbarligen höll på med väldigt coola saker. 

\ditem[Eskimåspov]\label{eskimaaspov}
 (\textit{Numenius borealis}) är en jätteovanlig spov i familjen fåglar. Det kan till och med hända att den är så ovanlig att den blivit utdöd. Ingen har i alla fall säkert sett någon sen 1980-talet. Om den finns kvar så bor den i Kanada och Alaska på sommaren och på Pampas slätter\ref{slaett} på vintern. Den kvittrar klara visslande toner och har en lätt nedåtkrökt näbb som den kör ned i jorden för att hitta mask och annat den felaktigt tror är mat. Den är tillräckligt liten för att få plats i en normalstor cigarrlåda.

\ditem[Ett kille]\label{ett kille}
 är en ung man som objektifierats av en eller flera kvinnor, kanske ett tjejgäng som varit ute och druckit vin och skrattat tillsammans. Killet blir sett inte som den person det egentligen är, utan som ett ting, en kropp för kvinnor att konsumera. Sven-Otto Littorin, även känd som \quotetext{Dangerzone2010}, har länge närt en längtan efter att bli ett kille och har lagt ner både tid, pengar och karriär på detta. Än så länge har han inte varit framgångsrik i sin strävan.
\quotetext{Haters gonna hate} mumlar han när man frågar. 

\ditem[Etta]\label{etta}
 Ettan är ett svenskt snus som finns i varianterna lös, portion och vit portion. Receptet har varit det samma sedan pangea.

\ditem[Eulalia II]\label{eulalia ii}
 är Åsa-Nisses bil\ref{bil}. Modellen är en 1932 års Ford B och skänker Åsa-Nisse stor glädje men ibland också bekymmer. Man vet aldrig riktigt vad som kommer hända när man åker med Eulalia II.

\ditem[Eva Ekeblad]\label{eva ekeblad}
 (född 10 juli 1724 och död 15 maj 1786) var en svensk vetenskapskvinna, som idag främst ihågkoms för att ha informerat almogen hur man kokar brännvin av potatis.

\ditem[Evert Taube]\label{evert taube}
 (1890-1970) är en ärkesexist som tillhör sveriges mest älskade musiker och författare. Håkan Hellström är ett uttalat fan av denna man. Annars är den enda grupp i det svenska samhället som verkligen inte ogillar Taube straffade pedofiler. De tycker att han sätter saken i ett visst perspektiv.

\uline{Exempel ur Evert Taubes konstnärliga verksamhet}

\begin{itemize}
\item Flickan i Havanna
\item Fiorella från Carmella
\item Rosa på bal
\item Flyg till Pampas
\item I najdernas gränd
\item Mina damer och herrar
\end{itemize}

\uline{Evert Taubes arv till eftervärlden}

\begin{itemize}
\item Innevånarna i Roslagen, de mytomspunna \quotetext{rospiggarna,} har låtit anlägga en nöjespark i nationalskaldens ära, den kallas Evert Taubes värld\ref{evert taubes vaerld}
\item Myntade i senare tonåren uttrycket lurmus efter att både använt sig av Feminist-knepet och bjudit på päronhalva\ref{paeronhalva} men ändå inte fått ligga. Evert Taube skall icke förväxlas med snälla killar som aldrig får ligga\ref{snaella killar som aldrig faar ligga}
\end{itemize}

\ditem[Evert Taubes värld]\label{evert taubes vaerld}
 är en nöjespark i Roslagens famn. Här kan barn i alla åldrar dansa vals med Calle Schewen, eller bara se Rönnerdahl virvla sina lurviga ben om man hellre vill det. Efter det kan man ta en åktur med karusellen \quotetext{Briggen Blue Bird från Hull} och sedan gå in till Den Glade Bagaren i San Remo för att äta det bästa bröd som fås, och det är bäst just för att han är så glad. Skulle inte dessa aktiviteter vara nog så kan man bara ligga på Sjösala äng och titta på alla vackra blommor som slagit ut, nämligen gullviva, mandelblom, kattfot och blå viol.


%%%%%%%%%%%%%%
\newpage
\null
\\
\null
\\
\Huge
F
\normalsize
\\
\null
\\
\null
%%%%%%%%%%%%%%

\ditem[Facebook]\label{facebook}
 När folk talar om sociala medier på nätet är det med största säkerhet facebook de menar. Att förakta facebook är något väldigt fint som berättigar hataren till avsevärd social status. Att hata facebook är 2010-talets svar på den elisabetanska tidens bleka hud och komplicerade håruppsättningar och visar liksom dessa statusmarkörer att man inte delar andras behov (i det elisabetanska fallet att kroppsarbeta och i facebookfallet social interaktion). Hataren kan mycket väl ändå inneha ett facebookkonto men \quotetext{kommer aldrig ihåg att kolla det} eller \quotetext{förstår sig inte på hur det fungerar}. Hen kan inte förstå varför man skulle vilja veta eller berätta att man druckit morgonkaffe eller gått ut med hunden.och blir  topp tunnor rasande av anblicken av sina gamla mellanstadiepolare eller fd. arbetskompisar. Hen misstänker att riktiga möten människor mellan allt mer ersätts av virtuell interaktion, som aldrig kan ersätta riktiga möten och det ickeverbala kroppsspråk som, påpekar hen, utgör omkring 80\% av all kommunikation. Att förakta facebook är att jämföra med att vara less på julen och alla krav och all kommersialism som den medför.

\ditem[Fagersta]\label{fagersta}
 är Sveriges\ref{sverige} Ruhrområde. Befolkat huvudsakligen av amfetaminister, nazister och en icke förringansvärd andel finnar\ref{finland}. Fagersta har förutom festivalen Våryran även ett bandylag samt en fabrik där man gör tunn metalltråd ännu tunnare.

\ditem[Fagersta-Posten]\label{fagersta-posten}
 är lokalblaskan i Fagersta, Norberg och Skinnskatteberg. Kravet för att få anställning är enligt dess förra redaktör att kunna rimma på Ingrid och Sture. Trots sitt skrala innehåll har tidningen ändå ett stort antal prenumeranter, vilket antas bero på att många människor i norra Västmanland eldar med ved och behöver tändningspapper på morgonen. Hösten 2009 tog Fagersta-Posten steget in i 2000-talet och lanserade en hemsida på Internet.

\ditem[Fakta]\label{fakta}
 är en världsberömd musiker och Fagerstas svar på Brian Wilson. Lika stor som Wilson gjorde surfrocken för 50 år sen, gjorde Fakta surfpunken på 90-talet. Fakta heter Fakta för att han jobbade en tid på en livsmedelsbutik med samma namn. Vet ni inte vem Fakta är vet ni inget om musik.

\ditem[Faktoid]\label{faktoid}
 Sanningar som förnekas av förståsigpåare\ref{foorstaasigpaaare}, kulaker\ref{kulaker} och andra besserwissers. I folkrepubliker där man insett sagans förmåga att ingjuta kraft i en tappert kämpande befolkning är faktoidbegreppet sedan länge avskaffat och istället gläds man gemensamt åt Idi Amin Dadas världsrekord på 100 meter frisim och Den Käre Ledarens prickfria golfrunda. I Sverige representeras faktoidbegreppet främst av Prof. Etiennes\ref{prof. etienne} självbiografier och Sven-Otto \quotetext{Dangerzone2010} Littorins akademiska kometkarriär. Visst kan man ha invändningar men haters always gonna hate.

\ditem[Falafel]\label{falafel}
 Hoppressat sågspån stekt i olja. Anses av vissa vara en maträtt.

\ditem[False metal]\label{false metal}
 är metal som är falsk, till skillnad från till exempel true Norwegian black metal. Kanske har metallen spelats på radio. Kanske spelar man för att det är kul och inte för att man hatar allt mänskligt liv. Något gör i alla fall att metallen i fråga avviker från det ledord som en enligt en enig frikår stavas: \quotetext{No synths, no pedals, no wimps - Just Power, Metal and Might}.

\ditem[Fasta nycklar]\label{fasta nycklar}
 det är riktiga grejer det. Inget jävla larv med med skiftnyckel\ref{skiftnyckel} som sliter ner muttern och måste ställas in på nytt inför varje vridmoment. Nej, fasta nycklar ska det vara!

\ditem[Fax]\label{fax}
 en är ett redskap för telekommunikation. Den möjliggör kommunikation i realtid mellan stora avstånd. Ett viktigt faxmeddelande som skickas från Malå\ref{malaa} kan, om inget går fel, printas ut ur en faxmaskin i Australien\ref{australien} redan en minut senare. Många olika faxmaskiner har uppfunnits, några prototyper så tidigt som på 1880talet, men den fax som vi idag har svårt att föreställa oss ett liv utan kommersialiserades under 70- och 80-talen. Den blev snabbt en populär telekommunikationsteknik bland unga urbana människor som är \quotetext{on the go,} som vill \quotetext{vara med} och som värdesätter ett brett informationsnät. Faxen användes till en början mest av större företag men, som står att läsa på ett rivaliserande internetbaserat uppslagsverk, \quotetext{numera är fax mellan privatpersoner inte alls ovanligt.} Telefaxen har jämte den minst lika populära digitala kommunikationen bidragit till att skapat en global kultur där information utbyts på bråkdelen av ett par sekunder, vilket inte minst blir märkbart då man beställer undermåligt andrasorterings-tegel från shady firmor i Baltikum. 

\ditem[Fejkspons]\label{fejkspons}
 är ett fenomen nära besläktat med dubbel- och trippelmerching och är vanligt bland människor som är lite väl intresserade av märkeskläder. Professionella idrottsutövare, musiker och andra yrkesgrupper där individer blir till varumärken får ofta spons av olika företag. De får då betalt för att gå runt i kläder av ett visst märke så att andra ska köpa likadana kläder. De som gör det kan då råka ut för att råka fejksponsa ett företag, som inte är detsamma som att ha t.ex. en Nike t-shirt, utan att verkligen gå till överdrift. Säg att du spelar i ett trallpunkband med viss lokal status och gillar märket Haglöfs av någon anledning. Säg att du köper ryggsäck, sovsäck, byxor, fleecetröja, keps\ref{kepsar med olika fooretagslogotyper}, shorts, plånbok\ref{haesthandlarplaanbok}, axelremsväska samt jacka med Haglöfs karakteristiska logotyp, ja, då är du fejksponsad.

\ditem[Feliks Dzerzjinskij]\label{feliks dzerzjinskij}
 Även känd som Tjekans tjusigaste skägg, grundade densamma i december 1917. Levde rövare hos borgare och kreti och pleti fram till 1926 då han plötsligt dog i en hjärtattack efter att ha hållit ett två timmar långt tal för partiets centralkommitté. Som den professionella kåkfarare han var under tsartiden fick han tokstryk av plitar och andra lakejer i diverse fängelser och läger. På grund av detta var hans käke permanent ur led under den senare tiden av hans liv, vilket föranledde en redig fetor ex ore\ref{fetor ex ore} som var så tilltagen att dess jämlike sällan eller aldrig har skådats.

\ditem[Feminism]\label{feminism}
 är ett slags politisk inriktning som grundar sig på idén att även kvinnor är människor. Tanken uppfanns av sådana som Mary Wollstonecraft och Emma Goldman men möttes till en början med viss skepsis av diverse manliga förståsigpåare\ref{foorstaasigpaaare}. Sedemera har dock fler och fler anslutit sig till idén om att det vore ganska bra om även kvinnor behandlades på ett sjysst sätt.

I Kristdemokraterna\ref{kristdemokraterna} håller man inte riktigt med. Ånej, män är män och kvinnor är kvinnor! ryter man på partikansliet så att det ekar i sakristian och med en sådan inlevelse att Jungfru Maria-figuren, som förärats en plats strax brevid (snett bakom) bysten av Josef, tycks gråta av lycka. Eller så är det kanske bara det dunkla ljuset som löper in från medeltidskapellets blyinfattade fönster som spelar en ett spratt\ref{jaevelskap}.

\ditem[Feministknepet]\label{feministknepet}
 är att som kille bedyra att man minsann är feminist, med avsikt att få ligga. En A-kurs i genusvetenskap kan vara behövligt för att backa upp påståendet. Kanske en F!-knapp, alternativt en venussymbol på rockslaget.

\uline{Berömda exempel}

Socialdemokraten Göran Persson gjorde detta knep för att försöka ligga med hela svenska folket. Han lyckades med Anitra Steen i alla fall. Manliga SSUare har inte mycket annat än detta knep att förlita sig till, vilket är en förklaring till den dåliga tillväxten i partiet.

\ditem[Femma]\label{femma}
 En femma är ett vanligt viktmått när man köper hasch.

Se även: Etta\ref{etta}, Tvåa\ref{tvaaa}, Trea\ref{trea}, Fyra\ref{fyra}, Sexa\ref{sexa}, Sjua\ref{sjua}, Åtta\ref{aatta}, Nia\ref{nia}.

\ditem[Femtusen invånare-regeln]\label{femtusen invaanare-regeln}
 Alla som bor i ett samhälle med exakt femtusen invånare eller mindre måste heja på varandra, oavsett om de är tjenis eller ej.

\ditem[Fet och grisig mat döpt efter lyxiga ställen/personer]\label{fet och grisig mat doopt efter lyxiga staellenpersoner}
 är ett klart uttryck för klassförakt. Tegaren\ref{tegare} i Umeå är en korv i bröd dränkt i alla dressingar gatuköksinnehavaren kan uppbåda, döpt efter stadens finare kvarter. Parisaren\ref{parisare} är i Skellefteå en maträtt som består av en svettig korvslant. Den har döpts efter den mest fashionabla staden i världen. Wallenbergaren är en svensk klassiker: en pannbiff gjord i stort sett uteslutande på grädde, döpt efter en förmögen häradshövding. Plassarn är en i Arjeplog förekommande maträtt bestående av underdelen av ett hamburgerbröd, en stor parisare\ref{parisare}, två mosklickar samt räksallad. 

\ditem[Fetma]\label{fetma}
 kan orsakas av åtminstone två av de sju dödssynderna. Fetma blir allt vanligare i världen. I somliga kulturer hålls fetman för något åtråvärt. Nordkorea valde exempelvis nyligen en fet ung man till högste kamrat på just denna merit. I takt med att fetman blir allt vanligare blir bristen på en tydlig terminologi tydlig. Att enbart säga: \quotetext{Du vet han den tjocka} är inte längre tillfylles. Ett anspråkslöst förslag således:

\uline{Barnfetma}

Barnfetma är vanligast hos barn\ref{barn}. Den svåraste skiljelinjen är här mot mobbarfetman men vanligtvis brukar ett par slappa handleder avslöja innehavaren.

\uline{Blekfetma}

Blekfetma grasserar på de brittiska öarna, Sydafrika, Förenta staterna och Norge. Helt okänt öster om Åland. Utan fordon förväxlas ofta en blekfet man med en mopedfet dito.

\uline{Mopedfetma}

Mopedfet äro den ståtlige man som kombinerar sin fetma med ett offentligt liv på moped.  Mopedens storlek och modell är av stor betydelse för mopedfetmans gränser. Onslow i \quotetext{Keeping up appearances} skulle exempelvis vara mopedfet på en elegant Crescent kompakt, i fetmans limbo på en svart Zündapp och endast blekfet på en trimmad DT.
 
\uline{Oi!Fetma}

Oi!fetton hittas normalt i polarens inrökta etta\ref{etta} mellan alkisschäfern\ref{alkisschaefer} och traven med pizzakartonger. Avgörande är kammobyxorna och huvtröja med pitbull/nazi/antinazi/ACAB/fotbollstryck.

\uline{Veganfetma}

Veganfetma återfinns nästa alltid hos veganer. Denna typ av fetma är av mycket distinkt karaktär – en tanig kroppshydda kombinerad med en ordentlig bukfetma bestående av 95\% transfett. Veganer lever i villfarelsen att allt \quotetext{veganskt} är nyttigt och äter därför kopiösa mängde Tofuline, chokladbollar och andra onyttigheter.

\ditem[Fetor ex ore]\label{fetor ex ore}
 \textit{Fetor ex ore} är den latinska termen för dålig andedräkt. Detta symptom orsakas av sådant som lungsjukdomar, gasbildning vid nedbrytning av matrester, en övre magmun\ref{oovre magmunnen} som står på vid gavel, skadad tunga eller trasiga tänder, samt rubbad matsmältning föranlett av leverskador. I Sverige\ref{sverige} är fetor ex ore främst associerat med Lars Leijonborg som uppvisar samtliga av ovanstående åkommor, och mer därtill.

\ditem[FFSSB]\label{ffssb}
 Föreningen för storspovens bevarande, fordom tida arrangör för fredagsyran i Åkerby. 
På senare år har föreningens aktivitet avtagit på ett beklagligt vis. Dess logotyp visar inte helt olikt de hos många statliga verk i USA en ståtlig Storspov\ref{storspov} i profil. 

\ditem[Fiacre]\label{fiacre}
 är taxichaufförernas skyddshelgon och föddes på Irland på 600-talet. Detta är faktiskt, i motsats till mycket annat i detta infiormationstäta uppslagsverk, helt sant.

\ditem[Filateli]\label{filateli}
 är den renaste formen av kärlek.

\ditem[Filipinsk apörn]\label{filipinsk apoorn}
 Den filipinska apörnen är en blandning av örn och apa och ställer till förtret i Filipinerna genom att segla in i hus och bodar och skrika för full hals. Den filipinska apörnen är enligt auktoriteterna på ämnet utrotningshotad. Och det kan man ju förstå.

\ditem[Finland]\label{finland}
 I Finland finns mycket som är typiskt för just detta landskap, så som det finska språket, finländare, läckerheter så som olika sorters rotsakslådor och korvsås, och ett samhälle som i mycket påminner om hur livet såg ut i resten av norden under 50-talet, det vill säga helt OK för de flesta, men ganska segt. Att det skulle finnas människor med mörk hy avfärdar man i landet bestämt eftersom man inte sett någon sådan. För män anses det svårt socialt stigmatiserande att ha någon slags frisyr (dvs inte rakat huvud eller stubb). Har man det blir man förr eller senare anklagad för att inte vara heterosexuell och utdragen på gatan av arga folkmassor. Men även om landet fortfarande har framför sig att kliva in i 2000-talet finns där mycket att se och uppskatta. Finländaren musicerar lika ofta som gärna och ägnar sig även åt idrott på vinterunderlag. Hen njuter av ledigheten genom att kasta stövel, spela trummor i ett av landets många kängpunk-orkestrar eller genom att stapla ved. 

\ditem[Finljuga]\label{finljuga}
 är att samtala under lugna former med en viss grad av fria spekulationer. Samtalet får gärna röra sig obehindrat mellan ämnen också. Med fördel finljuger man sittandes på en bänk i solen. Det kan låta ungefär så här:

-\quotetext{Troru he bi nå lingon i år?}
-\quotetext{Njaaaa.}
-\quotetext{Nä, prästn' troddent dä häller.}
-\quotetext{Jaså?}
-\quotetext{Jo han hade så fullt opp me annat.}
-\quotetext{Di säger du?}
-\quotetext{Gården å mor sin å påsken\ref{paask}.}
-\quotetext{Ja du hör.}
-\quotetext{Va?}
-\quotetext{Mmmm.}

\ditem[Finsk inställning till rock]\label{finsk instaellning till rock}
 är ett psykologiskt fenomen och en politisk åskådning som är vanlig hos österlänningar. Den finska inställningen till rock är svårdefinierad, men tar sig uttryck då män med mustasch i ursinne sliter ner musiker från scenen i festrummet på färjor som trafikerar färjeleden Åbo-Stockholm och skriker åt dem att spela Hurriganes eller nån annan hetrosexuell finsk rock. Rock för den finske mannen är nämligen inte detsamma som vi här i Sverige\ref{sverige} tänker oss det. Rock för den finske mannen är ett slags homosocial sfär vid sidan av vardagens alla krav och bekymmer, i vilken man tillåts bära bandana  och ko-chaps, stå avsides och ursinnigt vråla med i en låt om att ha det trevligt i grupp samt supa tills man svimmar i sina kroppsvätskor. Denna inställning har visats driva många manliga finska musikorkestrar till att enbart spela covers av klassiska heavy metal-låtar men med finsk text. Andra hänger sig åt att skapa musik som handlar om bortskämda kvinnor som vägrar gifta sig med finska pappersbruksarbetare\ref{finsk pappersbruksarbetarkraut}, och som istället envisas om att drömma sig bort över havet skummande vågor, drömma sig hela vägen till Örnsköldsvik där de vill skapa sig ett nytt liv bland stadens \textit{parnasse}. Den finska inställningen till rock har också vissa konsekvenser för manliga svenska rockers som besöker detta pungformade område, Finland\ref{finland}. Bland dessa konsekvenser utmärker sig att bli utskrattad, knäad i skrevet och inlåst i en städskrubb på S/S Singoalla. Där dör man, gråtandes i mörkret, hopkrupen och avmaskuliniserad.

\ditem[Finsk pappersbruksarbetarkraut]\label{finsk pappersbruksarbetarkraut}
 är en underavdelning till Krautrocken och framförs i första hand av arbetare vid finska\ref{finland} pappersbruk. Genrens två största band är Circle och Pharaoh Overlörd, som har samma medlemmar. De har också spandexbraller, bandana och nitarmband, vilket är för krautrocken ovanliga accessoarer. 

\ditem[Finsk sommarsoppa]\label{finsk sommarsoppa}
 3 dl Kosken i en blommig djuptallrik.

\ditem[Finskt pannband]\label{finskt pannband}
Det är tidigt 1980-tal, en tid då Björn Borg klädde sin frontallob i ett underverk av högelastiskt frottétyg, som kunde hålla hans örontoppar varma i arktisk kyla, utan att för funktionens skull svika sin tids högsta estetiska ideal – det fluffiga och samtidigt färggranna.

Ja, 1970- och 1980-talen var för det svenska pannbandsmodet vad 1490-talet var för aztekernas blodindränkta imperium, en guldålder. Under samma era begagnade sig även svenskens rotfruktsivrande grannar i öst av det uppburna Pannbandet. Men där svensken svepte in sina vingmuttrar i det mäktigaste som tygindustrin i Ruhrområdet kunde uppbåda, bar finnarna (ständigt i kontrast till svenskarna) skosnören eller tunna läderremmar hårt knutna runt huvudet. Det var särskilt populärt bland finska råpunkare, som år 1984 utgjorde 43 \% av Finlands totala population. Stilikoner som Jakke i Kaaos spred läderremmens lov både i de tusen sjöarnas land, så väl som till andra delar av världen, genom gryniga bandfoton i ölindränkta skivkonvolut. Stil över funktion var lagen som rådde. Och visst får man medge att det finska huvudbonadsmodet var mer respektingivande än svenskt dito, som tydligast representerades av Jörgen i Mob 47, med sin minnesvärda toppluva med inskriptionen \quotetext{Lake Placid 1980}, tillsammans med OS-ringarna och det årets två OS-maskotar, tvättbjörnarna Ronny och Roni.

\ditem[Fiska kräfta med ficklampa]\label{fiska kraefta med ficklampa}
 Det är olagligt att fiska kräfta med ficklampa, men vad är å andra sidan inte olagligt nu för tiden? Man får varken besikta utan ljuddämpare eller ta bilen hem från Kickis vägkrog. Att fiska kräfta med ficklampa går ut på att man tar roddbåten ut på sjön där det är grunt och lyser med ficklampan på kräftorna på botten. De blir då paralyserade av ljuset och ba FTW?! och då plockar man upp dem.

\ditem[Fiskeredskapsaffär]\label{fiskeredskapsaffaer}
 är affärer som tillhandahåller redskap användbara vid fiske och efterföljande tillredningsprocesser, så som rensning och rökning. Fiskeredskapsaffären drivs i nittionio fall av hundra av en manlig föreståndare vid namn Kjell-Åke. Fiskeredskapsaffärs-näringen bidrar med cirka fyra\ref{fyra} och en halv procent av Sveriges\ref{sverige} totala bruttonationalprodukt, vilket ungefärligt motsvarar den skattefinansierade infrastrukturella uppbyggnaden i den norra halvan av landet (inräknat anläggandet av tågbanor, vägar, broar etc). En av de många anledningar som föreligger detta näringsområdes höga omsättning är att konsumenter ges incitament\ref{incitament} till relativt stora investeringar i spön, agn och tillbehör genom att en Abu Garcia-keps\ref{abu garcia} utan avgift byter hand om konsumenten handlar för mer än en viss, ganska hög, summa. Då Abu Garcia-kepsar är ett av de mest eftertraktade bytena inom det moderna sportfisket ökar denna försäljningsstrategi inte sällan omsättningen med över fyrtiotusen miljarder\ref{fyrtiotusen miljarder} procent.

Se även kepsar med olika företagslogotyper\ref{kepsar med olika fooretagslogotyper} och Abu Garcia\ref{abu garcia}

\ditem[Fisljud]\label{fisljud}
 Att göra fisljud med händerna var en mycket vanlig sysselsättning bland pubertala glopar\ref{glop} på landsbygden före internet\ref{internet} och spelkonsoler ledde ungdomen bort från brännvin\ref{braennvin} och a-traktor\ref{a-traktor}.

\ditem[Fixed gear metal]\label{fixed gear metal}
 , eller hipstermetall (numer även Stockholmsprofilsmetal), bör undvikas i den mån det är möjligt. Fixed gear metal spelas av sådana band som hipsters, som ofta cyklar på fixed gear bikes, lyssnar på. Med anledning av detta är det en förvirrande och splittrad upplevelse att själv lyssna på dessa band. Hipstern lyssnar inte på någon annan metal, utan väljer ett eller två band som hen sedan påstår sig ha lyssnat på sedan Moses låg i vassen. I resten av skivhyllan/hårddisken förvarar han/hon dubstep, postrock och hip-hop\ref{hip-hop} etc. Exempel på fixed gear metal är The Sword och SunnO))).

\ditem[Flaka]\label{flaka}
 Att flaka är att snatta nånting på en affär med hjälp av en kundvagn. Man placerar det man vill snatta, låt oss säga en back lättöl, under kundvagnen. Sen handlar man som vanligt förutom att man \quotetext{glömmer} att påpeka artikeln under kundvagnen vid kassan. Gamla människor gör det här utan att veta om det.

\ditem[Flanera]\label{flanera}
 Att flanera innebär att man utan någon direkt agenda går runt. Ofta tillsammans med någon man har ett amoröst/erotiskt intresse av. Flanören omgärdas av ett romantiskt skimmer då begreppet förknippas med ungdomar. Om man ser ett par äldre personer gå runt utan mål och syfte är det mer troligt att de gått vilse än att de är ute och hård-flanerar som kidsen gör.

Än så länge har inget etymologiskt samband mellan flanera och materialet flanell kunnat påvisas, men evighetsakademikern Prof. Etienne\ref{prof. etienne} kungjorde i en intervju i Femina (Nr. 3 2009) att han inte skulle sova en blund innan han publicerat ett paper som grundligt redovisade kopplingen orden emellan.

\ditem[Flatologi]\label{flatologi}
 (lat. \textit{flatus} ung. \quotetext{vind}, \quotetext{vindpust}) är den gren av den moderna medicinen som studerar flatulens, det vill säga pruttar. Många tycker att flatologi och flatologer mest är något att skratta åt, men man sätter lätt skrattet i halsen då man ser de ledsna ögonen hos en ung vegan som inte kan ta det minsta steg utan att det slinker ut en liten rackare. Förutom att dag och natt arbeta för lindring hos denna krisgrupp är flatologins vita val det välbekanta problem som uppstår då två människor som står inför möjligheten att inleda en relation tittar på film tillsammans. Liksom mången gång inom astrofysiken har man vid ett flertal tillfällen trott sig ha löst problemet, bara för att när den första uppståndelsen lagt sig inse att man missat någon viktig detalj. Redan Linné\ref{carl von linné} påstod sig till exempel i sin \textit{Flatologia Laponika} (1742) ha kommit på en lösning, men denna avfärdades ganska snart av mustiga tyska vetenskapsmän med skäpparkrans. Under kalla kriget talades det om en kapplöpning mellan vetenskapsmän i öst respektive väst om vem som först skulle lösa problemet och det har spekulerats om att man i Sovjet faktiskt kom på en lösning som efter kalla krigets slut ledde till Gazproms rekordsnabba intåg på marknaden. Som vanligt döljs sanningen av ogenomträngliga rökridåer. Tills vidare är man i både öst och väst hänvisade till det gamla beprövade knepet att börja röka så man kan gå ut på balkongen och prutta i smyg. Bäst läge har som så ofta dansken, för i vårt tomtenisseformade grannland\ref{danmark} i sydväst anses det artigt och romanitiskt att när man kommer till sitt gemål bara slänga sig i soffan och prutta lite slappt medan man kollar med ett halvt öga på nån repris av en gammal Eddie Murphy-rulle.

\ditem[Flerväxlad cykel]\label{flervaexlad cykel}
 Ett väldigt åtråvärt fordon som under nittiotalet gav upphov till en kapprustning där cyklister tävlade i vem som hade flest växlar. Den gamla innernavsväxlade varianten tillät upp till femton, det senare systemet med utvändiga kedjeväxlar såg betydligt tuffare ut och tillät betydligt högre antal växlar. På en del flerväxlade cyklar måste man sluta trampa när man ska växla, medans en del är precis tvärtemot.

\uline{Erfarenheter av turer på en blå tjugoettväxlad mountainbike}

\begin{itemize}
\item Cyklar man på ettans växel är det lätt att cykla på bakhjulet, men man ser ut som ett slan\ref{slan} för man måste trampa så otroligt snabbt.
\item På tjugoettans växel så kan man komma upp i skrämmande höga hastigheter, men hamnar man i en uppförsbacke får man växla ner för annars finns en överhängade risk att man blir ansträngd.
\item Nånstans kring elfte växeln är bäst om man bara vill glida runt.
\end{itemize}

Micke på Brånet hade flest växlar. Han sa att han hade en femtioväxlad cykel, men alla visste att det inte låg ett uns sanning i detta. För det var egentligen en femväxlad cykel som han ritat dit nollor efter varje siffra på.

\ditem[Flisbil]\label{flisbil}
 är en lastbil som transporterar flis. Är man riktig snål och vill ta sig mellan Malå\ref{malaa} och Piteå\ref{piteaa} kan man alltid åka med en sån.

\ditem[Flodkanin]\label{flodkanin}
 (\textit{Bunolagus monticularis}) är en kaninart som bara finns i Sydafrika. Den är väldigt utrotningshotad, typ bara 200 individer kvar, vilket antagligen beror på att det är en dålig kombination att vara kanin och bo i en flod i ett land där det finns vithajar.

\ditem[Flundra]\label{flundra}
 är en fisk som till skillnad från alla andra fiskar bor i halsen på Ernst Billgren\ref{ernst billgren}.

\ditem[Flytväst]\label{flytvaest}
 Fungerar ungefär som ett bilbälte, fast på vatten.

\ditem[Fläsksvålar]\label{flaesksvaalar}
 eller\textit{ flæsgæsvælær }som det heter på danska\ref{danmark} är höjden av dansk kokkonst. Fläsksvålens historia börjar hos bonden Preben Madsen som hade fått höra talas om den nya flugan om pommes frites och bestämde sig för att själv tillverka denna delikatess. Prebens snedsteg/genidrag att istället för att bruka alla skandinaviers favoritknöl potatis så använde han den överblivna och orakade svålen från gårdagens rimmade sidfläsk och stoppa detta i fritösen. Stay classy, Preben. Sedan dess har danskar från Jylland till Fyn\ref{fyn} blivit alldeles till sig när en påse fläsksvålar öppnas och sprider sin karakteristiska doft av hundmat i rummet.

Fläsksvålar är det livsmedel som stoltserar med överlägset högst andel transfetter om man inte räknar ren ister och kokosfett. Personer med svagt hjärta bör hållas långt borta från denna charkuteri, och detta kan vi på Nissepedia\ref{nissepedia} inte understryka nog.

\ditem[Flöjt]\label{floojt}
 är som en liten visselpipa, men med en massa hål i. Den ligger ofta i ett flöjtfodral\ref{floojtfodral}.

Tuffa hårdrocksband som Hawkwind kan balansera upp sin massiva hårdhet genom ett sju minuter långt flöjtparti. Detta är förbehållet riktigt hårda band,annars blir det lätt musikhögskolemusik\ref{musikhoogskolemusik}.

\ditem[Flöjtfodral]\label{floojtfodral}
 är en det hölje eller den skida vari flöjtisten för ner sin flöjt\ref{floojt} då han eller hon vilar och inte spelar på sitt lilla instrument. Flöjtfodralet uppfanns av den svenske författar'n, assäist'n, krönikör'n och debattör'n Oskar Sverre Lucien Henri Jan Guillou.

\ditem[Fnysning]\label{fnysning}
 Folkpartistisk\ref{folkpartiet} nysning. Sällsynt tråkig, och rensar inte näsan alls.

\ditem[Folk födda före 1970]\label{folk foodda foore 1970}
 Denna kategori människor har alla gemensamt att de passerat 40-strecket. En gemensam egenskap som särskiljer dem från yngre människor är att de talar in röstmeddelanden när någon de ringer inte svarar på mobilen. Vad detta beror på kan den moderna vetenskapen inte svara på, men den har gett oss så mycket annat så ingen ska börja gnälla bara därför! Det finns såklart en nedre gräns för denna praktik och det är folk som aldrig brytt sig om att lära sig hur en mobiltelefon fungerar för att de ändå har en fot i graven.

\ditem[Folke Pudas]\label{folke pudas}
 var en svensk taxichaffis mellan 1930 och 2008. Hans tre månader långa hungerstrejk i en trälåda på Sergels torg bör inspirera varje människa som önskar driva ut den svenska rättsrötan. Under 00-talet försökte SIBA:s VD Fabian Bengtsson återlansera konceptet med lådan, men då han saknade Pudas skicklighet att hantera media kom han mest att framstå som en rik gnällspik och vann inga sympatier från arbetarklassen.

\ditem[Folketinget]\label{folketinget}
 är Danmarks\ref{danmark} parlament. Där bestäms vilka lagar som ska gälla i Danmark, om man ska försöka erövra Skåne en gång till, vilka artister som ska bokas till Roskildefestivalen och andra, för danskar, viktiga frågor.

Vilka som ska sitta i folketinget bestäms genom allmänna val där alla danskar skriver sin kandidat på en papperslapp som läggs i en Tuborg-flaska\ref{tuborg} och slängs i närmsta glasigloo. Jens Spendrup vinner varje gång men han är upptagen med att tappa upp öl så han avsäger sig alltid uppdraget.

Namnet folketinget kommer sig av att vaktmästaren som släcker och stänger där varje kväll heter Folke.

\ditem[Folkhjälte]\label{folkhjaelte}
 En folkhjälte är någon som gjort något storslaget för ett folk, gärna mot det folkets förtryckare och utsugare.

Exempel:
\begin{itemize}
\item Engelbrekt Engelbrektsson
\item Pia Sundhage
\item Folke Pudas\ref{pudaslaada}
\item Sigvard Thurneman\ref{sigvard thurneman}
\item Pelle Svensson\ref{pelle svensson}
\item Anckarström
\item Eva Ekeblad\ref{eva ekeblad}
\item Frantz Fanon
\item Jurij Gagarin\ref{jurij gagarin}
\item Pelle Fosshaug\ref{pelle fosshaug}
\item Ivar Bryntse
\end{itemize}

\ditem[Folkkök]\label{folkkook}
 Namnet till trots är det inte ett kök för folk utan ett ställe där veganer träffas för att få magsjuka och dra politiskt korrekta vitsar.

\ditem[Folkpartiet]\label{folkpartiet}
 är ett parti för förbittrade marxister som tappat tron på socialismen. Många av dess medlemmar är egentligen godhjärtade men orkar inte längre engagera sig utan vill ha snabba lösningar. Vissa medlemmar kommer dock från annat håll. Jan Björklund\ref{jan bjoorklund} är exempelvis fortfarande uttalad nazist.

\ditem[Folkölsfylla]\label{folkoolsfylla}
 Ett sexpack trefemmor eller två sexpack tvååttor är tillräckligt. 

Öppna en vid klockan två (folkölsfylla är ofta synonymt med dagsfylla)\ref{dagsfylla}, du kanske diskar och lyssnar på P1. Blir lite arg över nån korkad grej men tar en klunk och tänker, jaja c'est la vie antar jag. Den första tar rätt lång tid och ljumnar lite, men det gör inte så mycket.

Öppna nästa ca 40 minuter efter den första. Nu är du lite törstigare. Känner dig rolig nog för att skriva en artikel i Nissepedia\ref{nissepedia} och börjar fundera på middag. Det blir nog fiskpinnar och spaghetti igen. Hmmm, kanske med remouladsås. Ja, det är gott, tänker du och halsar den sista fjärdedelen.

Den tredje ölen behöver du snacks till. Vinägerchips småäts samtidigt som du spelar quizkampen i din mobiltelefon, eller dota 2 på din dator. Det går i alla fall skit i båda spelen. Du blir lite sur, men tänker, \quotetext{Life just be that way, I guess.} och sveper vad du har kvar i den tredje burken.

Nu är det dags att gå på affärn! Du tar med den fjärde ölen och smådricker på vägen. Beroende på hur glad/ledsen du är, känner du dig som en stigmatiserad bärshagga\ref{kalaskula}, men nu kan du inte sluta dricka, så det är skit samma. När du kommer fram till din kvartersbutik ställer du din öl på elskåpet utanför och glider självsäkert in. Väl inne förser du dig med fiskpinnar och remouladsås och två folköl till. Dumt att chansa, liksom.

När du kommer ut har du redan glömt ölen du lämnade på elskåpet. Men vad gör det? Du har ju fler. Nu går det undan. Du klunkar girigt i dig nummer fem innan du kommit hem. 

Det råddas en del vid spisen. Troligtvis lyssnar du på Armored Saint eller Tank på helgvolym\ref{helgvolym}, trots att det bara är tisdag! Spaghettin kokar fast i kastrullens väggar och fiskpinnarna är rätt svarta när du är klar, men när du står och hojtar med en spaghettistump hängandes ur mungipan bryr du dig inte så jävla hårt.

Nu är du folkölsfull på riktigt. Klockan är typ sex och dina polare kommer snart över. Antingen gnäller du till dig att nån av dem handlar ett sexpack till dig, eller så går ni på lokal och blir fulla på riktigt. Vad som sker från den här punkten spelar i alla fall inte så stor roll - du är framme.

\ditem[Folkölsförädling]\label{folkoolsfooraedling}
 Om man slänger in en folköl i frysen en timme och sen tar ut och dricker jättefort får man samma effekt som av en starköl.

\ditem[Forskningsinstitut i Schweiz]\label{forskningsinstitut i schweiz}
 verkar bedriva ohemult\ref{ohemul} mycket forskning om karies, tandkräm och andra områden relaterade till dentalhygien. Ingen har dock någonsin lyckats lokalisera ett sådant institut.

\ditem[Fotboll]\label{fotboll}
 Tjugotvå män/kvinnor som alla vill sparka på en boll. På gräs eller grus.

Fotboll är världens största sport sett till antalet utövare, som är så stort att det är något alldeles makalöst, och en uppbygglig fritidssysselsättning. Månghövdad är den skara som tycker om att se på fotboll på TV liksom i verkliga livet och bland de som föredrar det sistnämnda alternativet finns de som gärna passar på att ge motståndralagets ivriga påhejare en dagsedel\ref{dagsedel}, medelst slagträ eller, helt sonika, en knuten näve.

Fotboll går till som så att två lag om tio eller, för småglin, sex spelare plus målis med fötter eller andra kroppsdelar föser bollen mellan sig och sedan, helst, in i motståndarnas mål. Man får inte vidröra bollen med händerna såvida inte man är målis, domare eller bollkalle, för då kallas det handboll och det är mycket mindre populärt. Varje gång bollen kommer in i motståndarlagets mål får man en poäng och man tar av sig tröjan och kramas. Sedan blåser domaren i sin lilla flöjt\ref{floojt} och då är det dags att köra igång igen. Så fortskrider matchen i två gånger fyrtiofem minuter och sen sätter sig domaren ner och räknar ihop alla poäng som gjorts och utser ett vinnarlag. Om inget vinnarlag kan utses blir det oavgjort eller så måste man fortsätta att spela eller lägga straffar.

Exempel på framgångsrika fotbollspelare är George \quotetext{Georgie} Best (bla Manchester United FC), Glenn Hysén (bla Liverpool FC och IFK Göteborg) och, sägs det, Julio Iglesias.

\ditem[Fotbollspundare]\label{fotbollspundare}
 är ständigt på jakt efter en fix. Fixen i detta fallet är en plan. Är fotbollspundaren riktigt sugen kan planen se ut lite hur som helst: lerig, asfalterad, ojämn och/eller i en sluttning. Är läget mindre desperat är siktet inställt på en fotbollsplan som faktiskt har någon form av mål.

Fotbollspundarna använder olika tekniker för att få tag på nästa plan. Det rör sig främst om spanare, kontakter eller det kommunala bokningssystemet. Bokningsystemet används dock inte som hederliga medborgare använder det utan för att finna tomma planer. På samma sätt som tjuvar ringer innan de bryter sig in för att se om någon är hemma.

Fotbollspundarna är inte direkt en utsatt grupp i samhället utan betår i regel av män och något färre kvinnor ur medelklassen och den universitetsutbildade arbetarklassen. De anser sig dock lite för fina och speciella för att vara med i en vanlig förening som faktiskt betalar och bokar tider hos kommunen.

Missbruket kommer alltid förr eller senare att leda till kriminalitet. \quotetext{Brytfotboll} (alt. \quotetext{brytträning}) är den vanligaste formen av brottslighet. Ett gäng fotbollspundare bryter sig då olovligt in och tar över en plan för att spela. De tar sig in genom att klättra över grinden eller klippa upp låset. I Umeå är Minervaskolans konstgränsplan särskilt utsatt.
Snabbaste vägen ut ur beroendet är en ordentlig korsbandsskada.

\ditem[Fotografering]\label{fotografering}
 Har du en slumrande kreativ potential som endast får komma till utlopp då det ska sättas upp påskris och julstrumpor? Då kan fotografi vara något för dig. Med dagens teknik behövs inga dyra framkallningsvätskor och mörkrum och du kan ta med dig din kamera ut i den friska luften. Det första du behöver tänka på är vad ditt motiv ska vara, och som vanligt kommer Nissepedia\ref{nissepedia} till undsättning:

\uline{Förslag på motiv}

\begin{itemize}
\item \textbf{Nedgångna industrilandskap}. Många gånger kan ett förfallet industrilandskap vara minst lika vackert som naturen. Patinan hos ett nedlagt bruk\ref{nedlagda industribyggnader} i Bergslagen kan vara ett nog så suggestivt motiv men det krävs att hitta precis rätt bild. Arbeta med kontraster mellan skugga och ljus, och mellan olika ytor, till exempel en betongvägg och rostiga rör.
\item \textbf{Dig själv}. Ta bilder på dig själv i svartvitt. Du kanske sitter på golvet med huvudet lite på sned och armarna kring benen: Du fäster din lite sorgsna blick på en punkt strax bakom och ovanför kameran. Du kanske är lite i oskärpa. Skapa en blogg med svart bakgrund, The Smiths-citat och enigmatiska uttalanden om dig själv och publicera dina bilder där. Stäng av kommentarsfunktionen för att förhindra att kommentarerna ökar lavinartat.
\item \textbf{Graffiti}. Ta gärna färgfoton på ytor täckta av graffiti. Genom att göra det blir du automatiskt en gatukonstnär med känsla för storstadens puls. Du är så ett med staden att den uttrycker dina urbant odefinierade tankar och perspektiv på samtiden. Du ställer ut på caféer där det säljs komplicerade, kalla kaffedrinkar och får ligga med attraktiva tjejer som är betydligt yngre än du själv om du är en kille eller flata, eller äldre män om du är tjej eller bög.
\item \textbf{Dina tokiga punkarkompisar}. Dina kompisar är så spektakulära att de utgör ett outsinligt material. De har saker på huvudet som egentligen är till för något annat. De sitter uppflugna på offentliga skulpturer. Någon har somnat och råkat ut för ett spratt som alla dock kan skratta åt tillsammans i slutändan. Framkalla bilderna och klistra ihop dem till ett collage som liksom får stå för er oupplösliga vänskap utanför samhällets normer och krav.
\item \textbf{Mat}. Vare sig du lagat den själv eller betalat för den finns alltid ett stort intresse hos folk att konsumera stillbilder på den mat du sätter i dig. Tänk på att ha väldigt kort skärpedjup så att det mesta av maten befinner sig i oskärpa medan fokus ligger endast på en liten späd dillkvist, en glatt grön wasabisträng eller en grå-skär falukorvsnärt.
\item \textbf{Blad som flyter på vatten}. Bladen representerar oss människor, vattnet livet.
\end{itemize}

\ditem[Foucaultfingret]\label{foucaultfingret}
 uppfanns av den franske filosofen, historikern, sociologen tillika flintbrillot Michel Foucault. Fingret används när man är på väg att lura in en meningsmotståndare i ett intellektuellt bakhåll.

När du märker att den du debatterar mot är på väg att göra ett utlåtande som du krossat tusen gånger om i din hemmasnickrade akademiska dojo\ref{dojo}, är det lätt att för tidigt avlossa en salva shurikenvassa motargument. Det bästa i det läget är att istället tålmodigt vänta på att ens motståndare helt fullföljt sitt resonemang innan du skiljer dennes huvud från sin kropp, med hjälp av ditt katanaskarpa intellekt. Annars finns risken att din motståndare undkommer sargad från striden, med möjligheten att återvända senare med sina roninpolare \textit{en masse}.

Men hur ska du hindra dig själv från att öppna flabben och avlossa alla pilar i din verbala armborstpistol för tidigt? Foucault hade lösningen. När du hör din motståndare börja fälla provocerande korkade utlåtanden, för ditt pekfinger in i din mun och alternera mellan att gnaga och suga på det tills tiden är mogen för att fullkomligt pulverisera din motståndare, utan att ge hen en möjlighet att undfly med själslivet i behåll. Banzai!

\ditem[Framtiden]\label{framtiden}
 bär saker i sitt sköte, likt en känguru.

\ditem[Fransk gubbstoner]\label{fransk gubbstoner}
 Det krävs avancerade instrument för att skilja fransk stoner från gubbrock\ref{gubbrock}. De två genrerna är svåra för en lekman att skilja av ett flertal anledningar.

\begin{itemize}
\item Som en i publiken överväger man spontant att handla något man inte vill ha från merchbordet, som ett sätt att säga \quotetext{Tro på er själva! Låt ingen säga till er att ni inte kan}.
\item Franska stonerband har ofta med sig egen rökmaskin.
\item De har också allt som oftast tryckt upp turnétshirts med tourdates på ryggen.
\item Up close är det inte ovanligt att sångaren är till förvillelse lik Dominique Pinon.
\item Emperi visar att franska stonerband inte reflexmässigt skyggar tillbaka för användandet av sliderör.
\end{itemize}

\ditem[Franska svordomar]\label{franska svordomar}
 Frankrike har världens lamaste svordomar. Den vanligaste är \quotetext{sacré bleu} som betyder \quotetext{heliga blå}, vad fan är det liksom? Önskar man att det ska börja regna på sin fiende, eller att hen ska ha slut på varmvatten när det är dags att bada, eller vad är grejen? En annan fransk klassiker är \quotetext{baguette moisis}, som betyder ungefär \quotetext{din mögliga baguette}. Gud va jobbigt att bli kallad det... inte...

Vill man lära sig att svära på ett annat språk vänder man sig istället med fördel till de italienska svordomarna\ref{italienska svordomar}.

\ditem[Frasses]\label{frasses}
 är en norrländsk hamburgerkedja utan några större ambitioner. Inte heller personalen på Frasses har några större ambitioner. De är ofta finniga och flottiga också. De är å andra sidan inte hjärntvättade, åtminstone inte vad gäller hamburgare, vilket hamburgerkedjepersonalspersoner normalt är, till exempel på den andra norrländska hamburgerkedjan, den med ambitioner. Frasses känns okapitalistiskt på något sätt, eller åtminstone bara lite. Det är rätt skönt.

Urfrasses låg på Storgatan i Luleå, strax nedanför kyrkan och inte allt för långt från Hermelinsskolan (gymnasium). Där serverades på sjuttiotalet skrovmålen i plåtlådor. Extra dressing fanns bara en sort, och absolut inte i små knapsu plastburkar. Frasses har än idag den godaste orginaldressingen, vilket inte alla håller med om.

\ditem[Fredag]\label{fredag}
 är den dag i veckan när man går hem från arbetet/studierna och skriker \quotetext{BLACK SABBATH!} och helgen börjar.

\uline{Fredagen inom judendomen}

Ett lustigt sammanträffande\ref{maerkliga sammantraeffanden} är att sabbaten firas just under fredagen inom judendomen.

\uline{Fredag i nyliberalismens folkhem}

I nyliberalismens folkhem utmärks fredagen som den dag då man ägnar sig åt sk. \quotetext{fredagsmys\ref{fredagsmys},} vilket innebär att barnfamiljen vräker i sig chips och coca-cola och att nyförälskade par äter avokadohalvor med räkor på.

\ditem[Fredagslyx]\label{fredagslyx}
 En kasse starköl och en falukorvsring.

\ditem[Fredagsmys]\label{fredagsmys}
 är ett påhitt av detaljhandeln i samarbete med nyliberala institut och tankesmedjor\ref{institut och tankesmedjor} och går ut på att barnfamiljer sitter i en soffgrupp, kollar på störd TV och smäller i sig chips och läskeblask som om det icke fanns någon morgondag. Vissa påpekar dock att detta bara är en kommersiell variant av firandet av slutet av arbetsveckan, vilket har pågått sedan kristendomen institutionaliserades under antiken. Inkluderas även dessa sätt att ägna sig åt fredagsmys finns det många olika sätt på vilka man kan koppla av under fredagskvällen. Några av dessa är att:

\begin{itemize}
\item Rulla hatt\ref{rulla hatt}
\item Ta av sig byxorna\ref{sans pants}
\item Lyssna på rymdrock och röka gräs\ref{stenad}
\item Slåss utanför dansbanan
\item Ägna sig åt en hobby
\item Fira den judiska sabbaten.
\item Spela spel
\item Laga mat med polarna
\item Leka charader
\item Kolla på fotboll\ref{fotboll}
\item Spela på travet
\item Måla små, små tennfigurer och sedan leka att de krigar mot varandra (Warhammer)
\item Dricka läsk och lyssna på Minor Threat
\item Åka skateboard och ställa till med jävelskap\ref{jaevelskap}
\item I sällskap med sina ungar äta sig spymätt på onyttigheter och kolla på Jokkis
\item Lyssna på Sabbath och byta kammrem på en skrutt-Ford
\end{itemize}

\ditem[Fredrik Reinfeldt]\label{fredrik reinfeldt}
 är en moderat\ref{moderat} som på tok för länge var Sveriges statsminister. Han tillåter inte ens sju procent av den del av Sveriges\ref{sverige} befolkning som gör rätt för sig genom att arbeta och samtidigt dricka bira, vilket till och med fascisterna i Danmark\ref{danmark} gör. Reinfelt är också ansiktet utåt för de onda i J.R.R Tolkiens\ref{j.r.r tolkien} Sagan om Ringen och mentor åt Crang i Teenage Mutant Ninja Turtles\ref{teenage mutant ninja turtles}. Enligt rykten ska Fredriks Reinfeldts farfar ha tillhört zulufolket och ha praktiserat kannibalism. Sonsonen har dock övergett denna tradition för gammal hederlig utsugning istället.

\ditem[Freikörperkultur]\label{freikoorperkultur}
 är tyska och är en sammansättning av orden för \textit{frihet}, \textit{kropp} och \textit{kultur}. Som du säkert redan befarat rör det sig alltså om tysk nudism, vilket innebär att man vid alla tillfällen som givs sliter av sig sitt bomullsfängelse och går omkring naken, att man uppmanar andra att ta efter detta gåtfulla beteende och att man tvingas sitta i koncentrationsläger när ens landsmän kommit på att de ska rösta in en galen massmördare från Österrike som rikskansler. Freikörperkultur är omåttligt populärt i Tyskland, speciellt i forna DDR, och är, paradoxalt nog, en mångmiljonindustri. Riktigt var i konceptet att gå runt naken alla dessa pengar ligger krävs det antaligen en \quotetext{frei} tysk att reda ut, men å andra sidan kan nog de flesta tänka sig att fortsätta förbli okunniga på detta område. Freikörperkultur är ett klockrent exempel på den tyska mustigheten\ref{den tyska mustigheten} och är besläktat med Kalle anka-konceptet\ref{kalle anka}.

\ditem[Fri rörlighet]\label{fri roorlighet}
 Ett modernare ord för slavhandel. Används ofta av tokliberaler\ref{tokliberal} för att förgylla Litauiska svartarbetare.

\ditem[Fri sikt]\label{fri sikt}
 är ett uttryck som ibland används i sammanhang som har med trafiksäkerhet att göra. Termen används dock betydligt oftare i det sammanhang det är avsett för: diskussioner om hur lätt det är att se bandet på olika sorters spelningar. Sikten från piten är nämligen olika i olika musikgenrer. Den som av någon anledning uppskattar reggae bör till exempel antingen vara lång eller komma till spelningen tidigt. Många reggaefans har nämligen stora frisyrer och/eller mössor, vilket effektivt skymmer sikten för de som står bakom. Är man kort eller tidsoptimist är det därför lätt att glida in på Oi!-punk istället för karibisk folkmusik. Man skulle nämligen kunna tro att Oi!-punk är den bästa musiken för den som vill kunna se bandet, eftersom Oi!-entusiasten i regel är väldigt korthårig. Men tyvärr är Oi! en genre som drar till sig män med högt body mass index, så därför hamnar den scenen på plus/minus noll när det gäller sikt och visibilitet. Av all faktorer som kan väga in i ett beslutet att följa och engagera sig i en musikscen är hänsyn till sikten på spelningar troligtvis bland de minst vanliga. Mer vanligt är hur lätt och önskvärt det är att få ragg på spelningar.

\ditem[Fridhem]\label{fridhem}
 är ett bostadsområde i Gävle känt för sina välklippta gräsmattor, välskötta rabatter och som \quotetext{stadsdelen där alla har ett leende på läpparna}. Området byggdes som en exakt kopia på Walt Disneys bostadsområde \textit{Celbration}, Florida och har marknadsförts som ett boende där trygghet och familjen står i fokus. Trots den trygghet och den livskvalitet som invånarna där känner så är området inte fritt från kritik. Dels har kritik framförts mot själva idén av att en välbesutten medelklass isolerar sig från omvärlden och därmed också stänger ute allt som inte passar enligt deras snäva definition av normalitet men på senare år har också en kritik framförts över vad som blivit av de nu vuxna personerna som fått växa upp i denna artificiella såpbubbla. Det har visat sig att flera av dessa har haft svårt att anpassa sig till världen utanför, som för dessa personer ter sig skrämmande och hotfull. Det är inte ovanligt att personer uppvuxna här ofta försöker hävda sig inom matchokulturer som t.ex att hävda att de är bra på kampsporter eller helt enkelt bra på att slåss, men där sanningen uteslutande ligger i att de endast \quotetext{slåss} i skyddade miljöer så som anonyma skribenter på internet eller i rollspelet World of Warcraft.

\ditem[Friedrich Hegel]\label{friedrich hegel}
 (1770 - 1831) var en tysk filosof som funderade fram dialektiken\ref{dialektik}. Det går inte nog att understryka hur tysk denne man var då han led något kopiöst av den tyska mustigheten\ref{den tyska mustigheten} och det storhetsvansinne\ref{storhetsvansinne} denna medför. Det fick honom att hålla en föreläsning om filosofins historia där han började med antiken\ref{de gamla grekerna} och tesen Platon mot anti-tesen Aristoteles och jobbade sig dialektiskt genom filosofins historia fram till den slutgiltiga syntesen, nämligen Hegel själv. Hans böcker kallas, skämtsamt, Hegelstenar.

\ditem[Frihet]\label{frihet}
 är bara ett annat ord för att inte ha nåt kvar att förlora.
Frågar man en tokliberal\ref{tokliberal} vad frihet är så handlar det om något som har med pengar att göra, eller typ att få behandla folk som arbetar som boskap. Nåt i den stilen.

\ditem[Frikyrkligt lycklig]\label{frikyrkligt lycklig}
 Vi vanliga syndare kan endast drömma om hur det är att vara frikyrkligt lycklig, för om vi inte först börjar knarka och leva satan och sedan blir pånyttfödda i herren kommer vi aldrig att få möjlighet att vara det. Att vara frikyrkligt lycklig är nämligen endast förunnat medlemmar av frikyrkorörelsen, som har utomhusgudtjänst, spelar gitarr\ref{gitarr} och sjunger lovsånger med ett slags extatiskt leende över anletet. Detta leende följer med baptisten eller pingstvännen in i konsumbutiken\ref{konsumbutik}, biblioteket, skolan, kontoret eller vad det vara må. Den frikyrklige gläds åt skapelsens prakt och all den kärlek som utgå från den förlåtande fadern. Vi andra, människorna i allmänhet, ser inte detta utan oroar oss över arbetslivets små motgångar, sexuella relationer och politik. Vi går i strejk, grälar på partnern, bygger ut huset och går klädda i obekväma kläder, medan den lyckliga - ja näst intill euforiska - frikyrkomedlemmen ser i den eleganta böjningen av ett grässtrå herrens skapelsekraft och ser att detta är gott.

\ditem[Frilans]\label{frilans}
 Ett finare ord för ganska arbetslös, särskilt inom kreativa yrken.

\ditem[Front row banger]\label{front row banger}
 En front row banger är en person som står längst fram på en hårdrockskonsert och headbangar\ref{headbanga}. Uttrycket populariserades i och med bröderna Nifelheims härjningar. Att stå längst fram vid kravallstaketet och headbanga visar att man är extremt hängiven till artisten man tittar på, på samma sätt som att göra ledingreppet\ref{ledingreppet} på en Tomas Ledin-konsert eller att dansa våldsam pogodans när Sex Pistols kommer till stan.

\ditem[Frukt]\label{frukt}
 är ett slags godis, har forskarna nu konstaterat efter noggranna studier. Frukt skiljer sig dock från andra sorters godis genom att växa på träd och buskage. Vanliga former av denna konfektyr är:

\begin{itemize}
\item Banan\ref{banan}
\item Äpple
\item Apelsin
\item Päron\ref{paeronhalva}
\item Sviskon
\end{itemize}

\ditem[Frukt är gott]\label{frukt aer gott}
 är den mest spelade sången i Sverige genom alla tider. Alla som vill lära sig ett instrument inom den kommunala musikskolan får börja med den låten. Bakgrunden är antagligen att låten anses enkel att lära sig med sina tre toner. Faktum är dock att \textit{Teenage Superstars} med Vaselines är ännu enklare med sina två toner så egentligen borde man lära sig den.

Texten till Frukt\ref{frukt} är gott: \textit{Frukt är gott, Frukt är gott, Äpplen och bananer smakar jättegott.}

\ditem[Frukters sociopsykologiska dimension]\label{frukters sociopsykologiska dimension}
 Givetvis är det så att frukter inte bara är mättande, välsmakande och ofta estetiskt löjeväckande (lex päron). De är också bärare av vissa kulturella föreställningar. Vilken frukt man äter är tydligt kopplat till vilken personlighetstyp man är, eller vill utge sig för att vara. Nedan följer en kort genomgång av fruktsorter och de associationer de bär med sig.

Bananen är kopplad till låg självkänsla. Den vulgäre frestas göra en totalfreudiansk analys, men den energi som strålar från en som äter banan är den hos en mjuk själ. Någon som varsamt skalar, och försiktigt nafsar i sig av den söta, mjuka, gula frukten. Halsen är för ömtålig av gråt för att klara av äpplets hårda kanter. Förtäring av banan syns ofta i lekplatsers utkanter, där de ensamma barnen med sandblandat snor under näsan tuggar på, med blicken vänd mot marken.

Äpplet å andra sidan utstrålar självförtroende. Blivande statsråd, företagspampar och motivationstalare gnuggar äpplets glansiga hårda yta mot sitt kavajslag innan de hänsynslöst tar ett bett in i det hårda, syrliga fruktköttet. I metropolernas bakgator rullar ledarna för gatubarnen äpplen över sina axlar, i en kaxig gest som säger till världen, I’m simply the best!

Apelsinen, slutligen, är det neurotiska geniets frukt. Utan ett maniskt temperament är det i stort sett omöjligt att ta sig igenom apelsinens oinbjudande skyddande skal. Och efter att fruktens härdade exoskelett utplånats, väntar en komposition så komplex att den utgör ett sorts inverterat Teodicéproblem (hur kan Gud finnas om det finns ondska i världen?). För utan en gudomlig intelligens, hur kan något så intrikat som denna naturens egna Rubiks kub ha kommit till? Apelsiner och den hög av avskräde dess förtäring lämnar efter sig, kan återfinnas på kontor, bibliotek, högskolor och konstnärsateljéer världen över.

\ditem[Fruktsallad]\label{fruktsallad}
 innehåller olika sorters frukt\ref{frukt}, men ingen sallad.

\ditem[Fryntlig]\label{fryntlig}
 Välvillig, tjock och röd - i ett ord!

\ditem[Ftw]\label{ftw}
 betydde på det ascoola/brutala 90-talet Fuck The World och var då ett populärt uttryck bland kåkfarare, bikers och medlemmar i Driller Killer. Betydelsen ändrades under det urmesiga 00-talet till For The Win och är nu ett väldigt populärt uttryck bland internetungdomar och andra fjantiga grupper.

\uline{OBS!!}

Slarva inte så det blir FTV. Det betyder Folktandvården och har aldrig, inte ens på 90-talet, varit populärt.

\ditem[Fudge]\label{fudge}
 är en populär konfekt som folk trugar på en inte minst kring jultid. Konfekten är oftast guldbrun, kubformad och har en konsistens som kan uppfattas som lite suspekt, vilket man dock har överseende med eftersom den har en ganska okej smak. Fudge uppfanns av psych-coverbandet Vanilla Fudge för så länge sedan att det är svårt att föreställa sig.

\ditem[Fulhybris]\label{fulhybris}
 Hybris hos en ful person. Det må så vara att skönhet ligger i betraktarens öga. Men det finns människor vilka de flesta betraktares ögon inte gärna tittar på. Fula personer. Hos dessa fula personer finns en utbredd uppfattning om att de, genom deras utseende, dragit en kosmisk nitlott. I samma sociala kategori (fula), är tron på ett karmaliknande system, som kompenserar den fule personen med gudaliknande egenskaper lika utbredd. Oftast handlar det om att den fule personen ser sig själv som extremt intelligent. Detta i sin tur leder ofta till ett extremt jobbigt beteende\ref{storbossnoord}, vilket vidare bidrar till den fules sociala isolering, och därmed tillförs mer bränsle till den fules hybris. För hur smart måste inte Gud ha gjort en om man både är ful och ensam? Enligt Nissepedias källor har ingen kausalitet kunnat påvisas, där ful=smart.

\ditem[Fulsnygg]\label{fulsnygg}
 är en person som både är ful och snygg på samma gång, något som kan skapa förvirring hos den som betraktar fulsnyggingen. Skolboksexemplet på fulsnygghet är den franske skådespelaren Gerard Depardieu, vars utseende, hans gnistrande leende och milda ögon till trots, alltid kommer att framkalla en känsla av ambivalens, då han i stället för en näsa har vad som förefaller vara en mandelpotatis storlek XL monterad mitt i fejset.

\ditem[Furu]\label{furu}
 Att ha åtminstone en del av sitt hem (helst en källare eller gillestuga) klädd i furu är för inredning vad det är för namn att heta Andersson eller för bilister att köra Volvo - Det svenskaste som finns. Att komma ner i en källare med väggar som lyser i det ljusa trämaterialet signalerar trygghet på samma sätt som att hitta Black Sabbath-vinyl på en skivloppis. Dalahästar och broderade dukar med kristna budskap på kan ta sig. Snackar vi svensk inredning snackar vi furu.

\ditem[Fyllevolontära]\label{fyllevolontaera}
 När man är lite på lyset är det lätt att man kommer på idéer som verkar bra för stunden men som i backspegeln\ref{inre backspegel} ofta kan te sig mer tveksamma. Ett klassiskt exempel på detta är att fyllevolontära.

\uline{Pedagogiskt exempelscenario}

Vi befinner oss på Augustibuller där en gråtfärdig 14-åring står vid campingentrén och inte blivit avlöst på två dagar för att träskpunkaren\ref{traeskpunkare} som skulle gjort det har drunknat i en barnpool fylld av urin\ref{urin}. Förbudet mot glasflaskor efterlevs mycket sporadiskt men barnets\ref{barn} möjligheter att göra något åt saken är ganska obefintliga. Men här kommer du, raj-rajande efter en lyckad spelning med (till exempel) Varukers, och får syn på den tårögda tonåringen. Dina solidaritetskänslor är starka efter att under en halvtimme ha sparkats på smalbenen av crustknytna\ref{crustknytning} kängor så du erbjuder dig genast att sköta barnets jobb medan denne letar på en vuxen. Du slår dig ned i en brassestol med knäckt rygg (alltså stolens) och vinkar glatt medan kavalkaden av slan\ref{slan} som saknar festivalarmband tågar in på campingen. Barnet har tagit första bussen hem och din mage börjar kurra efter ett tag så du drar och klämmer en burgare.

\ditem[Fyn]\label{fyn}
 är en ö tillhörande konungariket Danmark\ref{danmark}. Om man betraktar Danmark som en stor röv ligger Fyn precis där ändtarmen normalt har sin mynning. Detta återspeglar sig också i öns ekonomi som till stor del bygger på att medelst håv och dansk kanot\ref{dansk kanot} fånga in de fekalier som flyter runt i havet efter att Roskildefestivalens bajamajor tömts. De näringsrika slaggprodukterna torkas sedan och blir till gödningsmedel på haschplantagerna\ref{hasch} som genererar öns huvudsakliga exportprodukt: vit reggae. Eftersom den globala efterfrågan på denna vara är ganska begränsad går Fyn regelbundet i statskonkurs och måste räddas av den välbärgade haschtomten\ref{tomten} och vinylfantasten Lars Krogh\ref{lars krogh}.

\ditem[Fyra]\label{fyra}
 En fyra är en liten spritdrink och ett armgrepp i brottning. Av någon anledning känns denna kombination rättvisande och välmotiverad.

Se även: Etta\ref{etta}, Tvåa\ref{tvaaa}, Trea\ref{trea}, Femma\ref{femma}, Sexa\ref{sexa}, Sjua\ref{sjua}, Åtta\ref{aatta}, Nia\ref{nia}.

\ditem[Fyrhjulsdrift]\label{fyrhjulsdrift}
 är den enskilt största anledningen till dödsolyckor i trafiken. Efter ihärdig propaganda från bilindustrin har vanliga konsumenter sakta invaggats i en falsk trygghet om att fyrhjulsdrivna bilar är säkrare att köra. Förblindad av dogmer om ABS, antisladd, styrservo, etcetera övervärderar föraren sin förmåga och slutar allt som oftast i en bergvägg eller röven på en älg.

Den trafikant som värdesätter riktig säkerhet väljer istället bakhjulsdrivet. Med detta alternativ får du en tydlig indikation vid minsta halka och kan dessutom häva en sladd själv utan att behöva ställa dina förhoppningar till en millimeterstor kiseltransistor som någon storbossnörd\ref{storbossnoord} med flottiga fingrar lödit ihop.

\ditem[Fyrtiotusen miljarder]\label{fyrtiotusen miljarder}
 är ett kvantifierande kraftuttryck som myntades av Sverigedemokraten Margareta Sandstedt som skämtade till det lite i riksdagskammaren genom att räkna om 400.000 Euro till just fyrtiotusen miljarder SEK. Exempel på språkliga kontexter där uttrycket kan användas:

-\quotetext{Jag hade tänkt ta med mig en flaska vin\ref{vin}, men det var fyrtiotusen miljarder folk före mig i kön så jag struntade i det.}
-\quotetext{Jag hade gärna följt med på utflykt, men jag har fyrtiotusen miljarder saker att hinna med idag.}
-\quotetext{Max, Karl och Emile Andersson!! Jag har sagt åt er fyrtiotusen miljarder gånger att den som plockat fram legot\ref{lego} tar undan legot när den är klar!! (springer skrik-gråtandes mot vedboden)}

\ditem[Färskost]\label{faerskost}
 är en diet som går ut på att enbart äta saker gjorda på färs. Vanligtvis består färsen av kött från djur eller fiskar, men det enda riktiga kriteriet för att en färs ska få kallas färs verkar vara att den är hackad eller mald så för allergiker\ref{allergi} torde det gå utmärkt att köra på till exempel gröt- eller gurkfärs. Dieten har utvecklats av Nissepedias\ref{nissepedia} medarbetare Ronny och är en hårdare variant av hans gamla koncept \quotetext{bärs och burgare}. I valet mellan att hacka och mala rekommenderar han hackat, eftersom man får använda kniv då. Vad dieten ska vara bra till är oklart.

\ditem[Fåglar som går]\label{faaglar som gaar}
 är något av det mest fåniga och skrattretande som finns. Ofta kan man se dem på stan - duvor och kajor som går utmed marken och letar mat. De har så små ben och otjänliga fötter. Vissa fåglar kan inte ens gå, som vi förstår ordet, utan liksom hoppar jämfota framåt eller \quotetext{knixar\ref{knixa}} där de står.

Vissa fåglar, så som familjen vadare, har fått sitt namn genom ovanan att gå istället för att flyga. Andra har gått så mycket att evolutionen har gett upp på dem och tagit bort deras flygförmåga. Sådana fåglar har fått den träffande benämningen \quotetext{gåfåglar}. De ojämförligt största gruppen av fåglar som går är dock helt vanliga fåglar som kan flyga men som av en eller annan anledning väljer att promenera istället.

\ditem[För nit och redlighet i rikets tjänst]\label{foor nit och redlighet i rikets tjaenst}
 är en belöningsmedalj som delas ut till den som oförvitligt tjänat staten i över 30 år. Den präglas i 23 karats guld, väger 11 gram och har en diameter av 27,5 mm.

\uline{Valfrihet}

Den som inte önskar få belöningen i form av en medalj kan istället välja på följande alternativ:

\begin{itemize}
\item En vacker skål av glas från Kosta Boda.
\item Ett förgyllt armbandsur.
\item Motsvarande värde i Abu Garcia-kepsar\ref{abu garcia}.
\end{itemize}

\uline{OBS!}

För nit och redlighet i rikets tjänst är ingen leksak! Den ska därför inte behandlas som en sådan, utan placeras på en väl synlig plats där den inte kan komma till skada. Undantaget är om man väljer Abu Garcia-alternativet. Abu-garcia-kepsar är lite som leksaker och kan behandlas hur som helst.

\ditem[Fördelar med att bo i gryt]\label{foordelar med att bo i gryt}
 Gryt är en boendeform som funnits mycket längre än populära nymodigheter som bostadsrätt och villakollektiv. Trots den långa etableringstiden är det ändå nästan ingen människa som bor i gryt idag. Man kan tycka vad man vill om detta och vi på Nissepedia\ref{nissepedia} tycker att det är himla synd. Att bo i ett gryt har nämligen, bland annat, följande fördelar:

\begin{itemize}
\item Det är gratis.
\item Det är svalt på sommaren.
\item Tak, väggar och golv är av jord så inget blir smutsigare än det var från början.
\item Ibland kommer det maskar och mullvadar krypandes i jorden så om du inte är vegan serverar middagen sig själv.
\item Ibland händer det att jägare går på grytjakt så om du faktiskt är vegan kommer ett ypperligt tillfälle att skrämma slag på din antagonist.
\item Du behöver aldrig putsa fönstren eller måla om fasaden.
\end{itemize}

Som läsaren märker finns det i princip bara fördelar med att bo i gryt. Ska man säga något negativt skulle det i så fall vara att det kan finnas grävlingar där, och grävlingar biter tills det knakar.

\ditem[Förebyggande skallgångskedja]\label{foorebyggande skallgaangskedja}
 Förebyggande skallgångskedjor bildas för att se till att personer som ännu inte försvunnit förblir oförsvunna. Den förebyggande skallgångskedjan drar runt på stan i armkrok tills de får syn på en tant med misstänkt förvirrande hatt eller en gammal gubbe som inte verkar fatta någonting. Den förebyggande skallgångskedjan tar då rygg på personen och följer efter den tills man bedömer att den trots allt är på rätt spår. Skulle det dröja för lång tid innan några positiva tecken visar sig fångas personen in och transporteras tillbaka till torget där den först upptäcktes för att få ett nytt försök. Den förebyggande skallgångskedjan har fullföljt sitt uppdrag och kan fira med att ta ännu en snaps.

\ditem[Förlovad]\label{foorlovad}
 En gång i tiden förlovade man sig och sedan gifte man sig inom ett år.

\textit{\quotetext{Vi ska gifta oss men inte just nu}} så att säga.

I takt med att ungdomen valde sus och dus istället för Lars Levi Laestadius\ref{lars levi laestadius} skrifter blev en förlovning nån slags tonårsgimmick.

\ditem[Första sjuan]\label{foorsta sjuan}
 Inom de flesta obskyra musikstilar börjar de flesta banden med att släppa en sjua\ref{sjua}.

Den här är alltid bättre än nästkommande släpp oavsett hur det låter, i alla fall om du frågar en konnässör. Framförallt så är den jättedyr och skivsamlare får halvfralla/smygväta om de skulle bläddra fram en sådan i en skivback.

Exempel på första sjuor som är bättre än resten av bakkatalogen:

\begin{itemize}
\item Tragedy - Can we call this life?
\item Totalitär - Multinationella Mördare
\item Mob 47 - Kärnvapenattack
\item Minor Threat - Minor Threat
\item Earth Crisis - All Out War
\item Rolling Stones - Come on b/w I want to be loved
\item Atomångest - Kasta gatsten
\end{itemize}

\ditem[Första skivanalibi]\label{foorsta skivanalibi}
 är mycket vanligt bland skinheads\ref{skinhead}, framförallt vad det gäller Skrewdriver.

\quotetext{Men dom var inte nazister på första skivan} säger alltid den strykrädde skinnskallen som konfronteras av en grupp glada\ref{spritfylla} lulebor.

Förstan skvianalibit fungerar även för den anti-auktoritäre socialisten med smak för progg, och batiktomten hävdar då att \quotetext{Dom var inte stalinister på första skivan} när hen ombeds förklara sin t-shirt med Knutna Nävar för en liberal ledarskribent.

Första skivanalibi kan även gälla kassetten \quotetext{Gällivarevisor} som faktiskt spelades in 20 år innan Nisse Hennix blev nazist.

\ditem[Förstklassig cricket]\label{foorstklassig cricket}
 är den ädlaste klassen av nationell cricket och består av matcher som spelas under minst tre dagar. Lagen är två till antal och består av vardera elva spelare. Förstklassiga matcher kan bara spelas mellan lag tillhörande samma nation. Den ädlaste klassen av matcher mellan nationer kallas istället \quotetext{test cricket} och kan bara spelas av nationer som tilldelats \quotetext{test status} av Internationella Cricketrådet (fram till 1966 kallat \quotetext{Imperiets Cricketkonferens}). Termen förstklassig cricket definierades första gången 1947 under ett möte med Imperiets Cricketkonferens och klassificeringen kan bara ges särskilt kvalificerad lag.

\ditem[Förståsigpåare]\label{foorstaasigpaaare}
 Det enda yrket som står till buds för någon med över 300 hp. Yrket tillkom under regeringen Persson som en arbetsmarknadsåtgärd för att avhjälpa den allt mer kritiska situation som det ökande antalet personer med kandidatexamen ledde till. Förståsigpåare får inte förväxlas med anställda hos institut och tankesmedjor\ref{institut och tankesmedjor}

\ditem[Försåvitt]\label{foorsaavitt}
 är ett väldigt pråligt ord som alla borde använda mycket oftare. Det är synonymt med \quotetext{om} och \quotetext{ifall}. Ett exempel på hur det kan användas: \textit{\quotetext{Termen globalisering avser väsentligen den tänjningsprocess försåvitt förbindelserna mellan olika sociala sammanhang eller regioner blir till ett nätverk över hela jordens yta}}(Giddens 1996: 66). Ett annat exempel som är lättare att tillämpa i vardagen: \textit{\quotetext{Kära surkärring jag är gift med, jag vägrar komma ut från mitt hemliga gömställe försåvitt det inte står tolv frasvåfflor, en hink blåbärssylt, sju deciliter vispgrädde (halvvispad - sådär som jag tycker om det) och tre Arboga 10.2 \% på middagsbordet inom 30 minuter}} (Etienne\ref{prof. etienne} 1992: 88).

\ditem[Förvirring]\label{foorvirring}
 Befinner du dig i en situation där du inte riktigt förstår vad som händer? Har fenomen som vart och ett är fullkomligt rimliga kombinerats på ett sätt som inte går att förklara? Har du svårt att sätta ord på vad du ser? Då är det troligt att du drabbats av förvirring.

Förvirring är ett utbrett fenomen som drabbar nästan alla som inte förstår vad som händer. Exakt vad det är har debatterats länge och någon full konsensus råder tyvärr inte. Fysiker förklarar det vanligtvis med att DNA:t hos den förvirrade har för små atomer, medan hippies tror benhårt att det beror på att det är människans litenhet i förhållande till universum. Postmodernister tror som vanligt att allt egentligen förhåller sig tvärt om.

Förvirring existerade redan långt innan den första hominiden vandrade ut över Pangea och slog ned sina bopålar. Redan under dinosauriernas tid ledde förvirring till att snurrigare arter blev dödade av rationellare. Tyvärr finns det inget helgarderande försvar mot förvirring. Sprit kan fungera ett tag för att få allt att verka rimligt, men tyvärr brukar förvirringen ofta komma tillbaka med dubbel kraft efter ett tag. Att meka bilar fungerar också skapligt så länge kärran inte är för ny. Den som drabbas av förvirring bör för sin egen och andras skull hålla sig undan.


%%%%%%%%%%%%%%
\newpage
\null
\\
\null
\\
\Huge
G
\normalsize
\\
\null
\\
\null
%%%%%%%%%%%%%%


\ditem[Galna ko-sjukan]\label{galna ko-sjukan}
 Well, never mind. We are ugly but we have the mu-sick.

\ditem[Gamle Ole]\label{gamle ole}
 är något så ovanligt som en dansk ost. Man tänker att det inte borde finnas ost i Danmark\ref{danmark} eftersom det enda pålägg danskar har på sina smörgåsar är godis och alla åkrar är belamrade med besökare till Roskildefestivalen. Men det finns faktiskt några hektar oförstörd betesmark på norra Jylland (vilket förmodligen beror på att det ligger en tandläkare och en klädaffär i närheten, och danskar hatar som bekant att borsta tänderna och att inte vara nakna) och där betar några trötta kor. Osten karraktäriseras framförallt av sin starka lukt men har också en distinkt smak. Ska man bjuda på ostbricka rekommenderas att man lämnar Gamle Ole utanför eller ställer på ett separat fat i ett eget rum så att den inte tar över smaken på andra ostar. Mads Mikkelsen\ref{mads mikkelsen} har kallat Gamle Ole \quotetext{mye, mye dejlig}. Vem Ole är eller var förtäljer inte historien, men kanske luktade han riktigt mycket ost.

\ditem[Gammpojkar]\label{gammpojkar}
 En man som trots sin ålder behållit sitt oberoende, man framlever alltså sina dagar som ogift och barnlös.

Han bor allmänt kvar i fäderneshemmet (ofta är han hemmansägare) ibland tillsammans med en bror som även han är gammpojk. Som gammpojk har man gott om tid att lägga på hembyggda vedklyvar, stövarjakt, starka drycker\ref{braennvin} och radiosporten.

Man färdas stundom längs byavägen på moped eller möjligen i dieselbil med lastgaller;\ref{dieselbil med lastgaller} på väg till konsumbutiken\ref{konsumbutik} för att handla kokkaffe, ärtsoppa på burk och kanske en korvsnärt\ref{snaert}.

\uline{Gammpojkar i kulturen}

\begin{itemize}
\item Esko Männikö har skildrat gammpojkar i norra Finland\ref{finland} i en fin bok.
\item Euskefeurat har gett röst till gammpojken Leonard Larsson på plattan \quotetext{Hipp Happ}
\item Frantz Kafka skrev uteslutande om gammpojkar, även om dessa var kontinentala sådana.
\item Leif Bäckström i den svenska filmen \textit{Jägarna}, skickligt spelad av Lennart Jähkel.
\end{itemize}

\uline{Synonymer till Gammpojk}

\begin{itemize}
\item Gammelgoss
\item Ungkarl
\item Grovsingel
\end{itemize}

\ditem[Gammstinta]\label{gammstinta}
 En gammstinta är en kvinna som inte hittat någon lämplig karl att gifta sig med. Det kan bero på att hon har jättehöga krav eller att alla karlar i världen har en massa brister, eller någonstans mittemellan. Man kan tycka att varje gammstinta borde hitta sig en gammpojk\ref{gammpojkar}, men det händer väldigt sällan.

\uline{Gammstintor i kulturen}

\begin{itemize}
\item Bridget Jones
\item Elaine Benes
\end{itemize}

\ditem[Geezer Butler]\label{geezer butler}
 Terence Michael Joseph \quotetext{Geezer} Butler är en brittisk man som inte alls är butler utan basist i hårdrockens grand old band Black Sabbath. I intervjuer är det lätt att missta herr Butler för en helt vanlig hippie\ref{hippie} som puffat på pipan lite för länge men om man läser texterna han skrev för Sabbath om religion, magi, krig och andra otäcka saker förstår man att han inte är en snubbe att leka med. Räcker inte det som bevis kan man googla fram bilder från runt 1980 där Geezer ofta manifesterade sin råhet med svarta spandexdräkter med eldflammor på. Känsliga rockers har varnats.

\ditem[Gengas]\label{gengas}
 är en gas som uppstår vid ofullständig förbränning av trä och kol. Men hjälp av ett gengasaggregat kan gasen renas och regleras på ett sätt som möjliggör att använda den som drivmedel till bilar. \quotetext{Det här låter ju som rena storfiskarhistorien}, tänker du nu. Men under det stora fosterländska kriget\ref{det stora fosterlaendska kriget} fanns i Sverige faktiskt fler än 70.000 bilar som drevs på detta sätt. Det är lätt att föreställa sig en romantiserad bild av hur föraren märker att mätaren börjar närma sig rött och då bara ler, öppnar handskfacket\ref{handskfack}, plockar ut ett vedträ som han slänger in i kaminen i baksätet. Riktigt så fungerar inte gengas men man önskar att den gjorde det. I realiteten sliter gengas något så djävulskt på motorn och det händer ibland att aggregaten sprängs. Så sker till exempel i Åke Hodells bok \textit{Elddopet}, en bortglömd klassiker i den svenska litteraturen.

\ditem[Genuint snåla människor]\label{genuint snaala maenniskor}
 är sådana som arbetar inom offentlig sektor och röstar på moderaterna\ref{moderat} för att de vill undgå att betala skatt. Typiskt kälkborgeri\ref{kaelkborgare}.

En annan kategori genuint snåla människor stöter du på i sådana hem där pappersinsamlingen förvaras inne på toaletten. Samma kategori värmer bara upp ett rum i huset och klagas det på kylan kommer svaret \quotetext{Men gå till varmrummet då!}.

\ditem[George Everest]\label{george everest}
 Lantmätare från Wales (4 juli 1790 – 1 december 1866, från 1861 \textit{Sir} George Everest). För allmänheten känd som den person Mount Everest\ref{mount everest} är uppkallad efter. För sina vänner främst ihågkommen som han som alltid slet av sig byxorna när han var full.

\ditem[Georgij Zjukov]\label{georgij zjukov}
 var en sovjetisk militärgeneral som bland annat listade ut hur man skulle inta Berlin och återta Leningrad. För detta och lite annat blev han utnämnd till \textit{Sovjetunionens hjälte} hela fyra gånger, två fler är självaste Jurij Gagarin\ref{jurij gagarin}. För att öka hajpen ytterligare lät kommunistpartiet ge ut serietidningen \textit{Rött Inferno}, som handlade om Zjukovs bedrifter. Precis som Haile Selassies\ref{haile selassie} mausoleum är Zjukovs dito idag en offentlig toalett.

\ditem[Gertrud]\label{gertrud}
 är ett gammsvenskt namn som är en sammansättning av gammsvenskans ord för \quotetext{bjär} och \quotetext{skrud,} det vill säga OP-klänning.

\ditem[Gitarr]\label{gitarr}
 Ett lika vanligt som älskat stränginstrument vars form påfallande mycket påminner om hur kurviga kvinnor målas och skulpteras av gubbar som påstår sig vara livsbejakande \quotetext{konstnärer}. Många medelålders män har också ett slags nostalgiskt och känslofyllt förhållande till gitarren, trots att den är ett relativt vanligt förekommande instrument och en ägodel som påträffas i typ vartannat hem i det västerländska samhället. Detta gäller även, eller kanske framförallt, om personen i fråga knappt kan spela gitarr och kanske aldrig ens ägt en sådan. Även om inte mannen kan spela gitarr för att rädda sitt liv hindrar detta honom inte från att tala om klassiska gitarrer som \textit{Fender Stratocaster} och Gibsons \textit{Les Paul} och påminna omgivningen om världskända gitarrister med mindre imponerande förmågor när det gäller att faktiskt producera ny, kreativ musik - så som Eric Clapton och Sting. Det faktum att gitarren påminner om en kvinnokropp och samtidigt är en fallossymbol ger en viss hint om varför så många män ser på den med nostalgi och med en längtan tillbaka till det förflutna, vilket dock inte gör den komplexa sexual-psykologiska problematik som ligger bakom detta fenomen mer begriplig. 

\ditem[Gitarrkille]\label{gitarrkille}
 En gitarrkille har ofta hatt, väst och problem med sin självinsikt. Han kan vara i vilken ålder som helst, och nästlar sig gärna in på alla slags tillställningar. Han har inte hört talas om jantelagen och bör lugna ner sig och bete sig som folk. Han slaktar gärna den ena covern efter den andra, utan någon som helst tanke på att folk kanske önskar att han behöll sin kreativitet för sig själv och sitt pojkrum. En äldre gitarrkille kallas med ett finare ord för trubadur, och/eller ses som frontman i ett coverband. En gitarrkille bör hanteras hårt och bestämt; knip tillgångarna (gitarren) och höj musiken på stereon när du anar vad som kan komma att ske. Uppmuntra honom inte på något sätt om du vill ha en fortsatt trevlig kväll.

\ditem[Gitarrmys]\label{gitarrmys}
 är en populär aktivitet på lärarlagsfester och liknade tillställningar och går ut på att killen med musikerambitioner i lärarlaget plockar fram en gitarr och spelar lite låtar som de andra kan sjunga med i. Oftast inmundigas alkohol före, men å andra sidan är det just gitarrmyset som förhindrar att alla blir grisfulla\ref{grisfull} och idkar samlag med varandra. Många gånger startar gitarrmyset lite oväntat, eftersom folk varit för indragna i något samtal som börjar bli lite gapigt, när gitarrkillen plötsligt utbrister Halt!..........(Paus för att de andra på festen ska förstå vad som händer) .... Här får ingen passera! Här kommer ingen förbi!" och så är gitarrmyset igång. Senare kommer någon Cornelis Vreeswijk-låt och kanske, om man har tur, nån Tomas Ledin-bit.

\ditem[Gittan]\label{gittan}
 är ett kvinnonamn som betyder \quotetext{den som är en jävel på att lira ackegura}.

\ditem[Glassbutt]\label{glassbutt}
 (på engelska \textit{tupperware}), är ett kärl speciellt framtaget för att transportera lunchmat från hemmet till en annan plats. Ett genomsnittligt kärl rymmer mellan 0.5 och 1 liter. Överstiger kärlet 1 liter kallas det lunchtråg\ref{traag}. Glassbutten uppfanns av en ren slump av den finske industridesignern Uri Marimekkonen (1912-1996) när han skulle konstruera en begravningsurna i det nya trendmaterialet plast. För att lansera den nya produkten genomförde Uri ett PR-jippo där han lät fylla de 500 första kärlen med glass. Produkten blev en succé och reklamkonceptet används fortfarande på sina håll i världen. IKEA försökte som vanligt plagiera designen och lanserade den billigare varianten glasspaketet. Det är inte på långa vägar lika funktionellt att transportera mat eller människostoft i.

\ditem[Glassbåt]\label{glassbaat}
 är en skapelse som kombinerar det bästa av två världar. Glass: godisets närmaste släkting men inte alls lika tabubelagt. Båt: symbolen för drömmen om det fria livet på sjön. Med ett ytterhölje av kex kan du hålla i den utan att bli kladdig och slippa bry din lilla hjärna med vad du ska göra av pinnen. Till utseendet påminner den om Noaks ark, såsom den avbildades i Gustave Dorés bibel. Och det kan ju gå an, så väl som den smakar.

\ditem[Glasse]\label{glasse}
 är företaget Triumfglass’ maskot. Han är en isbjörn och det bästa han vet är glass. Han har haft en lång och krokig karriär men mår just nu ganska bra. Triumfglass startades 1946 och strax där efter gjorde Glasse sitt intåg, till många barns\ref{barn} lycka. År 2003 försvann han dock till synes spårlöst och sommaren var sig inte längre lik. Det visade sig att han kidnappats av det norska klasskonsortiet Diplom-Is, som lagt rabarber på Triumfglass och ersatt Glasses glada nuna med den betydligt fulare Eskimonika. Reaktionerna lät inte vänta på sig och i februari 2010 bytte Diplom-Is namn till Triumf Glass AB och Glasse är populärare än någonsin. Så kan det gå om man bråkar med Glasse.

\ditem[Glasögon]\label{glasoogon}
 är en attiralj som i varierande grad förbättrar din syn.

Motsatsen till glasögon kallas metanol.

\ditem[Glenn]\label{glenn}
 Titeln Glenn tilldelas framstående göteborgare och relaterade personer inom bollsporter såsom gruppbollsparkning, bollkastning och ölbrännboll. Namnet påstås ha irländska rötter men verkligheten är något helt annat. Glenn som namn kommer från det fornsvenska ordet för varvsarbetare, ett i Göteborg traditionellt högt aktat yrke. Detta ledde till att Glenn övergick i en titel, då svensk varvsindustri gick i putten efter att regalskeppet Wasa satte industrin på pottkanten, nådens år 1628.

\ditem[Glesbygdsball]\label{glesbygdsball}
 I vissa tätorter betonar man sin metropolitiska avart genom att uttalat tycka att det är \quotetext{genuint} att komma från orter med få invånare. Landsbygden ses som genom ett flor som skänker flärd och romantik till en tillvaro som i själva verket ofta präglas av patologisk inåtvändhet, sexuell frustration och förekomnsten av villfarelsen att Jula är en klädaffär. Vissa cashar in på detta genom att starta reggaeband som besjunger detta livsöde och då kan man bli inbjuden att spela på TV4s nyhetsmorgon om man har tur. Kanske får man ligga med kulturstockholm\ref{ligga med kulturstockholm}.

\ditem[Glida under radarn]\label{glida under radarn}
 Den provisoriska titeln på ett manuskript till landsförrädaren Stig Berglings ännu outgivna spänningsroman. Innehållet sägs vara av huvudsakligen självbiografisk karaktär, med färgstarkt skildrade internationella intriger och livsbejakande erotiska anekdoter.

Enligt källor existerar texten i ett enda (ofullständigt) fysiskt exemplar, vilket tillföll Berglings polska ex-hustru i samband med parets andra skilsmässa 2004.

\ditem[Glima]\label{glima}
 är en form av ganska störtlöjlig vikingabrottning som går ut på att man sliter varandra i bältet tills den ena tävlanden ramlar, eller blir \quotetext{omkullglimad}. Sporten utövas idag främst av nazister.

\ditem[Glimröv]\label{glimroov}
 är vikingatidens motsvarighet till\ref{hockeyroov}, och kännetecknas precis som hockeyröven av sin imponerande radie. 

\ditem[Globen]\label{globen}
 är ett slags enorm golfboll som man kan gå in i för att kolla på ishockey eller U2. Bollen står i södra Stockholm\ref{stockholm} och kan ses enda från Danderyd en klar dag. Globen anses sedan 2007 vara ett av Sveriges sju underverk\ref{sveriges sju underverk} eftersom den är så rund och golfboll-lik. De som inte varit i Stockholm har förmodligen ändå sett Globen på TV i slutscenen till Björn Skifs gamla kioskvältare \textit{Joker} som visas på TV4 med jämna mellanrum.

\ditem[Globetrotter]\label{globetrotter}
 En globetrotter kan vara en person som reser väldigt mycket och har genom det fått en självklar världsvanhet som är väldigt irriterande för globetrotterns nära och kära. De kan haspla ur sig grejer i stil med \quotetext{Åh det här är som när jag var i Tokyo förra hösten...} och resten av sällskapet tvingas nicka och le instämmande och känner sig som att de aldrig varit utanför 50-skyltarna.

Globetrotter är också namnet på en av Volvos lastbilshytter som tillverkas vid företagets Umeåfabrik. Den är stor och rymlig och som gjord för att åka jorden runt i, men det är inte därifrån den fått sitt namn. I själva verket så är det Volvos smygkommunistiska fackledning som döpt den till Globetrotter som en hyllning till Leon Trotsky\ref{trotta} och den globala revolutionen.

\ditem[Glop]\label{glop}
 Ett stadie i en människa av mansköns liv vilket inträffar mellan gosse och karl.

Moderna glopar hänger utanför Coop och har ofta wct-byxor\ref{wctbyxa}, huvtröja samt spottar obscent mycket på marken.

\ditem[Glädjevetenskaper]\label{glaedjevetenskaper}

 är akademiska forskningsområden som skänker folk stor förnöjsamhet och hopp om en vackrare värld. Den största glädjevetenskapen är ortnamnsforskning som syftar till att utröna de etymologiska betydelserna bakom nordiska ortnamn. Ett enda ortnamn, som exempelvis Lövånger, kan sysselsätta en ortnamnsforskare i flera år och ändå skänka denne stor tillfredställelse när historien äntligen är klarlagd. Det finns ju flera hundra Lövånger i Sverige! Ortnamnsforskningens vita val är Medelpad. Vart ligger egentligen meden mellan padarna? Så där håller det på, till allas glädje.

\ditem[Gnagare]\label{gnagare}
 En gnagare är någon som, av någon anledning, alltid vill att AIK (Allmänna IdrottsKlubben, Solna, Stockholm) ska vinna när det är match. AIK vinner dock inte alltid, vilket gör gnagaren ledsen och nedstämd. Gnagaren går då hem till sitt krypin någonstans i Solna eller stannar ute och slåss med kniv. Det är vanligt att gnagaren är nynazist.

\ditem[Gnussa]\label{gnussa}
 är en term som används inom den samtida pardans-kulturen. Att gnussa innebär att två danspartners liksom gnider och smeker sig mot varandra medan de dansar. Speciellt kontakt mellan partnernas ansikten är viktigt för att det ska röra sig om gnussning.

\uline{Trivia}

Det är strängt förbjudet att ovälkommet gnussa någon utanför dansbanan.

\ditem[Godisautomat]\label{godisautomat}
 En godisautomat är en smart mojäng som gör det möjligt att få en helt annan sorts godis än den man betalat för. Godisautomater finns i två utföranden: En som tar enkronor och en som tar femkronor. Man stoppar in det angivna myntet i en springa och vrider på ett handtag och vips så kommer fel sorts godis ut ur en lucka.

\ditem[Golf]\label{golf}

 är en skotsk sport där utövaren slår allt vad den orkar med en pinne på en liten boll för att se vart den tar vägen. Världens bäste golfare hette Kim Jong-Il, som enligt egen uppgift snittade 3 hole-in-one per runda.

\ditem[Gore-Tex]\label{gore-tex}
 är ett material som är vind- och vattenavvisande, men ändå \textit{andas}. Av Gore-Tex kan man bl.a. göra regnkläder. Man kan undra varför kläder behöver \textit{andas}, men prova jogga i galonbyxor så kanske ni förstår. Har en person mycket kläder med Gore-Tex i eller på är den personen friluftsintresserad, vegan, båda eller en vanlig människa utan sinne för ekonomi.

\ditem[Grafologi]\label{grafologi}
 går ut på att studera sambandet mellan utseendet på handskriven text och egenskaper hos personen som skrivit. Den moderna grafologin uppfanns av Carl von Linné\ref{carl von linné} efter att denna tröttnat på anklagelser om att vara usel på partytrick. Om många ord slutar på \quotetext{E} är det troligt att en kille skrivit medan många ord som slutar på \quotetext{A} tyder på att det är en tjej. Skriver man skrivstil har man troligtvis glasögon. Korta meningar med ett hårt tryck mot papperet tyder på att författaren haft bråttom eller varit arg. Är texten skriven med sprayburkar har personen stora byxor.

Kända grafologer: Hans Scheike, Max Pulver.

\ditem[Grand Funk Railroad]\label{grand funk railroad}
 , eller GFR som fansen kallar det, är ett rockband från Flint i USA. De har gått till rockhistorien för att de var först med:

\begin{itemize}
\item Distad bas
\item Att dra distratten förbi 7
\item Långt hår
\item Cowboyhatt
\item Jeans
\item Strupsång
\item Gitarrsolon
\item Dubbeltramp
\item Att ha en trummis som dör under mystiska omständigheter
\item Powerackord
\item Sjungande trummis\ref{sjungande trummis}
\end{itemize}

\ditem[Grekiska statsobligationer]\label{grekiska statsobligationer}
 är ett oförtjänt baktalat sorts värdepapper. För den ekonomiske lekmannen räcker som förklaring att pappret berättigar ägaren till utdelning vid försäljning i den mån den grekiska ekonomin går med förtjänst. Här har vi ett land och örike som kan sägas vara den europeiska kulturens vagga, där demokratin och den logiska filosofin föddes, ett land vars skönhet lockat mången konstnär och skulptör att för en tid bosätta sig i ett högt beläget åsneskjul för att med detta som utgångspunkt kunna förkovra sig i den kulturskatt som dväljs i riket Grekland. Knossos, Sokrates, fetaost.. många är de uppslagsord som Grekland producerat sedan antikens storhetsdagar. Varför skola då just grekiska värdepapper icke vara tillförlitliga så som sparform? Om detta säger Prof. Etienne följande kloka ord: \textit{Köp, vid filosofens skägg! Köp!}

\ditem[Grind]\label{grind}
 Ett slags dörr eller jättesnabb punk.

\ditem[Grisfull]\label{grisfull}
 Kraftig alkoholberusning. En grisfull person har vanligtvis svårt att tala i sammanhängande meningar och kontrollera sin urinblåsa. Om flera grisfulla personer vistas på en liten yta är chansen stor att handgemäng kommer uppstå.

\ditem[Grissini]\label{grissini}
 är en slags brödpinnar som består av mjöl och inget mer. Det smakar inget, man blir fet men inte mätt av det och det smular. Grissini står ofta framdukat på finare restauranger redan när man tar plats vid bordet. Tanken med detta är att det ska kännas lyxigt att man erbjuds detta gratis. Yes, en pinnformad mjölbit som kostar mindre än en fis i rymden, fan vad generöst. Den medvetne konsumenten bojkottar grissini och köper istället dess delikata kusin salta pinnar som smakar utsökt, är fria från fett och höjer stämningen på alla fester.

\ditem[Gruk]\label{gruk}
 är, som läsaren förmodligen redan gissat, ett danskt versmått. Snubben som uppfann det hette Piet Hein (1905-1996) och var på äkta danskt vis förutom poet även vetenskapsman, matematiker, filosof, författare, uppfinnare, konstnär och den som formgav Sergels torg. En klassisk gruk är kort och koncis och bör innehålla ett paradoxalt eller satiriskt inslag. Under Tysklands\ref{tyskland} ockupation av Danmark\ref{danmark} blev gruken ett viktigt redskap för att gjuta mod i motståndsrörelsen (tyskar förstår som bekant inte ironi och man kunde därför trycka grukar helt öppet i dagstidningarna). Hein själv skrev över 7000 grukar som han gav ut i mer än 20 böcker. Sedan hans död har nyproduktionen av grukar kraftigt minskat, liksom Danmarks motstånd till nazism.

\uline{En klassisk gruk}

\textit{Til antropologiske forskere}
\textit{i det inden- og udenlandske}
\textit{kaster jeg hermed en handske:}
\textit{Jeg paastaar, at Nordmænd er norskere}
\textit{end nogen Danske er danske.}

\ditem[Grunka]\label{grunka}
 (verb, infinitiv; avlett substantiv, \textit{grunk}) är en multipel handling där utövaren gråter högljutt samtidigt som den onanerar. Fenomenet har blivit vanligare på senare tid i och med att ens sexpartner avvisar alla närmanden. Gråten måste vara ärlig.

Grunka kan även användas synonymt med pryl, vilket kan leda till språkförbistringar.

\ditem[Grå eminens]\label{graa eminens}
 Person som lever efter familjen Wallenbergs devis om att \textit{\quotetext{verka utan att synas}}; den som håller till i kulisserna och drar i trådarna. Ofta har den grå eminensen inte någon officiell makt utan styr sin marionett med andra medel. Gemensamt för hela skrået är att man har någon form av diabolisk plan som man skrattar diaboliska skratt åt varje gång man kommer ett steg på vägen. Den första grå eminensen var kapucinmunken Joseph Le Clerc du Tremblay som var kardinal Richelieus rådgivare och arbetade för Frankrikes deltagande i trettioåriga kriget. Honom har ingen hört talas om, men om man istället tar Saruman och den där gamla kungen han förtrollat så förstår nog de flesta. 

Sveriges mäktigaste grå eminens är kungens polare Noppe Lewenhaupt. Postens frimärksserie \quotetext{Hästsport} ska till exempel ursprungligen varit Lewenhaupts idé - han ska till och med ha skrattat diaboliskt vid lanseringen.

\ditem[Gråmelerad T-shirt]\label{graamelerad t-shirt}
 Känner du fortfarande att du är ung och har koll på läget? Tror du att tonåringarna som morsar på dig gör det för att de tycker att du är cool som är något år äldre? I själva verket är det tio år sedan Korn hade en hit och ungarna hälsar bara som en ironisk kommentar till den där tribalen du har på armen. Hur kunde det bli så här? När tappade du greppet? Svaret stavas: den dag du började bära gråmelerad T-shirt utan tryck. Bekväm, neutral, billig. Där och då tog du steget och nu kan du aldrig återvända. Glöm inte MedMera-kortet nästa gång du ska fylla på garderoben.

\ditem[Gubbafint]\label{gubbafint}
 En gubbafint är en klassisk handbollsfint, lika enkel som briljant. I anfallsposition låtsas spelaren helt enkelt gå förbi försvararen på ena sidan, men byter sen ben och går på andra. Den som behärskar gubbafinten kan även använda den på fritiden. Till skillnad från kampsporter är det nämligen tillåtet att bruka handbollsrörelser utanför planen.

\ditem[Gubbrock]\label{gubbrock}
 (av sv. 'gubbe' ung. \quotetext{äldre man} och eng. 'rock' ung. \quotetext{vaggande rörelse} el. \quotetext{klippa}) är rockmusik som spelas av gubbar och uppskattas av människor med tendenser mot gubbighet. Även om gubbrocken av naturliga skäl inte tillhör rockens mest vitala grenar är den dels en av de största inriktningarna inom rocken eftersom alla rockband som består av män förr eller senare börjar spela gubbrock. Dels är det den mest pondusfyllda rocken eftersom gubbar har spelat rock längst (bortsett från gummor, som kan ha spelat lika länge) och därför vet mest. Exempel på gubbrock är Ulf Lundell, John \quotetext{The fog\ref{the fog}} Fogertys soloprojekt, Roger Waters och David Gilmoures dito, Eldkvarn, AC/DC (som automatiskt blev gubbrock när den nya sångaren\ref{den nya saangaren} började), allt som någonsin har spelats på Droskan i Umeå, Rolling Stones samt allt som Chips Kiesbye någonsin tagit i. Motörhead är inte, och kommer aldrig att bli, gubbrock. Motörhead är arbetarklassrock\ref{arbetarklassrock}.

\ditem[Gubbsova]\label{gubbsova}
 Att gubbsova är att ta en liten tupplur, utan att för den skull sluta med den aktivitet man håller på med, t.ex. läsning eller datoranvändande. Sovställningen är nästan alltid oergonomisk då den intas i stundens hetta och inte efter noggrann planering. Anledningen till att det heter just \quotetext{gubbsova} är att det är populärt bland gamla människor, främst av hankön, s.k. \quotetext{gubbar}.

\ditem[Gubbsäker]\label{gubbsaeker}
 Att vara gubbsäker innebär att man kanaliserar själva essensen av att vara gubbe - en luttrad ovilja att bli påverkad av omvärlden och en utopisk önskan om att existera som en ö, avskiljd från samhället. Det kan handla om att bestämt förneka sanningshalten i vetenskapliga rön om att det är ohälsosamt att äta två paket basonfläsk och dricka fem old ox\ref{old ox} om dagen, att vägra sluta säga negerboll\ref{alternativa namn paa bakverk} eller att bestämma sig för att tatuera Betty Boop på axeln vid 55-års ålder - till synes bara för att jävlas. 

I diskussioner tar sig gubbsäkerhet uttryck genom ett tvångsmässigt behov att agera djävulens advokat. Först lite för att retas, men sen när man inser att verkligen ingen tycker som en själv, stå fast vid korkade åsikter som att hålla med folkpartiet\ref{folkpartiet}, hylla abortmotståndare, eller hävda att Metallica gjorde sig bäst på Load och Reload, till det bittra slutet. För att kontra gubbsäkerhet är det säkraste kortet att lägga ner diskussionen. Detta brukar åtföljas av en ursäkt en eller två dagar efter att diskussionen tagit plats. Kanske hade till exempel Olof Rudbeckius d.ä. då kunnat säga till sin vördnadsvärde gäst Jean-François Regnard: \textit{\quotetext{- Eh, det där jag sa om att det mytomspunna öriket Atlantis faktiskt är det samma som Sverige\ref{sverige}... Alltså fan, jag hade druckit tio old ox och var bara gubbsäker. Atlantis är nog bara en saga ändå, och även om det fanns var det inte Sverige.}}

Men dessa ord yttrades aldrig, vilket resulterade i ett av de mest passivt-aggressiva vänskapsförhållanden världen någonsin skådat. Men det är en annan historia.

\ditem[Gud]\label{gud}
 är redaktör till den bästsäljande antologin Bibeln\ref{bibeln}. Boken har skapat kontrovers då Gud själv författade flera kapitel i den tillsammans med spökskrivare och det anses svårt att utröna vilka kapitel Gud ursprungligen avsåg att ta med, flera reviderade utgåvor förekommer. Debatten har försvårats av att Gud väldigt sällan ger intervju.

\uline{Andra bedrifter}

Envisa rykten säger att Gud ska ha uppfunnit den moderna sjudagarsveckan.

\uline{Familj}

Gud är enligt de flesta pappa till Jesus\ref{jesus}, men vissa påstår att han har avsevärt fler barn än så.

\uline{Populärkulturella referenser}

Gud nämns vid namn i följande sånger:

\begin{itemize}
\item Ebba Grön - Häng gud
\item Mob 47 - Religion är hjärntvätt
\item Nick Cave \& the Bad Seeds - God is in the house
\item David Sandström Overdrive - The god thing
\item Eyehategod - My Name is God (I Hate You)
\item Hillsong - God is Great
\item Swans - Children of God
\item Om - Unitive Knowledge of the Godhead
\item Slayer - Disciple
\end{itemize}

\uline{Övrigt}

Ej att förväxla med Mob 47-Åke

\ditem[Gunborg]\label{gunborg}
 , förstår den som har lyssnat ordentligt på engelskalektionen, betyder \quotetext{kulsprutenäste} och är ett vanligt namn hos flickebarn till ivriga hemvärnsfrivilliga. Det vill säga till reaktionära bönder\ref{boonder}.

\ditem[Gurka]\label{gurka}
 En gurka är en grönsak som faktiskt är grön. Den består mest av vatten och smakar inte så mycket. EU har lagt ner ofantliga resurser på att reglera gurkodlandet i Europa.

\ditem[Gurkmajonnäs]\label{gurkmajonnaes}
 Efter att Jean-Francois Lyotard på 1970-talet skapat räksallad\ref{raeksallad} och hävdat postmodernismens intåg, kom på 1990-talet en replik från den filosofiska underdog-nationen Storbritannien. Det var socialdemokratins frälsare/Beelzebub Anthony Giddens som gick i polemik med Lyotards idéer i och med skapandet av gurkmajonnäsen, i boken \textit{Modernitetens följder}.

Giddens menar som bekant att vi idag lever i en högmodern samtid som bär drag av postmodernism, men ändå inte är fullt så postmodern som Lyotard påstår (de stora narrativens död, alla kategoriers upplösning etc). Därmed var räksalladen, postmodernismen fulländad, något som inte kunde stå obesvarat. Giddens, alltid villig att kompromissa, skapade således gurkmajonnäsen. Den är på ett sätt (modernistiskt) ärlig med sitt innehåll, då majonnäs skrivs fram tydligt i namnet. Men den innehåller också en (postmodern) lexikal orimlighet, då \quotetext{gurk} kommer före \quotetext{majonnäs} trots att gurka inte är huvudingrediensen. Således bär gurkmajonnäsen spår av både modernitet och postmodernitet, utan att direkt tillhöra någondera.

\ditem[Gurkvatten]\label{gurkvatten}
 I årtionden har gurkvatten varit en bestående del i det svenska cuisinet. Till vissas glädje. Till andras förtret. Här följer ett recept på gurkvatten. Skiva ca 1/3 gurka av EU-standardstorlek i lagom tunna skivor. Stoppa ner gurkskivorna i en karaff. Häll på iskallt vatten. Sen är det klart. Och meningslöst.

\ditem[Gustav Vasa]\label{gustav vasa}
 Enkel och praktisk frisyr. Låt håret växa så att det värmer öron och nacke. Klipp ett rakt snitt strax ovanför axlarna så att håret inte flottar ner din jacka (du tycker schampo är en myt). Klipp något kortare på framsidan så att du har fri sikt. Klart. Gjort på mindre än fem minuter och kräver inget gesällbrev för att få till.

\ditem[Gylfa]\label{gylfa}
 Att gylfa är att snatta med hjälp av gylfen. Förövaren stoppar byxorna i sockarna och stoppar sedan in prylar i gylfen. Denna metod fungerar skitdåligt med Cheap Monday-jeans och boxershorts.

\ditem[Gå och köpa tidningen]\label{gaa och koopa tidningen}
 Att gå och \quotetext{köpa tidningen} är något som svenska män ofta gör på självaste julafton. Barnen tror att fadern gör just detta, medans frugan tror att de ska gå ut i garaget, ta på en tomtemask och komma tillbaka med julklappar. I själva verket går de över till sin andra familj för att fira julen där. Handlingen upprepas där för att gå tillbaka till ursprungsfamiljen, och sådär håller det på.

\ditem[Gå och snickra]\label{gaa och snickra}
 Kod för att gå ut i snickarboa och dricka ur en kvarting som man har gömd i ett kikarfodral. Populärt under släktmiddagar och andra krävande situationer.

\ditem[Göra rätt för sig]\label{goora raett foor sig}
 Att göra rätt för sig är att alltid göra saker som man själv inte har någon glädje av, och att undvika sådant som att studera ämnen som kan anses vara oviktiga men intressanta. Bland saker man inte ska studera ingår alla humanistiska ämnen. Man måste tjäna sitt levebröd på sådant som inte är roligt. Alltså kan man inte samtidigt göra rätt för sig och arbeta som mångsysslarpensionär\ref{maangsysslarpensionaer}, barberare och så vidare. Det säkraste sättet för den som vill göra rätt för sig är att ta en sådan anställning genom vilken man kan bli medlem av ett fackförbund anslutet till Landsorganisationen, LO. Är man det är det i princip omöjligt att inte göra rätt för sig och då är man också värd päronhalva\ref{paeronhalva} till efterrätt. Ett bra steg på vägen mot att göra rätt för sig är att ta för vana att alltid använda handjagare\ref{handjagare}.

\uline{Att göra rätt för sig i litteraturen}

\textit{Storstrejken? Ingen har några minnen av något speciellt vad gäller Josefina Markström. Hon ogillade den väl, naturligtvis. Hon ansåg att alla skulle göra rätt för sig. Jo, ett kritiskt yttrande från henne om Bolagsledningen: He\ref{he} var int rätt. Vad var inte rätt? Att sänka löningen med tjugofem procent. Då gjorde inte arbetsgivaren rätt för sig. Men en skall så, en annan skall skörda, tillade hon gåtfullt}

\textit{Musikanternas uttåg} av P.O Enquist

\ditem[Göran]\label{gooran}
 är ett klassiskt svenskt namn som betyder \quotetext{lägga till båten}.

\ditem[Göras till åtlöje inför hela svenska folket]\label{gooras till aatlooje infoor hela svenska folket}
 Att göras till åtlöje inför hela svenska folket är det faktiskt inte många som varit med om, och som med allt annat som är exklusivt ska naturligtvis kungahuset ta för sig\ref{ta foor sig} även av detta. Kungen har till och med skapat en tradition av att på nyårsafton göra bort sig inför kanske inte precis alla svenskar, men en mycket stor del av dem i alla fall.

\uline{Folk som gjorts till åtlöje inför hela svenska folket}

Trots det exklusiva elementet i denna aktivitet finns det ett och annat driftkucku på denna lista, som omfattar:

\begin{itemize}
\item Fotbollslandslaget (herrar)
\item Hon som spydde i direktsändning på ZTV
\item Leila K
\item Junilistan
\end{itemize}

Skall ej förväxlas med Rikspucko även om likheterna är många. Ett rikspucko kan medvetet offra sin själ för att få uppmärksamhet. Här återfinner vi karaktärer som Robinson-robban, Alex Schulman m.fl.

\ditem[Gösta]\label{goosta}
 betyder tjuvfiskare. Det var länge ett rent nidingsnamn men nu kan man ju heta lite vad som helst.

\ditem[Gösta Snoddas Nordgren]\label{goosta snoddas nordgren}
 (1926-1981) var en framgångsrik bandyspelare som bland annat lirade med Bollnäs GIF och det svenska landslaget. Han var också Sveriges\ref{sverige} första, och mest framgångsrika shockrockare\ref{shockrockare}, spred kaos och orsakade hjärtstillestånd i folkparkerna med sin syndiga och djävulsdyrkande refräng \quotetext{haderian hadera} i låten \textit{Flottarkärlek}. \quotetext{Snoddas} ska enligt sägnen ha segnat ner död under en innebandymatch han spelade med handikappade ungdomar på ett sjukhus i Vänersborg. Passande, på nåt vis.

\ditem[Göteborg]\label{gooteborg}
 , också känt som \quotetext{Lilla London}, är rikets\ref{sverige} andra stad. Här uppfanns den moderna ordvitsen och klasshatet. Många är de artister som besjungit staden och mest känd är förmodligen Troublemakers dänga \textit{Staden Göteborg}.

%%%%%%%%%%%%%%
\newpage
\null
\\
\null
\\
\Huge
H
\normalsize
\\
\null
\\
\null
%%%%%%%%%%%%%%


\ditem[Ha bärs]\label{ha baers}
 Att ha bärs är det enklaste sättet att försäkra sig om en trivsam kväll/eftermiddag/älgjakt/skiftlagsfest/etc. På Systembolaget kan man köpa nästan hur många bärs man vill, men ändå händer det ibland att man har slut när man vill ha en. Då måste man skaffa bärs på annat sätt och det lättaste är att fråga sig runt. Det låter ungefär så här:

-\quotetext{Öhh, ha... haru bärs?}
Eller:
-\quotetext{Haru bärs!?!! Ja får en va? ...du! ...amen du!}

\uline{Kända personer som har eller har haft mycket bärs}

\begin{itemize}
\item Jens Spendrup
\item Torsten Flink
\item Homer Simpson
\end{itemize}

\ditem[Habbadixen!]\label{habbadixen!}
 Habbadixen är en interjektion som används vid bevittnande av sport på TV för att få en idrottsutövare att misslyckas med sitt företagande. Det uppfanns mig veterligen av min morfar, Anders Selldén, under en tennismatch då trollformeln ledde till flera dubbelfel. Det har sedan dess med varierande grad av funktion använts vid olika idrotter, exempelvis skidskytte och dart.

Exempel:
\quotetext{Nu ska jag habbadixa Federer så att han gör ett dubbelfel; HABBADIXEN!}

\ditem[Hacka]\label{hacka}
 En hacka är ett spetsigt redskap som används för att gröpa ett hål i något; till exempel en bergvägg, en sjöis eller en fiende i kampen om makten i kommunistpartiet. Liksom många andra av människan uppfunna redskap kopierar hackan en teknik som redan finns utvecklad i djurriket. Hackspetten har, ända sedan den levde på havsbottnen, använt sin näbb till att picka hål i saker och äta upp vad som finns däri. Den första prototypen till den mänskliga hackan togs fram på medeltiden\ref{medeltiden} och bestod av en vass sten som stoppades in i munnen. Den fungerade inget vidare. Det verkliga genomslaget kom först i och med uppfinnandet av träskaftet. I senare tid kan hacka anspela på en icke ansenlig summa pengar, ibland förtjänad utom riksskatteverkets kontroll.

\ditem[Haiku]\label{haiku}
 är en versform som är populär inom japansk poesi, men som har sitt ursprung i nordkinesisk poesi från tang-dynastins dagar. Haikun består av tre rader med två olika versmeter, den ena (rad ett och tre) på fem stavelser och den andra (rad två) på sju. Haikun är normalt avdelad i två led - rad ett och två bildar ett led och rad tre ett. Ofta byggs en viss stämning eller scen upp i den första raden, medan den andra introducerar ett nytt element i dikten, ofta som en oväntad förändring som på ett omedelbart och till och med dramatiskt vis kastar om den bild som läsaren har skapat i sitt inre. I någon av leden förekommer traditionellt en så kallad \textit{kigo}, det vill säga ett ord som på ett mer eller mindre subtilt vis situationerar dikten i en av årstiderna. I modern Haiku förkommer dock dikter utan kigo minst lika ofta som dikter med. Kigon kan vara en direkt referens till året, så som en månad, helgdag, festival eller annan speciell dag. Den kan också vara den mer subtil referens, så som närvaron av en viss, säsongsbunden växt (så som körsbärsblom), en referens till färgen av ett löv, immande andedräkt, närvaron av eller sången hos en säsongsbunden fågel och så vidare. Den medeltida Japanska diktaren Basho framhålls generellt av poesihistoriker och haiku-älskare som den mest genialiske diktaren i haikuns långa historia.

\uline{Exempel på Haikudikter med kigo}

Farsan, i gummistövlar,
står på gårdsbacken och svär -
packad och förkyld

Isblandat störtregn
har gjort jumpapåsen blöt -
Friluftsdag i mars

Ingen vill komma
på min födelsedagsfest -
För då är det påsk\ref{paask}

Det luktar fiskrens
ur gubbens mörka munhål-
Svanarna flyttar

\uline{Exempel på Haikudikter utan kigo}

Farsan och farfar
drämmer näven i bordet -
ryter åt farmor

Brevbärarens fru
Blir kanske inte så glad nu -
För han ser död ut

Halkar på död fisk
och slår huvet i stenen -
Farmor skrattar glatt

Jag är förlamad
ända från midjan och ner
men ingen bryr sig.

\ditem[Haile Selassie]\label{haile selassie}
 bör främst kommas ihåg för sin långa titel:

\quotetext{Hans kejserliga majestät, kejsare Haile Selassie I, Lejonet av Juda, Guds utvalde, konungarnas konung av Etiopien.}

Selassies gamla flygplan stod länge vid flygfältet i Kärrgruvan men är i dagsläget nerbrunnet. Hans gravplats är numer en allmän toalett\ref{ornaesstugans dass}.

\ditem[Haka (vanlig)]\label{haka (vanlig)}
 En vanlig haka är en kroppsdel som är lite spetsig eller lätt rundad och återfinns allra längst ner och ut på ansiktet. Dess funktion är inte klart definierad, men på män kan den fungera som fästpunkt för ett pipskägg. Ovanför hakan finns munnen.

\ditem[Hakkors vi minns]\label{hakkors vi minns}
 De landmärken som hjälpt vagabonder på vägarna att hitta skydd under en dragig ladas tak en regnig sommarnatt. Liksom forna tiders luffarristningar. På en klippväg, eller var det en stor sten, minnet sviker, där stod den. Som en Bergslagens Tanums hede, sin mening höljd i dunkel: svastikan följd av texten \quotetext{SAAB}. Som så mycket annat gick just denna svastika ett grymt öde tillmöte - den mötesfria vägen - nollvisionen. Kommer människan någonsin igen kunna orientera sig mellan Norberg och Avesta?

\ditem[Halland]\label{halland}
är enligt flera källor Sveriges\ref{sverige} minst nödvändiga landskap. Geografiskt ligger det ivägen mellan Malmö och Göteborg, så utan Halland skulle det bara ta någon enstaka timme att köra mellan dessa, vilket både ekonomin och miljön skulle ha tjänat på. Kulturellt har Halland skänkt oss artister som Per Gessle och Basshunter och inom politiken var det här Carl Bildt föddes och tog sina första bestämda steg mot att slå sönder allt vad välfärd heter. Som läsaren märker finns det inte mycket gott att säga om Halland och det bästa vore kanske att bara asfaltera hela skiten och sedan gå vidare med våra liv. Just nu finns inget politiskt parti som driver frågan, men en motion är inskickad till Sjöbopartiet.

\ditem[Ham]\label{ham}
  född juli 1959 i Kamerun, död 19 januari 1983 på North Carolina zoo, var den första hominida apan i rymden. Två år gammal flög Ham till rymden i 16 minuter och 39 sekunder. Bedriften kan te sig ganska klen om man exempelvis jämför med Belka\ref{belka}, fast det här var ju å andra sidan USA. Ham blev stor stjärna när han återvänt och spelade bland annat in film med Evel Knievel.

\ditem[Hamlet]\label{hamlet}
 är en pjäs av författaren och flintskallen William Shakespeare, Storbritanniens svar på Astrid Lindgren.

\uline{Synopsis}

Hamlet är prins av Danmark\ref{danmark}, det är medeltid\ref{medeltiden}. Han tycker inte att tillvaron är sådär jättesoft. Hamlets pappa har nyss dött och återvänder som spöke för att berätta för Hamlet att han blivit mördad av sin bror Claudius. Med det namnet kan man tycka att prins Hamlet borde ha listat ut att Claudius är en skurk, men tydligen ska hans farsa behöva återvända från andra sidan graven för att upplysa honom om detta. Ungdomar alltså, nåväl. Claudius har alltså dödat sin bror kungen och gift sig med Hamlets morsa Gertrude för att således bli kung själv. Förr i tiden krävdes det inte mer för att få den yttresta maktpositionen, jämte Gud fader, i ett land. Sjukt. Hamlet svär på sin fars skalle att hämnas denne och döda Claudius. Det hela blir dock rörigt tack vare Polonius, Claudius rådgivare, och far till Ofelia som Hamlet kärat ner sig i. Polonius försöker få reda på vad Hamlet har i kikaren genom att gömma sig bakom en gardin och tjyvlyssna när han och Ofelia pratar. Hamlet får för sig att det är Claudius skor som sticker fram där under gardinen och sticker ner honom med sin värja. Polonius dör, Ofelia och hennes bror Laertes svär att hämnas. Man skulle kunna bli filosofisk och kalla det en metahämnd, men det är inte säkert att det är så. Claudius övertalar Laertes att han och Hamlet ska duellera, och genom att förgifta Laertes värja ska Polonius jämna ut oddsen. Under duellen utbryter ett sånt där rökmoln som när de slåss i tecknade filmer och Hamlet lyckas på något vis sticka Laertes med hans egen förgiftade värja. Med sina sista andetag berättar Laertes om Claudius plan, varpå Hamlet dödar Claudius, bara så där. Sen dör Hamlet också av nån anledning. Sedan är pjäsen slut. Nej, justja, Gertrude dör också mitt i allt under oklara omständigheter. Så. Nu är det slut.

\ditem[Handjagare]\label{handjagare}
 är sådana verktyg och redskap som endast drivs av handkraft. Är man en riktig, redig karl/kvinna och ingen Stockholmsborgare använder man handjagare, så är det bara. Medan skruvdragare och elhyvlar inte vill starta bara för att man glömt att ladda batteriet, som man också tappat bort, sviker handjagaren en inte förrän den gått sönder och då är det bara för att någon annan har använt den på fel vis eller låtit den ligga ute och bli rostig. När man använder en handjagare \quotetext{handjagar} man och detta leder ofta till att man får indianmuskler\ref{indianmuskler}.

Husqvarna fick mycket kritik då de inrättade särskilda läger dit handjagande maskiner forslades för att användas tills de gick sönder. I efterhand har det visat sig att deras plan var att deras motordrivna diton sedan skulle regera över hushåll och trädgårdar världen över. Vissa handjagarredskap monterades t.om. medvetet ned, för att delarna sedan skulle användas som komponenter till Husqvarnas egna maskiner.

Exempel på handjagare är spade, yxa, fogsvans, hammare, skruvmejsel och hacka\ref{hacka}.

\ditem[Handskfack]\label{handskfack}
 är ett utrymme i bilar som fungerar precis som bakluckan, fast inne i kupén och i mindre skala. Där förvarar man kartor över Gotland, skruvar som lossnat från interiören, ett avmagnetiserat kassettband med Mikis Theodorakis\ref{mikis theodorakis}, eltejp, sand\ref{sand}, nyckelringar, isskrapa halvtäckt av smält kexchocklad, fiskeflöten och kvitton. Hos rutinerade bilägare ligger där också en rostig morakniv och ett par repiga läsglasögon. Allt detta är till för att förenkla förarens möjligheter att ratta fordonet på ett säkert sätt. I finare bilar, som till exempel Volvo 740\ref{volvo 740}, har handsfackets lucka försetts med en uppfällbar sminkspegel och en jättedålig mugghållare, så att även den som sitter shot gun kan nyttja fackets praktiska möjligheter.

\ditem[Handvass]\label{handvass}
 Ett dialektalt ord från Västerbotten\ref{vaesterbotten} som signalerar att en person är jävligt stark i nyporna. Handvassa personer lyckas alltid öppna syltburkar och kan vrida upp en vanlig kapsyl som vore det en skruvkapsyl\ref{skruvkapsylool}.

\ditem[Hardware]\label{hardware}
 är en post-apokalyptisk skräckfilm från 1990. Den handlar om en kille som går på skrotloppis och köper en gammal häftig robotdel åt sin konstnärliga flickvän som gärna bygger skulpturer av gamla robotdelar och skrot. Det visar sig att denna robotdel är en självreparerande och väl fungerande bit metall som snabbt bestämmer sig för att börja mörda allt levande i sin omgivning. Sen har konstnärstjejen en granne som fulonanerar och tittar på henne genom fönstret med en kikare\ref{kikare}.

Filmen hör till sub-genren skräckfilm med självreparerande robotar som skräckmoment. Den påminner i upplägget om filmen Virus som handlar om en båt där en självreparerande robot av misstag kraschat i sin resa till jorden för att utplåna mänskligheten innan vi hinner påbörja vår virusliknande spridning ut i rymden.

\ditem[Hasch]\label{hasch}
 (Fornpersiska, uttalas som Ernst-Hugo Järdegård skulle ha uttalat det) är ett slags brun koda från mellanöstern. Den är väldigt förbjudet i Sverige\ref{sverige} eftersom man blir så fnissig och lite pratig av att röka den. Som tur är kan man hinka i sig brännvin\ref{braennvin}, köra motorbåt och åsamka en gammal tant hjärnskakning genom att slå henne över huvudet med en metallstång under en redig spritfylla\ref{spritfylla} istället, och på så vis åka i fängelse den lagliga vägen.

\ditem[Hasselbackspotatis]\label{hasselbackspotatis}
 En stackars bintje\ref{bintje} eller King Edward\ref{king edward} som någon sadist hackat upp till hälften för sitt eget höga nöjes skull.

\ditem[Havsmunk]\label{havsmunk}
 (\textit{Monachus monachus}) är en sälart\ref{saelar} som främst brukar ligga och glassa\ref{glasse} i Medelhavet och östra Atlanten. Den kan väga nästan 300 kg och bli uppåt 30 år. Tyvärr blir de flesta havsmunkar bara någon månad gamla för naturen har försett arten med den rätt korkade vanan att föda ungarna i grottor där ingången ligger under havsytan. Vid höga vattennivåer översvämmas sådana grottor ganska ofta och familjen drunknar. Hårt och inte särskilt rättvist. Till skillnad från många andra sälar saknar havsmunken faktiskt öron.

\ditem[Hawaii-pizza]\label{hawaii-pizza}
 tar dig med till en tropisk värld där livet är enklare, gladare och där ljumma nätter vibrerar av exotisk sensualitet. När man får sin Hawaii-pizza serverad är det därför bara naturligt att för en stund sluta sina ögon samtidigt som man drar in doften av fett och mjuk ananas genom näsborrarna och drömma sig bort till orangea solnedgångar, det avlägsna ljudet av oljefatstrummor, stålsträngad gitarr\ref{gitarr} och ljusa kvinnostämmor, till söta frukter och lekande delfiner\ref{delfin}, till bar hud och djupa suckar.

Av förklarliga anledningar är hawaii-pizzan populär som bakfyllekäk efter en redig vinfylla\ref{vinfylla}.

\ditem[He]\label{he}
 (verb, infintiv) är ett ord på västerbottnisk bondska. Det är ett ersättningsord för alla verb som kretsar kring att placera ting på olika platser, och mycket, mycket mer. Det böjs He Hedde Hett.

\uline{Exempel (Inom parentes översättning till rikssvenska):}

\begin{itemize}
\item He på lyset! (Slå igång den elektriska inomhusbelysningen!)
\item He igång biln'! (Starta automobilen!)\ref{bil}
\item He de hem! (Gå hem!)
\item He de borda hygget! (Förflytta dig ifrån detta nyligen avverkade skogsområde)
\item He int schwartmyren i ryggsäcka! (Stoppa inte svartmyror i min ryggsäck)
\item He på nå pären! (Var god börja koka potatis.)
\item Han derna Reinfeldt\ref{fredrik reinfeldt} skull ha va hedd bort för länge sen. (Vår statsminister, Fredrik Reinfeldt, borde ha avgått för länge sedan)
\item Nä, om man sku ha hett se hemåt vägen. (Nä om man skulle ha tagit och gått hem)
\item N' Holger ha hett se åt Missenträsk. (Holger har begett sig till Missenträsk)
\end{itemize}

En dialog mellan två västerbottniska potatisbönder kan således se ut så här:

-\quotetext{Ha du hett ner nå King Edward\ref{king edward} i pärbänka i år?}
-\quotetext{Nä, ja hedde bara ner Bintje\ref{bintje}.}
-\quotetext{Tro du he vall nå bra?}
-\quotetext{Nä fan heller.}

Märk väl att he inte är ett verb i den tredje strofen, utan en lokal variant för ordet \quotetext{det}.

\uline{Övriga betydelser}

He kan också vara det onomatopoetiska ljud som kommer efter att en superskurk avslöjat sin diaboliska plan. Till exempel:

\textit{-You see, I have a gift. An instinct for sensing people's weaknesses. Yours is women. Hers and mine are winning, whatever the cost. So when I arranged for that fatal overdose for the true victor at Sydney, I won myself my very own MI6 agent, using everthing at my disposal - her brains, her talent, even her sex, he he he!} (Gustav Graves i \textit{Die another day})

I engelskan betyder \textit{he} \quotetext{han}. Exempel, \quotetext{He dislikes trousers} (Sv. Han ogillar byxor).

\ditem[Headbanga]\label{headbanga}
 Att headbanga är att våldsamt nicka med huvudet i takt med hårdrocksmusik\ref{haardrock}. Vad nickandet signalerar är ett intensivt medhåll från lyssnarens sida. Vid varje takt nickar lyssnaren, som för att säga till artisten, \textit{ja det där var bra, och det där, och det där också, och den där med, och oj den var fräck, och den, och den, och den, och även det taktslaget fångade mitt intresse, och det, och det, bra så, bra så, japp, japp, japp} osv.

Är man riktigt intresserad av att visa sin uppskattning för artisten står man längst fram och är en front row banger\ref{front row banger}.

\ditem[Hefaistos]\label{hefaistos}
 var smidets gud\ref{gud} i det gamla Grekland. När han föddes tyckte hans morsa att han var så vanskapt att hon kastade honom från Olympen, han föll i nio dygn och landade i havet. Fatta att ramla i nio jävla dygn. Jag tror inte att det finns någon aktivitet som är rolig i nio dygn - allra minst att ramla.

\ditem[Hegemoni]\label{hegemoni}
 Enligt Antonio Gramsci, som lanserade begreppet i politisk bemärkelse, handlar hegemoni om idéer som har en dominant ställning i samhället. Dessa hegemoniska idéer ses som sunt förnuft, eller av naturen givna sanningar i samhället. Hegemoni är inom marxismen ett verktyg för de med makt att skapa samtycke om att det rådande samhället är det bästa. Men, eftersom idéerna är kulturella konstruktioner och inte fysiska institutioner (även om de för all del kan ta sig uttryck i sådana), kan man attackera de rådande idéerna utan att bli skickad i fängelse eller bli skjuten i huvudet av en snut, vilket händer om du attackerar statens fysiska institutioner. Nedan presenteras en kortare lista över konstruerade sanningar som måste brytas ner för att åskådliggöra verklighetens egentliga beskaffenhet.

\begin{itemize}
\item Det är viktigt att ta för sig
\item Ditt liv är speciellt
\item Cadillac är finare än volvo
\item Att organisera sig fackligt leder inte till något
\item Det är finare med chorizo än falukorv
\item USA\ref{united states of america} är coolare än Tyskland\ref{tyskland}
\item Om du jobbar tillräckligt hårt kan du bli vad du vill
\item Metallica är världens bästa band
\end{itemize}

\ditem[Helgarderat efterliv]\label{helgarderat efterliv}
 Vill man vara säker på att ha något att göra efter döden så är det bäst att helgardera. Ingen vet vilken av våra fem världsreligioner som har rätt så därför är det bra att inte stryka någon av dem mothårs.

\uline{Kristendomen}

Vi börjar med kristendomen och då kan en bra idé vara att följa de tio budorden. Det första budordet \quotetext{Du skall icke ha andra gudar jämte mig} blir i denna hypotes lite knepigt att efterfölja, men om man ändå inte tror på någon av religionerna så har man trots allt ingen gud jämte någon annan. Ateism är alltså här att föredra, hur absurt det än låter. De andra budorden är väl inte helt omöjliga att efterfölja, hålla sig i skinnet och softa på söndagar\ref{soondag}. Det här med att man inte får stjäla får väl ses i perspektiv\ref{perspektiv} mot hur mycket borgarna stjäl av oss hela tiden. Om ni måste svära så se till att göra det ordentligt och inte säga \quotetext{Herregud!\ref{gud}} och annat trams.

\uline{Judendomen}

I egenskap av abrahamitisk religion så har judendomen också tio bud, men dessa kompletteras av hela 603(!) bud till. Tack och lov så handlar precis vart enda av dessa 603 om hur och vilken mat som man får äta. Alla matvaror som behandlas av dessa bud handlar om djur så vegetarianism är en väldigt enkel väg att gå, men mer om det senare.

\uline{Islam}

I dagens samhälle är det svårt att be en massa gånger om dagen, så det är bäst att slå på stort. Ta en semester till Mekka och delta i circle piten runt kaban så kommer nog Allah tycka att du är en helt ok lirare. Vill man verkligen vara på den säkra sidan kan man uppfylla de tre andra pelarna också. Allmosan är inga problem om man bor i ett land där man betalar skatt som går till samhällets utsatta. Än så länge gör vi det i Sverige\ref{sverige}, men se upp så att det inte blir blåval\ref{blaaval} igen. Fastan låter väl inte jätterolig, men tänk på att man får äta på natten. Se till att vända på dygnet under Ramadan. Trosbekännelsen blir knepig om man är ute efter att vara nere med alla religioner, så kompromissen får bli att man skriver den på en lapp och stoppar i bakfickan\ref{bakficka}. Den sista pelaren är bönen och den hinns som sagt inte med i dagens samhälle, det får baske mig Allah ha överseende med.

\uline{Hinduismen}

Den här religionen fungerar extremt bra att kombinera med de andra, då den faktiskt inte menar att den är överlägsen någon annan, väldigt sympatiskt. Däremot har den inget efterliv i samma mening som de abrahamitiska religionerna, utan man återföds. Dock kan man återfödas som olika saker som ändå får lov att ordnas hierarkiskt med kloakdjur\ref{kloakdjur} längst ner, gulsvansad ullapa någonstans i mitten och uv\ref{uv} längst upp. Det ultimata ska tydligen vara att inte återfödas alls, men det låter ju ta mig tusan helt befängt. Vill man slippa ett liv i träck så kan man ta till sig följande regel: Kor är heliga, så vegetarianism är fortfarande ingen dum idé. Sen har vi konceptet karma som kort går ut på att om man beter sig extremt osoft så lutar det åt att man blir ett kloakdjur\ref{kloakdjur}. Följer man budorden från de abrahmitiska religionerna så borde man redan ha ganska bra karma, så det är lugnt.

\uline{Buddhism}

De två österländska religionerna är ganska lika så genom att vara nere med hinduismen så får man buddhism på köpet. För att vara störtsäker kan man ta till sig de fyra sanningarna: Det finns osofta företeelser, anledningen till dessa är begär, begär kan och måste tas bort och detta görs genom den åttafaldiga vägen. Till vårt försvar så är det väldigt lätt att gå fel på en åttafaldig väg så det sista får nog bero.

\uline{Slutsats}

Fixar ni det här så kommer ni inte hamna i helvetet eller återfödas som något obehagligt. Är man däremot ute efter ett drägligt jordeliv rekommenderas en marxistisk tolkning av världen.

\ditem[Helgvolym]\label{helgvolym}
 Uppskattningsvis 25 \% högre än vardagsvolym.

\ditem[Helvetet]\label{helvetet}
 är en plats som det finns många teorier kring. Den knastrande croissangen Jean Paul Sartre menade att helvetet var andra människor. Men det är det inte alls. Den italienske författaren Dante Alighieri var god nog att efter en resa till platsen i fråga skissa upp vad helvetet är och vad som finns där, i en av delarna i sin bokserie Den gudomliga komedin.

Enligt Dante ligger helvetet under marken. Antagligen börjar helvetet efter jordlagren där man i genomskärning kan se skattkistor och dinosauriefossiler ligga och vila. Helvetet består av nio cirklar. Den yttersta cirkeln heter Limbo. Där hamnar alla som i Guds ögon är ganska snälla men inte kristna - så kallade nobla vildar. Stora personligheter som Orpheus, de sju dvärgarna i snövit och Don Dokken promenerar runt där nere, lätt konfunderade över vad fan de håller på med och när de får åka därifrån. Svaret är aldrig.

Den andra cirkeln är den plats där de som syndat genom Lust befinner sig. Cleopatra och Paul Stanley (vid frånfälle) huserar i den obekväma miljön. Antagligen försedda med kyskhetsbälte, bara för att jävlas.

I cirkel tre finner vi frossarna. Theoden ståthållare av Gondor och Micke Dubois drar fräckisar där i en evighet.

I fyran hamnar alla giriga. Joakim von Anka och Guldivar Flinthjärta är i cirkelns centrum, locked in eternal combat, som de säger i engelskspråkiga länder.

\quotetext{På femman} som de säger där, är de arga. Alla som någonsin rage-quittat efter en keff omgång Starcraft sitter där. Många fjortonåringar födda på 90-talet vandra dessa dystra hallar.

Sjätte cirkeln inhyser kättare. Euronymous och Karl Marx hinkar bärs under ett plommonträd.

Cirkel sju är avsedd för våldsverkare. Hulk Hogan och Henry Kissinger går oändliga mängder ronder.
I åttan är alla bluffmakare. Där sitter Göran Persson för att ha kommunaliserat skolan.

Och i den innersta nionde cirkeln sitter satan själv, med Judas Iscariot och andra renommerade förrädare som Jussi och Grima Ormstunga.

Och det är Helvetet.

\ditem[Hemliga koder]\label{hemliga koder}
 är sätt för slutna sällskap att kommunicera med varandra utan att utomstående förstår vad det är frågan om. Stonerskins\ref{stonerskin} använder sig till exempel av frisyrer hämtade från \textit{Dödligt vapen}-serien för att signalera tillhörighet. För en oinvigd ser det bara ut som ett märkligt sammanträffande\ref{maerkliga sammantraeffanden} men fyller alltså ett djupare syfte. Om du ser en svensk stålåsna\ref{volvo 740} försedd med en bildekal\ref{bildekal} innehållandes gamla sosserosen vet du, om du känner till hemliga koder, att det sitter en av Konsums\ref{konsumbutik} stolta ägare\ref{aegmaestare} bakom ratten. \quotetext{Vad roligt!}, tänker du säkert nu, \quotetext{om jag lär mig dessa kryptiska signaler kan jag skaffa mig nya kamrater}. Visst kan du det, MEN tänk på att också fienden använder samma taktik. Om du till exempel möter en medelålders man på flerväxlad cykel\ref{flervaexlad cykel} är han med stor säkerhet medlem av någon illuminatiliknande sekt, storfräsare\ref{storfraesare} eller ICA-personal i förklädnad.

\uline{Andra vanligt förekommande koder}

\begin{itemize}
\item Om du ser den gamla sosserosen i något annat sammanhang betyder det att personen har stomipåse.
\item Om du ser en regnbåge är det förmodligen en leprechaun som signalerar till sina kumpaner att hen har grävt ner en ny kruka med guld. 
\item Om du är på PRO-möte och någon plockar ut löständerna och börjar putsa på guldtanden vet alla invigna att det blir päronhalva\ref{paeronhalva} efter kaffet så det gäller att tömma stomipåsen innan.
\end{itemize}

\ditem[Hemmansvärde]\label{hemmansvaerde}
 Ett hemmansvärde är en diffus måttenhet populär i Västerbotten\ref{vaesterbotten}. Den åsyftar värdet på hemmanet, alltså familjens gård, komplett med kräk, fuse och loge.

Exempel:

\quotetext{Ska du köpa ny kaffebryggare igen?}
\quotetext{Ja det är ju inget hemmansvärde direkt.}

\ditem[Hemmets Härold]\label{hemmets haerold}
 är Pingstkyrkans skivbolag som faktiskt funnits i nästan hundra år. Utan Hemmets Härold skulle landets begagnatskivbutiker bara ha hälften så många sumpiga femkronorsbackar som står och tar plats helt i onödan. Från 1938 till 2007 var Svante Widén från Avesta producent på de flesta av bolagets inspelningar. Samarbetet avbröts dock när Svante dog.

\uline{Klassiska namn på Hemmets Härold}

\begin{itemize}
\item Pelle Karlsson\ref{pelle karlsson}
\item Gitarrbröderna Värnamo
\item Ny-David
\item Pärla
\end{itemize}

\ditem[Herrkläder]\label{herrklaeder}
 är en gren på mode-rikets brokiga träd. Inom gruppen herrkläder ingår sådana plagg som den modemedvetne i första hand associerar med mannen, men som mycket väl också kan bäras av andra, till exempel Sven-Otto \quotetext{Dangerzone2010} Littorin. Typiska herrklädesplagg är läderväst, belt-buckle, boots, skogshuggarskjorta, gummistövlar (utan färgglatt mönster), kepsar med olika företagslogotyper\ref{kepsar med olika fooretagslogotyper}, speedos\ref{speedos} och ett par skitiga jeans.

Utlottning av herrkläder brukar kombineras med älgköttsoppa och snapsvisor vid fester i hembygdsgårdar.

\uline{Önskar du uppnå succé genom att klä dig i herrkläder?}

Då har du ett svårt dilemma framför dig, för detta är nämligen en självmotsägande paradox. Om du medvetet vill uppnå succé genom att klä dig så har du redan gjort det omöjligt att lyckas med ditt företag. Om du däremot slänger på dig det du hittar i tvättkorgen, bakar en snus och kliver ut i världen kommer du, med lite tur, att anses vara nonchalant maskulin så som till exempel Bon Scott, Thåström och Stor-Anders\ref{stor-anders}.

\ditem[Hertigen av Rothesay]\label{hertigen av rothesay}
 är prins Charles hovtitel i Skottland. I England kallas han \quotetext{prins av Wales}, men det tycker skottarna är att ta i lite väl. Charles bästa ovän är hans mamma som aldrig vill dö.

\ditem[Hertsökullar]\label{hertsookullar}
 är ett par högar av typ bajs och lera på bostadsområdet Hertsön i Luleå. På Hertsökullar är de populäraste aktiviteterna på vintern pulkaåkning och hundrastande medan sommarens aktiviteter består av haschrökande\ref{hasch} och att tända eld på grejer. Från Hertsökullars högsta topp cirka 2\ref{tvaaa} meter över havet kan man se så långt som till SSAB ibland.

\ditem[Hillevi]\label{hillevi}
 är ett svensk kvinnonamn som betyder ungefär \quotetext{Joe Hills minne ska alltid leva}.

\ditem[Hip-hop]\label{hip-hop}
 är en musikstil som är minst lika bred och mångfacetterad som doom, men handlar till skillnad från doomens symbolekonomi av himlakroppar och mörk tomhet om damer med stora behag samt pengar\ref{valuta}.

\uline{Hiphop och samhället}

Hip-hop är främst populär hos medelklassen som tycker att genren är \quotetext{genuin}. Överklassen är som vanligt helt världsfrånvänd och har inte hört talas om den.

\uline{Hiphopens utveckling}

Hip-hopens fader är, som alla vet, Evert Taube med sin låt \quotetext{Den kinesiska muren\ref{kinesiska muren}}. Höjdpunkten inom genren nåddes dock 1992 av den amerikanska duon Kris Kross med låten \quotetext{Jump}.

\ditem[Hippie]\label{hippie} 
 kort för hipster. Vard.: ett nedsättande ord för allehanda träd-, varg- och kattkramare. På 1920-talet då hela hippievågen startade var det hippt med arier. På den vitala delen av kontinenten samlades entusiaster i små grupper som gemensamt benämndes \quotetext{Völkisch-}. Det hela slutade i ett abrupt nederlag efter ett kort men maffigt hippie-crescendo. Förgrundsfigurerna är fortfarande populära i hippie-kretsar med namn som Adolf Hitler och Madame Blavatsky. Den första hippien med medial genomslagskraft var onekligen ovan nämnde Hitler. Gemensamt för hippies av både igår och idag är den bottenlösa relativismen. Ofta ingår en vurm för nakenhet och ungdomlighet. En illa dold fallenhet för blodsmystik, naturlighet och renhet är heller inte okänd. Hippierörelsen fick under 60- och 70-talen en nytt uppsving då under devisen \quotetext{All you need is love}. Stora tänkare från denna andra våg är Charles Manson och John Lennon vars tankar korsbefruktades intensivt. Vad som börjar med att unga pojkar brottas nakna på scout-liknande läger slutar inte sällan någonstans i Sydamerika med hopp om att sprida den germanska rasen medels incest. Den s.k gröna vågen kan sägas vara en syntes av dessa två första generationer av hippies. Storstadsbor en masse flyttade ut på \quotetext{landet} och rollspelade lokalbefolkning.

Ett annat typiskt särdrag för hippies är den totala handfallenheten. Att man inte kan någonting döljs genom den spindelnätsliknande tankevärlden där \quotetext{rätt} och \quotetext{fel} inte existerar. Hellre än bra tycks vara parollen. Detta gör att hippies ofta återfinns i pseudoområden: exempelvis reklam och design. Här syns ständigt återkopplingen till den tidiga hippierörelsens frontfigurer, Göbbels och Hitler. Hitlers banbrytande gärning inom designen går knappast att ifrågasätta i korridorerna på designskolorna. Volkswagenbubblans hisnande och på samma gång läckert aerodynamiska linjer. Eller det legendariska samarbetet kring SS-uniformerna som aldrig tycks sluta influera film-, mode- och designvärlden. Man tror på \quotetext{något} - varför inte \quotetext{Blut und Boden}?

Detta fortgår ständigt men det måste få ett slut.

\ditem[Historiska händelser i badrum]\label{historiska haendelser i badrum}

\begin{itemize}
\item Archimedes luktar så illa att han måste bada och upptäcker då sin egen princip.
\item Gustav Vasa\ref{gustav vasa} beträder Ornässtugans dass\ref{ornaesstugans dass} och Sverige\ref{sverige} påbörjar resan mot modern nationalstat.
\item Frank Zappa får idéen till det där skivomslaget där han sitter på muggen.
\item Den jätteäckliga scenen i filmen \textit{Trainspotting}.
\item Polisen upptäcker varför Plura Jonsson alltid är så glad fastän han gör så tråkig musik.
\item Ett av bibelns\ref{bibeln} märkligare avsnitt, \textit{Jesaja 36:12\ref{jesaja 36:12}}.
\item Evelyn Waugh skyndar sig på påskdagen 1966 in på dass efter att ha deltagit i högmässan och avslutar där, sittandes, ett av Storbritanniens stilistiskt viktigaste författarskap genom att dö av ansträngning.
\end{itemize}

\ditem[Hobofobi]\label{hobofobi}
 En hobofob är en person som är sjukligt rädd för att få sexuella inviter från småvuxna människor med stora, håriga fötter. Antagligen för att hen inte riktigt vet hur man artigt säger \quotetext{Nej tack.} i sådana situationer. Vetenskapen har hittills gått bet på att förklara hur hobofobin hos somliga individer kan blir så kraftfull, men man antar att det har en freudiansk koppling till att själv vilja ha riktigt håriga fötter men inte våga stå för det. Hobofobi räknas till de sociokulturellt inlärda fobierna, till skillnad från till exempel orchofobi, som har en starkt genetiskt-evolutionär bakgrund.

\ditem[Hockey]\label{hockey}
 är en sport för killar där man kissar på varandra i duschen och runkar i grupp. Hockeyspelare har taskig musiksmak och klär sig ungefär som Peter Jihde.

\ditem[Hockeypulver]\label{hockeypulver}
 är ett frätande ämne som uppstår som en slaggprodukt\ref{slagg} vid framställning av batterisyra. Vid torkning av ämnet bildas ett finkornigt pulver som används till att späda benmjöl vid industriell framställning av foderpellets.En liten del försvinner dock ut på den svarta marknaden och säljs i kiosker över hela landet.

\ditem[Hockeyröv]\label{hockeyroov}
 En hockeyröv kännetecknas av att den är väldigt bred. Det enda som kan matchas i hur bred den kan bli är hur bred glipan kan bli, alltså avståndet mellan skinkorna. 

\ditem[Holland]\label{holland}
 Ett stycke mark under havets yta. Holland är mindre än Bodens\ref{boden} kommun så att erkänna denna markplätt som suverän nation är endast löjligt. Precis som sin granne Belgien\ref{belgien} är det mycket pedofiler och ravedansare i Holland.

\ditem[Holmsunds tropikhus]\label{holmsunds tropikhus}
 är enligt dess hemsida \quotetext{ett kunskapscentrum där man kan fascineras av livets stora, unika mångfald} som vill verka för en \quotetext{nyfiken natursyn som också tar hänsyn till det levande livet omkring oss.} Inte minst är Holmsunds tropikhus ett klockrent föremål för en Nissepediaartikel\ref{nissepedia} efterdom det uppfyller alla önskvärda kriterier för en sådan: Den har en ganska rolig hemsida , är en lite provinsiell företeelse och har med djur att göra.

\uline{Historia}

Holmsunds tropikhus låg förr på bottenplanet i gamla sporthallen i Holmstund men har sedermera flyttat till egna lokaler, varpå den gamla sporthallen brann ner (eller upp, beroende på perspektiv)\ref{perspektiv}. Man skriver på hemsidan (notera festligt syftningsfel) att \quotetext{på den tiden var det en uppskattad utställning med enbart uppstoppade djur som flitigt besöktes, inte minst under sommarmånaderna.}

\uline{Nutid}

Nu kan dock besökare av Holmsunds tropikhus, förutom uppstoppade djur, se levande ormar och \quotetext{olika slags leddjur}. Det kostar 40 spänn för ungdomar att gå in. För vuxna vill de ha 60. Man har stängt måndag\ref{maandag} - tisdag\ref{tisdag} och öppet lördag\ref{loordag} - söndag\ref{soondag}. Vad som sker där emellan förtäljer inte hemsidan, vilket lätt gör en lite misstänksam och orolig.

\ditem[Holmund]\label{holmund}
 är ett slags mun\ref{mun} som har fått sitt namn efter Lennart Holmlund\ref{lennart holmlund}, fd. socialdemokratiskt kommunalråd i Umeå\ref{umeaa} kommun. Den skiljer sig från en vanlig mun i det att den ofta uttalar sig av sig själv, utan direkt koppling med hjärnans olika centra för logisk avvägning och framtidsplanering. Holmunden uttalar sig om saker den inte vet något om, säger korkade saker och förolämpar folk åt höger och vänster och blir speciellt aktiv i närheten av olika former av lokala media. Medan hjärnan under en intervjusituation processar input och skickar varningssignaler till olika delar av de kognitiva och motoriska systemen (\quotetext{Varning! Här krävs taktisk försiktighet!}) uttalar sig Holmunden vitt och brett om allt mellan himmel och jord: Whitney Houston, offentlig konst, italienska kvinnor, Håkan Juholt\ref{haakan juholt}, Ahlgrens bilar, skidlegenden Assar Rönnlund och det nuvarande beståndet av fiskmås, allt har holmunden något att uttala sig om, och inget vet den.

\ditem[Homi K. Bhabhas son]\label{homi k. bhabhas son}
 Den extremt supersmarta och coola akademikern Homi K. Bhabha har en son. Sonen, Satya Bhabha, är skådespelare i Hollywood och kan bland annat ses i filmen \textit{Scott Pilgrim Vs. the World}. Nissepedias redaktion ringde upp Homi K Bhabha, som agerar agent åt sin son, för att få höra det senaste om Satyas framgångar. Tyvärr ville inte Homi prata om det alls. Han ville bara prata om Mimikry.

\ditem[Horgalåten]\label{horgalaaten}
 är en svensk hambo utan känd upphovsmakare. Tillsammans med \textit{Idas sommarvisa}, \textit{Skala banan}, \textit{Den blomstertid nu kommer} och \textit{Hasta mañana} utgör den stommen i svensk grundskolas musikundervisning alltsedan krigsslutet. Den saknar ursprungligen sångtext men det är idag vanligt att man sjunger om hur ett gäng ungdomar i Hårga by i Hälsingland ger sig ut på dans en midsommarafton där de förtrollas av en speleman med bockfot som får dem att dansa sig till döds.

Kebnekaise har spelat in en utmärkt version av låten på sin skiva \textit{II}.

Iron Maiden blev så gripna av melodin att man gjorde en hel temaskiva om Horgalåten år 2003.

\ditem[Hudiksvall]\label{hudiksvall}
 är en stad i Sverige, unik i sitt slag för att det är den enda stad i Sverige som upplevt ett regelrätt krig i modern tid.

\ditem[Hugo Alfvén]\label{hugo alfvén}
 Musiker som 1954 blev den första svensk att ge ut en stereoinspelning på skiva. Låten var \textit{Midsommarvaka} och bolaget Swedish Society Discofil.

\ditem[Hundkäx]\label{hundkaex}
 är en växt som det finns fullt av i svenska ko- och fårhagar. Den växer och frodas i hela Norden med sin gröna stjälk och vita blommor. Då växten är giftig är det jävligt dumt att äta den men om man prompt måste göra det rekommenderas att först koka den ca 45 minuter för att få bort det bittra ur smaken. Däremot går det utmärkt att mata kaniner och andra gnagare med den. Ser man svinstora hundkäx är det antagligen i själva verket björnloka. Hundkäx kan med fördel ingå i en midsommarstång eller som en av sju blommor att sova med under kudden. På vissa platser kallas hundkäx för hundshecks, provocerande nog.

\ditem[Hundra sätt att få ligga]\label{hundra saett att faa ligga}
 \textit{Hundra sätt att få ligga} är ett verk av den boklärde småskurken\ref{smaaskurk} Prof. Etienne\ref{prof. etienne}. Som titeln antyder rör det sig om en handbok i konsten att få till det sexuellt. Prof. Etienne lägger stor vikt vid att använda rakvatten, hävda sin sociala rang genom att ljuga som en häst galopperar och samtidigt tränga sig på så mycket det bara går. Ett annat knep som förs fram är att åka till Danmark\ref{danmark} och medverka i skapandet av friluftsporr.

\ditem[Hur man ritar ett snyggt lodjurshuvud]\label{hur man ritar ett snyggt lodjurshuvud}

 Många har skrivit in till Nissepedia\ref{nissepedia} och undrat hur man ritar ett snyggt lodjurshuvud\ref{huvud} och nissepedia-teamet hörsammar så klart läsarnas önskningar. Här kommer en enkel, pedagogisk beskrivning av hur detta går till.

\uline{Förberedelser och grundstruktur}

\begin{enumerate}
\item Ta fram ett bra papper (se pappersform) och en eller flera bra blyertspennor.
\item Rita en boll (cirkel). Detta är din prototyp till lodjurshuvudet. Om du vill kan du här använda ett runt föremål som mall, till exempel en femma\ref{femma}.
\item Rita nu öron ovanpå huvudet. Öronen ritar du genom att efterlikna uppochnedvända V:n.
\item Rita djurets nos genom att göra ett litet o eller en uppochnedvänd triangel mitt i den större cirkeln.
\item Rita djurets ögon strax ovanför nosen genom att efterlikna små mandlar. Om du ritar ett stort lodjurshuvud kan du rent av använda mandlar som mallar för ögonen. OBS! Rita dem inte för högt upp! De ska nästan vara i linje med nosens högsta punkt.
\item Nu är det dags att rita djurets mun\ref{mun}. Åstadkom detta genom att rita ett lite utdraget uppochnedvänt V under djurets nos.
\end{enumerate}

\uline{Detaljer}

Nu har du grunden till ditt lodjurshuvud. Nästa steg är att lägga till sådana detaljer som verkligen får lodjurshuvudet att \quotetext{leva}.

\begin{enumerate}
\item Börja med att dra horisontella streck från nosen och ut mot cirkelns kanter. Detta är lodjurets morrhår.
\item Rita pupillerna i djurets ögon. Tänk på att de måste vara på samma sida om ögat så att lodjuret uppfattas som att det ser på något byte i fjärran och inte är vindögt.
\item Rita lodjurets kindhår och karaktäristiska örontofsar. Låt pennan löpa upp och ner respektive från sida till sida med lätt tryck. På så vis ser strecken ut som hår.
\item Lägg på skugga under och på ena sidan av lodjurshuvudet.
\end{enumerate}

\uline{Övning, övning, övning!}

När du har utfört dessa steg har du ett klart huvud. Tänk dock på att det tar tid att bli skicklig på att rita snygga lodjurshuvuden. Du kanske inte blir nöjd med ditt första försök, men fortsätt träna! Studera bilder och logotyper på skoterkepsar där lodjurshuvuden förekommer och prova dig fram. Efter många timmars träning kommer du att kunna åstadkomma ett lodjurshuvud som är så skickligt utfört att det ser ut som att det kan \quotetext{kliva ut} ur teckningen vilken sekund som helst.

\ditem[Huvud]\label{huvud}

 et är den del av kroppen som sitter längst upp. Det är i huvudet som det mesta pågår - tankar, ätande/drickande, hörsel, balans, syn, nickmål och så vidare. I vårt mindre grannland Danmark\ref{danmark} används huvudet som närstridsvapen och kallas då \quotetext{dansk skalle.} Så blev de också ockuperade av tysken under det stora fosterländska kriget\ref{det stora fosterlaendska kriget}.

\uline{Symbolism}

Den som vill vara i fred eller signalera till sin omgivning att hen menar allvar kan med fördel låta trä upp ett antal avhuggna huvuden på pålar på balkongen eller i trädgården.

\ditem[Huvudduk]\label{huvudduk}
 Att uppbära huvudduk är icke förbehållet muslimska damer, vilket många tycks tro. Huvudduken finns i alla upptänkliga färger och former och bärs av bikers, folk som lagt sig till med det ryska babusjka-stuket, Suicidal Tendencies, rappare\ref{hip-hop}, sjörövare, cancerpatienter och tennisspelare. Huvudduken består ofta i ett slags tygstycke som läggs på hjässan och sedan binds ihop någonstans, vanligtvis i nacken. Huvudduken skyddar bäraren från att utsättas för starkt uv-ljus\ref{uv-ljus}, från lättare fallande föremål och människors dömande blickar (om man till exempel, likt skådespelaren och TV-personligheten Ulf Larsson, är skallig). Den som enkelt och smidigt vill skaffa sig en huvudduk kan leta fram en gammal snusnäsduk från tiden då man bar sådan tillsammans med jeansjacka, vika den som en triangel, lägga den på huvudet\ref{huvud} och sedan knyta samman de tre ändarna så att näsduken bildar en liten \quotetext{hätta}.

\ditem[Hyena]\label{hyena}
 Hyenan (\textit{Hyaenidae}) är det djur som är mest crust av alla djur som man kommer på sådär på rak arm. Till utseendet ser det ut att vara ett hunddjur men tillhör egentligen grenen kattliknande rovdjur (\textit{Feliformia}). Arten delas upp i underarterna jordvarg, brun hyena, strimmig hyena och fläckig hyena. Pälsen är grov, fläckig eller strimmig och kan ibland vara försedd med små lappar som det står Conflict, Extinction of Mankind eller Discharge på. Den kan inte dra in sina klor och har en skalle som på många vis skiljer sig från liknande rovdjurs, inte minst vad gäller de grova käkarna. Ordet hyena kommer från grekiskans \textit{hýaina} som är ett nedsättande ord för svin.

\uline{Föda}

Jordvargen lever uteslutande på termiter och är immun mot det gift som vissa djurarter utsöndrar - Ja, hyenan till och med uppskattar det, sägs det. De andra underarterna lever mest av att sno kadaver från andra djur. De behöver inte dricka eftersom vätskebehovet täcks av födan (ruttet blod).

\ditem[Hyperhidros]\label{hyperhidros}
 \textit{Hyperhidros} är den latinska termen för kraftiga svettningar. Detta symptom orsakas av sådant som feber, fetma, förvirring\ref{foorvirring}, sömnbrist, korpfotboll och ansträngning\ref{goora raett foor sig}. I Sverige\ref{sverige} är hyperhibros främst associerat med Lars Leijonborg som uppvisar samtliga av ovanstående åkommor, och mer därtill.

\ditem[Hytta med näven]\label{hytta med naeven}
 Att hytta med näven är ett klassiskt tecken för att visa missnöje. Blir du omkörd av en sportbil? Hytt med näven! Kallar snorungar dig för dinosauriefossil? Hytt med näven! Har bolaget höjt priset på Kir?\ref{kir} Hytt med näven! Hejar din måg\ref{maag} på Djurgården? Hytt med näven! Säger någon att den nya sångaren\ref{den nya saangaren} ät bättre än Bon Scott? Hytt med näven!

Det finns alltid en anledning att hytta med näven!

Att hytta med näven ska inte förväxlas med att veva med kängnäven\ref{veva med kaengnaeven}.

\ditem[Háfrónska]\label{háfrónska}
 , även kallat högisländska, är en konstgjord form av isländska\ref{island} som har som mål att vara helt fri från utländskt inflytande i form av låneord, böjningar etc. \quotetext{Det är väl någon stollig ordförande i Vatnajökulls hembygdsförening som hittat på det här}, tänker ni säkert nu. Men i själva verket är det den inte alls särskilt isländske men väldigt stofila belgaren\ref{belgien} Jozef Braekmans som satt ensam på sin kammare i typ tio år och snickrade på detta. Att Braekmans både är belgare och vurmar för att rena det nordiska kulturarvet gör naturligtvis att Nissepedias\ref{nissepedia} läsare, som vi alla vet till största del består av kulturmarxister, genast börjar associera honom med nyliberalism\ref{nyliberalism} och glima\ref{glima}. 

\ditem[Häcklefjäll]\label{haecklefjaell}
 Svenskt namn på den isländska vulkanen \textit{Hekla}, vars krater enligt nordisk folktro är porten till helvetet. Namnet låter rätt larvigt men stället är alltså egentligen ganska ball.

\ditem[Hällefors]\label{haellefors}
 är en kommun i nordvästra Västmanland. Den är framförallt känd för att ha gett upphov till hälleforskavajen, ibland felaktigt benämnd Helly Hansen-jacka.

\ditem[Hänga på låset]\label{haenga paa laaset}
 När någon suktar att inhandla något, eller befinna sig, på en plats som har en låst dörr och etablissemanget inte riktigt hunnit öppna så har konsumenten bara ett val, att hänga på låset. Denna praktik är socialt accepterad på t.ex. Konsum\ref{konsumbutik}, kombinationsaffärer och butiker som bara säljer skoter-kepsar men brukar ses snett på vid anrättningar som serverar eller säljer alkohol\ref{alkohol}, t.ex. Rött\ref{roott} och Systembolaget. Det är synnerligen populärt vid reor.

\uline{Etmyologi}

Idag utförs sällan praktiken till fullo, dvs de flesta står bara och slår dank\ref{dank} utanför inrättningen. I uttryckets begynnelse var det annorlunda. Under ransoneringens dagar flödade inte brännvinet så som det gör idag, utan tillgången var strikt begränsad. För att få ut det mesta av detta så började bonddrängar att dricka upp sin sprit kvällen innan nästa ransonering började gälla och gick därför ner till handlaren och somnade, hängandes på låset, i väntan på att få ut mer brännvin\ref{braennvin} och kunna fortsätta bläckan.

\ditem[Häst]\label{haest}
 Gulligt djur med hovar och mjuk och luddig päls, gillar att äta hö och att reta andra djur.

\ditem[Hästhandlarplånbok]\label{haesthandlarplaanbok}
 En hästhandlarplånbok är en valutaförvaringsaccessoar\ref{valuta} av det rejälare slaget. Ett typiskt exemplar är fullt till bristningsgränsen med visitkort, medlemskort, lånekort, bankomatkort, bilder på tjejer i kortkort, körkort, plåster, fiskelina, snabbmatsförpackningar med salt och peppar, lite blandade diversesaker och fyrtiotusen miljarder\ref{fyrtiotusen miljarder} kvitton. 

Namnet kommer sig av att det främst är hästhandlare och andra skojare som begagnar sig av den. Tanken är att när kunden återvänder med länsman, rosenrasande över att den präktiga draghästen hen trott sig köpa visat sig vara en blind mulåsna, ska hästhandlaren kunna blåneka till att något avtalsbrott begåtts. Hästhandlaren tar upp sin hästhandlarplånbok som med sin väldighet inger respekt och visar att handlaren minsann har alla sina papper\ref{viktiga papper} i ordning. Han gräver runt en stund i börsen och säger sedan självsäkert att han inte hittar något kvitto över affären. Detta förklarar han med att hästskrället var så eländigt malätet att han inte brydde sig om att ta betalt utan faktiskt gav bort den till nya ägaren av ren vänlighet.

Vill man göra sin egen hästhandlarplånbok tar det ungefär fem år att samla på sig tillräckligt mycket trovärdigt skräp\ref{skraep} att stoppa i den.

\ditem[Hästkista]\label{haestkista}
 Innehav av hästkista berättigar en person (ibland med sällskap) till företräde i kön till det etablissemang där hästkistan är utfärdad. Systemet infördes på begäran av väktarfacket när det blev förbjudet att ras-, köns- och åldersdiskriminera. Det finns inga glasklara regler för hur man går till väga för att erhålla hästkista, men vissa sätt har större utdelning är andra. Det enklaste sättet är att på ett diskret vis hinta om att man kan tänka sig att ligga med någon i bandet. Vill man inte det kan man istället ligga med kulturstockholm\ref{ligga med kulturstockholm}, det är lite krångligare men fungerar i slutändan lika bra. Båda dessa knep kräver dock ett visst förarbete, vilket man inte alltid har tid med eftersom det är förfest. Det bästa knep som återstår då är att helt sonika hävda att man står på hästkistan fastän man inte gör det. Stega kavat fram till vakten, morsa och säg \quotetext{Tjena chefen! Martin Lundvall, jag står på hästkistan}. Det är inte helt säkert att namnet du drar till med finns med, så nu gäller det att ta det kallt. Skulle det saknas måste du nu på ett självsäkert sätt kräva att få prata med någon i bandet för att reda ut detta missförstånd. Bandmedlemmar tjänar ändå aldrig några pengar på entré så får du tag på en kommer hon eller han med stor sannolikhet att legitimera din scam. Går inte det återstår bara att ta sats och springa. Göm dig i en folksamling, vänd jackan ut och in och hoppas på det bästa. Samtliga knep fungerar bästa om man är lite full\ref{vaeltagravstensfull}.

\ditem[Håbroa]\label{haabroa}
 är en bro över sjön Gäsen utanför Kärrgruvan. När man fiskar vid Håbroa får man alltid napp. Brogrunden utgörs av två betongfundament som fixerar två stålbalkar över sjön. Stålbalkarna täcks på ovansidan av brädor som gör att bron kan nyttjas av både fotgängare, bilar och mopeder. Vid normalt vattenstånd går det att ro med en vanlig eka under bron, så länge man duckar för stålbalkarna. Bron är varken någon arkitektonisk skönhet eller milstolpe i ingenjörskonst men är väl värd ett besök ändå.

\ditem[Håkan Juholt]\label{haakan juholt}
 hade en god chans att bli Sveriges\ref{sverige} nästa stadsminister som partiledare för Arbetarepartiet Socialdemokraterna (sic!). När detta tillkännagavs jublade Sveriges samlade satirtecknare och karikatyrmålare, och firade sedan i dagarna tre. Idag är dock Juholt inte längre partiledare på grund av att han åt den sista tårtbiten på Sossarnas kongress och blev sedan systematiskt utmobbad med stor hjälp av massmedia.

\uline{Goda skäl som talade för Juholt som statsminister}

\begin{itemize}
\item Karln har mustach
\item Han är typ inte höger
\item Hans favoritfågel är pingvin
\end{itemize}

\ditem[Hålkort]\label{haalkort}
 . Hålkort används inom informationsteknologin för att hantera stora mängder information. Det uppfanns av IBM som sålde hålkort och tillhörande maskiner till Nazityskland, som använde dessa i logistiken kring koncentrations- och fångläger för att hålla reda på vilka fångar som förts var och vilka som var vid liv respektive döda. Idag används hålkort i huvudsak inom den civila sektorn. Uppkomsten av hålkortsnätet, som tillåter hålkortsoperatörer från skilda delar av jorden att utbyta information med varandra, har radikalt förändrat förutsättningarna för denna informationstekniks fortlevnad och har gjort att den idag är en av det franska Minitels stora utmanare.

\ditem[Hårdrock]\label{haardrock}

 är ett slags sjukt rå rock för den lite tuffare publiken, men det är mycket mer än så. Det är vänskapsband som sträcker sig över kontinentalplattor och en lukrativ marknad för de som broderar backpatches. Hårdrockaren är en person som går sin egen väg och inte låter sig kuvas av samhällets normer och förväntningar. Medan samhället talar arbetslinjen\ref{arbetslinjen}, amorteringar och anställningsbarhet bara garvar hårdrockaren, bär jeansshorts och lyssnar på Saxon\ref{anglosax}. Detta särskiljer hårdrockaren från andra invånare i den semirurala bruksort där hårdrockaren bor. Men den kanske tydligaste sak som skiljer hårdrockaren från vanliga människor är de picturediscs som hårdrockaren lägger sina surt förvärvade pengar på. För den genomsnittliga människan är en picturedisc en lite fånig och ofta betydligt sämre skiva i jämförelse med normala plattor, men inte för hårdrockaren, som sätter värde på att kunna ta fram skivan och njuta av den genom att titta på den coola och lite skrämmande bilden.

\uline{Hårdrockare i utbildningsvärlden}

Hårdrockaren är i regel en bra student i de flesta akademiska ämnen eftersom hårdrockaren letat sig till akademin pga sitt intresse för medeltiden, nordiska språk osv och har skaffat sig språkkunskaper genom att läsa liner notes och analysera låttexter medan andra studenter sprungit på disco och fjantat sig. Paradoxalt nog är det som hindrar hårdrockaren från en akademisk karriär det faktum att hen är upptagen just med att läsa liner notes, analysera låttexter, titta på picturediscs och klippa av ärmar och ben på div denimplagg.

\uline{Trivia}

Ingen hårdrockare har någonsin kallat AC/DC för AC/DC, Judas Priest för Judas Priest, Black Sabbath för Black Sabbath eller Iron Maiden för Iron Maiden - för hårdrockaren är det DC, Priest, Sabbath och Maiden som gäller.

\ditem[Hårdrockare med gomspalt]\label{haardrockare med gomspalt}
 En hårdrockare med gomspalt är en hårdrockare vars gom inte vuxit ihop på normalt vis under graviditetsvecka 10-12. Hårdrockarens gomspalt kan omfatta både den mjuka och den hårda gommen, bara den mjuka gommen eller muskulaturen i denna (så kallad Submuskulös gomspalt, förkortat SMG). Tudelad gomspene förekommer också i sällsynta fall. Den moderna plastkirurgin är idag så avancerad att den i de flesta fall, ibland med hjälp av logoped, kan förhindra nedsatt talutveckling så väl som synliga ärrbildningar hos hårdrockaren. I svårare fall kan hårdrockaren få en protetisk obturator, vilket är en anordning av härdat silikon och titanium som stänger spalten mellan de nasala och orala kaviteterna. Även en Lathmansoperation kan vara nödvändig. En Lathmanordning opereras då in i munnen\ref{mun} då hårdrockaren är i fyra-\ref{fyra} eller femårsåldern och fixerar de två delarna av gommen så att de slutligen växer samman, varvid anordningen kan avlägsnas.

\ditem[Hårdrockare och vitaminer]\label{haardrockare och vitaminer}
 För att hårdrockaren ska må bra och kunna göra sin grej är det viktigt att hårdrockaren har ett bra vitaminintag. Vitaminer är livsnödvändiga organiska ämnen; en del är sådana som hårdrockarens kropp inte kan bilda själv utan måste intagas genom hårdrockarens föda. Brist på något vitamin ger upphov till specifika bristsjukdomar hos hårdrockaren, som därmed kan missa Motörhead i ishallen i Gävle. Vitaminer kan delas in i två olika typer, vattenlösliga och fettlösliga, där överskottet av de fettlösliga kan lagras i fettet i hårdrockarens kropp medan de vattenlösliga försvinner med hårdrockarens urin\ref{urin}. Hårdrockaren bör få i sig rekommenderade nivåer av A-, D-, E- och K-vitamin (0,8 mg, 10 µg, 10 µg respektive 80 µg) inklusive alla undergrupper till dessa. Det viktigaste för hårdrockaren är att ha ett varierat näringsintag och att äta minst en frukt\ref{frukt} om dagen, gärna två. Vitamintillskott kan också vara aktuellt, speciellt om hårdrockaren har en specialdiet eller inte får så mycket solljus, kanske för att hårdrockaren satt upp en sån där kombinerad flagga och affisch med omslaget till \textit{Powerslave} över fönstret. Sådana preparat kan dock vara dyra och innehåller inte alltid ämnesnivåer som passar den enskilda hårdrockaren. Det är därför bra om hårdrockaren konsulterar sin läkare eller dietist, som kan hjälpa hårdrockaren att finna det tillskott som funkar bäst.

\ditem[Hårdrockism]\label{haardrockism}
 är ett samlingsnamn för åsikter, beteenden och politiska uttryck som på ett eller annat vis diskriminerar eller har för avsikt att skada hårdrockare. Extrema grupper av hårdrockister finns i vårt samhälle idag, men de är ganska få och egentligen inte den stora faran för hårdrockaren\ref{haardrock} - det är vardagshårdrockismen som är det verkliga hotet mot vårt demokratiska samhälle: När hårdrockaren går in på H\&M får hen veta från en kylig expedit att alla jeansvästar plötsligt tagit slut både på lagret och hos leverantören. När hårdrockaren makar sig fram till DJ-båset på Berns och vänligt ber att \textit{South of Heaven} eller \textit{Wolverine Blues} spelas på helgvolym\ref{helgvolym} blir hen bryskt avvisad och inte sällan hånad. Hårdrockaren kanske kommer till Emmabodafestivalen och får sig en kalldusch när det visar sig att inte ett enda av alla band på festivalen spelar dieseldoftande thrash metal. Exemplen är oräkneliga.

\uline{Anti-hårdrockism}

Den enda riktigt framgångsrika sättet att bekämpa hårdrockismens fula tryne/månghövdade hydra är, förutom att möta människor ansikte mot ansikte och sprida kunskap, den amalgamerande gemenskap och enighet som hårdrocksvärlden erbjuder. I den finner många styrkan att mot alla odds fortsätta åka på Whitesnake på Gavliahallen, bära jeanshorts i oktober och köpa repade picturedics för ohemula\ref{ohemul} summor.

\ditem[Högtalartips]\label{hoogtalartips}
 Om du i vänskapskretsen har en person som ser sig själv som högtalarexpert och du bjuder in denna person (som i nittionio fall av hundra är en man) kan du få högtalartips. När din bekante kliver in i lägenheten eller stugan eller vad du nu bor i säger han att du har högtalarna på fel ställe, skulle få fetare bas om du bara bytte element, att du måste ha bättre kablar och nya slutsteg. Om du gör som han säger, bedyrar han ihärdigt, kommer allt bli så bra, så bra. Högtalarexperten känns igen på att han äger högst fyra skivor och att ungefär hälften av dessa (2) är samlingsskivor.

\ditem[Hönsgård]\label{hoonsgaard}
 En hönsgård är ett inhägnat område avsett för höns. En bra hönsgård bör innehålla pinnar som hönsen kan sitta på, en dammig plats som hönsen kan lerbada i, ett tak som hönsen kan sitta\ref{sitta} under, samt ett tråg\ref{traag} som hönsen kan äta ur när de ledsnar på att picka efter mask. Är det många tjejhöns kan det vara bra med ett rede som dom kan värpa i också. Eftersom i princip alla andra djur är hönsens naturliga fiender bör hönsgården vara inklädd i någon form av stängsel eller nät både på sidorna och ovanpå som skydd mot inkräktare. Vanligast är att man använder hönsnät till det. Brädorna som håller stängslet på plats ska alltid vara röd\ref{roott}. Förr eller senare kommer man dock garanterat tröttna på att bara äta ägg\ref{aegg} varje dag och vilja göra sig av med hönsen och därför kan det även vara praktiskt med en dörr eller lucka på hönsgården som man kan öppna för att låta fjäderfäna irra sig ut i skogen. Rockgruppen Imperiet belyste problematiken med att äga för många höns i sin sång \textit{Var e vargen}.


%%%%%%%%%%%%%%
\newpage
\null
\\
\null
\\
\Huge
I
\normalsize
\\
\null
\\
\null
%%%%%%%%%%%%%%

\ditem[I'm so tired I could sleep on a clothesline]\label{im so tired i could sleep on a clothesline}
 Detta ursprungligen londonesiska uttryck kan te sig svårbegripligt för den utan inblick i det tidiga nittonhundratalets brittiska samhälle. I Londons east end levde under denna tid de särskilt fattiga, så som prostituerade och sjömän med träben. Inte sällan var dessa två samhällsgrupper så barskrapade att de inte hade någon fast bostad utan var tvungna att driva runt och sova varhelst de kunde. Ett alternativ, om man hade en penny på sig, var att gå till en lokal där man satte sig ned på en bänk, flera personer i rad, för att sova. Vad som höll de sittsovande personerna upprätt var ett klädstreck, fastbundet vid en punkt på vardera sida av de sovande. När tiden för väckning var inne, knöt helt sonika den som ägde etablissemanget upp en av knutarna som höll klädstrecket på plats och lät de slumrande falla till golvet.

\ditem[Ica prästost]\label{ica praestost}
 Denna prästost är matvarukedjan icas egna. Den är sämre än alla andra prästostar som finns till. Efter att ha inmundigat en leksands knäckemacka med smör, ica prästost och röd paprika på, ska estradören, renässansmannen tillika vetenskapsmannen Carl von Linné ha skaldat:

\textit{"Thänna prästost från icander}
\textit{göra vålhd på smakens löhkar}
\textit{then smaaka som a4-ark med snorloska på}
\textit{thät ej alls i munhålan pöhkar."}

\ditem[Incitament]\label{incitament}
 är ett påhittat ord av dom rika för att dölja för arbetare vad chefer egentligen håller på med.

Studier har visat att om man säger till en arbetare att man tänker ta hens pengar för att man vill ha mer av dom själv utan att göra mer blir arbetaren arg. Om man istället säger till arbetaren att man behöver hens pengar som ett incitament för att sköta företaget är chansen större att arbetaren inte förstår vad det handlar om och låter det hela bero.

\ditem[Incitatus]\label{incitatus}
var namnet på den häst som kejsare Caligula av det romerska imperiet tillsatte som senator år 40 e.Kr. I historieskrivningen har Incitatus politiska karriär beskrivits som en produkt av Caligulas nedåtgående spiral av vansinne. Alternativa historieforskare med fokus på att uppvärdera djurs osynliggjorda historia, har dock påstått sig hittat pergament som bevisar att Incitatus redan innan sin tillsättning som senator var en folkkär lokalpolitiker som genom personkryss tog sig ganska långt i valet till landstinget i provinsen Narbonensis (dagens Franska riviera). Bland Incitatus mer framgångsrika politiska initiativ återfinns några för den tiden tämligen progressiva motioner om skattelättnader för import av torrfoder, och en hovfull protesterande skrivelser till lokalpressen om den förtryckande \quotetext{No pants policy} som på den tiden rådde innanför imperiets gränser. Det råder fortfarande stränga drakoniska lagar kring byxbruk i offentligheten i dagens Italien. Men den moderna italienarens idoga vägran att bära underkläder kan ses som en dold, ödmjuk hyllning till Incitatus radikala ställningstaganden.

\ditem[Indianmuskler]\label{indianmuskler}
 får man när man bedriver högre form av träning så som cykling, löpning, skidåkning, spjutkastning, fäktning, lägring, jakt av bäver och annat kritter, bergsklättring m.m.

\ditem[Indiska]\label{indiska}
 är en kombinationsaffär\ref{kombinationsaffaer} som tillhandahåller kläder, heminredning och enklare livsmedel till genusvetare, konstskolestudenter och liknande. Som sig bör säljer man endast damkläder och hänvisar eventuella män i butiken till Dressmann\ref{dressmann} strax intill.

\ditem[Individ]\label{individ}
 kan man va när man redan har det bra.
Individer utbrister gärna saker som:

-\quotetext{Måste vi vara kön? Kan vi inte vara individer?}
-\quotetext{Klass hit och klass dit, kan vi inte bara vara individer?}

När detta inträffar bör man vara på sin vakt för då har du antagligen att göra med en tokliberal\ref{tokliberal}.

\ditem[Inga lejon]\label{inga lejon}
 Det korrekta att svara när någon uttalar det självklara påpekandet \quotetext{vackert väder} på en solig dag.

\ditem[Ingvar Carlsson]\label{ingvar carlsson}
 (1934 - när som helst nu) är en svensk socialdemokratisk politiker som var Sveriges\ref{sverige} statsminister mellan 1986 och 1991. Efter avklarat uppdrag som folkvald upptäckte han punken och började ställa till osämja i grannskapet genom att spela Black Flags \quotetext{Damaged} (ibland, men mycket sällan, även \quotetext{First four years}) på helgvolym\ref{helgvolym} dygnet runt. Han ska även ha setts på en av Henry Rollins spelningar i Stockholm, försökandes stagedive:a iklädd en pappskiva med partilogotypen.

\ditem[Ingvar Kamprad]\label{ingvar kamprad}
 (a.k.a Ingvar Mein Kampfrad) är en av världens rikaste kapitalister. Han är det tack vare sina företagarskills, men framför allt tack vare stulet mervärde från arbetare över hela världen, inte minst i Asien. Trots att han är mångmiljardär så har han fått ett rykte om sig att vara snål. Snålheten är, i bästa fall, vansinnigt ambivalent. Han har klippkort på den lokala restaurangen där han bor i Schweiz, men har inga problem med att köpa en svindyr vargpälskappa. Hans image inbegriper också en folklighet, även den påklistrad. Han vill framstå som vilken människa som helst, förutom då att han har fyrtiotusen miljarder\ref{fyrtiotusen miljarder} på banken. Han fick problem med detta när det uppdagades att han med kreativ bokföring sket i att betala skatt och likförbannat ville ha pension från svenska staten. Att folk blir förvånade över att en riking skiter i att betala skatt är konstigt i sig. Men Kamprad är bannemig inte känslokall, trots att han är kapitalist. Han har en gång uttalat sig om hur ledsen han blir när de som arbetar åt honom går med i facket då han tycker att IKEA \quotetext{är som en stor familj}.

\ditem[Inititativ Anusmark]\label{inititativ anusmark}
 Initiativ Anusmark är en aktionsgrupp och politisk lobbyförening som verkar för att orterna Anumark och Ansmark, som ligger norr respektive söder om Umeå, ska slås ihop till en gemensam kommun vid namn Anusmark.

\ditem[Innebandy]\label{innebandy}
 är en lek som går ut på att man med hjälp av en plastklubba ska fösa en bisarr plastboll in i ett mål. Som hinder har man ett motståndarlag bestående av glaslirare som skriker att domaren är idiot så fort man nuddar dom utan att bli utvisad. För att lag från hela landet ska kunna träffas har man hittat på en klubb som heter Superligan som ser till att alla får leka med alla. Fastän det heter Superligan har man inte superhjältekläder på sig utan färdigslitna jeans och Champion-hood. Innebandyklubbor finns att köpa i alla välsorterade leksaksaffärer.

Musikgruppen Takidas medlemmar lärde känna varandra när dom lekte innebandy på samma ställe, vilket speglar musiken väl. 

\ditem[Inre backspegel]\label{inre backspegel}
 kallas den spegel på bilar som sitter inne i kupén. Formen är ofta rektangulär och färgen på fästet svart eller krom. Dess funktioner är att man enkelt ska kunna spana på snygga passagerare i baksätet och ha en central punkt att fästa sin wunderbaum\ref{wunderbaum} på.

Enligt §18 i Trafikbalken är det lag på att alla bilar i Sverige ska ha inre backspegel. Skulle detta saknas får man dock ingen påföljd. Förmodligen för att man inte vill jävlas\ref{jaevelskap} med alla ägmästare\ref{aegmaestare} som kör dieselbil med lastgaller\ref{dieselbil med lastgaller}.

I sydeuropeiska länder ser lagen annorlunda ut och den inte backspegeln ses mer som ett schysst komplement men inte alls nödvändig. Mången spanjack eller italienare bryter därför bort den när bilen rullar ut från försäljaren för första gången med motiveringen \quotetext{Det som är bakom är redan passerat.}

\ditem[Institut och tankesmedjor]\label{institut och tankesmedjor}
 Sammanslutningar för bilförsäljare, hästhandlare, kulaker\ref{kulaker} och pissliberaler. Använder någon, i exempelvis en insändare, en hänvisning till institut, allra helst stavat på engelska, då har man att göra med en charlatan. Motsatsen är givetvis rediga och hederliga sammanslutningar, exempelvis Landsorganisationen. Ofta så har dessa institut och tankesmedjor engelska namn, trots att de kanske håller till i Schweiz. Andra symptom kan vara blekta tänder. Ofta använder man sig gärna av akademiska titlar som kanske var viktiga någon gång då Per Albin levde, exempelvis Fil. kand. i idéhistoria mfl.

\ditem[Insändarsignaturer]\label{insaendarsignaturer}
 Insändare är, som alla vet, det effektivaste sättet att bilda opinion runt en viktig fråga. För att ge extra trovärdighet åt ditt inlägg är det alltid en bra idé att skriva under meddelandet med en snygg signatur. Den klassiska \textit{Vän av ordning} fungerar naturligtvis fortfarande, men i en globaliserad värld kan det ibland löna sig att prova något nytt. Slösa dock inte tid på att återuppfinna hjulet utan låt Nissepedia\ref{nissepedia} guida dig genom de nya slagfärdigheterna:

\begin{itemize}
\item Jag och många andra
\item Missnöjd skattebetalare
\item Guds sändebud på jorden
\item Aldrig mera sosse
\item Djurvän och varghatare
\item Stolt köttätare
\item Lennart Holmlund\ref{lennart holmlund}
\item Lär av de äldre
\end{itemize}

\ditem[Intellektuell regression]\label{intellektuell regression}
 är en företeelse som förekommer periodsvis i västerlandets historia och består i att alla plötsligt blir dumma i huvudet\ref{huvud}.

\uline{Under medeltiden}

Medeltiden\ref{medeltiden} anses vanligtvis innebära en sådan period eftersom man under denna tid lät sig styras helt av Bibeln\ref{bibeln} och påven, istället för att som hos de gamla grekerna\ref{de gamla grekerna} låta det intellektuella samtalet och kritiskt tänkande stå i första rummet. Som tur var följde renässansen på detta mörka kapitel i västerlandets historia och spred åter vetenskapens och det självständiga tänkandets ljus över de kristna träskaften.

\uline{Under republiken och restorationen}

I den engelskspråkiga världen ledde Oliver Cromwells republik och den efterföljande restorerandet av monarkin, med Karl IIs kröning, till ett bakslag för det allmänna intellektuella klimatet. Puritanerna lät efter att de tillsammans med sina bundsförvanter gjort sig av med Karl I inte vänta på sig vad gällde att skrota alla musikinstrument i kyrkor och kapell eftersom de hatade musik, liksom att stänga alla teatrar och kaféer, eftersom de hatade kultur och sällskapliga samtal om livets många knepigheter. Efter Cromwells död och efter att hans son jagats på flykten öppnades teatrarna igen och man skapade kultur som handlade om överklass-snusk så som ytliga urbana charlataner som jagade in lika ytliga men kurviga damer bakom skärmar i kulisserna. Man skrev ironisk eller lätt erotisk poesi om lättidentifierade överklasstanters utseenden och började så smått knåpa ihop grunden för den moderna rasismen.

\uline{I postmodernismen}

Nästa stora period av intellektuell regression började efter det andra världskriget, men blommade inte ut förrän det kalla krigets slut 1989. Nu annonserade Fukuyama \quotetext{historiens död} och nyliberaler världen över ansåg sig sitta på den enda sanningen om hur allt ska vara, vilket de, konstigt nog, uttryckte som att ideologierna var döda. Denna sanning består i att \quotetext{åt de som har skall vara givet} och \quotetext{åt de som inte har ska vara givet napalm och fosforbomber}. Det intellektuella samtalet består idag av förfasande över frekvensen av mörkhyade i västvärlden, att lönearbetare ställer krav på att faktiskt bli betalda för sitt arbete och att vänstern har mage att föreslå en lite annorlunda världsbild än den som går ut på ett slags förvirrad läsning av Darwins teori om det naturliga urvalet (i vilken själva \quotetext{urvalet} går ut på militära angrepp från skyn och ett slags långsam svält orsakad av \quotetext{frihandel}). I Sverige\ref{sverige} tillhör bland annat Dagens Nyheter\ref{dagens nyheter} denna intellektuella regressions mest högljudda tillskyndare.

\ditem[International cloud atlas]\label{international cloud atlas}
 är en bok med bilder på olika molnformationer. Den publicerades första gången 1896 men har sedan dess kommit i flera nyutgåvor med fler moln\ref{moln}. För att ett moln ska hamna i atlasen måste det ha en helt unik formation och godkännas av \textit{Cloud appreciation society}. Senast det skedde var 1951 när man valde in formationen \quotetext{Cirrus intortus}. Just i detta nu förs diskussioner om att välja in den nyupptäckta formationen \quotetext{Undulatus asperatus}.

\ditem[International harvester]\label{international harvester}
 är ett amerikanskt företag som tillverkar traktorer och andra redskap för jordbruk. Man tillverkade även bilar\ref{bil} och lastbilar fordom. Bilar som är heta som tusan, därför ofta röda\ref{roott}.

Det är även det ursprungliga namnet på psychrockgruppen Träd, Gräs och Stenar\ref{traed, graes och stenar}.

\ditem[Internet]\label{internet}
 Av pederaster och sociopater - för pederaster och sociopater.

Internet byggdes av den amerikanska armén med syfte att föra ett lågintensivt krig mot god smak. Såhär retrospektivt så kan man säga att man lyckades med sina intentioner. Vi lever numera i en gränslös värld. Då inte på det fina viset där man slipper visa pass och där herrar på rad skriver på ett konvolut. Nej, utan på det där lite olustiga och ganska svårgreppbara viset. Att var och varannan männska inte bara är singer/songwriter utan dessutom swinger eftersom alla de traditionella instanserna för kvalitetskontroll av både Neil Young-epigoner och sex helt försvunnit. Man kan säga att internet är ett solipsistiskt paradis för världens alla rape-dejtande datanördar. Visst folk har väl under längre tid haft mer eller mindre exotiska böjelser men nu har man fräckheten att prata om dom. Det värsta med hela den vämjeliga historia som kallas internet är att det inte längre går att vara i opposition mot något - allt är ju okej.

\ditem[Irländsk parkering]\label{irlaendsk parkering}
 Att parkera på irländskt vis är att låta omgivningen, t.ex. träd, tegelväggar och andra bilar, sköta inbromsningen.

\ditem[Isaac Johannes Lamotius]\label{isaac johannes lamotius}
 (1653–1710) var den siste europe som rapporterats ha sett en levande dront.

\ditem[Isebrogen]\label{isebrogen}
 är Islands\ref{island} flagga och har så varit allt sedan självständigheten från Danmark\ref{danmark} 1944. Under kolonialtiden var alla riktiga flaggor förbjudna och danskarna tillät bara att man hissade sopsäckar eller boxerkalsonger i XXL, såsom man gjorde hemma på Själland respektive Jylland. Motivet föreställer Islands enda fyrvägskorsning, omgärdat av fyra glaciärer.

\ditem[Island]\label{island}
 är ett nordiskt land som ligger i vattnet brevid Norge. På Island finns gejsrar, polacker, glaciärer och en anrik historia med en massa vikingar i.

\uline{Ekonomi}

Islands ekonomi är baserad på fiske, turism och att låna pengar från Storbrittanien (UK) som de sedan inte betalar tillbaka.

\uline{Politik}

Island är i likhet med de andra nordiska länderna socialdemokratiskt, skrällen är att de har en kvinna som statsöverhuvud. Parlamentet utses genom allmänna val, men eftersom alla islänningar hatar valar är det nästan ingen som röstar. Reykjavik-borna har valt en borgmästare vars första punkt på dagordningen är att bygga ett äventyrsbad\ref{aeventyrsbad} i Keflavik.

\uline{Kultur}

Islands kultur består till 80\% av Björks förvånansvärt bra techno-pop och till 20\% av Sigur Rós postrock\ref{posten}.

\uline{Island i sportvärlden}

Island deltar sällan i sportevenemang, men har spöat Sverige i både fotboll och handboll en gång. De är dock hejare på glima\ref{glima}. Den ständigt kontroversielle schackspelaren Bobby Fischer spelade schack mot dåvarande världsmästaren Boris Spassky på Island '72. Fischer, slut som människa, återvände till Island 2005 efter att ha blivit landsförvisad från resten av världen.

\uline{Kända islänningar}

\begin{itemize}
\item Björk
\item Snorre Sturlasson
\item Egil Skallagrimsson
\item Halldór Laxness
\item Bubbi Morthens
\end{itemize}

\ditem[ISO 216]\label{iso 216}
 är en internationell standard för pappersform. Tyvärr är den inte internationell på riktigt då Nordamerika inte är med på tåget, vilket är helt särske\ref{saerske}.

\ditem[Italienska svordomar]\label{italienska svordomar}
 är rejäla grejer. Där är det inget larv med något \quotetext{heliga blå\ref{franska svordomar}} eller \quotetext{skam på torra land}. Till och med amerikanernas \quotetext{könsligt umgänge med en moder} står sig slätt\ref{slaett}. Eftersom italienarna över lag är ett väldigt religiöst folk blandar man gärna in gud\ref{gud} eller jungfru Maria i sammanhang kring sex och avföring. \quotetext{Dio cazzo porco cristo culo!} betyder ungefär \quotetext{guds kuk i Jesus\ref{jesus} grisarsle} och används mycket i allmänna sammanhang på fotbollsläktare\ref{fotboll}. För direkta uppmaningar riktade till personer är \quotetext{Vai in culo!} (\quotetext{Gå in i arslet!}) och \quotetext{Cazzo durro dell cavallo!} (\quotetext{En hård hästkuk (i din mun)})\ref{mun} mycket vanliga. Som ni märker är det lite annan nivå än \quotetext{kyss Karlsson} och \quotetext{far åt Häcklefjäll\ref{haecklefjaell}}.

För den som vill konstruera sina egna italienska svordomar kan följande lilla ordlista vara mycket användbar:

\begin{itemize}
\item Dio (Gud)
\item Cristo (Jesus)
\item Madonna (jungfru Maria)
\item ostia (hostia, nattvardsbröd)
\item cazzo (kuk)
\item fica/figa (fitta)
\item chiavare (knulla)
\item coglione (testikel)
\item culo (röv)
\item merda (skit, som massord)
\item pirla (skit, om person)
\item boia (bödel)
\item cane (hund)
\item porco/porca (svin)
\item troia (sugga)
\end{itemize}

\ditem[Ivar Lo Johansson]\label{ivar lo johansson}
 (1901 - 1990) var sveriges första punkare. Hans arv förvaltas idag av hans sonson Johan Johansson.

\ditem[Ivriga små bävrar]\label{ivriga smaa baevrar}
 är centerpartiets\ref{centerpartiet} officiella begrepp för vad som normalt kallas egenföretagare. Där många ser en kulturarbetare med viscerala stressyndrom ser alltså centerpartisten en ca 75 cm hög amfibisk gnagare\ref{gnagare} som har stor negativ inverkan på biotoper där den lever.

%%%%%%%%%%%%%%
\newpage
\null
\\
\null
\\
\Huge
J
\normalsize
\\
\null
\\
\null
%%%%%%%%%%%%%%


\ditem[J.R.R Tolkien]\label{j.r.r tolkien}
 John Roland Razorhoof Tolkien (1892-1973) var en brittisk författare och språkvetare som skapade de populära böckerna om Midgård (Middle Earth). Böckerna handlar om ett antal fantastiska väsen, hoberna, som endast Tolkiens gränslösa fantasi kunnat föda fram. Hoberna är korta människor med hår på fötterna. Trilogin om härskarringen anses av finniga nördar i femtonårsåldern vara kärnan i Tolkiens författarskap. Den handlar om en ring som några hober ska kasta ner i en vulkan. En trollkarl är med och så finns det orcher (upphottade troll). Sex är det dåligt med, men en viss del våld kan som tur är utlovas.

\uline{Trivia}

\begin{itemize}
\item De flesta av Tolkiens böcker handlar om ringar.
\item Tolkien rökte pipa.
\item Åke Ohlmarks\ref{tolkien och den svarta magin} har länge svurit på att Tolkien gick på Subutex och intog sin Metallica post-Burton.
\end{itemize}

\uline{Sagt om Tolkien}

\textit{The man once wrote: Do not meddle in the affairs of wizards, for they are subtle and quick to anger. Tolkien had that one mostly right.}
\textit{Jim Butcher}

\textit{Asså Tolkien är ju så jävla dum i huvet }\ref{huvud}
\textit{M. Frygell}

\ditem[Jackie Howe]\label{jackie howe}
 (26 Juli 1861 – 21 Juli 1920) var en australiensisk\ref{australien} fårklippare och fackföreningsledare. Howe är framförallt känd för att han år 1892 slog världsrekord i antal klippta får under en dag respektive en vecka. Dagsrekordet Howe satte uppgick till 321 får på strax under 11 timmar och stod sig ända fram till 1950 när Ted Reick slog det. Reick använde dock elektrisk klippare medan Howe på sin tid enbart jobbade med handkraft\ref{handjagare}. Hans veckorekord på 1437 klippta får är fortfarande obesegrat. En bronsstaty föreställande Howe klippandes ett får finns rest i det lilla australiensiska samhället Blackall.

\ditem[Jackson Pollock]\label{jackson pollock}
född 28 januari 1912 i Cody, Wyoming, USA, död 11 augusti 1956 i Springs, New York, var en amerikansk \quotetext{målare}, centralfiguren inom den abstrakta expressionismen med sin \quotetext{action painting}. Pollocks teorier byggde i stort på att göra sitt liv så enkelt och bekvämt som möjligt och en stor del av detta mål var att försöka övertyga konstvärlden om att han var duktig. Man kan likna Pollocks verk lite vid produkten en fotograf producerar\ref{fotografering}. Medan fotografens jobb består i den komplicerade konsten att trycka på en knapp tog Pollock lättja till en ny nivå. Han lät placera duk på golvet och gick sedan mest runt i sin studio med färg droppande från olika saker. På grund av detta passerade det abstrakta måleriet fotografiet efter den klassiska devisen \quotetext{mest utdelning för minst ansträngning} då han kunde göra en bild under kortare tid än en fotograf kan i och med den komplicerade fotoframkallningsprocessen som består i att gå till en butik och få bilder utskrivna. Än idag kan man se resterna efter Pollocks runtflanerande på olika museum.

Jackson Pollocks action-painting var en föregångare till action-filmen (som uppfanns under 80-talet i och med filmen Terminator) men idag är denna koppling mer eller mindre bortglömd och oftast ses därför tv-serien COPS som action-filmens urfader.

\ditem[Jacques Touillaud]\label{jacques touillaud}
 var gaffare vid inspelningen av James Bond-filmen \textit{Moonraker} (1979). Han var även chefselektriker vid inspelningen av den betydligt mindre framgångsrika \textit{En fluga i soppan} (Brust oder Kule) från 1976, men denna films begränsade försäljningsframgångar ska på inget vis härledas till Jacques, vars arbetsinsats ska ha varit så nära felfritt som det går.

\ditem[Jag ska bara bli full först]\label{jag ska bara bli full foorst}
 Sex vackra små ord som kan hjälpa dig i de flesta situationer. Ursprungligen var det Epikuros\ref{epikurism} som myntade uttrycket, fast på grekiska.

Uttrycket och sysslan att faktiskt bli full innan man gör viktiga saker har sedan dess blivit en smash-hit världen över. Ofta kan man höra folk säga det innan de byter däck på bilen, gräver ett dike, tillagar moules frites, diskutera fotboll med fans från andra lag, eller innan ett parti intensiv men slafsig älskog. En viss backlash kom i och med att skolungdomar började bara bli fulla först innan de gjorde läxan. Detta resulterade i en förlorad generation illitterata suputer som försett Sverige med två borgerliga regeringar... i rad!

Men trots det fortsätter folk att säga, \textit{Jag ska bara bli full först}, innan de borstar tänderna, kör hem svärmor från släktmiddag, renoverar badrummet, eller packar in ungarna i familjens volvo 740\ref{volvo 740} och drar till Legoland\ref{lego} för lite dansk semester\ref{dansk semester}.

\ditem[Jan Björkblund]\label{jan bjoorkblund}
 är en sagofigur som ofta förekommer i godnattsagor och vaggvisor. Grundberättelsen, av vilken det finns olika varianter, går så här: När det lilla barnet\ref{barn} har borstat sina små vita mjölktänder, kysst mor och far godnatt, repeterat sin katekes och lagt sig mellan lakanen flyger Jan Björkblund in genom det öppna fönstret och ställer språkkrav. Han läser från en lista med krav på motprestationer på invandrare och för varje steg på listan blir barnets ögonlock lite tyngre. Den lille, som hört allt detta fyrtiotusen miljarder\ref{fyrtiotusen miljarder} gånger i sitt oansenliga liv, slappnar av och drömmer sig bort. Nu somnar barnet in och Jan Björkblund, som ser att sitt arbete är gjort, utsvävar populistiskt genom fönstret igen för att ställa språkkrav i en barnkammare någon annanstans.

\ditem[Jan Björklund]\label{jan bjoorklund}
 (fp) är riksdagspolitiker och uppvisar den intressanta kombinationsdiagnosen \quotetext{liberal} och \quotetext{sjuk i huvudet}.

\ditem[Jan Wilsgaard]\label{jan wilsgaard}
 Estetikens urfader. Designade bland annat Amazon, 240\ref{volvo 240-serien} och 740\ref{volvo 740}.

\ditem[Japan]\label{japan}
 är två öar längst bort i Asien, om man mäter från London som man brukar. Där bor de flesta av världens japaner i nybyggda höghus av gummi eller i anrika samurajtempel av trä. När som helst kan hela skiten sopas bort av en flodvåg eller jordbävning så japanerna lever sina liv i ett sinnessjukt tempo för att ändå hinna med så mycket som möjligt. Dom har typ växlar på rulltrapporna och ris som kokar sig självt efter att man svalt det och robothundar som bara behöver rastas en gång om året och såna sjuka grejer. Riktigt hektiskt. Konjekturalt\ref{konjektural} nog har detta stressande gjort att typ alla japaner blir 150 år gamla. Ekonomin består främst av guldpengar som samlas in i tv-spelen \textit{Super Mario Bros} 1-3. Utöver detta finns även yrkena \quotetext{valforskare} och kamikazepilot, som också verkar vara ganska stressigt.

\ditem[Je ne sais quoi]\label{je ne sais quoi}
 (franska, ungf. \quotetext{det där lilla extra}, uttalas: /\textipa{Z@n@ sE 'kwa}/) är ett begrepp som syftar till att uppmärksamma den där lilla obeskrivliga som gör något perfekt. Blixten i AC/DCs logga, gitarrsolot på \textit{Warmachine}, pungen under pelikanens \ref{pelikan} näbb, avloppsdragningen på engelska tegelhus. Alla utgör de på sitt eget sätt det där särskilda som gör varje uppräknat fenomen fascinerande unikt. Eftersom den amerikanska medelklassen tror sig vara jordens medelpunkt kan den inte ett ord franska och har därför uppfunnit den egna termen \quotetext{X factor} istället, vilket hade varit jävligt töntigt om det inte vore för att Iron Maiden gjort en skiva med det namnet.

\ditem[Jeansröv]\label{jeansroov}
 Det som nittio procent av all amerikansk rockmusik från 80-talet handlade om.

\ditem[Jens]\label{jens}
 är ett mansnamn som är relativt vanligt hos personer som är med på olika sorters läger, så som pingisläger, scoutläger och skidläger, och där är föremål för ens odelade irritation.
Namnet är en korrumperad form av ordet \quotetext{jeans\ref{jeansroov}}.

\ditem[Jerry Williams]\label{jerry williams}
 Sveriges\ref{sverige} motsvarighet till Lemmy. Har turnerat sedan koppardaler var en accepterad valuta\ref{valuta} och skiter i hur ljudet låter så länge det svänger. Allt du vet om rock har du lärt dig av Jerka.

\ditem[Jesaja 36:12]\label{jesaja 36:12}
 Men Rab-Sake svarade: \quotetext{Är det då till din herre och till dig, som min herre har sänt mig att tala dessa ord? Är det icke fastmer till de män som sitta på muren och som jämte eder skola nödgas äta sin egen träck och dricka sitt eget vatten?}

\ditem[Jesus]\label{jesus}
 Kristus (f.0 - d.33) var son till gud\ref{gud} och \quotetext{Jungfru} Maria. Med sådana inflytelserika föräldrar var det naturligtvis inget problem för Jesus att komma in på den tidens byggprogram och sedan att få jobb som snickare. Jesus gick som prao och handjagade\ref{handjagare} ett tag men fick sedan arbete som informationsansvarig på sin farsas firma och reste runt på firma-åsnan och höll presentationer och föredrag för judar och fariséer på konferensmiddagar. Vid ett sådant tillfälle ska Jesus ha bjudit på odeklarerade mat- och starkvaror. De romerska myndigheterna hade tampats med detta utbredda problem i flera år och var nu så jävla less att de ville statuera ett exempel. Man ställde därför Jesus inför rätta, men än en gång kom farsan till undsättning och såg till så Jesus fick en plats som vice VD i vad som då blev familjeföretaget.

\ditem[Johan Skytte]\label{johan skytte}
 (1577-1645) var friherre och allmän vis man. Han undervisade bland annat Gustav II Adolf, men glömde ge denne lektionen om vikten av att undvika krig i dimma. Skit som händer. På 1600-talet gick det bra att syssla med lite vad som helst bara man var smartare än allmogen, eller i alla fall låtsades att man var det. Johan Skytte nyttjade detta privilegium till att, bland mycket annat, inrätta Skytteanska skolan i Lycksele. Skolans uppgift var att utbilda missionärer som hade till uppgift att konvertera så många samer som möjligt, och på det viset kolonialisera hela Umeå Lappmark. Det lyckades med facit i hand ganska väl.

\ditem[Johann Neumann]\label{johann neumann}
 , född 10 september 1949 i Österrike. För många känd som \quotetext{Iprenmannen}, men framförallt ihågkommen för sin fantastiska rollprestation som \quotetext{Ödlan} i \textit{Jönssonligan på Mallorca}. Då det ibland kan gå något decennium innan Neumann hittar ett nytt filmmanus som är bra nog för att han ska ställa upp händer det att han får det lite kärvt ekonomiskt. Därför är han även utbildad konservator och urmakare. Två typiskt österrikiska sysselsättningar som med lätthet kan kombineras om en kund till exempel beställer en uppstoppad uv\ref{uppstoppad uv} med tidtagning.

\ditem[Johannes Brost]\label{johannes brost}
 Sveriges grand old man när det kommer till a-hootin' and a-hollerin'. Började på 60-talet med att supa med Rolling Stones och fortsatte med det fram till det glada 80-talet när Stones började suga och han gick över till att äga upp\ref{aegmaestare} alla andra i \textit{Gäster med gester} istället. Efter en hektisk inspelning av succéfilmen \textit{Black Jack} behövde Brost varva ned och tog jobb som bartender på en finlandsfärja där han stannade i nästan 10 år. Som om inte allt detta vore nog har han dessutom med sina åtta gånger svenskt rekord i att medverka i \textit{Fångarna på fortet}.

\ditem[John Kellogg]\label{john kellogg}
 var en amerikan som är mest känd för att ha uppfunnit konceptet frukostflingor. Vad han är mindre känd för är att han var en ihärdig förespråkare för yoghurtlavemang, någon som han menade lämnade tarmen \quotetext{skinande ren}. Kellogg tyckte även att man borde hindra flickor från att onanera genom att utsätta klitoris för frätande syra. För pojkar menade Kellogg att det var lämpligt att sy ihop förhuden. Han var även, föga förvånande, vegetarian.

\ditem[Johnny Takter]\label{johnny takter}
 (f. 1958) är en svensk travkusk och alla spelares skräck. Tippar man Takter som vinnare kan man ge sig sjutton på att han kör bort sig, men om man räknar ut honom lyckas han alltid vara en mullängd före tvåan. Därav uttrycket \quotetext{vara lika opålitlig som Johnny Takter}. Han ser ut som en vanlig travkusk med hjälm och ridstövlar och allt sånt där, men ändå är det något som inte stämmer. Kanske har han en enäggstvilling som heter Assar Takter som tar hans plats ibland och kör som Farmor Anka. Eller så hatar han det monetära samhället och går emot oddsen med flit. Eller så har han helt enkelt bara lite för roligt på förfesterna innan loppen ibland och råkar dricka för mycket hemohes.

\ditem[Jolle]\label{jolle}
 En jolle är en liten båt man hoppar över till om man är på en stor båt som börjar sjunka. Den är jätteliten så alla får inte plats i den, tyvärr. Men så är det ibland i det jordliga livet.

\uline{Alternativa betydelser}

Jolle kan också syfta på en marijuana-cigarett\ref{cigg}. Den räcker oftast till alla som vill ha och kan paradoxalt nog vara anledningen till att båten sjunker.

\ditem[Jonathan Guy]\label{jonathan guy}
 var den första engelsmannen att födas i Kanada. Han var son till Nicholas Guy och dennes fru (namn tyvärr okänt), vilka tros vara de första engelsmännen att göra barn i Kanada.

\ditem[Jordbruksort]\label{jordbruksort}
 En jordbruksort är ett antal åkrar där det också finns en väg och minst två postlådor.

\ditem[Joseph Lucas]\label{joseph lucas}
 (1834-1902) var en brittisk affärsman som genom sin firma Lucas Electric tillverkade elkomponenter åt finbilsmärken\ref{putsbilar} som Rolls-Royce, Jaguar, MG, Rover och Triumph. Vissa läsare känner honom förmodligen bättre vid hans smeknamn \quotetext{mannen som uppfann mörkret}. Som alla andra brittiska uppfinningar var hans elsystem nämligen onödigt komplicerade och sämre än exempelvis tyska motsvarigheter. Bland annat använde sig Lucas av positiv jord, vilket gjorde att elsystemen blev extra känsliga för korrision och lätt fick glappa kontakter (inte för att vi egentligen har en aning om varför negativ jord skulle vara bättre men någons farsa påstod det så det får man ju tro på). Till en början var detta inget större problem eftersom bilarna kördes av anställda chauförer som ofta hade en hel del dötid att meka på medan gentlemännen spelade hästpolo, hade möte med äventyrsklubben eller duellerade. Lucas själv avfärdade synpunkter på hans bilbelysning med att \textit{en gentleman kör inte bil efter mörkrets inbrott}. Lucas Industries levererade elsystem åt brittisk bilindustri in på 1980-talet då eldelen av firman såldes.

Lucas avled den 27 december 1902 i sviterna av den tyfus han ådragit sig efter att ha avböjt ett glas vin och istället druckit smittat vatten. Hans barn ärvde dock firman och såg till att den dåliga tekniken levde vidare.

\ditem[Jubal]\label{jubal}
 är en biblisk\ref{bibeln} figur som enligt samma bok är stamfader till alla som spelar harpa och röker pipa. Att inte fler frifräsare predikar om honom tycker vi på Nissepedia\ref{nissepedia} är ganska B.

\ditem[Juridisk rådgivare]\label{juridisk raadgivare}
 När man tittar på en film från USA är det alltid poliser\ref{polis} med, och så snart någon blir gripen kräver hen alltid att få ringa sin advokat. Vem fan har sin egen advokat? Advokater är ju snordyra och ber dig dra åt helvete\ref{helvetet} så snart du slutar betala. Den som av någon anledning behöver gå i strid med lagen gör istället klokt i att skaffa sig en juridisk rådgivare. Till skillnad från advokat är juridisk rådgivare nämligen inte en skyddad titel så vem som helst kan verka inom skrået utan att behöva förhålla sig till dammiga idéer såsom hederskodex och yrkeslegitimation. Generellt behöver juridiska rådgivare inte heller kvitto på sina tjänster utan tar gärna ersättningen i trivselskrot\ref{trivselskrot}, snus eller kattungar\ref{kraeftbete}. De flesta har ganska mycket dötid att göra av med så det behöver inte bli så fasligt dyrt.

Storheten hos en juridisk rådgivare ligger kanske inte främst i sin kunskap om paragrafer och vägledande domar utan mer i sin känsla för att ta ut svängarna och uppmuntra till sådant som andra inte skulle ha tänkt på. Den juridiska rådgivaren är dock ett tveeggat svärd och man bör vara beredd på att uppdragen ofta följer modellen \textit{ett steg bak, två steg fram}. Man bör också vara beredd på att framstegen kan te\ref{te} sig tvetydiga för utomstående. Till exempel när din juridiska rådgivare deklarerat att du inte accepterar löneförhöjningen på 2,5\% utan vill ha en ökning i samma takt som industriavtalet, får detta beviljat för att sedan inse att industriavtalet är lägre än 2,5\%. Då är det osäkert vad man vunnit av att göra som man blivit åtsagd av rådgivaren. Observera att din juridiska rådgivare aldrig kan hållas ansvarig för dåliga råd. Som man bäddar får man ligga och sover man i sovsäck får man sällan ligga.

\ditem[Jurij Gagarin]\label{jurij gagarin}
 (1934-1968), sovjetisk stridspilot och kosmonaut, blev den 12e april 1961 den första människan i rymden. I urvalsprocessen, där Gagrin ställde upp mot 20 andra hoppfulla ansökande, stod till sist Gagrin mot German Titov, som dessvärre fötts in i ett medelklasshem och därför fick stanna hemma på jorden medan Gagarin garvandes körde \quotetext{no hands}-grejen  någonstans högt, högt ovanför stilla havet. Gagarin var nämligen son till kolsjos-bönder\ref{boonder}, så han passade perfekt för jobbet tyckte kommunistpartiet. Tråkigt nog dog Gagarin i en flygkrasch för flera år sen. Vila i frid kamrat.

\ditem[Järnspett]\label{jaernspett}
 är ett gammalt beprövat verktyg som man kan använda till allt mellan himmel och jord - bland annat att göra hål i backen, använda som tyngd vid tillverkningen av pressgurka och kasta på grannar. Men dess verkliga syfte är att luta mot glappa dörrar till vedbodar, ladugårdar och dass.

\ditem[Järvhägn]\label{jaervhaegn}
 En del människor har så mycket att göra att dom inte har tid att lönearbeta. Smörjgropen måste gjutas om eftersom den första blev för bred och döptes om till fågelbad, det går att spara flera hundra på att göra sitt eget snus hemma i ugnen, Biltema har öppnat nytt i Borlänge och bjuder på korv och kaffe, osv. Tiden räcker helt enkelt inte till för att även dra in nytt kapital till hushållet.

Känner du igen dig i beskrivningen ovan är det bästa du kan göra att ta pengarna du fick i skrotbilspremie och investera i ett begagnat viltstängsel. Åk ut en bit i skogen och spika upp skiten i närheten av ett älgtorn, sen är det bara att fara hem och sätta upp en skylt bredvid postlådan där du noggrannt textat JÄRVSAFARI. Förr eller senare kommer någon stanna till och fråga om du har öppet. Stäng då av vinkelslipen och avsluta modifieringen av släpkärran senare. Istället vandrar du fram till staketet och förklarar att du egentligen har stängt för säsongen men är villig att göra ett undantag just idag eftersom järvarna blivit utfodrade på morgonen och då brukar vara extra aktiva. Eftersom säsongen är över har du kopplat ur kortläsaren men kontant betalning går bra. Åk före ut i skogen och peka på älgpasset innan du förklarar att du har ett snabbt ärende att uträtta. Åk hem till Marklunds och stanna där till sent. Om turisterna mot förmodan kommer förbi ditt hem även dagen efter beklagar du att dom inte fick någon utdelning men att järvar är väldigt skygga djur så det går inte att garantera något ens när dom befinner sig i hägn.

\ditem[Jättemyrslok]\label{jaettemyrslok}
 (\textit{Myrmecophaga tridactyla}) är ett djur inom familjen myrslokar, den största faktiskt. Den bor antingen i träskmarker eller på savannslätten\ref{slaett}. Med svansen inräknad kan den blir nästan två meter lång. Den har inga tänder men den kompenserar detta med en jättelång tunga, som den använder till att äta myror med. När en unge föds ställer sig mamman på den och börjar riva den med sina enorma klor, precis som hon gör när hon ska riva upp en avokado. I Sverige finns det två vuxna jättemyrslokar, dom heter Benita och Rozinski. 

Jättemyrslokens avsmalnande huvud gjorde att den är omöjlig att hålla kopplad och den har därför inte hunnit bli så populär bland barnfamiljer än.

\ditem[Jättemyrslokssele]\label{jaettemyrslokssele}
 är en uppfinning som gör det möjligt att hålla en jättemyrslok\ref{jaettemyrslok} kopplad. Selen uppfanns av Jan Guillou i samarbete med ett forskningsinstitut i Schweiz\ref{forskningsinstitut i schweiz} och har gjort att jättemysloken blivit sjukt populär hos barnfamiljer.

\ditem[Jävelskap]\label{jaevelskap}
 är sånt man kan hitta på när man vill busa. Ibland händer det att det drabbar andra och det är lite olyckligt men ofta oundvikligt. Det första kända jävelskapet utfördes av en lokal byfåne i Chan Chan, nuvarande Peru\ref{peru}. Byfånen lär ha fångat ett bältdjur och sedan gömt det i sin farfars mockasinlåda.

\uline{Välkända jävelskap:}

\begin{itemize}
\item Lasse Brandeby lurar Robert Gustavsson att dricka sig full i tv-serien Rena Rama Rolf.
\item Jöns Skalman får i uppdrag att rita ett första utkast till Vasaskeppet och lägger till ett extra kanondeck. Ingen upptäcker sprattet innan det är för sent.
\item Den första januari 1994 intar över 3000 beväpnade zapatister en rad olika städer och byar i Chiapas, Mexiko, för att dagen efter lämna dem igen.
\item Noshörningskungen Rataxes låter aldrig Babar vara ifred.
\item Umeå kommun anlitar balticgruppen för att skämta till det lite med sina invånare inför \textit{Kulturhuvudstad 2014}. Ingen är riktigt klar över vad skämtet består i. Men det är ju så det är med postmodern kultur så allt är i sin ordning. Att pengarna går in i fickorna på familjen Olsson är en del av det postmoderna skämtet.
\end{itemize}


%%%%%%%%%%%%%%
\newpage
\null
\\
\null
\\
\Huge
K
\normalsize
\\
\null
\\
\null
%%%%%%%%%%%%%%



\ditem[Kaffekask]\label{kaffekask}

Recept på rusdrycken kaffekask.

\begin{enumerate}
\item Fixa kaffe.
\item Lägg ett mynt i en kopp, förslagsvis en tia\ref{tia}
\item Slå i kaffe tills myntet inte syns.
\item Slå i klar sprit tills myntet syns igen.
\item Fyllna till.
\end{enumerate}

\ditem[Kaknästornet]\label{kaknaestornet}
 är tillsammans med Globen\ref{globen} Sveriges svar på Paris' Eifeltorn, Londons Big Ben och Kuala Lumpurs Petronas Twin Tower. Med sin majestätiska höjd på 155 meter tornar byggnaden upp sig över Ladugårdsgärdet i Stockholm. Det invigdes 1967 av den dåvarande kommunikationsministern Olof Palme\ref{olof palme} och blev snart ett attraktivt besöksmål för turister fråm hela världen. Tornet är brunt och gjutet av betong. Förutom att placera Stockholm på världskartan och fungera som ett av rikets viktiga och identitetsskapande landmärken är tornets funktion att fungera husera landets sämst belägna restaurang. 

\ditem[Kalaskula]\label{kalaskula}
 kallas en mage som är alldeles uppspänd, hård och rund. Kalaskulan är ett attribut som i dagens samhälle ofta innehas av herrar i 40-årsåldern. Ursprungligen var kalaskulan en sorts dyrkan av urmodern, Gaia. I jägar- och samlarsamhället skaffade sig alla som kunde en ordentlig kalaskula för att hylla kvinnans fertilitet och förmåga att skapa nytt liv. När en medlem (man eller kvinna) i stammen skaffat sig en mage som motsvarade den hos en kvinna i åttonde månadens graviditet samlades stammen för att dansa och äta. Man dansade så ogenerat som man bara kunde göra i en tid utan modebloggar och popfanzines och förlustade sig efteråt på saltat grävlingkött, blötlagd lök och dryck gjord på jästa sviskon.

Av oklara anledningar har den kvinnohyllande delen i att ha en kalaskula försvunnit under historiens gång. Snarare är kalaskulan idag en hyllning till den fetlagde gubben, då en ogenerad övervikt hos herrar sägs signalera \quotetext{pondus}, samtidigt som den hos en kvinna signalerar \quotetext{stigmatiserad bärshagga}.

\ditem[Kalixare]\label{kalixare}
 En kalixare är att kupera en kortlek genom att ta ut en bunt kort ur mitten av kortleken och placera dessa längst upp. Det är annars brukligt att kupera genom att lägga den nedre halvan av kortleken längst upp. En bra kalixare har lite snärt och flärd.

\uline{Etmyologi}

Kalixare heter som det gör för att människor från Kalix brukar kupera kortlekar så, i alla fall på jägarregementet i Kiruna på 50-talet.

\ditem[Kalle anka]\label{kalle anka}
 är ett sätt att klä sig där överkroppen är påklädd och underkroppen bar. Klädseln är flitigast använd på festivaler och svensexor men passar i princip till alla tillfällen. Motsatsen till att klä sig Kalle Anka är att klä sig Mimmi Pigg\ref{mimmi pigg}.

\ditem[Kalle Ankas pocket]\label{kalle ankas pocket}
 (hädanefter refererad till som KAP) är en tidskrift och systerpublikation till \textit{Kalle Anka\ref{kalle anka} och CO}. I KAP erbjuds dock betydligt mer djuplodande porträtt av Kalle och hans vänner, och det ställs därmed också större krav på läsarens intellekt. I KAP räcker det inte att bara titta på bilderna, som i en vanlig trerutorsstripp där kusin Knase halkar på ett bananskal\ref{banan}. Istället förväntas läsaren själv delta i skapandet av upplevelsen genom att läsa i pratbubblorna och memorera vad som hände mer än tio sidor bakåt. I gengäld blir helhetsintrycket starkare och trogna läsare som verkligen ansträngt sig berättar om spirituella upplevelser där de verkligen trott sig sitta i gamla 313 eller stulit en paj hos Farmor Anka. De lite tjockare pärmarna på KAP gör också att de klarar sig betydligt längre än andra magasin inne på muggen innan de blir vattenskadade. Man får heller inte glömma att den hängivne samlaren, utöver en litterär resa, får en mycket dekorativ bokhylla om böckerna ställs i kronologisk ordning med ryggarna mot betraktaren. Likt atomerna i en DNA-sträng bildar de nämligen en gemensam helhet.

\ditem[Kanadensisk frack]\label{kanadensisk frack}
 Jeans, jeansskjorta och jeansjacka.

\ditem[Karbinhake]\label{karbinhake}
 En karbinhake användes förr i tiden, precis som namnet antyder, för att haka fast gevär (karbiner) på. Idag används karbinhaken i stort sett uteslutande av veganpunkare vilka genom att fästa sina nycklar i haken och sedan haka fast den i sina byxor använder nycklarnas högljudda rasslande som ett sorts parningsläte, då användarna hoppas att oväsendet ska dra omgivningens uppmärksamhet till deras oansenliga ändalykter. 

\ditem[Karel Gott]\label{karel gott}
 (f. 1939), även kallad \quotetext{The Golden Voice of Prague} och \quotetext{Sinatra of the East}, är en tjeckisk musiker som kom på plats 13 i Eurovisionschlagerfestivalen 1968.

\ditem[Katolik]\label{katolik}
 En katolik, eller papist som man också kan kalla dem, är en kristen person som inte vill göra avkall på två av sina största intressen, nämligen att dricka alkohol och ha så mycket oskyddat sex som är praktiskt möjligt. Katoliken ägnar sig åt detta i vanlig ordning och biktar sig sedan för sin präst och blir då förlåten för att ha sex och vinfylla\ref{vinfylla} som huvudintressen. Den katolskt kristne kan kanske ses som en mindre seriös troende av sina protestantiska bröder och systrar, men detta är inte helt sant. Katoliken är nämligen en engagerad motståndare till homosexualitet, abort, kondomer, icke-traditionella familjekonstellationer och mycket annat som, kan man väl tycka, det krävs lite jävlaranamma för att vara emot så här ett decennium efter millenieskiftet. Katoliken är dock bergfast i sin beslutsamhet när det gäller dessa saker, bara inte just när det gäller att dricka alkohol\ref{alkohol} och knulla runt med folk (av motsatt kön).

\ditem[Katt]\label{katt}
 (obestämd form singular; plural kateter) är ett fyrbent djur inom familjen morrhåringar. Mången katt har lurvig päls vilket gör det lätt att även som lekman kunna åldersbestämma den: ju tovigare päls desto äldre. Katten föds genom att modern efter befruktning hostar upp ungen i form av en hårboll som sedan planteras i kattsand för att ligga där och gro några veckor. Träffar man en katt bör man tilltala den Misse eller Jamis då arten haft väldiga svårigheter att inrätta sig i 1900-talets Du-reform. På landet är följande kattnamn frekvent förekommande: Fräsen, Sotis, Maja / Majsan, Pelle, Katta, Kattskrället, Kattfan samt alla namn som innehåller ett eller flera S då katter lystrar till detta.

\ditem[Kattbrosch]\label{kattbrosch}
 Kattbroscher är ofta av lackat trä med en tydlighet över sig. Det kan vara ett par distinkta morrhår och eller en färgglad detalj. Kattbroschtillverkarna vänder sig i första hand till den konstnärliga och lekfulla kundkretsen.

\ditem[Kattguld]\label{kattguld}
 är en populär sulfidmineral som finns lättillgänglig över större delen av jorden. Precis som vanliga katter\ref{katt} och vanligt guld framstår kattguld vid en första anblick som väldigt fint. Men vid en närstudie märker man att det är rätt överskattat och inte värt besväret. Carl von Linné\ref{carl von linné} var så klart tidigt framme och forskade i kattguldets mysterium och kom bland annat fram till att det inte är särskilt hälsosamt att äta och att barn börjar gråta om man slänger bitar större än en knytnäve på dem. Den franske upptäcktsresanden Jacques Cartier (1491-1557) föddes tyvärr innan Linné upptäckt dessa egenskaper hos metallen och trodde att han hittat en förmögenhet när han under en resa i nuvarande Kanada seglade förbi en strand täckt av kattguldsklimpar. Han släpade hem en hel skeppslast till Frankrike där kungen blev måttligt imponerad vid upptäckten att han langat upp en brakmiljard för att Cartier skulle segla halva jorden runt och hämta hem stenar lika uppskattade som ringaren i Notre-Dame. Från denna utflykt härstammar de skämtsamma uttrycken \quotetext{guld från Kanada} och \quotetext{Fransk upptäcktsresande}.

\ditem[Kattsläkt]\label{kattslaekt}
 familjeband längre bort än kusinnivå.

\ditem[Kattstrypare]\label{kattstrypare}
 är en typ av plastband som är nästan lika användbara som gaffatejp. Bandet har en liten ögla i ena ändan, och när man drar den andra ändan igenom går den inte att dra tillbaka sen. Det är väldigt användbart om man vill spänna fast någonting men inte är så bra på att knyta - till exempel stänkskärmar, handtag, valurnor eller trumstativ. Men hur kan det komma sig att det heter kattstrypare? undrar kanske den vetgirige. Om det bara går att dra bandet åt ett håll kan man ju bara strypa en (1!) katt, sen är bandet oanvändbart. Det är förvisso sant, men plast är fortfarande så pass billigt att banden säljs i storpack och har ett styckpris på några få ören så det behöver inte bli så dyrt.

\ditem[Kelsey Grammer]\label{kelsey grammer}
 är en känd underhållare vars repertoar i huvudsak går ut på att imitera den parsisik-indiske postkoloniale teoretikern Homi K. Bhabha, vars son Satya Bhabha\ref{homi k. bhabhas son} många lärt känna och kommit att älska genom rikstelevisionen. Grammer har flera gånger turnerat med en uppsättning där han förklädd till Bhabha, iförd flaskbotten-glajjor och lösskägg, råkar ut för de mest dråpliga och slap-stickartade situationer. Bland annat fastnar \quotetext{Bhabha} med foten i en hink och spiller vindaloo på skjortslaget. I slutakten ser vi ett bejublat inhopp av Bhabha Jr. som sin fars bästsäljande bok, \textit{The Location of Culture} (1994). Under 2012 turnerade uppsättningen i Europa och sålde bland annat ut Millenium Arena och Holmsunds tropikhus\ref{holmsunds tropikhus}.

\ditem[Kent]\label{kent}
 är framförallt ett förnamn som är exklusivt reserverat för män. Det är en kortform av det äldre namnet Kentaur som betyder ungefär \quotetext{hästarsle}. Dessutom är det ett grevskap i Storbritannien där canterburyscenen skapade magisk progg på 60-talet, ett ciggmärke speciellt framtaget för den som bara vill röka fimpar och ett anskrämligt popkoncept som går ut på att sjunga slumpmässigt utvalda ord ur SAOL med ljus stämma - i Eskilstuna.

\ditem[Kepsar med olika företagslogotyper]\label{kepsar med olika fooretagslogotyper}
 är klassiska accessoarer för den som vill framhäva en avslappnad och skön stil. Vilket företag som återges på kepsen spelar mindre roll men logotypen bör gärna vara utfasad och inaktuell, exempelvis rekommenderas Lantmanna istället för Odal butik (Numer är båda namnen dock godtagbara då företaget haft den dåliga smaken att åter byta namn, denna gång till Granngården). Skärmen bör antingen helt sakna böjning eller ha så kallad \quotetext{sadeltaks-böj}. Tidigare hade många kepsar genomskinliga plastskärmar men detta har tyvärr försvunnit i modern tid. Passformen justeras bakpå av ett elastiskt band på billigare modeller medan större företag föredrar att justera sina kepsar med kardborre eller plastknäppning. Blir man erbjuden en keps med företagslogotyp bör man alltid ta emot den då man aldrig kan få för många. Detta kan te sig något komplext då man helst alltid bör bära samma keps år ut och år in. Men så kan det vara.

\ditem[Kerry King]\label{kerry king}
 är en förebild för tunnhåriga män världen över. När han inte är det spelar han, till skillnad från Larry David, i Slayer.

\ditem[Keytar]\label{keytar}
 är ett musikinstrument som kombinerar digitalsynthens plastiga bingo lotto-ljud med gitarrens utseende och attityd. Tack vare dess axelband kan keytaren hängas på musikanten, som nu får samma rörelsefrihet som en gitarrist. Det inbjuder exempelvis till att gunga i takt med basisten och gitarristen på refrängen i \textit{Rocking all over the world}, eller att gunga med i groovet på en bingo lotto-version av \textit{Smoke on the water} (RIP Jon Lord). Keytarens främsta uppgift är nämligen att underlätta för musikanten att på ett naturligt sätt få sträcka på benen ibland. Keytarmusiker är påfallande ofta äldre herrar som spenderat större delen av karriären sittandes på en svettig skinnpall lite i skymundan bredvid trummisen. De feta åren är sedan länge förbi men gubbens kroppshydda\ref{kroppshydda} växer däremot stadigt. Så åderförkalkningen är ett faktum och inte blir det bättre av att turnera i halvfulla ishallar. Framåt sista tredjedelen av setet halar keytarmusikern därför fram sin keytar och stegar ut på scenen för att efter 25 år äntligen få känna på det värmande rampljuset.

Det tydligaste exemplet är naturligtvis Robert Wells som varit dålig i mer än ett decennium och därför borde fått sjukpension för länge sen. Men en nitisk handläggare på Försäkringskassan lusläste de nya sjukskrivningsreglerna och skickade på stackaren en keytar istället.

\ditem[Kicki Danielsson]\label{kicki danielsson}
 En drink bestående av billigt vitt vin (helst ur bag-in-box) och 7-up (uttalas /ʃup:/), gärna med en falukorvssnärt\ref{snaert} på kanten. Den är extremt lättdrucken och bäddar för en helt vansinnig fylla, som i nio fall av tio slutar med att du står och skrålar om exotisk frukt tillsammans med en förtidspensionerad tandläkare med konstnärsambitioner. 

\ditem[Kikare]\label{kikare}
 är en av de mest psykedeliska uppfinningar den mänskliga hjärnan hittat på. Genom att titta på en sak genom en kikare blir den (saken) genast mycket större - fast den egentligen inte är det. Tar man bort kikaren går allt tillbaka till det normala och saken får sin ursprungliga storlek igen. Men tittar man i kikaren igen så, vips!, är saken jättestor igen. Det behövs inte ens någon konverteringstid. Enligt den tyske\ref{tyskland} paleontologen\ref{paleontologi} Wilhelmus Wurst kan detta mycket väl vara orsaken till att dinosaurierna var sådana jättar. Alla dinosaurier gick runt med kikare och till sist upphörde verkligheten att existera och ersattes av jättevärden.

\uline{Missbruk}

Om man vänder kikaren bak och fram och tittar blir alla saker istället jättesmå.

\ditem[Kineseri]\label{kineseri}
 är ett ålderdomligt uttryck som innebär att krångla till saker i onödan. Ordet härstammar från uppfattningen att kineser strävar efter att göra det lätta extremt svårt och krångligt. Även om ordet idag kan framstå som korkat på gränsen till rasistiskt, går det inte att förundras över saker som den Kinesiska muren. Istället för att bjuda över mongolerna på lite ödla på pinne\ref{oodla paa pinne} och té för att prata om saken, skulle kineserna prompt bygga en fet jävla stenmur runt hela sitt land. Kina är för den som missat det typ hur stort som helst.

\ditem[Kinesiska muren]\label{kinesiska muren}
 är en rap av MC Evert och handlar om hur Chi-Huang-Ti, kung av Tsin, lät bygga den kinesiska muren. Texten rappas över minimalistiska beat\ref{beat} bestående av bjällror på fingrarna och gong gong. Flera teorier florerar om att texten i själva verket är en dold kritik mot modernare härskare. Vidast spridd är teorin att texten i själva verket behandlar Göran Perssons instiftande av myndigheten Forum för levande historia.

\ditem[King Edward]\label{king edward}
 är ett brittiskt mansnamn som betyder ungefär \quotetext{kung över alla som heter Edward}. Vanliga smeknamn är \quotetext{Kingen} och \quotetext{Eddie baby}.

Det är också namnet på en könlös potatissort. Läckergommen Carl von Linné beskrev rotfrukten i ett brev till vännen Peter Artedi med orden: \quotetext{Thänna potät äro mycken ghod att ät. Ock brennvin brygg när käringa trät}.

\ditem[Kinneviksliljan]\label{kinneviksliljan}
 Tidigare mera känd som Flikenliljan (1865)och efter det som Fagerstaliljan (1886). Man har felaktigt antagit att liljan föreställer Ormbärsörten, \textit{Paris Quadrifolia}. Senaste forskning visar dock att \quotetext{liljan} är flertalet blodkärl som förenas i en maffig propp i aorta.

\ditem[Kir]\label{kir}
 - enkelhet med fransk finess och svensk tradition.

\ditem[Kissemiss]\label{kissemiss}
 Urin\ref{urin} på golvet bredvid klosetten.

\ditem[Kiwi]\label{kiwi}
 är en ganska menlös, men gullig, fågel. I motsats till typ alla andra fåglar kan den inte flyga, inte ens hoppa kan tänkas. Den är också utrotningshotad tack vare att dess bruna och ludna ägg är så söta och fulla av vitaminer.

\ditem[Kladdi Mittänän]\label{kladdi mittaenaen}
 är Finlands sämsta konstnär. Hans tavlor är så dåliga att Prof. Etienne\ref{prof. etienne} lät publicera bilder på flera verk för att använda som praktiska övningar i sin bok \textit{Barnagans förträffliga pedagogik\ref{barnagans foortraeffliga pedagogik}}. Till försvar för hans misslyckade livsval lär hans mor ha myntat begreppet \quotetext{det är inte utsidan som räknas}.

\ditem[Klasse Möllberg]\label{klasse moollberg}
(f. 1948), måste nog de flesta hålla med om, passerade sitt zenith med det klassiska barnprogrammet Trazan och Banarne som han gjorde tillsammans med Lasse Åberg, och hans kändisskaps sol har stadigt dalat sedan dess, med vissa perioder av tillfällig uppgång så som under 2009 då Möllberg medverkade i TV4s underhållssatsning \quotetext{Hjälp, jag är med i en japansk tv-show}. Många TV-tittares favorit-\quotetext{asse} har därför varit \textbf{L}asse och inte \textbf{k}lasse. Detta kan vissa andra tycka är Klasse oförtjänt, inte minst eftersom Klasse till skillnad från Lasse enligt ett konkurrerqande uppslagsverk har svensk, fransk och USAsk skidlärarexamen. Vidare är han en hejjare på gura - det var ju trots allt Klasse som svingade yxan i Electric Banana Band tillsammans med Janne Schaffer.

\uline{Kontrovers}

Vintern 2004 uppstod rykten om att Trazan och Banarne skulle vara ett näsligt varumärkesintrång på britten och dubstep-fantasten Edgar Rice Burroughs klassiska böcker om Tarzan. I en utdragen rättsprocess som bevakades väl i etermedia avkrävdes Möllberg stora summor i skadestånd av Burroughs dödsbo. Han friades dock i högsta domstolen och den anklagande sidans advokater fick näsor och öron avskurna, i enlighet med djungelns lag.

\ditem[Klassikerbilssjälvmord]\label{klassikerbilssjaelvmord}
 Ett klassikerbilssjälvmord är en mycket vanligt förekommande mansdöd. Många är de män som när fyrtio- eller framförallt femtioårskrisen inträffar skaffar sig ett bilvrak med klassikerstatus. Garaget städas ur och proffsverktyg (migatronicsvetsmaskin, speedglasshjälm och en stor uppsättning makitaverktyg) för en förmögenhet införskaffas. Nu påbörjas arbetet med en sprudlande entusiasm. Motorn monteras ned och tvättas in i minsta detalj, ramen lyfts och underredet skrapas rent från underredsmassa. I stort sett hela bilen monteras ned och placeras i godislådor och banankartonger. Ungefär vid detta läge börjar dock entusiasmen tryta, och problemen hopa sig. Man har skaffat sotningssats till en 57a när man i själva verket äger en 55a och nyansen på den dyra lackfärgen man köpt är ju en liten ton fel. Det var inte så självklart och enkelt som det från början verkade att renovera en klassisk bil till nyskick. Framförallt var det kanske inte det man behövde för att komma igenom sin femtioårskris. I själva verket så blir den bra mycket värre och grubblerierna och prestationsångesten växer lavinartat. Det hela känns oöverskådligt och utmynnar i en destruktiv spiral. Det är här jaktgeväret eller i vissa fall hängsnaran kommer in i bilden. Mången kortklippt permanentad hustru har funnit sin älskade hängande i takbjälken bredvid en avstannad renovering.

\ditem[Klo]\label{klo}
 är foten på en uv\ref{uv}. Den består av spetsiga, farliga tår som kan slajsa upp fejset på den som vill ha bråk hur lätt som helst. Den kan också lyfta upp saker, typ sköldpaddor att släppa i huvudet på Aischylos\ref{de gamla grekerna}. Då kallas det gripklo. Det kallas även gripklo om det är klon på det mytologiska monstret grip som åsyftas. I Skellefteåtrakten, där man alltid ska va så jävla speciell, betyder dock klo skithus. Så nu vet ni.

\ditem[Kloakdjur]\label{kloakdjur}
 är benämningen på en undergrupp av däggdjur. Det som särskiljer kloakdjuren från de andra däggdjuren är att kloakdjuren lägger ägg\ref{aegg}. Anledningen till att de kallas just kloakdjur är att honornas urogenitala system och anus mynnar ut i samma öppning. Öppningen började först skämtsamt kallas kloaken av Carl von Linné, för att sedan bli, inte bara öppningens namn, utan benämnande för hela subgruppen.

Idag finns endast två sorters kloakdjur, myrpiggsvinet och näbbdjuret. Båda toppade flera gånger Carl von Linnés\ref{carl von linné} lista \quotetext{Thät mäst bhuskis diur uti thenna mystiska wärld} som årligen publicerades i \textit{Then Swänska Argus}.

\ditem[Klä ut sig till ett djur]\label{klae ut sig till ett djur}
 Att klä ut sig till ett djur är ett av världshistoriens äldsta spratt. Sprattet går ut på att skämtaren iklär sig kläder eller en dräkt som påminner om ett djurs dräkt. Poängen är inte alltid att folk ska missta skämtaren för ett djur, utan att skämtaren ska framstå som lite lustig eftersom han eller hon ju i själva verket inte alls är ett djur.

\uline{Skämtet i mytologin}

Att klä ut sig till ett djur är ett väl använt skämt i mytologiska texter som Gilgamesh-eposet från Mesopotamien, i vilket Gilgamesh klär sig i ett lejonskinn, och i skandinavisk mytologi, i vilken halvguden Loke klär sig i fågelskrud och flyger omkring och således får de andra asarna att vråla av skratt. I Bibeln\ref{bibeln} återfinns berättelsen om spilevinken Jona som skojar till det genom att liksom \quotetext{klä sig} i en valfisk.

\uline{Skämtet i modern tid}

I modern tid har vi sådana exempel som när Lukas Moodysson tog emot ett pris för \quotetext{Fucking Åmål} på guldbaggegalan iförd ett slags diadem med djuröron på och ett spratt av fotbollslaget Arsenal i vilket några av lagets stjärnor iklädde sig djurdräkter.

\ditem[Klädsamt ful]\label{klaedsamt ful}
 Att vara klädsamt ful är när en person är ful, enligt konstens alla regler, men att det ändå lyckats bli deras \quotetext{grej}. Cronos i Venom är typexemplet av någon som är klädsamt ful. Sveriges\ref{sverige} motsvarighet till Cronos, Janne Schaffer, är ett annat slående exempel. Engelsmän är i regel, men med undantag, klädsamt fula.

\ditem[Knixa]\label{knixa}
 Att knixa är att rytmiskt och snabbt böja på knäna medan man står på stället.

\uline{Folk som knixar}

\begin{itemize}
\item Tjackisar
\item Strömstarar
\end{itemize}

\ditem[Knocking on heavens door]\label{knocking on heavens door}
 är en låt med den brittiska musikgruppen Mungo Jerry. Den skrevs ursprungligen av den amerikanske knarkhippien\ref{hippie} Bob Dylan, som åkte runt med en glapp elgitarr och spelade den 20 år tidigare. Det är det dock ingen som kommer ihåg idag. Låten återfinns på Mungo Jerrys skiva \textit{Together again} från 1981, varifrån den släpptes som singel. Den nådde bland annat förstaplatsen på Sydafrikas topplista, och denna historiska placering gjorde att Mungo Jerry nu kunde stryka a cappellaversionen, rapversionen, midiversionen och discoremixen av \textit{In the summertime} från sina greatest hits-skivor, vilkets sågs som en delseger för världens alla Mungo Jerryhatare\ref{mungo jerryhatare}.

\ditem[Kodnacke]\label{kodnacke}
 är ett skadligt värk- och anspänningstillstånd i halskotpelaren. Det uppstår när man återkommande, med huvudet i onaturlig vinkel, stirrar stint in i t.ex. tuggummihyllan på Konsum för att inte råka lära sig andras kortkoder och lösenord. Är man utbränd riskerar man dessutom att blanda ihop vems kod som är vems, och kanske få gå hem utan ketchup och pölsa.

\ditem[Kokt]\label{kokt}
 Blöt och sladdrig, istället för stekt och god. OBS. Gäller dock ej ägg\ref{aegg}.

\ditem[Kolonialdricka]\label{kolonialdricka}
 Länge var kolonialdricka per preference gin och tonic. Ibland kunde en och annan kopp te slinka ner i en sentimental brittisk strupe, drömmandes om afternoon tea någonstans i Kent\ref{kent}. På senare tid har kolonialdrickat Indian Pale Ale blivit populärt, främst bland hippies. 

\ditem[Kombinationsaffär]\label{kombinationsaffaer}
 En kombinationsaffär är en affär som erbjuder en ofta osannolik kombination av varu- eller servicekategorier. En legendarisk kombinationsaffär är Randolfos affär på Haga i Umeå. I den säljer han hårdrock\ref{haardrock} och akvariefiskar. Ett annat exempel på kombinationsaffär är \textit{Årsta mobilteknik och inredning} som säljer mobiltelefoner och stentroll. Anledningen till sådana kombinationer är att de ofta drivs av ett gift par som på ålderns höst, på grund av ett kognitivt fel, har bestämt sig för att driva affär. Maken sysslar med sitt stora intresse, till exempel mobiltelefoner, medan fruntimret sysslar med sitt, vilket ofta är stentroll. Ett annat riktigt skruvat exempel är \textit{Elvisboden} i Norberg som förutom att saluföra Elvis Presley-memoribilia också är begravningsbyrå.

\ditem[Kommunanställd småpåve]\label{kommunanstaelld smaapaave}
 Den kommunanställde småpåven är kort och gott en korrupt, rättshaveristisk, maktmissbrukande myndighetsanställd. Desto längre kommunen ligger från centrum (Stockholm\ref{stockholm} och det finansiellt blomstrande Skåne)\ref{skaane}, desto mer svängrum finns det för småpåven att göra sin grej i periferin. I många kommuner i bergslagen och Västerbotten finns berättelser om kommunalt anställda män och kvinnor som utnyttjat sina maktpositioner för att få fördelar, på samma sätt som byråkratin fungerade i Ryssland innan det stora frihetskriget 1917.

\uline{Exempel}

Låt oss säga att du heter Berit och bor i Bjurholm. Du skulle behöva extra sophämtning eftersom din alkisschäfer\ref{alkisschaefer} Kari producerar så mycket avfall. Så du kontaktar en byråkrat, låt oss kalla honom Dennis\ref{dennis}, i hopp om att få din ansökan behandlad så snart som möjligt. När du kommer hem på kvällen har du ett meddelande på din stationära telefon, där en flåsande Dennis frågar om det inte stämmer att du har en mjärde i Lill-armsjön som brukar ge väldigt bra avkastning? Dennis fortsätter och berättar hur mycket han gillar att äta smörstekt gädda med vitlöksås på torsdagar. Du äcklas men fortsätter lyssna. Dennis röst blir väsande när han säger att han kan göra det skönt för dig om du gör det skönt för honom, så att säga. Han kräver att få vittja din mjärde en gång i veckan, annars blir det ingen extra sophämtning. Du hinner knappt tänka att du ska skicka in meddelandet till polisen innan Dennis fortsätter genom att kvittra jovialiskt att han är supapolare med Göran\ref{gooran} på länsstyrelsen, och att de på skoj brukar utfärda order om att avliva husdjur under sina vilda spritfyllor\ref{spritfylla}. Och det vore ju trist om Kari råkade illa ut. Du somnar gråtandes.

Dagen efter går du igen till kommunen. Dennis sitter och snurrar på sin kontorsstol samtidigt som dokumentstrimlaren går på högvarv. Hade den varit mänsklig hade den svettats nästan lika mycket som Dennis, tänker du. Lysrörsskenet når knappt igenom Dennis överkamning, men det räcker för att ge hans huvud en glänsande discokulaliknande karaktär. Du säger att du går med på hans avtal. Han skrockar förnöjt och säger att ja, det brukar de göra. Du vänder dig om och skakar av vrede hela vägen hem. Du har blivit ägd av en kommunanställd småpåve.

\ditem[Kommunist]\label{kommunist}
 är ett skällsord som ofta används av tokliberaler\ref{tokliberal} och som etymologiskt betyder \quotetext{någon som frivilligt delar med sig} och \quotetext{någon som ogillar krig}. Med detta ord vill tokliberalen antyda att givmilda personer som ställer sig tveksamma till krig som affärsidé på ett eller annat vis är skyldiga till folkmord och förtryck.

Kommunist är motsatsen till småborgare.

\ditem[Kommunistglasögon]\label{kommunistglasoogon}
 Runda brillor, vilka skänker bäraren en aura av upplysthet. Han/hon har skådat ljuset som vanliga pantade knegare\ref{vanliga pantade knegare} inte kan se.

\ditem[Kommunslogan]\label{kommunslogan}
 Alla kommuner i Sverige\ref{sverige} värda namnet har en kommunslogan\ref{kommunistglasoogon}. Här kommer de bästa eller sämsta beroende på perspektiv\ref{perspektiv}.

\begin{itemize}
\item Alingsås - Utan oss vore korv och mos bara korv
\item Ska du handla i Ludvika imorgon? Handla i Ludvika idag istället!
\item Ljusterö - Just de'ru!
\item Fullt blås, Borås!
\item Kristinehamn - Staden med massor av p-platser i centrum
\item Motala - Why not?
\item Fagersta – Här får du livstid
\item Nordmaling - Sveriges godaste dricksvatten
\item Vindeln - En sista utväg
\item Bjurholm - Bäverstaden
\item Skellefteå - Inte bara en sjukdom
\item Malå\ref{malaa} - Snart vänder det
\item Norrtälje\ref{norrtaelje} - varken bra eller dåligt
\item Trollhättan - där drömmar dör!
\item Lilla Edet - en svag del av Trollhättan
\item Filipstad - inte bara den blåsta nazisten på youtube!
\item Robertsfors - Sveriges bästa kommun om cirka 85 år!
\item Ale - En härlig blandning av kemikalieindustri och riktigt pantade nazister!
\item Täby - entreprenörskap!! entreprenörskap!! entreprenörskap!! wohooouu uläääää!!!11!
\item Piteå - Lilla Frankfurt
\item Borås - Borås – of course not
\item Krokom - Center of Innovation
\item Sundsvall - Rasta hundens hemkommun!
\item Östhammar - kärnkraftverk och radioaktivitet är också kultur!
\end{itemize}

\ditem[Konjektural]\label{konjektural}
 Hypotetisk möjlig men ändå helt otänkbar. Som Anton Abele\ref{anton abele}, med andra ord.

\ditem[Konsum på Haga]\label{konsum paa haga}

\ditem[Konsumbutik]\label{konsumbutik}
 En Konsumbutik är en butik som drivs av Konsum och som innehåller allt man kan tänkas behöva. Cocktailpinnar, sandaletter, luftfuktare, svanväska, hackebiff, djungelolja, bingolotter, järnspett\ref{jaernspett}, picknickbog\ref{picknickbog}, gitarrlektioner, speedos\ref{speedos}, Kongo-fett, spikmatta, svag mat\ref{svag mat}, alkisschäfer\ref{alkisschaefer}, kubb, luktsudd, uvgodis\ref{uvgodis}, uvsvane\ref{uvsvane}, bil- och cykelparkering, svanskrove, transkrove, makadam, luffarschack, tråg\ref{traag}, frisedel, dagisplats. Allt finns i Konsumbutiken.

\ditem[Konventikelplakatet]\label{konventikelplakatet}
 är en förordning från den 12 januari 1726 som förbjöd bönemöten i hemmet (undantaget familjeandakter).

\ditem[Kopi Luwak]\label{kopi luwak}
 är världens dyraste kaffe för att bönorna har gått igenom en mungos tarmsystem. Hur dyrt är det då, frågar man sig? Jo, det är jättedyrt, närmare bestämt 6100 kr/kg.

\ditem[Kopparorm]\label{kopparorm}
 är den fjärde ballaste sortens orm. Skalan för balla ormar är precis som de olympiska medaljerna klassificerad efter metallens ballhet. Guldplatsen har spottkobran eftersom den har coolast tänder, avståndsattack och hypnotisk förmåga. Silverplats har anakondan eftersom den är så stor, svart, sväljer saker utan att tugga och har fått en Nicholas Cage-film\ref{nicholas cage-film} uppkallad efter sig. Brons tar den svenska huggormen på grund av sitt uppkäftiga sicksack-mönster på ryggen. Kopparplatsen innehas av den så kallade kopparormen, som inte ens är en orm. Det är egentligen en ödla som saknar extremiteter, vilket man tycker borde diska den från ormlistan. Men i och med att kopparormen har utnyttjat ormens kanske mest framstående egenskap, lömskhet, för att mygla sig in på listan får den vara med ändå.

\ditem[Korp]\label{korp}
 Synonym för bovaktig, gemen, lömsk eller bedräglig person. Kan också beteckna ett klassiskt redskap vid arbete med gruvdrift. För information om fågeln, se körp\ref{koorp}.

\ditem[Kortbyxor]\label{kortbyxor}
 eller shorts som anglofiler kallar det, är byxor som inte sträcker sig till anklarna, utan slutar i höjd med knäna. Är kortbyxorna längre än knäet så är det för korta långbyxor, clamdiggers, highwater-byxor eller capribyxor. Den enda personen som borde tillåtas ha så långa \quotetext{kortbyxor} är Mob 47-Åke. De bästa kortbyxorna är avklippta jeans, eller sportkortbyxor i kulörta färger. Avklippta mjukisar är också med beröm godkänt. Tumregeln är ju kortare desto bättre, men syns könsorganet är det att ta i.

\ditem[Korv i smörpapper]\label{korv i smoorpapper}
 är den matvara som främst bidragit till konungariket Sveriges\ref{sverige} utveckling. Någon gång i mitten av 1900-talet amerikaniserades dessvärre korvätandet och smörpappret byttes ut mot ett bröd.

\ditem[Korv med bröd]\label{korv med brood}
 är en populär maträtt som är både billig och lätt att laga. Receptet uppfanns av en den tyske emigranten Charles Feltman (1841-1910) på Coney Island\ref{island} i USA år 1871. Upptäckten gjorde Feltman av en slump när han skulle skära en brödskiva men hade en så slö kniv att han bara kom igenom till hälften. Blind av raseri tog Feltman en kryddig bratwurst och mosade ner i brödskrevan. Hans lille son Horst kom då inknallande i köket och tyckte det såg \quotetext{wunderbar!} ut och övertygade sin fader om att saluföra rätten kommersiellt Från början gick affärerna inte så bra, mycket på grund av att han på tyskt übermenchmanér envisades med att sälja korvarna kalla. När han skrev om receptet så att det istället blev varm korv med bröd var succén dock snabbt ett faktum. Till Sverige kom korv med bröd först år 1897 i och med världsutställningen i Stockholm,.Sedan dess är inget sig likt och i dag finns korv med bröd i alla möjliga urspårade varianter, så som Tegare\ref{tegare} och French hotdog.

\uline{Recept på korv med bröd}

\begin{enumerate}
\item Klyv en bit bröd till hälften (numera finns också färdigkluvna, så kallade \quotetext{korvbröd}\ref{korvbrood} att köpa i välsorterade matvarubutiker).
\item Lägg korven i brödet.
\item Garnera efter egen smak. Senap och ketchup brukar vara populära kryddor, och allt utöver det är egentligen mest tänkt för storfräsare\ref{storfraesare}.
\end{enumerate}

\ditem[Korv-Ivars]\label{korv-ivars}
 var en legendarisk korvmojje vid busstationen i Skellefteå som enligt fleras utsago ska ha serverat landets finaste parisare\ref{parisare}, fram tills det att mojjen av oklara skäl behövde byta ägare och snart blev ett mer konventionellt gatukök med samma namn, men utan landets finaste parisare.

\ditem[Korvbröd]\label{korvbrood}
 är en speciellt djup röd som alla nämnvärda konstnärer har på palletten. Namnet kommer sig av att pigmentet i färgen traditionellt kommer från den högt järnhaltiga jorden i den ukrainska provinsen Korvb som gränsar till Ryssland. Numer framställs dock syntetiskt pigment industriellt, men vissa användare menar att denna färg troligen kommer att förlora sin intensitet efter ett antal år. Det syntetiska pigmentet har dock endast framställts sedan 1989, så än är det för tidigt att säga om denna risk föreligger eller ej.

\ditem[Krigsgrisen]\label{krigsgrisen}
 är en tecknad gris med stora betar och hjälm som sedan 1977 varit heavy metalbandet Motörheads maskot. Egentligen heter den Snaggletooth men det är det inte så många som kallar den. Valet av maskot tros hänga samman med att Lemmy är väldigt intresserad av krig och råa grejer. Ungefär hälften av alla hårdrockare har krigsgrisen tatuerad någon stans på kroppen (och den andra halvan har spader ess). Om man frågar Lemmy vad han tycker om att folk har deras krigsgris intatuerad svarar han belåtet att han tycker det är \quotetext{fuckin' great}.

\ditem[Kriminalroman]\label{kriminalroman}
 En kriminalroman, eller deckare som det också kallas, är en bok som heter något coolt och suggestivt som typ \textit{Innan frosten} eller \textit{Leopardens öga} men sedan visar sig handla om en föråldrad polis med njursvikt som alla andra kriminalromaner. Nästan alla svenska kriminalromaner är omåttligt populära i Tyskland\ref{tyskland} på grund av att man där ändrar omslaget så att det föreställer ett rött hus, för detta tilltalar av någon anledning den tyska mustigheten\ref{den tyska mustigheten}.

\ditem[Kristdemokraterna]\label{kristdemokraterna}
 är ett svenskt parti för människor som inte gillar:

\begin{itemize}
\item Kvinnor
\item Ungdomar
\item Bögar
\item Flator
\item Transpersoner
\item Queer
\item Ensamstående föräldrar
\item Vilda djur (framförallt rovdjur)
\item Arbetarklassen
\item Sjuka
\item Politiker (detta kan te\ref{te} sig en smula orimligt, men det är helt i deras linje)
\item Hedningar
\end{itemize}

De gillar däremot:

\begin{itemize}
\item Kyrkor
\item Pengar
\item Färgen beige
\item Familjer
\item Att tala i tungor
\item Verklighetens folk\ref{verklighetens folk}
\item Framgångsteologi
\end{itemize}

\ditem[Krokodiljägare]\label{krokodiljaegare}
 är en yrkesgrupp man väldigt sällan stöter på i de svenskspråkiga delarna av världen. Vissa menar att det beror på att det inte finns några krokodiler här, medan andra snarare anser att det är ett resultat av att svenska krokodiljägare gjort ett lite väl bra jobb. Hur man än ser på saken är alla överens om att krokodiler är jävligt balla eftersom dom är typ som dinosaurier fast inte utdöda.

\ditem[Kroppshydda]\label{kroppshydda}
 Namnet på den bastkjol Fantomens kamrat Guran går klädd i.

\ditem[Krus]\label{krus}
 Ett krus är ett dryckeskärl som är mer rejält än nått litet fjantigt champagneglas men också mer behändigt än ett tråg\ref{traag}. Kruset är ofta gjort av lera och är försett med ett redigt handtag där en smedslabb går in utan problem. Kruset kan användas i slagsmål om så behövs, men kan framförallt höjas mot taket så att ölet skummar i en skål för brödra- och dryckesskap.

\uline{Krus i litteraturen}

\textit{Hemsöborna} av absinthpundaren tillika författaren August Strindberg inleds med meningen: \quotetext{Han kom som ett yrväder med ett höganäskrus i en svångrem om halsen}. Vad som nu menas med det.

\ditem[Krypa ihop i soffan som en katt]\label{krypa ihop i soffan som en katt}
 Ofta kryper smala tjejer upp i soffor som katter. Där värmer de sina händer på tekoppar\ref{te}. De talar kanske franska och säger sig älska ost men ingen har sett dem äta någon.

\ditem[Krypa upp i soffan som en brugd]\label{krypa upp i soffan som en brugd}
 Att sitta i soffan, gapa och bara ösa in olika födoämnen i munnen\ref{mun}.

\ditem[Krypa upp i soffan som en uv]\label{krypa upp i soffan som en uv}
 Folk med storhetsvansinne\ref{storhetsvansinne} kryper ofta upp i soffor som uvar\ref{uv}. Där läser de tjocka bruna böcker och hoar för sig själva av förnöjdhet. De talar kanske tyska och äter mörkt, redigt bröd som de klappat på en ostkiva på.

\ditem[Kräftbete]\label{kraeftbete}
 är sådant som kräftor\ref{kraeftor} äter. Kräftfiskaren lägger kräftbetet i en kräftmjärde eller modifierad mörtsump som de läckra skaldjuren vandrar in i och sen inte hittar ut ur. För att många kräftor ska gå dit är det viktigt att ha ett bra kräftbete. Det absolut bästa är nyfödda kattungar\ref{katt} som dränkts dagen innan och hunnit ligga och götta till sig lite. Rutten fisk eller en skopa blandade sopor fungerar också, dock inte lika bra. Kräftbete är ett mycket viktigt inslag i att hålla den ekologiska balansen av sommarkatter under kontroll och kan därför berättiga till EU-bidrag.

\ditem[Kräftor]\label{kraeftor}
 är små röda gynnare som springer runt på botten av sjöar och nyper i saker. Dess blod är blått och dess syn mycket dålig. De lever i flock tills de blir riktigt gamla, då vandrar de ut till havet för att vara ifred och kontemplera. Såna individer kallas krabbor. Utvecklingsmässigt befinner sig kräftan på ungefär samma nivå som Europas människor under medeltiden\ref{medeltiden}, vilket bland annat märks på att den äter ruttna kadaver och kan drabbas av pest. Det är tillåtet att jaga kräftor med alla till buds stående medel.

\uline{Disambiguation}

Kräfta kan också vara en sjukdom, vilket förklarar uttrycker \quotetext{att stupa i kräfta.}

\ditem[Kränkt]\label{kraenkt}
 På 1900-talet innebar en kränkning typ att bli avklädd och piskad offentligt. Såhär 100 år senare innebär det att någon med enormt uppblåst ego tvingas möta den bistra verkligheten.

\ditem[Kukenkillar]\label{kukenkillar}
 gillar Dressmann\ref{dressmann} och hårpomada. Slitz är en jäla bra tidning med många intressanta artiklar och riktigt bra raggninsrepliker. Kukenkillar har svårt för andra än kukenkillar.

\ditem[Kulaker]\label{kulaker}
 Rika bönder\ref{boonder} som älskar fri rörlighet\ref{fri roorlighet}, sin oxpiska i äkta läder och drömmer om en återgång till medeltiden\ref{medeltiden}. Kulaker finner man huvudsakligen i Skåne\ref{skaane} där dom lever gott på EU:s arealsstöd. Man kan dock finna kulaker så långt norrut som Västerbotten\ref{vaesterbotten} där Centerpartiets\ref{centerpartiet} numera före detta överstepräst häckar.

\ditem[Kulturarbetare]\label{kulturarbetare}
Egentligen är det lättare att karakterisera kulturarbetare som psykologiskt tillstånd än som ett jobb med tydligt definierade sysslor. Som kulturarbetare kan man göra vad som helst, egentligen. Men gemensamt för alla kulturarbetare är att de tillbringar stora delar av sin vakna tid med att skriva ansökningar till obskyra EU-projekt för att klara livhanken en månad till.

Psykologiskt sett finns desto fler gemensamma nämnare för kulturarbetarskrået, där stressrelaterad panikångest är ett dominant inslag i vardagen. Huruvida panikångesten är ett resultat av att försöka livnära sig på att gestalta intersektionell analys i dockteaterform för 5-åringar, eller av händelser tidigare i kulturarbetarens liv, är hittills inte klarlagt.

\ditem[Kulturstökigt]\label{kulturstookigt}
 är ett sanitärt stadie som ofta påträffas hemma hos kulturintresserade och akademiker. Eftersom individerna i dessa grupper är så fruktansvärt upptagna hela dagarna med att ta del av saker som inte är så viktiga men väldigt intressanta så hinner de helt enkelt inte städa hemma. Vanliga saker som ligger framme i en stor röra är uppslagna böcker, tekoppar, en sån där krokidocka i ljust trä, akustisk gitarr (ibland), tvätt, skivfodral, en trasig stol från någon dyr designer. Eftersom dom som har kulturstökigt hemma är så upptagna hela tiden med intressanta saker så skäms dom inte för det.

\ditem[Kusin]\label{kusin}
 En kusin är en person som ofta upplevs som mycket stötande, men som du enligt lag måste träffa ett par gånger om året. Kusinen är intresserad av sådant som fyrhjulingar, techno och dumma idéer, om han är av manskön, och hästar\ref{haest} och kläder om han är av kvinnokön. Din farmor/farfar eller mormor/morfar verkar av någon outgrundlig anledning tycka mer om kusinen än dig, trots ovan nämnda, fullgoda skäl till det motsatta. Kusinen får alltid fetare julklappar än du.

\ditem[Kvicktänkt]\label{kvicktaenkt}
 kallar man personer som är lite bättre än andra på att lösa problem. Den kvicktänkte låter sig inte luras av borgerliga lösningar som att lämna in bilen på verkstad eller betala för snöskottning. Istället kopplar hen förbi elfelet med hjälp av ett gem och lägger ut en bro av lastpallar från dörren till trottoaren.

\ditem[Kvinnlig författare-knepet]\label{kvinnlig foorfattare-knepet}
 Ett beprövat knep hos unga män som önskar att få ligga med medelklasstjejer. Tillfrågad om vem som är den unge mannens favoritförfattare drar han till med en kvinnlig sådan, för att framstå som en fördomsfri och härlig person som man kan diskutera Mare Kandre och Virginia Woolf med under filten när hösten faller på. Detta, hoppas han, ske leda till att den unga kvinnan erbjuder sig att ligga med honom. Detta händer också, men inte så ofta. Det beror på hur han ser ut.

\uline{Exempel på kvinnliga författare}

\begin{itemize}
\item Sylvia Plath
\item Susanne Brøgger
\item Toni Morrison
\item Hanne Vibeke-Holst
\item Joyce Carol Oates
\item Margaret Atwood
\item Carolyn Cassidy
\item Ayn Rand\ref{ayn rand}
\item Selma Lagerlöf
\end{itemize}

\ditem[Kvinnligt alibi]\label{kvinnligt alibi}
 Ett kvinnligt alibi behöver varje organisation som vill framstå som modern och jämställd trots att makten till 99,9\% ligger i händerna på samma gubbar som i resten av världen. Inget är så uppfriskande som en fräsch tjej i kavaj och vit blus.

\ditem[Kvinnokläder]\label{kvinnoklaeder}
 är en gren på mode-rikets brokiga träd. De plagg som ingår i gruppen kvinnokläder är sådana som den modemedvetne samhällsmedborgaren i första hand associerar med kvinnan, men som i vår postmoderna värld ingalunda bärs uteslutande av kvinnor bara för det, skall understrykas. Typiska kvinnokläder är kjol, sjal, huckle, blus och små hattar.

\uline{Undergrupper}

\begin{itemize}
\item Heroin chic
\item Tantkläder\ref{tantklaeder}
\item H\&Ms \quotetext{rockiga} tjejkläder
\item Det ryska babusjka-stuket
\end{itemize}

\ditem[Kälkborgare]\label{kaelkborgare}
 är människor som tror sig vara lite förmer men egentligen sitter i samma skit som alla andra. De röstar gärna borgerligt, trots att de inte tjänar på det, för att det känns lite finare. Kälkborgaren älskar hockey\ref{hockey} och arbetslinjen\ref{arbetslinjen}. Martin Timell är en typisk kälkborgare.

\ditem[Källkritik]\label{kaellkritik}
 I nissepediasammanhang\ref{nissepedia} har detta begrepp endast och enbart att göra med folks gnäll på undermåligt bordsvatten.

\ditem[Källor till glädje]\label{kaellor till glaedje}
\begin{itemize}
\item På sommaren är solen uppe till sent på kvällen
\item När Lars Ohly pekade finger åt Maria Abrahamsson i TV4:s morgonsoffa
\item Fotboll just under korpnivå
\item Dricka Sofiero och lyssna på Pink Floyd
\item Geggamoja
\item Ortnamnsforskning
\item Chips
\item Lura poliser
\item Små djur med päls
\item Vältravad ved
\end{itemize}

\ditem[Känslo-Oi!]\label{kaenslo-oi!}
 är en så smal gren inom oi! och street-punken att endast ett band kan kategoriseras inom denna kategori, the Crack. Genrens fans lyssnar med andra ord mest på av utvalda, speciellt blödiga låtar från band som i vanliga fall spelar vanlig Oi! eller streetpunk.

\uline{Exempel på Känslo-Oi!}

\begin{itemize}
\item Cock Sparrer- Platinum Blonde
\item Bonecrusher - Don't give up on me
\item The Crack - I'll be there 
\item The Templars - My Saving Grace
\item Cock Sparrer - We're Coming Back
\end{itemize}

\ditem[Kärlek]\label{kaerlek}
 Den kosmiska kraft som får den älskande att gravitera mot älskade, patrioten till fosterlandet, och tysken till korv.

\ditem[Kökssoffan]\label{kookssoffan}
 är en bra uppfinning. Det är en kombination av soffa, köksstol, garderob, säng, bokhylla, bäddsoffa, förråd och fåtölj. Kökssoffan passar utmärkt för många saker, den är dock främst skapt för att ligga på. Många tror att den är till för att sitta på, det är en av anledningarna till att den är så skön att ligga på. Till skillnad från en vanlig soffa som man ligger \quotetext{i} är kökssoffan en möbel man ligger på. Den är även hårdare än en vanlig soffa vilket ger omväxling. Kökssoffan inbjuder till liggande ätning vilket kan vara nödvändigt i dagens högersamhälle. Man bör dock se upp med liggande ätning vilket kräver viss träning om man inte vill ha en fläck på alla kläder, strax under halskragen. Kökssoffan är en utmärkt plats för funderingar. Kökssoffan bör vara av trä och innehålla en tunnare madrass eller dyna. För att helt tjäna sitt syfte bör det även finnas en kudde och en filt till hands i anslutning till kökssoffan.

\ditem[Könsrock]\label{koonsrock}
är en musikgenre där Sverige faktiskt är världsledande. I korthet går könsrock ut på att omsjunga känsliga områden på ett obsent och kontroversiellt sätt. Zoofili, hån av minoriteter, förtäring av hembrännt, Martin Ljung och nazism är populära teman som blandas hej vilt med målet att underhålla öldrickande ungdomar och unga vuxna som valt att stanna i bruksorten där de föddes. Det hittills bästa genomslaget nådde könsrocken 1995 när socialdemokratiske riksdagsledamoten Inge Carlsson skrev till justitieminister Laila Fredivalds gällande \quotetext{Förbud mot kränkande musiktexter} sedan han hört Onkel Kånkels \textit{Pedofilernas volleybollturnering}. Som tidiga föregångare omnämns ofta Johnny Bode och Eddie Meduza men dessa verkade snarare för att reta etablissemanget och vände sig till en äldre publik, även om de också spelas flitigt i rikets A-traktorer. Redan sagda Onkel Kånkel är utan tvekan scenens största namn men även Snorleifs, Roy Rövmun och Binnike-Bengts Orkester, Anus Cancer, Vrävarna och The Kristet Utseende förtjänar omnämnanden.

\ditem[Köping]\label{kooping}
 (sööpiŋɠ) är Västmanlands största stad. I stadskärnan bor cirka 150.000 människor men om man även räknar in närförorterna blir invånarantalet nästan 250.000. En stor del av befolkningen härstammar från Finland\ref{finland}, men inflyttning sker även från Kolsva\ref{helvetet} och Arboga. GPS är att rekommendera för den som besöker staden för första gången och inte vill förirra sig bland de vindlande boulevarderna som utgör centrum. Stadens välstånd byggde länge på skivbolaget Birdnests världsledande ställning som leverantör av trollpunk\ref{trollpunk}, men detta förändrades snabbt när Ulke lade ifrån sig gitarren. Nu för tiden bärs ekonomin istället upp av Volvofabriken där majoriteten av världens kvalitetsväxellådor monteras. Birdnest har stängt sin affär, så den som önskar en komplett diskografi med Dennis \& dom blå apelsinerna är numera istället hänvisad till Tradera där Birdnest-Stempa säljer av restlagret av det en gång blomstrande imperiet.

\ditem[Körp]\label{koorp}
 (Corvus corax) är en stor, svart kråkfågel. Den återfinns över hela norra halvklotet och är därmed den mest utspridda av alla kråkfåglar. Körpen är en av de absolut intelligentaste fågelarterna och använder sin talang till att föra andra fåglar bakom ljuset och leva utanför samhällets normer i möjligaste mån.

\ditem[Közösülés]\label{koozoosülés}
 Ungerska för \textit{knulla}. Ordet används påfallande ofta i samband med förtäring av Ungerns nationaldryck, Törley gala\ref{toorley gala}.


%%%%%%%%%%%%%%
\newpage
\null
\\
\null
\\
\Huge
L
\normalsize
\\
\null
\\
\null
%%%%%%%%%%%%%%


\ditem[Laissez-faire]\label{laissez-faire}
 (Fr. \textit{laissez} ung. \quotetext{skövla} Fr. \textit{faire} ung. \quotetext{Afrika}) är något som diggas utav bara helvete på ledarplats i Dagens Nyheter\ref{dagens nyheter}.

\ditem[Landslagsuppehåll]\label{landslagsuppehaall}
 är när de bra fotbollsligorna för herrar tar paus för att spelarna ska hem för att lira med sina, i de flesta fall, betydligt sämre landslag. I Sveriges fall innebär det att spelarna får styrk av lag i andra B-nationer, spelar oavgjort med C-nationer, och vinner med 1-0 mot micronationer. En gång vann man över Spanien och en gång spelade man oavgjort mot Brasilien. De enda som gillar landslagsuppehåll är England eftersom dom alltid tror att dom ska vinna. På damsidan är landslagsuppehåll något helt annat eftersom Sverige både har bäst spelare och coolast tränare.

\ditem[Lappskojs]\label{lappskojs}
 Lika delar renskav och potatismos. I nödfall kan det rena ersättas med corned beef.

\ditem[Lars Krogh]\label{lars krogh}
 är en dansk man som ger ut garagerock och som ogärna inte röker stora mängder hasch\ref{stenad}. Han ska enligt envisa rykten vara extremt gammal och är, som sagt, dansk.

\ditem[Lars Levi Laestadius]\label{lars levi laestadius}
 Predikant från Jäckvik, Lappland men slog igenom stort i Pajala, Tornedalen. Startade även en rapkarriär under artistnamnet LL Cool L och hade en hit under sommaren 1820 med låten \quotetext{Preachin' on til da break of dawn}. Tappade helt koncepten av all uppståndelse efter att han av en olyckshändelse uppfann jeansbyxan.

\ditem[Lars-Åke Lagrell]\label{lars-aake lagrell}
 är ca ett millenium gammal och sedan 1991 ordförande för Svenska fotbollsförbundet. Han är den enskilt största anledningen till att Sverige missade fotbolls\ref{fotboll} VM i Sydafrika 2010.

\ditem[LAS]\label{las}
 (Lagen om anställningsskydd) är det som gör att man kan ha ett arbete och ändå kunna gnälla på chefen. När borgarna skrotar LAS är det hög tid att ta upp vapen mot kapitalet.

\ditem[Latinsk facebook-rocker]\label{latinsk facebook-rocker}
 Han är allestädes närvarande och finns på varje rockbands facebooksida. Likt en äldre släkting med för mycket fritid och endast ett dimmigt hum om vad posten handlar om dyker han ofelbart upp i kommentarsfältet. Och oavsett vad posten handlar om – bandet ska släppa en ny skiva, de ska spela på en rockklubb i Prag på lördag, gitarristens stärkare har blivit stulen under en europaturné, bandet har splittrats – skriver han alltid detsamma: \quotetext{Your music is great! Please come to Buenos Aires!!!!!!!! Rock on!!!!}

\uline{Utbredning och social dynamik}

Enligt Mark Zuckerberg består hela 62\% av facebook idag av latinska facebook-rockers. Ett stort uppsving för den latinske facebook-rockern var när det gick upp för honom att man medelst ett vanligt tangentbord kan avbilda the horns: \m/. Intresset för subkulturen latinsk facebook-rocker sköt i och med detta i höjden. Nu utmanar denna utbredda livsstil till och med subkulturerna mariachi och brasiliansk samba-donna. Detta, skulle man kunna tro, borde skapa vissa konflikter inom gruppen. Men på grund av den hederskodex som råder gör det inte det. Ser Juan att Miguel redan har skrivit kommentaren \quotetext{I have one of your records! Please come to Colombia!!!!11!!!! \m/} på posten om att Chris Hakius hoppat av OM så surfar han vidare och kommenterar istället på posten om att det var x antal år sedan Crass splittrades och uppmanar dem att ta sitt pick och pack och besöka detta högt belägna, latinamerikanska land.

}

\ditem[Ledingreppet]\label{ledingreppet}
 är när en kille står bakom en tjej med armarna runt hennes midja (kan med fördel också vila hakan på hennes axel och liksom titta fram) i samband med framträdanden av rockartisten Tomas Ledin, helst under någon riktigt fin och romantisk låt. Han har nog en tröja knuten i midjan också för att det blev lite varmt där efter \quotetext{Det finns inget finare än kärleken}. Ledingreppet kan utföras i vilka mysiga musikaliska sammanhang som helst (och kallas då folkparksgrepp), men är vanligast förekommande på Ledinspelningar.

\uline{Regionala skillnader}

Norr om Skellefteå kallas det Bo Kasper-greppet (verb: \quotetext{att Bo Kaspra}) pga BKA kommer från Piteå.

\ditem[Leggings]\label{leggings}
 är en postmodern variant av långkalsonger. Av någon anledning är det socialt accepterat att traska runt på offentliga platser iförd dessa leggings medan den som åker ner på macken i långkallingar blir bemött med stor misstro.

\ditem[Lego]\label{lego}
 är ett byggmaterial uppfunnet av dansken Godtfred Kirk Christiansen. Det uppfanns just efter andra världskrigets slut och var tänkt att ersätta tegelstenen, som Christiansen tyckte var skrattretande föråldrad, vid återuppbyggandet av ett Europa som låg i spillror. Danmark\ref{danmark} hade inte liksom Sverige\ref{sverige} järnmalm att sko sig på och man ville därför ta fram något som tillät att man fick sin del av den ekonomiska kakan. Innan så hann ske greps dock Christiansen på grund av alkoholkonsumtion och närheten till Tyskland\ref{tyskland} av storhetsvansinne\ref{storhetsvansinne} och lät utropa republiken Legoland och i dess hjärta slå upp en stor jävla giraff av lego. Sedan dess har Legoland varit en av landets stora attraktioner och har årligen tiotusentals besökare, inte minst på grund av försäljningen av droger, som sker helt öppet på republikens huvudgata, pusherstreet, har relationerna till Danmarks myndigheter tidvis varit ansträngda och hot om \quotetext{normaliseringsprojekt} har avlöst varandra.

\ditem[Lemmy-bas]\label{lemmy-bas}
 En Lemmy-bas är ett musikinstrument som har som uppgift att bringa med tyngd till en rockgrupp. En Lemmy-bas skiljer sig från en vanlig bas i det att den låter mer som ett brus än ett dunkande. För att uppnå detta är baskroppen utskuren ur en halvrutten tallstam som sedan konserverats med tapetklister. De fyra strängarna är uppifrån i tur och ordning gjorda av taggtråd, elstängsel, startkabel och gummisnodd.

\uline{Kända brukare av Lemmy-bas är:}

\begin{itemize}
\item Alla kängband\ref{lista oover dis-namn}
\item Lemmy
\end{itemize}

\ditem[Lena]\label{lena}
 är ett vanligt namn bland kvinnliga entreprenörer, speciellt inom detaljhandel i vilken kvinnokläder\ref{kvinnoklaeder} saluförs, men också inom kombinationsaffärs- och stentrollsaffärsmiljön\ref{stentrollsaffaer}. Speciellt framgångsrika entreprenörer får efter skriftlig ansökan till länsstyrelsen beviljat att tillägga ännu ett namn till huvudnamnet \textit{Lena}. Populära tillägg är Anna- och Eva-.

\ditem[Lennart Holmlund]\label{lennart holmlund}
 Även känd som V75-kungen. Denne man var på papperet kommunalråd i Umeå\ref{umeaa} från den paleolitiska perioden till och med kulturhuvudstadsåret 2014. När han inte åkte på solsemester till Thailand brukade han och hans bäste vän Krister Olsson bada bastu, finljuga\ref{finljuga} och styra Umeå med järnhand. Hans födointag består uteslutande av Ahlgrens bilar\ref{bil} och choklad.

\uline{Missförstånd}

Många har misstagit Angry Anderson, sångare i det australiensiska\ref{australien} outlaw-bandet \textit{Rose tattoo}, för att vara Lennart Holmlund, vilket orsakat Angry obeskrivliga svårigheter i sin karriär liksom i sitt sällskapsliv. Detta har många gånger förvärrats av att Holmlund utgivit sig för att vara Anderson - häromsistens på Wacken, då Holmlund dök upp tillsammans med Krister Ohlsson, Prof. Etienne\ref{prof. etienne} och två kommuntjänstemän. Tillsammans framförde de Rose tattoos \textit{We Can\'t be Beaten} framför en lyckligt ovetande samling mustiga tyskar.

\ditem[Lenin-Churchhill aka Mysgubbe]\label{lenin-churchhill aka mysgubbe}
 Lenin-Churchill är smeknamet på en flaska Březňák. Det kommer sig av att den till synes goa mysgubben på etiketten, hållandes ett pilsnerstop i ena handen och en tänd cigarr i den andra, ser ut som en korsning av de två historiska herrarna från förra seklet. Om än några dussin kilo tyngre.

\uline{Från tågkonduktör till varumärke}

Mysgubben hette egentligen Victor Cibich (1856-1915) och prydde Breznaks flaskor redan år 1906. Han var under sitt tämligen korta liv, Cibich dog vid en ålder av endast femtionio år, vida känd i sin by med omnejd för sin öl och matentusiasm. Tydligen skall Victor varit en så pass trevlig människa att ha att göra med så att styrelsen för Breznaks bryggeri föreslog honom som logo för dess pilsner år 1906, mot att han fick trettio stop i veckan i resten av sitt liv på deras bekostnad.

Bryggeriet valde att göra ett uppehåll med den till synes välbärgade Cibich (dock arbetade han under större delen av sitt liv endast som tågkonduktör) på etiketten under eran av statssocialism, 1945-1990.

\uline{Mysgubben till de svenska massorna}

De ölentusiaster som kan räknas in bland småfolket och studenter gladdes rejält när Breznak introducerades i Systembolagets ordinarie sortiment (nr 1611) hösten 2009 till det humana priset av 31.21 riksdaler per liter, 10.30 per flaska. Ett vakuum hade nyligen uppstått bland budgetpilsnern från Tjeckien då Primator höjts från 10.10 riksdaler flaskan till 12.50. Nu kunde man åter få en låda Tjeckisk pilsner för mindre än fem Jennys. Den är sommaren 2010 fortfarande den enda Tjeckiska pilsnern under tolv kronor flaskan.

\ditem[Libanon]\label{libanon}
 Libanons flagga innehåller en tall vilket såklart för tankarna till Malå\ref{malaa}, historiker menar att Beirut kan ha grundats av Malåbor vilka tagit en tur på rullskidor.

\ditem[Ligga med kulturstockholm]\label{ligga med kulturstockholm}
 Att ligga med kulturstockholm är en upplevelse som är möjlig för subkultur-trendiga norrländska killar, androgyna och undersköna invandrare uppfödda i en autentiskt risig förort, jugoslaviska teoretiker med stora hygienproblem, samt kvinnliga opublicerade poeter med flätor, stickade kläder och pipröst. Detta kan vara en perspektivgivande erfarenhet och tillföra visst förnyat självförtroende, men man gör det dessvärre bara en gång, för sedan är man förbrukad. Den enda möjligheten att få obegränsad sexuell tillgång till kulturstockholm är att döpa om sig till en siffra och starta en blogg.

\ditem[Likgömmarmössa]\label{likgoommarmoossa}
 En likgömmarmössa är det som en del människor kallar sotarrolle, alltså en vanlig mössa som inte går ner över öronen. En riktig likgömmarmössa är svart för att passa in med övrig klädsel som är bruklig när man pysslar med tvivelaktigheter.

\uline{Att ha mössan DC}

Att ha mössan \quotetext{DC} innebär att rulla upp sin likgömmarmössa ett halvt varv till och placera den långt bak på huvudet, lite som sovjetiskt marininfanteri bar sina baskrar fordomtida. Precis som straight edge kan man skylla detta mode på Ian Mackaye.

\ditem[Lillgammal]\label{lillgammal}
 Om ett barn kan kallas lillgammalt är detta enligt nyblivna föräldrar och idiotiska mor- eller farföräldrar något av en jackpot. Lillgamla barn kallas så för att de utan en strimma av skam smörar för äldre för att få uppmärksamhet. Det lillgamla barnet lägger huvudet på sned och uttalar (högt så alla ska höra) klokheter så som att man ska \quotetext{ta hand om varandra} och annan skit som den plockat upp från Disneyprogram men själv inte praktiserar.

\ditem[Lillnöjd]\label{lillnoojd}
 Känsla av vällust efter någon form av glädjande och lite oväntad bragd. Som att Hammarby IF bandy efter tio förluster på rad vinner en, i de flesta avseenden, obetydlig match. Eller att man än en gång lyckas laga sin International Harvester\ref{international harvester} med lite ståltråd och några välriktade hammarslag istället för att köpa nya delar. Eller att någon du ogillar räknar fel på fyrtiotusen miljarder kronor\ref{fyrtiotusen miljarder}.

\ditem[Limerick]\label{limerick}
 är en diktform populäriserad genom Hasseåtage. Tydligen kommer den från den irländska byn Limerick, men eftersom det inte är så många som hört talas om denna säkert fantastiska plats är det heller inte så många som associerar diktformsjäveln med den. Anyway,  en limerick består av fem rader och är lite rolig. De två första och den femte består av rimmande rader på åtta eller fler stavelser. De två där mellan består av fem\ref{femma} eller sex\ref{sexa} stavelser. Den första ska innehålla ett ortsnamn och en referens till en viss person som dikten handlar om.

Till exempel:

\textit{John Anscha va en grabb från Malå sta'}
\textit{Han skida' till Ume på under en da'}
\textit{Han läste en bok}
\textit{Blev besatt som en tok!}
\textit{Han hade upptäckt hur mäktig marxismen va'.}

\ditem[Linda Norrman Skugge]\label{linda norrman skugge}
 är liberal hypokondriker och nästan alltid arg på något. Först hatade hon alla som var normala och var typ familjer och hade barn\ref{barn} sen fick hon barn själv och nu hatar hon alla som inte älskar barn.

\ditem[Lista över dis-namn]\label{lista oover dis-namn}
 Ett dis-namn är ett namn en orkester tar för att hylla Stoke-On-Trents bästa band: Discharge. Här följer två listor på dis-namn, en med upptagna och en med lediga.

\uline{Upptagna}

\begin{itemize}
\item Discharge
\item Disclose
\item Dischange
\item Dispose
\item Disorder
\item Disaffect
\item Dispense
\item Diskelmä
\item Disträ
\item Disklass II
\item Disculpa
\item Disrupt
\item Diskonto
\item Disbrubtum
\item Disaccord
\item Disarm
\item Dissober
\item Diskent
\item Disbrutal
\item Diskrieg
\item Disfear
\item Disfornicate
\end{itemize}

\uline{Lediga}

\begin{itemize}
\item Diskursanalys
\item Diskir
\item Disbärs
\item Diskmedel
\item Distinktion
\item Distingerad
\item Diskettstation
\item Disneyland
\item Diskussion
\item Distributionsapparat
\item Disputationsfest
\item Disproportionerlig fördelning av jordbruksareal
\end{itemize}

\ditem[Ljudtekniker]\label{ljudtekniker}
 En ljudtekniker är en person med fördjupade kunskaper i samspelet mellan decibel, megahertz, bas, diskant och allt annat som bildar ljudens magiska värld. Utan en ljudtekniker som reglerade styrkeförhållandet mellan alla sälla toner skulle Discharges och Slayers skivor bara låta som en stor gröt av slammer och bang. Ännu viktigare är ljudteknikern på livekonserter då hen även innehar stor erfarenhet av hur man bäst rullar upp en gitarrsladd utan att orsaka kabelbrott, något som annars lätt kan skapa osämja bland unga musiker som lånat sina grejer av Studiefrämjandet. Dessutom är en erfaren ljudtekniker på grund av sin stegrande tinnitus alltid den bästa garantin för en ljudnivå som skrämmer bort alla posörer från scenkanten. En rutinerad ljudtekniker känns igen på sin grånande hästsvans så om du känner dig osäker: titta alltid på ryggen. Andra positiva faktorer är att hen uppskattar allt med King Crimsons frontfigur Robert Fripp och har många saker fastsatta på skärpet i byxlinningen. Tecken på att du har att göra med en charlatan som hittat in i skrået via högskoleutbildning: byxor med många fickor, trådlösa mickar, spelar System of a Down i pauserna mellan banden. 

\ditem[Lobotomobil]\label{lobotomobil}
 Lobotomobilen, medicinvetenskapens motsvarighet till batmobilen, transporterade Walter Jackson Freeman II land och rike runt i jakt på orginal, karaktärer och andra filurer.
Lobotomobilen var en fransktillverkad Citroën 2CV Fourgonnette, möjligen årsmodell 1952.
Med hjälp av lobotomobilen lobotomerade Freeman ca 3400 intet ont anande medborgare, däribland John F. Kennedys syster Rosemary.
Lobotomi går ut på att man skär av anslutningarna i prefrontal cortex. Namnet kommer från grekiskans \begin{otherlanguage*}{greek}λοβός\end{otherlanguage*} (lobos): lob; \begin{otherlanguage*}{greek}τομή\end{otherlanguage*} – (tomē): skära/kapa. Lobotomi botar enligt Psyciatric Dictionary 1970:
\quotetext{Prefrontal lobotomy is of value in the following disorders, listed in a descending scale of good results: affective disorders, obsessive-compulsive states, chronic anxiety states and other non-schizophrenic conditions, paranoid schizophrenia, undetermined or mixed type of schizophrenia, catatonic schizophrenia, and hebephrenic and simple schizophrenia. Good results are obtained in about 98 percent of cases, fair results in some 35 percent and poor results in 25 percent are thereabouts.}

Hur de fick ihop procentsatserna tvistar fortfarande de lärda om.
Freeman uppfann inte lobotomin men förfinade den. Orginalversionen uppfanns av en portugis vid namn António Egas Moniz som fick ett halvt nobelpris 1949 för besväret (tillsammans med Walter Rudolf Hess (yes han hette så).
Egas Moniz kallade sitt ingrepp leukotomi där leukos är grekiska för ren eller vit. Leukotomin var lite bökigare eftersom man var tvungen att borra hål i kraniet först.
Freemans ingrepp, transorbital lobotomi, gick ut på att man tog en syl och knackade in den i tårkanalen med en gummihammare.
Väl inne i prefrontal cortex så vispade man helt sonika runt lite med sylen. Lite som att borsta tänderna. Fast med en syl. I hjärnan.
Att Freeman inte var kirurg utan psykolog gjorde inte så mycket då han bara tog 25 dollar för besväret.

Trots att svensken Snorre Wohlfahrt utvärderade tidiga resultat redan 1947 och kom fram till att det inte var så smart lobotomerades 4500 svenskar mellan 1945 och 1966. Mest kvinnor.

Den amerikanska cyberneticspionjären Norbert Wiener kommenterade ämnet med att om man vill att patienterna skulle vara enklare att ha att göra med så blir det ännu enklare om man tar död på dem.
Sovjet förbjöd lobotomi 1950 eftersom de kom fram till att det \quotetext{gjorde galna personer till idioter}.

\ditem[Looppedal]\label{looppedal}
 En looppedal är ett verktyg för fattiga bandledare som inte har råd att hyra sig en egen orkester.

\ditem[Lothar]\label{lothar}
 är den engelska bondkatt\ref{katt} som blev den första dokumenterade Lolkatten. När han var ca två månader gammal så togs en bild på honom poserande på hans ägarinnas axel som sedan skulle bli känd som den första bilden föreställande en Lolkatt.

Lothar gillade de facto att posera med familjemedlemmarna men försökte under den senare delen av sitt liv tvätta bort den oseriösa stämpeln han fått genom de foton som togs av honom som ung. För Lothars del var det aldrig tal om att göra någon modellkarriär då han inte var en raskatt, pga detta var han periodvis deprimerad och missbrukade kattmynta för att orka med att sköta sina dagliga rutiner i hushållet såsom att jaga möss och garnnystan. Genom att ofrivilligt undvika en modellkarriär levde han dock i det stora hela ett förhållandevis lugnt kattliv och somnade in vid en ålder av arton år.

Lothars minne bevaras endast av Internet då han själv inte erkände sitt faderskap till några kattungar under sin livstid.

Lothar har även gett namn åt en av warcraftuniversats största krigare, sir Anduin Lothar, knight champion of the kingdom of Azeroth under hordens första invasion av hans hemvärld och Supreme commander of the armies of Loarderon under hordens andra invasion. Denne Lothar dog i strid, man mot man, med den väldige orchen Warchief Ogrim Doomhammer, just innan alliansen segrade och den mörka portalen stängdes.

\ditem[Lucia]\label{lucia}
 är en högtid som sanktionerats av svenska staten allt sedan folkskolereformen år 1842. Med anledning av att det blev obligatoriskt att gå i skolan behövde staten klargöra vilka ämnen alla svenska barn förväntades kunna. Eftersom det skulle bli så tråkigt att bara räkna matte och läsa om Nils Holgersson hela dagarna kom staten på att man också skulle ha undervisning i musik. På vårterminen bestämde man att eleverna ska öva på \textit{Idas sommarvisa}, \textit{Skala banan}, \textit{Horgalåten\ref{horgalaaten}}, \textit{Den blomstertid nu kommer} och \textit{Hasta mañana}. På höstterminen hade man dock ingen naturlig högtid att öva inför, och det var då någon kom på att man kunde damma av den gamla luciatraditionen. Tidigare hade lucia firats stort i Sverige men detta åkte ut med buller och bång när Gustav Vasa vevade igång reformationen. Nu började dock en intensiv propagandaapparat att arbeta för att återinföra traditionen. För att göra högtiden populär även hos vuxna (som egentligen tycker det är skittråkigt att lyssna på bräkande barnkörer) finslipade man den en aning och lade till att det var helt okej att sitta i särk\ref{saerk} och spy ner sig. Det blev braksuccé på en gång och lucia firas allt jämt sedan dess.

\ditem[Luciavaka]\label{luciavaka}
 Traditionen att fira lucia\ref{lucia} härstammar lite blandat från folktro och kristendom. De förkristna elementen är så klart ballare då man trodde att djuren kunde tala under lucianatten och att övernaturliga makter var i rörelse eftersom det är det mörkaste dygnet på hela året. För att inte råka ut för motpåven i Gränna\ref{motpaaven i graenna}, Anton Lavey\ref{anton lavey}, Jubal\ref{jubal} eller någon annan kristen mörkerman denna natt satt människorna uppe hela natten och vakade vid fönstret. Liksom alla andra förkristna högtider inkluderar även luciavakan att deltagarna dricker stora mängder brännvin\ref{braennvin} sittandes i särk\ref{saerk}. Genom sin popularitet har luciavakan även fött andra jultraditioner såsom häxblandning, bockbränning och skinnsbergslucia.

\ditem[Luffarskål]\label{luffarskaal}
 Att äta i luffarskål, eller irländskt hovporslin som det också kallas, är att knipa ihop låren medan man sitter och sedan hälla ner maten i knät och äta direkt ur skrevet.

\ditem[Luftgitarr]\label{luftgitarr}
 kallas den uråldriga blandning av teater, mim och dans som går ut på att medelst kroppen imitera solot i \textit{November rain} eller annat gitarrmättat musikstycke.

Ursprungligen handlade luftgitarr bara om framförande för utövarens egen höga njutning. Men sedan den borgerliga revolutionen trätt in och raderat feodalsamhället förändrades även luftgitarrens roll och blev nu en individens kamp med alla mot alla. Som i så många andra tävlingar där det visuella styr koras vinnaren i luftgitarrtävlingar av en jury. Vinner gör den som mest får det att se ut som att hon/han faktiskt håller i en gura och dessutom inte spelar fel. I juryn till luftgitarr-SM 1991 ingick bland annat Jan Guillou och Micke \quotetext{Svullo} Dubois, den senare i egenskap av flink luftbassist tillika komiskt geni. Vann gjorde det året Lena PH:s ex-snubbe Martin Björk. 1984 vann Kalle Moraeus.

\ditem[Luktagott]\label{luktagott}
 är sånt som storfräsare\ref{storfraesare} har på sig för att till och med lukta märkvärdigt. Som alla vet är de bästa dofterna egentligen de från bränd linolja och Kir\ref{kir}, men vitsen med luktagott är inte att det ska lukta bra utan dyrt och borgerligt. Det mesta luktagottet görs därför på exklusiva och dyra saker såsom flodkaninens\ref{flodkanin} hypofys, saffran, kopparrör och jordgubbar. För att ytterligare öka luktagottets status ges essenserna namn som leder in brukarens fantasi på mytologiska spår såsom \textit{\quotetext{daggvåt ängsmark}}, \textit{\quotetext{änglafjärt}} eller \textit{\quotetext{kungen av Danmark\ref{danmark}}}. Naturligtvis är det mest kälkborgare\ref{kaelkborgare} som ägnar sig åt detta.

\ditem[Lule]\label{lule}
 (Luleå i Storsvensk\ref{storswaensk} stavning) är en stad i Norrbotten\ref{norrbotten} som bebos av en hel del tuffa typer av alla kön. I Lule kan folk av olika trosinriktningar såsom veganism, lutheranism och frifräsare umgås under samma tak utan att handgemäng uppstår.

Övriga landet har ofta svårt att hantera närheten av en lulebo då de ofta är ett gladlynt och skämtsamt folk som gärna klär av sig.

Lulebon vägrar att kompromissa med sig själv och säger sin åsikt med rak rygg och högt hållet huvud. Således följer hen sitt Lule Hockey med en självplågares fulla entusiasm. Poporkestern Toto har spelat inte bara en utan två gånger i Lule. Medelpersonen från Lule har cirka 105 högskolepoäng utfärdade av Umeå Universitet i humaniora, beteendevetenskap, socialt arbete och/eller från en lärarutbildning.

\ditem[Lundgren]\label{lundgren}
 är en av de äldsta inventarierna i stadsbilden av centrala Norberg. Han har sedan inlandsisen smälte ansvarat för att kundvagnarna utanför Konsum och Ica kommer tillbaka på sin plats. Om arbetet utförs på konsultbasis eller ideellt är oklart. Det är också oklart om någon någonsin faktiskt bett honom att göra detta. Lundgren går alltid oklanderligt klädd i kavaj och kepsar med olika företagslogotyper\ref{kepsar med olika fooretagslogotyper} och skulle med lätthet kunna ta plats i vilken historia som helst om Kapten Stofil. I väntan på att nya kundvagnar ska köras tillbaka händer det att han unnar sig ett pipstopp och går en sväng med händerna på ryggen.

Enligt samstämmiga uppgifter från en källa var Lundgren mods på 1960-talet.

\ditem[Lundin Petroleum]\label{lundin petroleum}
 är en en gren av FNs Barnfond UNICEF och har alls inget med hänsynslös exploatering och etnisk rensning att göra. Organisation ägnar sig mycket åt att pumpa upp olja för att av den kunna tillverka mat åt fattiga barn i Afrika. Vår egen Carl Bildt är en av organisationens mest envetna tillskyndare och arbetar oförtröttligt i anletets svett för organisationens bästa.

\ditem[Lurkuk]\label{lurkuk}
 En lurkuk är en man, påfallande ofta heterosexuell och medelålders, som lovar kvinnor i sin närhet guld och gröna skogar mellan lakanen, men i slutändan inte får upp den. Detta beror ofta på fylla/trötthet, nervositet eller att åldern helt enkelt tagit ut sin rätt på svällkropparna. Det finns olika strategier för att kompensera detta, varav de vanligaste är lite tafatt oralsex eller att somna. Bland våra kändare lurkukar återfinns framförallt Ulf Lundell och även Max Weber\ref{max weber}.

\uline{Strategier för att upptäcka lurkukar}

En ganska felsäker tumregel är att; ju mer karlfan lovar och bedyrar sin virilitet trots eventuell fylla, desto troligare att det mesta han kommer att åstadkomma är att dregla dig lite i underlivet innan han somnar.

\ditem[Luís Figo]\label{luís figo}
är en före detta portugisisk fotbollsspelare som var väldigt duktig på det han gjorde. Han tröttnade på Portugal och började spela i den katalanska klubben FC Barcelona där han utvecklades till en riktig stjärnspelare i mittfältet. Spaniens näst största stad blev dock för liten för Figos svällande ego och han nappade på ett erbjudande att börja spela i Real Madrid. För att konceptualisera precis hur stora rivaler dessa klubbar är kan ni tänka er att det är troligare att en sparkad brittisk kolgruvearbetare ger chefen en kram och säger \quotetext{You did your best!} än att två supportrar till de olika klubbarna sätter sig ner och tar en bärs tillsammans. Frågan som spanska sportjournalister ställde var om ens herren vår Gud\ref{gud} själv skulle vara förmögen att förlåta Figo. Barcelonas fans visade vad de tyckte om beslutet genom att hiva ett grishuvud på Figo när denne lade en hörna under sin första match med Real mot Barca.

\ditem[Lycksele]\label{lycksele}
 är ett sött och lätt efterblivet samhälle i Västerbottens\ref{vaesterbotten} inland. Som så mycket annat ligger det vid vatten. Varje vinter gör man en tiotals meter hög och ständigt ejakulerande kommunal snökuk (officiellt kallad ispelare) ute vid stranden vilket ortens feminister ställer sig lätt frågande till. Det lokala patriarkatet tycker att feministerna har snuskig fantasi.

\ditem[Läppar som prinskorv]\label{laeppar som prinskorv}
 (uttalas med fransk brytning) är ett skinhead från det, i korvsammanhang, anrika landet Tyskland\ref{tyskland}. Närmare bestämt bor han i Hamburg där han lägger mycket tid på att följa det lokala fotbollslaget\ref{fotboll} \textit{FC St. Pauli} och på att kolla in kassa tyska punkband. En genomsnittlig dag dricker Läppar som prinskorv ungefär 5 bärs av det lokala märket Astra och röker ett paket cigg. Har han inget bättre för sig kan det också hända att han drar nån skabbig gaddning på sin askgrå hud. Precis som man föreställer sig är han ganska trind om buken. Det mest iögonfallande med Läppar som prinskorv är dock hans fylliga läppar som alltid är glansiga och väldigt stora. Man blir som lite nyfiken. Varför är dom alltid lika glansiga som flottiga prinskorvar? Men det vågar man inte fråga för Läppar som prinskorv är trots allt skinhead och dom brukar sällan uppskatta sådana frågor oavsett om dom är sharp eller inte.

\ditem[Lärare]\label{laerare}
 är ett ädelt yrke som länge hade hög status i och med historiska föregångare som Thomas av Aquino\ref{chapeau de paysan} och varma mediala representationer som Ingemar Bergmans gullgubbe Caligula i filmen \textit{Hets}. Med åren har lärarkårens status chanserat. Rent historiskt befinner sig nu lärare vid en kritisk nollpunkt i ryktbarhet. Nedan följer en historisk redogörelse för hur det västerländska läraryrket en gång var, och hur det sedermera kommit att bli.

\uline{Antiken}

De gamla grekerna\ref{de gamla grekerna} höll lärarna högt. Sokrates föreläsningar om ditten och datten gjorde honom till en togaklädd rockstjärna. Platon, en annan farbror som lärde, var så begeistrad av Sokrates att han utformade nästan alla sina kommande böcker i dialogform, där han och den då avlidne Sokrates hade påhittade samtal om gamla idéer. Det var även så fenomenet \quotetext{coverband} uppstod.

\uline{Medeltiden}

På medeltiden bodde de flesta lärare i kloster. Kloster på den tiden var mäktiga institutioner där alla var feta som broder Tuck och söp hela tiden. Runt omkring klostren brann ett pestdrabbat Europa. Att lära sig om det ingick dock inte i den högst begränsade läroplanen, som mest avhandlade korvstoppning och att ordagrant kopiera enorma textsjok för hand.

\uline{Renässansen}

Under renässansen flyttade lärarna ut från sina kloster och tillbaka ut i samhället. Eller en del av samhället i alla fall. De som hade råd med lärare på den tiden var italienska merkantila högdjur och tyska hertigar som fick läsa \textit{Fursten} av Machiavelli och lära sig räkna medelst abakus. Rika, inte fattiga, med andra ord.

\uline{Upplysningen \& Romantiken}

Här började det ta fart som satan för lärarna. Upplysningen innebar att \quotetext{vetenskap} var inne och att samhället förändrades en del (fler fick tillgång till skolor). Därför blev läraren det hetaste sedan nån spillde senap på en varmkorv för första gången. Alla ville bli lärare som gick runt och berättade saker för folk. Rika som fattiga, alla ville ha en lärare!

Romantiken var en idéströmning som löpte parallellt med upplysningen och hade lite andra ideal, typ att man var andlig och gillade naturen som estetiskt föremål - inte som ett monster som skulle betvingas med eld och järn. Men de gillade också lärare, i den känslosamma filosofiska formen, ungefär som Robin Williams i \textit{Döda poeters sällskap}.

\uline{Moderniteten}

Allt skulle industrialiseras och mätas och så. Läraren fortsatte rocka loss i samhället. Visst, nu var lärarna fler och behövde undervisa barn från lägre samhällsskikt, men deras rockstjärnestatus levde vidare. Det här skulle kunna ses som läraryrkets guldålder. Många kunde bli det, ingen ifrågasatte en och lönen var rätt bra. Man fick också slå barnen.

\uline{1970 - 2002}

En inflation av lärare orsakade en urholkning av läraryrkets guldiga aura. Dessutom hade tjejer börjat bli lärare, något som i regel pajar ett yrkes status i patriarkala samhällen. På 90-talet behövdes inte längre någon riktig utbildning för att bli lärare, bara en extrem hängivelse till en hobby. Typ om man hade en LUF-pin kunde man titulera sig mattelärare och om man hade håriga handflator och gillade pörr\ref{poorr} kom en heltidsanställning som biologilärare som ett brev på posten. Det var även här elevernas respekt för lärarna försvann, då folkpartister och porrsamlare sällan har någon längre utbildning i pedagogik.

\uline{Nu och vidare}

Vad framtiden bär i sitt sköte för läraryrket är svårt att sia i. Ex-major Jan Björklund har i alla fall bestämt att det ska finnas lärarlegitimation nu. Detta skulle kunna resultera i att lärarkåren blir mer professionell och håller högre standard. Men i och med att antagningspoängen för lärarutbildningen är lägre än dito för dikesgrävare, kommer antagligen ingen större förändring att ske ändå.

\ditem[Läsesalsflört]\label{laesesalsfloort}
 En läsesalsflört uppstår när en person fattar tycke för en annan person i en läsesal, oftast på ett universitet\ref{universitet}, och bestämmer sig för att göra slag i saken. Slaget i saken består de flesta gånger i att överlämna en lapp på sitt amorösa intresses läsebänk, där man skriver typ: \quotetext{du är söt, vill du ta en fika någon dag? :)} och sedan nedtecknar sitt telefonnummer. Nästan ingen av dessa lappflörtar resulterar i mer än en eftermiddags höjt självförtroende hos mottagaren och minst en veckas dödsångest hos avsändaren.

\ditem[Lätt misshandel]\label{laett misshandel}
 Hästbett, purple nurple, knuffar i kombination med kränkande skällsord yttrade med avsikt att såra, öppen handflata i ansiktet, baksidan av handen i ansiktet, tjuvnyp på kärlekshandtagen, hajkbox, solar plex-slag, att fällas medelst krokben eller tillhygge, luggning, spark i arslet, box på axeln eller i magen (inte för hårt), kasta ett litet föremål på nåns pung, skjuta någon med soft air-gun eller häftapparat (på långt avstånd) och att skrika någon ashögt i örat så det gör ont är alla exempel på lätt misshandel. Alla dessa former av lätt misshandel förekommer allt som oftast i fylleceller och på defensiva linjen under en handbollsmatch.

\ditem[Lättlagade festrätter från Anderssons skafferi]\label{laettlagade festraetter fraan anderssons skafferi}
 är en kokbok med enklare rätter som kan serveras såväl till vardag som till fest. Den gemensamma nämnaren är att de alla komponerats av den Malå-ättade\ref{malaa} hippiekocken Ramses Andersson med målet att skapa en spirituell dimension av ätande. Alla rätter rekommenderas att serveras till tonerna av Enya och/eller panflöjt\ref{panfloojt}. Nissepedia\ref{nissepedia} publicerar här ett urval av recepten men för en komplett sammanställning får ni köpa boken. Alla rätter är veganska, så när som på några som innehåller ägg\ref{aegg}.

\uline{Ägg i babaganoush}

\begin{enumerate}
\item Mosa en aubergine till oigenkännlighet.
\item Tryck ner fem skalade och hårdkokta ägg så att ungefär halva kroppen döljs i sörjan.
\end{enumerate}

\textit{Tips: Har du några sverigeflaggor från en gammal prinsesstårta sparade kan du sätta dessa i äggen så blir det ännu trevligare.}

\uline{Rutten frukt}

\begin{enumerate}
\item Köp ett nät klementiner med kort datum.
\item Låt stå i solen tills skalen bytt färg minst två gånger.
\end{enumerate}

\textit{Tips: Bär gärna kläder från Ed Hardy medan du äter ifall magen får svårt att processa alla smakupplevelser.}

\uline{Irish hotdog}

\begin{enumerate}
\item Borra ut ett hål i en rova.
\item Placera en halv purjolök eller annat korvsurrogat i hålet.
\end{enumerate}

\textit{Tips: Önskas dressing så dra en snorloska i hålet först.}

\uline{Ägg á la Slayer}

\begin{enumerate}
\item Koka några ägg riktigt hårda.
\item Skiva upp och lägg ut i ett pentagrammönster på en Slayerskiva.
\end{enumerate}

\textit{Tips: Lyssna på SLAYER!!!!}

\ditem[Lättnad]\label{laettnad}
 är en känsla som ofta infinner sig efter en kortare eller längre period av oro. Lättnadskänslan upplevs normalt som positiv eller mycket positiv, men medför ibland förlorad kontroll av den anala kroppsöppningen, så att en liten fis i vissa extrema fall kan slippa ut. Enligt forskare ska detta vara förklaringen till att extremt höga gaskoncentrationer uppmättes i skandinaviska storstadsregioner den dag då Ålandskrisen\ref{aalandskrisen} fick ett lyckligt slut.

\uline{Historiska exempel på situationer då man upplevt lättnad}

\begin{itemize}
\item Lasermannen grips
\item Maud Olofsson\ref{maud olofsson} avgår
\item Stig-Helmer får tillbaka sitt förlorade bagage i \textit{Sällskapsresan\ref{saellskapsresan}}
\item Nikolaj Valujev\ref{nikolaj valujev} slutar boxa folk i huvudet och börjar istället leta efter snömannen.
\item Det visar sig i den första boken om Spöket Laban att Laban är ett snällt spöke.
\end{itemize}

\ditem[Lättöl]\label{laettool}
 är en form av bärs\ref{ha baers}, vilken till skillnad från resterande släktträdets magnifika utlöpare, maximalt innehåller löjeväckande 2.25 \% alkohol. Lättöl inmundigas oftast av medelålders gubbar som gift sig, skaffat tre barn - men förbjudits av sin partner från att dricka starköl varje dag. Även om det vore fullt möjligt att hinka läskeblask, vatten eller svagdricka till fiskgratängen på tisdagarna, klamrar sig den medelålders gubben fast vid den enda kvarvarande spillran av sin, i forna dagar skimrande, maskulinitet - det faktum att han åtminstone \textit{älskar smaken} av öl. Ibland träffas medelålders gubbar på ett slags minikonvent, så kallade \quotetext{konferenser}, på vilka det är legio att dricka lättöl till lunchen och på kvällen köpa prostituerade.
I Danmark\ref{danmark} har svensk lättöl varit olagligt sedan 1920-talet och i Tyskland\ref{tyskland} har det aldrig funnits en enda lättöl.

\ditem[Lådaktivism]\label{laadaktivism}
 Vid flera tillfällen i Sveriges historia har olika grupperingar och personer testat varianter av att placera en person i en låda som medel för att nå sina mål. Folke Pudas använde sin pudaslåda\ref{pudaslaada} för att processa mot länsstyrelsen\ref{processa mot laensstyrelsen} och vid två tillfällen har terrorister gjort försök att kidnappa rika eller mäktiga personer för att sedan placera dom i liknande lådor, i dom fallen talar massmedia om lådterrorism snarare än om lådaktivism.

\ditem[Långfredagen]\label{laangfredagen}
 är en del av påsken\ref{paask} och är den dagen då Jesus\ref{jesus} Kristus hängde på korset och var nära på att dö av de pinor som Pontius Pilates utsatte honom för. För att fira detta låter folk sig själva och sina barn ha fruktansvärt tråkigt.

\uline{Tips på aktiviteter}

\begin{itemize}
\item Se Werner Herzogs \textit{Fitzcarraldo}
\item Spela svälta räv\ref{svaelta raev} och fia med knuff.
\item Lyssna med ett halvt öra på SRs Trädgårdsdags eller \textit{Vapen och ammunition} av \quotetext{Sveriges\ref{sverige} största rockband}, Kent\ref{kent}.
\item Skjut omkring farmor i \quotetext{finrummet} i hennes rullstol
\end{itemize}

\ditem[Löneförmån]\label{loonefoormaan}
 Allt du kan bära är löneförmån.

\ditem[Lördag]\label{loordag}
 är enligt många den bästa dagen, för då är man ledig och har dessutom ännu en hel ledig dag att se fram emot (dvs. söndagen)\ref{soondag}. Lördagen är enligt Christian Information Service Homepage den egentliga sabbatsdagen. Därför har man i Umeå med omnejd delat ut flygblad i vilka vikten av att fira sabbaten på lördagen framgår med all önskvärd tydlighet. På lördagen får man äta lördagsgodis och spela på tipset.


%%%%%%%%%%%%%%
\newpage
\null
\\
\null
\\
\Huge
M
\normalsize
\\
\null
\\
\null
%%%%%%%%%%%%%%


\ditem[Mackshopping]\label{mackshopping}
 Vanligt vid haschrökeri och sådana dagar då fulla människor inte i förväg planerat att bli fulla, vilka också råkar vara de enda två tillfällen då man fortfarande drar sticka för att bestämma vem som ska tvingas göra något jobbigt. Väl på macken köper den fulle hutlösa mängder folköl för jättemycket pengar. Den stenade\ref{stenad} har ett helt annat sätt att ta sig an situationen: hen tittar på allt i butiken och tar ett individuellt beslut för varje vara, dvs köpa eller inte köpa. Normalt slutar det med en jättepåse med plockgodis, en Pepsi Max samt en Nöt-Crème att suga i sig på vägen hem, men det är inte ovanligt att ett gäng karbinhakar, 20 m metrev samt en presentinslagen domkraft åker med. För folk som är nyktra och över fyrtiofem år fungerar bensinmacken på samma vis som skivaffärer fungerar för oss andra: Här finns ett ställ fullt med 20 olika compact discs med Jill Jonsson och Tomas Ledin\ref{ledingreppet} som det bara är att välja och vraka från.

\ditem[Mads Mikkelsen]\label{mads mikkelsen}
 (född 22 november 1965) är Danmarks\ref{danmark} största skådis. Han slog igenom 1996 med filmen \textit{Pusher}, som enkelt kan beskrivas som en dansk motsvarighet till \textit{Sökarna}. Därifrån var vägen till Hollywood spikrak och han har bland annat spelat in \textit{Nattens Engel} och en reklamfilm för reseföretaget Ving sedan dess. Hans fans kallas \quotetext{mikklare} och på Mads-konvent är det vanligt att dessa träffas nakna på en strand och super hejdlöst tills tidvattnet spolar bort dem. Han är dubbad pilsnerdræng\ref{danska hedersbetygelser} vid det danska hovet, som är den näst finaste utmärkelse en privatperson kan få i landet.

\ditem[Maginotlinjen]\label{maginotlinjen}
 En historisk, militär försvarslinje längs Franska gränsen mot Tyskland\ref{tyskland}. Under första världskriget stod denna linje som försvar mot ett tyskt anfall från öster men någon osedvanligt förslagen tysk general kom på genidraget Schlieffenplanen som gick ut på att gå genom Belgien\ref{belgien} och således kringgå denna försvarslinje. Belgarna bjöd som väntat inget motstånd och vips var Frankrike ockuperat. Nu blev inte det här sista gången som Frankrike behövde försvara sig mot den tyska mustigheten\ref{den tyska mustigheten}, utan två decennier senare fick de en andra chans. Tyskarna rustade på nytt för krig och fransmännen skulle försvara sig mot ännu en ockupation. En fransk general lär ha sagt \quotetext{Ok, de gick runt maginotlinjen förra gången, men jag tycker vi testar, varför skulle de gå genom Beligen igen?} Tyskarna lär ha skrattat under hela marschen genom Belgien ända tills stöveltrampen åter ljöd på Champs-Ellysés.

\ditem[Malå]\label{malaa}
 (umesamiska: Máláge) är en ort i Västerbottens\ref{vaesterbotten} inland.
Orten fick sitt namn efter att några samer kastat malätna renskinn i det närliggande vattendraget Malån, antagligen på fyllan.

Förr i tiden fanns det en blöjfabrik här, men den bommade igen. Numera kretsar denna industriort kring sågverket\ref{saagverk} och verkstadsindustrin Hultdins, som gör gripklor. Malå hade ett tag kommunsloganen\ref{kommunslogan} \quotetext{Makalösa Malå!}. Det roliga är den höga koncentrationen ungkarlar, eller som det heter på orten, gammpojkar\ref{gammpojkar}. Malå är även ett demokratiskt föredöme som får Sjöbo att blekna. Man folkomröstar gärna och folket röstar rätt, varenda gång! Ett stycke Schweiz mitt i de västerbottniska skogarna. Till exempel röstade enade Malåbor igenom ett bestämt NEJ till att sörlänningarna skulle dumpa skiten från deras kärnkraftverk här.

\uline{Bra suparställen i Malå}

\begin{itemize}
\item Valfritt vindskydd i skidbacken Tjamstan\ref{tjamstan}
\item Omklädningsrummen vid Solviksbadet
\item Bastun vid sågen
\item Hemma hos någon eller i någons bil
\item Malå Hotell
\end{itemize}

\ditem[Malålistan]\label{malaalistan}
 är ett politiskt parti verksamt främst i Malå\ref{malaa}. Partiets ordförande är moderaten\ref{moderat}, tillika spaägaren Arne Hellsten som ville verka under annan flagg. Rött och blått övergavs för en purpurnyans och partiet (eller listan) sa sig verka för Malås bästa. Malås bästa råkade som av en händelse vara samma sak som företagarnas bästa mest hela tiden. Mycket märkligt\ref{maerkliga sammantraeffanden}.

\ditem[Malåparkering]\label{malaaparkering}
 En Malåparkering är att parkera mot körriktningen, eller att parkera på huvudled. Bakgrunden till att Malåbor gör dessa parkeringar är att de är så oerhört sugna på att äta på korvkiosken att de inte har tid att följa diverse trafikförordningar.

\ditem[Malårca]\label{malaarca}
 Har ni tänkt på hur jävulskt drygt det är att åka på charter? Det hade i alla fall ett driftigt gäng i Malå\ref{malaa} som bestämde sig för att ta russinen ur kakan och ha charter hemma i byn istället. Hyr en industrilokal, fyll den med sand, höj termostaten och börja blanda paraplydrinkar. Vips: \textit{Malårca}

\ditem[Manet]\label{manet}
 är lätt ett av djurrikets mest doomiga djur. Det finns en manet som heter \textit{Nemopilema Nomurai} som kan bli 2\ref{tvaaa} meter i diameter och väga upp mot 200 panner. Maneter är gjorda av ett slags slime och påträffas mycket djupt ner i havet där det är kallt och mörkt och jävligt och det är så högt tryck att ens huvud\ref{huvud} sprängs om man länar sin ubåt och försöker simma omkring. Först sprängs cyklopet så att glaset far rakt in i fejset på en och sen sprängs, som sagt, huvudet. Djupt där nere pumpar sig maneten fram och dödar genomskinliga djuphavsfiskar medelst ett slags celler som fungerar som små mikroskopiska harpuner som pumpar in gift i bytet. Maneten har ingen mun\ref{mun} och behöver ingen eftersom den typ är en mage och bara behöver ligga intill sitt byte så smälter det och så suger maneten upp det och åker vidare på sin ständiga jakt på genomskinliga djuphavsfiskar.

\ditem[Mangel]\label{mangel}
 är ett mängdmått för stora kvantiteter. För att få full betydelse måste manglet kombineras som suffix med en betydelsebärande del. Om man exempelvis skickar ner ett helt paket korv i stekpannan har man ett korvmangel, om man skottar hela gården istället för bara en liten stig\ref{stig} har man ett snömangel, och om man ser alla avsnitt av Benny Hill på raken har man ett buskismangel. På danska betyder \quotetext{mangel} brist (på något), eftersom dansken alltid ska vara lite annars.

\ditem[Mango safe]\label{mango safe}
 är ett slags lösenord som räddar en fisande person från att bli slagen av människor som drabbas av fisen. Genom att uttala \quotetext{mango safe} innan någon av offren hinner säga \quotetext{door knot} upprättas amnesti för fisaren.

\ditem[Mani]\label{mani}
 Att ha mani på något är att vara nästan lite \textit{för} intresserad. Är man till exempel intresserad av insekter är det fullt normalt att tatuera in The Locust logotyp på armen och ha en affisch på en jättestor gräshoppa hemma. Om man däremot flyttar ut i skogen under sommarhalvåret för att bo i en myrstack börjar intresset övergå i mani. Tycker man att Creedence Clearwater Revival\ref{the fog} är ett svängigt band är det fullt normalt att köpa alla deras skivor och ha en av deras låtar som ringsignal. Om man däremot börjar odla helskägg och ha en racercykel inomhus för att snabbt kunna gestalta omslaget till \textit{Cosmos factory}, och börjar varje dag med att klättra upp på hustaket för att spela luftgitarr\ref{luftgitarr} och sjunga \textit{Rambel tamble}, ja då har det börjat övergå i mani. På portugisiska betyder mani \quotetext{kung}, ett tydligt tecken på hur rimligt det egentligen är med monarki. Att ha mani på något ska inte förväxlar med att vara en vanlig gammal hederlig stofil.

\ditem[Manowar]\label{manowar}
 Amerikanskt rockband som med lika delar wagnervurm, thule-covers och manlig fåfänga delar Nissepedia\ref{nissepedia} och världen liksom Sagån klyver Svealand i en bebolig del och den själsliga öken som breder ut sig med runstenarna. Bandets kreativa höjdpunkt nåddes med skivan \textit{Sign of the Hammer} där DeMaio briljerar i hednisk Mythos och mustiga bassguitarriff. Skivorna \textit{Fighting the world} och \textit{Kings of metal} slog ner som Balders död i åttitalets pojkrum. Dessa två skivor innehåller de flesta av bandets stora verk, \textit{Hail and Kill}, \textit{Carry On}, \textit{Metal kings} m.fl. Här börjar bandets kreativa motor Joey DeMaio närma sig allkonstverkets ideal som trots de tidigare skivornas geist inte kunnat uppfyllas. Nämnvärt är preludiet i \textit{Hail and Kill} där en aura av äkttysk violoinvibrato tycks sväva som en salig ande, den rättframma hyllningen och presentationen av övermänniskan: författaren själv. Konstnärligt inleder dock bandet nu en snart 30-årig ökenvandring. Mer eller mindre tramsiga rekordförsök sätter sordin på festen. Datorernas intåg i pojkrummen och nittiotalets vitmaktvåg har dock skapat ett nytt träsk i vilket bandet åter slagit rot. 

\ditem[Manslyssna]\label{manslyssna}
Under det senaste decenniet har alla som tagit del av svensk språkdebatt blivit medvetna om en uppsjö nyord som alla förklarar hur tjejer gör saker. De tjejgillar, de tjejlyssnar och tjejsamlar. Vad som är utmärkande för alla de här definitionerna är att tjejer enligt språkrådet inte riktigt tar saker på allvar. Tjejgillar man nåt så är det en kort romans, om man tjejlyssnar så lyssnas det maniskt på en låt i en vecka, sen kastar man den på historiens sophög, och tjejsamlar man på nåt så har man typ en av Star Wars-filmerna på blue ray-dvd, inte allihopa.

Vad innebär det då att göra saker som en man? Utan en vidare definition kan man få intrycket av att det automatiskt är bättre att mansgöra något. Men många av oss har levt och sett konsekvenserna av att göra saker som en man. Och det är långt från alltid vackert.

När man manslyssnar, då duger inget halvhjärtat. Tanken på att bara lyssna på \textit{Wish you were here} från Pink Floyds skiva med samma namn, och inte hela opuset i en sittning, orsakar svår frossa hos manslyssnaren. Ska man lyssna på Pink Floyd, då ska man först ägna fyra timmar åt att spika upp äggkartong i sin [[dojo]]/garage för att få perfekt ljudmässig upplevelse, och sen sitta hela skivan och fälla 1 - 7 tårar i hemlighet.

En manslyssning kan också helt förvränga identiteten hos lyssnaren, särskilt om det inträffar i ett känsligt emotionellt skede. Om nån börjar manslyssna på Black Flag efter ett romantiskt uppbrott, till exempel, resulterar det inte sällan i att personen rakar av sitt hår, börjar träna som fan och går runt barbröstad i Levis 501:or. Och varje gång personen introducerar sig med sitt namn (typ Tord), så lägger den tvångsmässigt till ett väsande \quotetext{and you're here with me now...}. Oftast slutar det med att personen förlorar sitt jobb och alla sina kompisar.

Särskilt stor skada kan åstadkommas om personen lyssnar på Bob Dylan under sina tonår. Det resulterar ofta i att personen börjar läsa keff beatlitteratur, skaffar Göbbelsbrillor, för stor manchersterkavaj, börjar röka (taffligt) rullade cigaretter och ser ut som en undermåligt klädd spastiker i sina försök att emulera Dylan eller hans svenska motsvarighet, Bruno K. Öijer. Det trauma som exponering för Highway 61 Revisited innebär är ofta irreparabelt och hemsöker manslyssnaren under hela dess liv.

\ditem[Manuel]\label{manuel}
 är namnet på den kypare som spelas av Andrew Sachs i den brittiska TV-serien \textit{Fawlty Towers}. Tack vare sin spanska brytning anses Manuel vara nästan lika festlig som en skotte\ref{skottar} i kilt, enligt den vita arbetarklass som de flesta Oi!-band härstammar från. Kanske blir brytningen extra festlig i och med att Sachs egentligen härstammar från Tyskland. Det kommer vi aldrig få veta säkert förens Nissepedia\ref{nissepedia} lanseras på spanska. Förmodligen är så inte fallet eftersom man i större delen av Spanien\ref{spanien} valde att dubba och döpa om honom till italienaren Paolo och i Katalonien till mexikanen Gonzales.

\ditem[Martine]\label{martine}
 var den första franska apan i rymden. 7 mars 1967 gav sig Martine ut på sin historiska resa till rymden där hon levde i flera dagar innan någonting gick snett. Spacerockbandet Yuri Gagarin har en skiva döpt efter Martine. 

\ditem[Margaret Thatcher]\label{margaret thatcher}
Kärring som knullade den brittiska arbetarklassen på det ena onämnbara sättet efter det andra. Hon har också, verkar det som, varit en förebild för Maud Olofsson\ref{maud olofsson}. Bara det är värt en hel del förakt.

Men kanske mest känd är hon för att ha varit med i gruppen på J. Lyons and co. som uppfann mjukglassen, eller i alla fall hur man blåste upp glass med luft så att folk får mindre götta för pengarna. Lika osympatiskt som resten av hennes gärning.

\ditem[Margit Sandemo]\label{margit sandemo}
 Författare som är Norges\ref{norge} svar på Storbritanniens J.R.R Tolkien\ref{j.r.r tolkien}. Sandemos böcker handlar om isfolket och innehåller till många läsares glädje en hel del ångande erotik. Bokserien om isfolket består av hela 47 böcker och gavs ut mellan 1982 och 1989, vilket betyder att Sandemo, vars namn för övrigt kommer från den impopulära norska maträtten sandmos, skrev i genomsnitt en miljon böcker om året.

\uline{Vid sidan av författandet}

Vad många inte vet om Sandemo är att hon vid sidan av författandet också är en av de främsta företrädarna för pösbyxemärket Wu-wear. Detta upptar sedan sent nittiotal all Sandemos tid. Hon åker runt i världen och för märkets talan i olika sammanhang. Från talarstolen i FNs generalförsamling häromsistens. 

\ditem[Maski Hallonen]\label{maski hallonen}
 (1921-1947) blev som tjugoåring tvångskommenderad av finska armén att utvandra till Sverige tillsammans med ett hundratal andra unga finnar. Deras uppdrag var att samla pengar till krigsinsatsen på hemmafronten, kosta vad det kosta ville. De unga männen sökte jobb på flera industrier, men hade ingen tur. I slutändan blev krigarna tvungna att plocka bär för att skaffa pengar till fosterlandet.

I Sverige har en ordvits uppstått kring Hallonen, där man kallar honom Finlands\ref{finland} sämsta bärplockare på grund av hans namns lustiga betydelse på svenska. Sanningen är den att Hallonen skötte sig bra. Inte bäst, men bra. Vad han tyckte om vitsandet uppdagades aldrig då han kort efter sin återkomst till Finland\ref{finland} avled i sviterna efter ett fall av parasitsjukdomen testikulär maggot.

\ditem[Masonitemuséet i Rundvik]\label{masonitemuséet i rundvik}
 är Sveriges enda museum som enbart handlar om masonit. Museet är baserat i en skolsal på övervåningen i Rundviks gamla skola. Det var här företaget Masonite, ägt av amerikanen William H. Mason (som var chefsingenjör hos Thomas Edison), licenserade Europas första fabrik för tillverkning av masonitskivor till Nordmalings Ångsåg Aktiebolag år 1929. Här kan besökare exempelvis se gamla reklamanslag som meddelar att masoniten ersätter inte bara trä, utan även plåt och marmor.

Masonitemuseet invigdes på Rundviksdagen 2004 och hålls öppet vid större arrangemang eller efter överenskommelse.

\ditem[Matematikmaskinnämnden]\label{matematikmaskinnaemnden}
 Statlig myndighet som fram till 1963 ansvarade för att förse Svea rike med datorer. Lades ned av Tage Erlander som en del i förarbetet inför valrörelsen 1964. Strategien var lyckad och fick Folkpartiets\ref{folkpartiet} storbossnörd\ref{storbossnoord} Bertil Ohlin att ragequitta flera debatter.

\ditem[Math metal]\label{math metal}
 är metal som till 99\% eller mer bygger på avancerade och snabbt spelade skalor. Math metal bör undvikas då konsumtion av genren kan leda till att man utesluts ur alla former av social gemenskap. Exempel på band insom genren: Opeth.

\ditem[Mats Lundgren]\label{mats lundgren}
 är en busschaufför i Umeå som lessnade på den outhärdliga sommarvärmen. Det fascistiska bussbolaget Nobina tillåter inte sina manliga busschaufförer att ha shorts på sommaren då det inte är en del av uniformen. Den driftige Lundgren upptäcker då att kjol är tillåtet och kör på det istället. Hade ett sådant pris funnits hade Lundgren snarast blivit tilldelad Folke Pudas-priset\ref{pudaslaada} för sina insatser i kampen mot byråkratin.

\ditem[Mattias Alkberg]\label{mattias alkberg}
 Mattias \quotetext{Matti} Alkberg är en musiker från Luleå, i Norrbotten. Mattias har spelat i nästan lika många band som han har barn. I en legendarisk intervju i Måndagsklubben påstod faktiskt Mattias att hans två stora hobbys här i livet var att starta band och att produktionsknulla\ref{produktionsknull}. Mattias bor i Luleå. Mattias gillar inte Umeå\ref{umeaa} så mycket, för han tycker det är töntigt där. Hans skivor ges ut av ett bolag i Umeå och han är ofta där och spelar. Ingen vet hur Mattias förhåller sig till att det mest är folk han tycker är töntiga som gillar hans musik. Han tycker att det är coolt med råpunk. Mattias har haft ett band som spelar råpunk, fast de var inte så bra så han började spela pop istället. Och nu ska han starta ett band som låter typ som Black Flag. Och det är ju rätt Umeå anno 2004 att göra, men jaja det blir nog fett.

\ditem[Maud Olofsson]\label{maud olofsson}
 Maud Olofsson (även smeksamt kallad Mao Näringslivsson) är en ivrig bäver\ref{ivriga smaa baevrar} från Robertsfors som hatar arbetarklassen. Tack vare sin roll som överstepräst i bondeförbundet\ref{centerpartiet} har hon lyckats dupera kulakerna\ref{kulaker} att stödja henne ett flertal mandatperioder. Tack och lov har hon avgått. En del firar detta med stora smörgåsar, dagobertmacka, smörgåstårta eller knäcke med dubbelsovel\ref{dubbelsovla}.

\ditem[Max Weber]\label{max weber}
 var en tysk professor som anses varit medgrundare till samhällsvetenskapen sociologi. Vad vanligt folk minns Weber för är dock inte hans akademiska arbete, utan för att han uppfann klotgrillen. Anledningen till att det var just Weber som uppfann detta var att han kom på att man kunde använda gallret från rationalitetens järnbur i grillarna.

Weber plågades hela sitt liv av problem med det sexuella, främst den egna potensen. När de flesta andra led i tystnad över dessa problem tyckte Weber att det var en bra idé att diskutera saken med sin dyra moder. Hade Sigmund Freud varit samtida med Weber hade antagligen denne gått bananas\ref{bananas} över detta.

\ditem[Medelklassvänner]\label{medelklassvaenner}
 En vänskap man gjort genom exempelvis ett gemensamt intresse för trädskällare eller modern litteratur. Allt känns bra tills man för första gången blir hembjuden till den nya bekantskapen. Har man inte en halv miljon i studieskulder, skjorta från Dressmann\ref{dressmann} och upplever ifyllandet av blanketter för ROT-avdraget som sitt största problem kan stämningen lätt bli lite stel. Det kan man lösa genom lite lekar och spel.

\ditem[Medeltiden]\label{medeltiden}
 var, som namnet antyder, ingen braksuccé. Man levde på kottar och bär, var sjuk hela tiden och var träl åt någon rik. \quotetext{Alla som gillar medeltiden är inte idioter, men alla idioter gillar medeltiden}, som Carl von Linné\ref{carl von linné} träffsäkert lär ha beskrivit det. På medeltiden leve Petrus de Dacia\ref{petrus de dacia} och Hildegard av Bingen.

\ditem[Merchband]\label{merchband}
 Ett merchband är ett band som tillhandahåller ett brett utbud av merch i form av tshirts, backpatches, muggar osv. men som sällan släpper nytt material eller mest spelar dålig rock.

\uline{Förtydligande exempel}

\begin{itemize}
\item Ghost
\item The Misfits
\item Exploited
\item The Adicts
\item Chicago Bulls
\end{itemize}

\ditem[Metalmynt]\label{metalmynt}
 är ett slags poletter av plast som används som valuta\ref{valuta} vid köp av öl på diverse hårdrocksfestivaler inom den Europeiska Gemenskapen. Hårdrockaren byter in sina vanliga konventionella pengar mot dessa metalmynt och kan sedan förse sig med starkvaror i barer och öltält runt om på festivalområdet. Utanför detta är metalmyntet värt väldigt lite och accepteras sällan eller aldrig som pengar i det vanliga civilsamhället. Den kände författaren och folkbildaren Professor Etienne\ref{prof. etienne} ska enligt egen utsago vid ett tillfälle ha försökt föra in falska metalmynt till ett värde av femton kubikmeter öl på den belgiska\ref{belgien} hårdrocksfestivalen Deathfest. Då interpol fått nys om kuppen tvingades han dock överge dessa i ett irrigationsdike och betala en flamländsk bonde att föra honom över gränsen till frankrike gömd i ett hölass.

\ditem[Micke Alonzo]\label{micke alonzo}
 Sångare i det legendariska punkbandet KSMB och Sveriges\ref{sverige} genom tiderna mest underskattade rockband: Stockholms\ref{stockholm} negrer. Han har mycket kort, nästan osynligt hår och har problem med vissa feminister\ref{feminism}, som han tycker har gått lite för långt.

\uline{Privatliv}

Stora obesvarbara frågor väller upp inom Micke Alonzo. Han känner att han är som en roderlös båt som guppar på ett ödsligt hav. Ibland går vågorna höga och hotar att kantra den lilla båten. Andra gånger ligger ytan blank som en spegel och havet, enormt och tomt, breder ut sig i alla väderstreck. Alonzo ser sin spegelbild i havet. Han rör vid sitt ansikte och frågar viskandes; Vem är jag? Vad är meningen med allt?

\ditem[Mikis Theodorakis]\label{mikis theodorakis}
 är Greklands främste tonsättare. Det spelar ingen roll vilken ton man frågar om så sätter han den på en gång. Man ba \quotetext{Ja men den där då?} och så pekar man på en svart tangent på pianot mitt bland alla andra. Han ba \quotetext{det är c-moll}, utan att blinka. Inte ens om man krånglar till det och säger \quotetext{amen rita ett G i fis-dur då}. Då tar han fram papper och penna och ritar \textit{exakt} som det ser ut. Theodorakis är lika grekisk som en grekisk bondsallad och spelas flitigt på alla rikets charterorter. Otaliga gånger har, på Retsina och Rhaki, grisfulla\ref{grisfull} brittiska kvinnor dansat på, och ramlat ner från, borden på bodegor till tonerna av Theodorakis \quotetext{Zorba}.

\ditem[Miljöbil]\label{miljoobil}
 Det lobbas hårt av både stat och kapital för att alla ska köpa sig en miljöbil. Anledningen till detta är att det inte längre går att blunda för att jorden håller på att gå under på grund av koldioxidutsläppen, men att göra något så vansinnigt som att producera och konsumera mindre, det kommer inte på fråga! Miljöbilar är enligt etablissemangets definition elbilar, etanolbilar och snåla dieselbilar med partikelfilter. Här manar Nissepedia\ref{nissepedia} till eftertanke och ber er ponera följande: Vilken bil är bättre för miljön? Den bil som körs eller den bil som står still? Självklart är det bilen som inte brukas och således är den sanna miljöbilen en riktig jävla rishög. Denna bil har en livslängd på max ett halvår efter inköp om inte dess ägare är en jävel på att skruva, sen är det raka spåret till skroten där den aldrig mer är Moder Jord till last. Slutet gott, allting gott.

\ditem[Miljöpartiet]\label{miljoopartiet}
 är ett politiskt parti skapat av den norske entreprenören Jam-Ole Rip Tit. Namnet \quotetext{Miljöpartiet} är ett anagram av grundarens eget namn men betyder egentligen ingenting.
Jam-Ole Rip Tit bor numera på Bahamas.

\ditem[Mimmi Pigg]\label{mimmi pigg}
 är ett sätt att klä sig där underkroppen är påklädd och överkroppen bar. För att Mimmi Pigg-klädnaden ska bli fulländad bör man även komplettera med ett par väldigt fula pumps.

\ditem[Min kära gamla soppeskål]\label{min kaera gamla soppeskaal}
 är en sång av Sveriges\ref{sverige} proggflaggskepp Philemon Arthur \& the Dung. Låten släpptes för första gången år 2002 på bandets skiva \textit{Får jag spy i ditt paraply?} (Silence records) och blev en omedelbar hit bland landets batikklädda befolkning. Sången premierades framförallt för sitt allmogebudskap och sin enkla refräng som var lätt att komma ihåg och sjunga med i. Philemon Arthur \& the Dung hade inte haft en sådan monsterhit sedan 1992 års \textit{Plocka päron\ref{plocka paeron}}.

På Österlen i Skåne fick sången sådant genomslag att Peps Persson bildade ett särskilt sällskap, Soppeskålsorden, vars medlemmar betraktade föremålet i visan som en helig graal som skulle återbörda alfalfagroddar på menyn i varje skolmatsal. Denna sammanslutning hade som mål att finna soppeskålen och föra hem den till Silence musikstudio i Koppom.

\uline{Text}

\textit{Min käre gamle soppeskål jag har dej så kär}

\ditem[Minusmat]\label{minusmat}
 Mat som ger mindre energi vid intaget än det går åt när man äter.
Resultatet blir att man blir smalare av att äta än att ligga på soffan.

Exempel:

\begin{itemize}
\item Ris (äta med pinnar)
\item Pommes frites (äta en och en)
\item Kräftor (jättepilligt med skalet)
\item Myror (för björnar)
\end{itemize}

\ditem[Missbrukare]\label{missbrukare}

När den moderna fabriken uppfanns tvingades folk sluta dricka ren sprit varenda dag i Sverige. Innan, i jordbrukssamhället, var det helt okej att vara packad hela tiden. Man behöver inte särskilt mycket finmotorik för att gå runt och stoppa rovor i marken, nämligen. Och för husmodern var det i stort sett ett krav att vara konstant [[grisfull]], hur annars skulle hon palla att ta hand om alla sina 1800-talsbarn?

Men i och med fabriken var det slut på det roliga. De stora maskinerierna kapade många arbetargubbars händer i början, innan man insåg att det inte längre var hållbart att hälla brännvin i gröten på morgonen. Och när gubbarna inte längre fick dricka blev de så avundsjuka på sina tanter som bara satt hemma och festade och gormade hela dagarna, att de tvingade dem att sluta också.

Sedan dess har det inte varit okej att vara full eller påverkad av andra substanser mer än två gånger i veckan i Sverige. Men det är en del ändå. Vissa kan kontrollera sitt drickande, vissa gör det tvångsmässigt, för att kunna hantera sin vardag. De senare kallas för missbrukare. Ofta är folk missbrukare för att de mår dåligt inombords. I vissa länder i världen tycker man lite synd om missbrukare och försöker hjälpa och rehabilitera dem på olika sätt.

I Sverige har en annan väg valts. Den svenska modellen när det kommer till synen på missbrukare är en elakartad kombination. Dels finns i Sverige en djupt rotad socialdemokratisk misstro mot alla som inte vaknar skrattandes 06.00 på vardagar och med ett leende på läpparna promenerar till sin pappersmassefabrik. Sen finns också en lång protestantisk tradition av att hata folk som är svaga och på något sätt står utanför majoritetssamhället. Så istället för vård utsätts missbrukare för en oerhörd skambeläggning och kriminalisering, som i många fall tar livet av missbrukaren. För om man först hatar sig själv och därför missbrukar, och sen ovanpå det inser att hela samhället hatar en för att man missbrukar, minskar inte direkt sannolikheten att man ökar dosen av det man missbrukar.

Den bästa chansen man har att leva ett gott liv som svensk missbrukare är att flytta till Danmark, som alla sina brister till trots, har anständigheten att se på missbrukare som människor som ska hjälpas, inte som kriminella monster som ska bekämpas.

\ditem[Mjukis]\label{mjukis}
 Ett adjektiv som ofta används för att beskriva någon vars främsta karaktärsdrag är snällhet. I de flesta fall används det på ett uppskattande sätt. I dessa fall främst av kvinnor om män, då kvinnor ofta redan antas vara mjukisar och det därmed blir överflödigt att kommentera eftersom det inte är ett brott mot en vedertagen norm. När det används av män om andra män är ofta tonfallet hånfullt, då mjukisen ofta \quotetext{går hem} hos det motsatta könet, vilket sällan uppskattas av andra män i mjukisens närhet.

Uppmärksammas bör att mjukis i obestämt plural, mjukisar, lätt kan blandas ihop med slangbenämningen på mjukisklädsel (mjukisar). Förvirring av begreppen bör särskilt undvikas då mjukisar inte alltid bärs av just mjukisar. Många gånger bärs mjukisbyxor av sportutövande ungdomar, en demografisk grupp som i stort sett aldrig uppvisar mjukisens mest prominenta egenskap; snällhet.

Den mjukis som oftast bär mjukisbyxor är den långtidsarbetslösa mjukisen, som nio gånger av tio har gått en estetisk utbildning på gymnasiet för att sedan komma ut i ett kargt samhällsklimat. Mjukisen misslyckas i samhället på grund av dess oförmåga att \quotetext{vässa armbågarna} och \quotetext{ta för sig}. När den sortens mjukis använder mjukisbyxor kallar den inte detta plagg för \quotetext{mjukisar}, som den sportande ungdomen, utan snarare \quotetext{fisbyxor}. Mjukiströjan försvann i stort sett helt från marknaden under 90-talet och används idag så gott som uteslutande av två grupper där den första är folk som tycker det är ballt med retro. Den andra gruppen som fortfarande storkonsumerar mjukiströjor (uteslutande i kombination med mjukisbyxor) är handbollsmålisar, vilka alltid har tjänat som fanbärare för mjukismode.

\ditem[Mjukpack cigg]\label{mjukpack cigg}
 är en benämning för den behållare som flera typer av cigaretter säljs i. Mjukpacket skiljer sig mot andra ciggpack genom sitt mjuka material, andra ciggpack är för det mesta hårda. Mjukpackets främsta egenskap utgörs dock av att det alltid finns minst en cigg kvar i det. Mjukpack går utmärkt att förvara i sin bakficka\ref{bakficka}.

\ditem[Mob 47]\label{mob 47}
 är ett svensk råpunkband från Stockholm som hållit på sen -81. Deras texter handlar mest om vad de tycker om saker. Det här tycker dom inte om: systemet, hela jävla samhället, David Bowie, rustning, bylingen, politiker, religion, krig, lögner, rasistiska regimer, maktmissbruk, kärnvapen och mycket mer. Det här tycker dom om: resning mot överheten, frihet, fred, cellbyggnader, nedrustning, ändringar i systemet, rättvisa, barn, att inte rösta och inget mer. Dom tycker också att alla djurförsök är mord. För att kunna sjunga om andra saker än vad dom tycker har dom ett sidoprojekt som heter Protes Bengt. Deras vanligaste logo är ett kranium med tuppkam, vilket förstås är jävligt coolt.

\uline{Medlemmar}

\begin{itemize}
\item Mob 47-Åke - Gitarr\ref{gitarr}
\item Chrille - Trummor
\item Johan - Bas
\end{itemize}

Det är flera andra som också varit med men de har slutat.

\uline{Diskografi}

\begin{itemize}
\item \textit{Hardcore attack} kassett utgiven 1984
\item \textit{Kärnvapen attack} 7" utgiven 1984
\item \textit{Dom ljuger igen} 7" utgiven 2008
\end{itemize}

Dom har även varit med på massa, massa samlingsskivor.

\ditem[Modem]\label{modem}
 Ett modem är en liten låda som gör att alla 1:or och 0:or som susar runt i kablar och luften förvandlas till roliga bilder på djur i din dator. Utan modem skulle inte hemsidor som Passagen och Blabbermouth ha några besökare och då skulle allt det ideella arbete som hårdrockare lägger ned på att få fram det senaste om Lemmys påstådda inkontinens vara förgäves.

\ditem[Moderat]\label{moderat}
 betyder ursprungligen återhållsam, men i politisk parlans benämner ordet medlemmar av högerpartiet Moderaterna. Det finns dock inte mycket som är återhållsamt hos detta gäng storkskurkar, borgarbrackor och stråtrövare\ref{straatroovare}. Moderaterna kännetecknas nämligen av en osund kombination av dåligt tålamod och fullkomlig oförmåga (eller ovilja) till att tänka utanför sammanhang som större än en normal kärnfamilj. Detta är anledningen till att de inte kan betala TV-avgift, inte förstår sig på skatter, offentlig sektor, klassperspektiv och så vidare, förordar svågerekonomi och glatt påhejar överklassens bärsärkagång runt om i landet.

\uline{Två sorters moderater}

Det finns två sorters moderater. Dessa är:

\begin{itemize}
\item De som är genuint onda.
\item De som är helt jävla slut i roten.
\end{itemize}

Den första kategorin modernater förstår mycket väl att deras politik förstör liv och gör vardagen omöjlig för många människor, men eftersom de är onda tycker de att detta är bra, i enlighet med modern satanism i vilken själviskhet och hänsynslöshet som bekant förordas. Den andra kategorin består av sådana som är så rudis att de inte förstår detta utan på allvar tror att privatiseringar, corporate social responsibility\ref{corporate social responsibility}, entreprenörskap, friskolor och allsköns jävelskap\ref{jaevelskap} är bra för allmänheten. Denna sorts moderater är ofta inavlade Täby-bor eller löst folk som har hjärntvättats medelst ett slags infernalisk maskin som partiets härförare Fredrik Reinfeldt\ref{fredrik reinfeldt} förvarar i en skokartong i ett mycket högt torn i sin borg \textit{Zhraflindhur} i Mordor, som han kom vandrande ifrån.

\ditem[Moderator]\label{moderator}
 Moderatkrigare med leveln \textit{Warlord} eller högre. Har hög attack i när- såväl som eldstrid och kan skjuta laser med sitt fruktade (s)kullcrusher-svärd, men saknar helt försvar och babblar istället om att motståndaren önskar ha det som i Gulag.

\uline{Grundvärden}

\begin{itemize}
\item Att. 10
\item Def. 2
\item Mana. 6
\item Char. 1
\end{itemize}

\uline{Levels}

Om du använder till exempel \quotetext{parrot attack} för att öka din moderators Mana till +6 kan den levla till \textit{Harbinger}, men om den uppnår -10 i karisma blir den automatiskt Maria Abrahamsson.

\ditem[Moln]\label{moln}
 är anhopningar av vattenbaserad kondens och inget annat. Molnen skapas av att solen värmer upp vatten som på så vis blir varmare än den omkringliggande luften, varpå vatten kondenseras och stiger upp genom atmosfären. En viss del av det kondenserade vatten som molnen består av kommer från Malå\ref{malaa} sågverk\ref{saagverk}. När molnfronter kolliderar överstiger fuktigheten och större vattendroppar bildas, dessa blir då så tunga att de faller ner mot marken. Detta kallas i folkmun för regn och är något helt normalt och inget att hetsa upp sig över. Vill man veta mer om moln kan man gå till sitt lokala bibliotek och låna International cloud atlas där några entusiaster skrivit ned allt som går att veta om fenomenet.

\ditem[Molotov cocktail]\label{molotov cocktail}
 känner ju alla till som växt upp utanför Lundsbergs låtsasvärld. Vad desto färre känner till är föregångaren \quotetext{Molotovs brödkorgar}, som finnarna kallade de klusterbomber Sovjet ovänligt nog släppte över landet och själva kallade \quotetext{matleveranser} när resten av världen frågade vad dom höll på med.

\ditem[Moona röv]\label{moona roov}
 Att moona röv är ett av de absolut äldsta sätten att driva gäck med sin omgivning. Byxorna halas helt enkelt ned så att röven\ref{praktarsle (positiv)} tittar fram. Ett lyckat moon resulterar oftast i att den som blir moonad hyttar ned näven\ref{hytta med naeven}, medan resten av åskådarna skrattande tittar på, och då har moonaren fått sin belöning. Nyckeln till skämtets popularitet är att det har så få variabler att det i princip inte kan utföras fel. Komikern Adde Malmberg\ref{adde malmberg} har exempelvis uteslutande kört denna repertoar de senaste 15 åren.

\ditem[Motpåven i Gränna]\label{motpaaven i graenna}
 Efter att påve Benedictus XVI meddelat sin avgång, den 11 februari i herrens år 2013, uppstod en schism i den kristna världen. I Vatikanen verkade allt vara business as usual - en ny påve skulle tillsättas, och kardinalerna lovade att han också skulle vara nazist, precis som den förra. Där kunde allt ha varit frid och fröjd, men i omvärlden kokade missnöjet. Den katolska kyrkan framstod för resten av världen som försvagad. För första gången på 800 år hade en påve frivilligt avgått. Vad sänder det för signaler, när killen just under Jesus i den kristna hierarkin inte tycker att det är så kul längre, och hellre startar en kombinationsaffär\ref{kombinationsaffaer} eller nåt?

Skolastiker skyndade till biblioteken för att finna prejudikat för motaktioner mot den romersk-katolska överheten. Och de som sökte, skulle också finna. Inom den katolska tron finns en stolt tradition av så kallade motpåvar, eller antipåvar. De är män som vägrar acceptera att de inte är minst lika balla som han i Vatikanen, och utser sig själva till påvar. Under historiens gång har en mängd motpåvar existerat, där de med starkast anspråk på riktig makt var de som tillsattes i Avignon, av den franske kungen Filip IV på 1300-talet. Nu för tiden är motpåvarna en sorglig skara sektmedlemmar, geografiskt avlägsna från maktens kärna. Men det skulle komma att förändras, 2013 i Gränna.

Varför står religionen utanför marknaden, frågade sig många kristna svenskar? Är inte förfallet inom katolicismen ett tecken på att det blivit för slappt rent konkurrensmässigt i den själsliga världen? I Gränna grunnade Alf Svensson, svensk kristendoms grand old man, tillsammans med sina rådgivare, över dessa frågor. Till slut fattade han ett beslut. Den 14:e februari 2013 utropade sig Alf Svensson till påve, under namnet Albion Venerabilis I. Runt sig hade han en konklav nytillsatta kardinaler, bestående av Marcus Birro och Paolo Roberto. Tillsammans skulle de utmana Vatikanen. Budskapet var att uppvärdera påvedömet, fixa nya kläder, bättre pensionsförmåner, lite mer sex appeal. Regeringen Reinfeldt\ref{fredrik reinfeldt} uttalade sitt aktiva stöd för initiativet, då det bara kunde innebära en frisk fläkt med lite gammal hederlig konkurrens i det där murkna örnnästet. En dag efter sitt tillkännagivande hade Albion Venerabilis I både hunnit erhålla en bannbulla från Vatikanen, sms:at en egen tillbaka, och meddelat sin medverkan i melodifestivalen 2013. Under deltävlingen i Skellefteå, i Sveriges enda sanna bibelbälte, skulle Albion Venerabilis I avslöja sina budord. Paulo Roberto skulle spela ackegura och Marcus Birro hade skrivit text. Spänd förväntan följde.

\ditem[Mount Everest]\label{mount everest}
 är med sina 8848 meter över havet jordens högsta berg. Namnet Mount Everest är dock kattpiss i jämförelse med det tibetanska namnet \quotetext{Jo-mo glang-ma ri} som översätts till det oerhört psych-iga och övermäktiga \quotetext{Universums moder}. Typiskt att nån jävla britt\ref{george everest} döpte det efter sig själv istället.

\ditem[Mun]\label{mun}
 En mun (alt. stav. \quotetext{mund}) är ett hål i ansiktet. Man hittar den nedanför näsan\ref{naesa} och ovanför hakan\ref{haka (vanlig)}. Man använder munnen till att tala, äta, vissla, spela dragbasun\ref{dragbasun}, pussas och snusa. Det enda dåliga man gör med munnen är att räva. Munnen bör rengöras morgon och kväll om man vill undvika fetor ex ore\ref{fetor ex ore}. Vissa målar munnen röd, medan enstaka människor uppgraderar den till näbbmun\ref{naebbmun}, men detta är ganska onödigt tycker många.

\ditem[Mungo Jerryhatare]\label{mungo jerryhatare}
 är människor som anser att den brittiska musikgruppen Mungo Jerry är den sämsta skit som hänt världen sedan Adolf Hitler. Dom tycker att \textit{In the summertime} är så fantastiskt värdelös att lobotomi plötsligt inte verkar så omodernt när dom hör den. Det mest provocerande, tycker Mungo Jerryhatarna, är att Mungo Jerry tack vare låten fått släppa mer än 20 skivor till av ren skit. Varför kunde inte medlemmarna nöjt sig med att leva på STIM-pengarna från all radiotid låten får? Måste de dessutom turnera år efter år och spela okända låtar i 80 minuter för att sedan köra \textit{In the summertime} som andra extranummer. Mungo Jerryhatarna blir helt tokiga när dom tänker på detta och känner för att skjuta alla som heter Mungo i förnamn.

\ditem[Muno]\label{muno}
 är en röd skabbräv som bor vid mynttorget i Norberg. Det är han som ser till att alla bussarna kommer och går i tid. Han lever på korv och mos från Myntgrillen och är allas vän.

\ditem[Musikhögskolemusik]\label{musikhoogskolemusik}
 Folk som spelar musik på musikhögskola gör inte det för att dom gillar musik eller ens är intresserade av musik. Vad de då gör där vet de inte ens själva så tiden fördrivs genom att bilda olika lösa konstellationer inom genren musikhögskolemusik.

\uline{Säkra tecken för att identifiera musikhögskolemusik:}

\begin{itemize}
\item Basisten har en bas med fler än fyra strängar.
\item Gitarristen bär sin fullkittade gitarr strax under hakan.
\item Matilda är en trevlig tjej, så man har flöjt\ref{floojt} med.
\item Stor vikt läggs vid avancerade arrangemang, framförande åsidosätts fullständigt.
\item Bandnamnet innhåller något i stil med, \quotetext{project}, en sifferkombination eller intern humor
\item Om man mot all förmodan spelar inför publik soundcheckar man i tre timmar och gnäller på allt tekniskt som går att gnälla på från telekablar till belysningen på toaletten.
\end{itemize}

\ditem[Musselini]\label{musselini}
 (utr. \textit{Mickeylini}) var en karaktär skapad på Walt Disneys initiativ år 1938, efter att Il Duce samma år infört raslagar i Italien. Då den senare inte stod i någon högre kurs hos allmänheten i USA tyckte anti-semiten Disney att en charmoffensiv kunde vara på sin plats där den Italienska regimen genom karaktären på helsidor i olika serietidningar förklarades vara en bra politisk förebild. Italienska propagandaministeriet avsade sig alla kopplingar till kampanjen för att inte de diplomatiska förbindelserna med USA skulle påverkas negativt, men under bordet postades bilder på Il Duce till Disney studios för att underlätta arbetet. Italienarna ville inledningsvis att deras slogan \textit{Libro e moschetto — fascista perfetto} (\quotetext{Bok och musköt - den perfekte fascisten}) även skulle figurera i kampanjen men detta avslogs av Disneys kampanjansvariga. 

Då kampanjen inte rönte någon större framgång återvände Disney redan år 1939 till att bädda in smygrasism och konservativa ideal i sina större produktioner, sin anti-semitism lät Walt ligga i skymundan fram till sin död 1966.

På sistone har extremhögern inspirerats av Disney och försökt med liknande propagandaknep i samma anda för att deras förebilder skall nå ut till en yngre skara av potentiella rekryter.

\ditem[Muttersvarvare]\label{muttersvarvare}
 Nedvärderande benämning på den svenska uppfinningen skiftnyckeln\ref{skiftnyckel}. Alla vet att det är fasta nycklar\ref{fasta nycklar} som gäller.

\ditem[Myntsamleri]\label{myntsamleri}
 Få men hängivna är de som ägna sig åt myntsamleriets ädla konst. Många tror motsatsen, då livet i västvärlden, som du kanske noterat, i mångt och mycket går ut på att skaffa sig pengar - antingen genom förvärvsarbete eller genom att kanalisera pengar som andra skapar genom förvärvsarbete ner i sin egen maffiga hästhandlarplånka\ref{haesthandlarplaanbok}. Detta är dock inte myntsamleri, utan helt andra företaganden, vars uppslagsord är löneslaveri respektive entreprenörskap. Myntsamleriet går dock också ut på att samla pengar, men dessa måste för att det ska röra sig om myntsamlande vara oanvändbara som valuta\ref{valuta}. Hittar myntsamlaren ett värdelöst mynt blir hen alldeles till sig och tar fram en pincett med vilken hen varsamt flyttar det till en för ändamålet avsedd bok som hen sedan trycker mot sin barm av glädje och själslig salighet.

\ditem[Myspyssockerkaka]\label{myspyssockerkaka}
 är en de bakverk vars recept ingick i det klassiska häftet Vegankokboken, författad av Kristoffer Åberg. Kakan innehåller bland annat vetegroddar och är perfekt när man vill \quotetext{myspysa}, det vill säga fisa ut nedbrytningsgaser i glada medelklassvänners\ref{medelklassvaenner} lag.

\ditem[Mystiska band]\label{mystiska band}
 Vissa band väljer i början av sin karriär att uppträda förklädda för att på så vis skapa en hype kring sig själva. Ingen vet vilka de är. De som är coolast i scenen har sina aningar men säger inget till oss vanliga töntar. Ofta hittar bandet också på skojiga rykten om sig själva, som man låter sprida så att folk ska tycka att man är tuff.

\uline{Exempel på band som är eller har varit mystiska}

\begin{itemize}
\item SunnO)))
\item Lordi
\item Onkel Kånkel
\item Final Exit
\item Slipknot
\item Goat
\item Syphilitic Vaginas
\item Ghost
\item Rednex
\item Hyenaz in the desert
\item Antirep
\item The knife
\end{itemize}

\ditem[Myt]\label{myt}
 En myt är en historia som inte är sann, den har likheter med skröna\ref{skroona} och det har egentligen ingen som helst betydelse vad skillnaderna är eftersom de båda rätt och slätt bara är hittepå eller bluff med i bästa fall en smula sanningshalt.

\uline{Exempel på Myter:}

\begin{itemize}
\item Om man kör 200 Km/h så kan man se sin egen bil i backspegeln.
\item Kör man i 400 Km/h så möter man sig själv på tillbakavägen.
\item Frukt är godis.
\end{itemize}

\ditem[Mäklarbricka]\label{maeklarbricka}
 En mäklarbricka är en bricka\ref{bricka} med vodka\ref{vodka} och groggvirke, samt en eller två champagneflaskor och är en populär beställning/klassmarkör på krogar runt Stureplan i Stockholm. Mäklarbrickan delas sedan med folk runt det bord vid vilket beställaren sitter, speciellt om bordsgrannarna är damer och beställaren herre. Mäklarbrickan säger som inget annat \quotetext{jag har pengar,} \quotetext{jag har social status,} \quotetext{jag har en lätt utvecklingsstörning.}

\ditem[Mäklarsvenska]\label{maeklarsvenska}
 Ett yrkesspråk som kan liknas vid knoparmoj. Syftet är att utomstående inte skall begripa innebörden. Omfattande Nissepediaforskning har uppenbarat några glimtar.

\textit{\quotetext{Äldre ytskikt}}- Du måste byta panel, antagligen köksinredning också.
\textit{\quotetext{Tillfälle för den händige}}- Vattenskada.
\textit{\quotetext{I populära.....}}- Det kan bli svårt att sälja detta hus.

\ditem[Märkliga sammanträffanden]\label{maerkliga sammantraeffanden}
 är när två\ref{tvaaa} eller flera saker på något sätt hänger ihop, till synes utan rimlig anledning. Finns en rimlig anledning är det bara ett vanligt sammanträffande. Nya märkliga sammanträffanden inträffar hela tiden men det gäller att vara uppmärksam för att upptäcka dem. Ser man ett bör man rapportera det snarast till berörd myndighet.

\uline{Exempel på märkliga sammanträffanden}

\begin{itemize}
\item Maud Olofsson\ref{maud olofsson} deklarerar att hon ska lämna offentligheten dagen innan Clarence Clemons dör under mystiska omständigheter.
\item Många chefer kallas för \quotetext{Adolf}.
\item Olof Palmes\ref{olof palme} gata ligger nästan precis där han blev mördad.
\item Göran Greijder och King Buzzo i Melvins har samma frisyr
\item Oljegarker\ref{oljegark} som delar ekonomiska intressen med Carl Bildt (m)\ref{moderat} brukar normalt inte stöta på patrull i svensk utrikespolitisk.
\item Den amerikanske skådespelaren och skämtaren David Koechner är i princip identisk med Ulf Larsson, svensk skämtare, skådespelare och före detta programledare för \textit{Söndagsöppet}.
\end{itemize}

\ditem[Måg]\label{maag}
 En måg är en man som envist framhärdar med att han på något krångligt vis är din ingifta släkting. Ofta kräver mågen saker av dig, inte sällan är det nåt slags jovialiskt sällskap han är ute efter. Han pratar med dig om takisolering eller segelbåtar när han ser dig på släktmiddagar, medan du i ditt stilla sinne tänker \quotetext{vem är du och vad gör du här?}

\ditem[Målvakt]\label{maalvakt}
 En målvakt är en idrottsutövare som hatar bollar och puckar. Så fort en boll eller puck kommer nära målvakten blir den förbannad och gör allt den kan för att schasa bort eller fånga in föremålet. Väl infångat tar målvakten i allt vad den kan för att slänga bort bollen/pucken till en annan målvakt, så det är inga direkta vänskapsband kollegor emellan. Tekniken varierar inom olika sporter men vanligast är att målvakten använder händer och fötter för att få vara ifred. Det avgrundsdjupa hatet mot allt bollformat bottnar oftast i att målvakten var tjock som barn och brukade bli retad genom att kallas boll- eller puckformat pucko.

\ditem[Måndag]\label{maandag}
 är veckans första och kanske sämsta dag (vissa menar, med starka argument, att söndagen är sämst) samt en låt av rockgruppen Stockholms Negrer, i vilken Micke Alonzo\ref{micke alonzo} stod för sången.

\ditem[Mångsysslarpensionär]\label{maangsysslarpensionaer}
 En mångsysslarpensionär är en person som på grund av sin ansenliga ålder inte arbetar på riktigt utan bara på låtsas. Mångsysslarpensionären fyller sin tillvaro med ett stort antal ofärdiga projekt, så som båtrestauration, mattväveri, nedtecknandet av hembygdshistoria, låtskriveri, hembryggeri, dagdriveri och myntsamleri\ref{myntsamleri}.

\uline{Mångsysslarpensionären i kulturen}

Vissa har i J.R.R Tolkiens\ref{j.r.r tolkien} karaktär Gandalf sett själva arketypen av en mångsysslande pensionär.
Andra omnämner Lasse Åberg som en typisk sådan.

\ditem[Mördare]\label{moordare}
 är ett yrke som går ut på att döda andra människor. Ni kanske tänker att det inte är särskilt lukrativt med mord, men ni anar inte. Det finns många olika sorters mördare och störst av alla mördare är Jacob Wallenberg.


%%%%%%%%%%%%%%
\newpage
\null
\\
\null
\\
\Huge
N
\normalsize
\\
\null
\\
\null
%%%%%%%%%%%%%%


\ditem[Naturens dialektik]\label{naturens dialektik}
 Det som inträffar när en blomma växer i betong.

\ditem[Naturhistoriska museet]\label{naturhistoriska museet}
 Låt dig inte luras av den i jämförelse pampiga fasaden. Ta istället första chansen du har att kolla vilken matsäck som skickats med dig. Detta är utgångspunkten för dina val från nu till ni åker hem.

\uline{Klassamhället}

Risig matsäck = Glida iväg från gruppen och försöka hitta rådjuret som har två huvuden.
Helt okej matsäck = Häng på. Har du tur kommer du hamna i ett läge där du kan dra in på shoppen och använda resten av de pengar dina föräldrar kastar kring sig till att köpa en ihålig gummidinosaurie. Gör en mental notering - de i klassen som drog iväg för att kolla på rådjuret istället för att äta matsäck kommer förr eller senare spöa skiten ur dig.

\ditem[Nedlagda industribyggnader]\label{nedlagda industribyggnader}
 är betongkolosser som byggdes för länge sedan innan Socialdemokraterna gick på myterna om \quotetext{tjänstesektor} och \quotetext{incitament\ref{incitament}}. Utan nedlagda industribyggnader skulle Sverige\ref{sverige} inte ha några konsthallar och ungdomar skulle inte ha någon stans att öva sig på att panga rutor. Dessutom skulle landets fotografer\ref{fotografering} inte ha några motiv att ställa ut i konsthallarna.

\ditem[Nedsatt sikt]\label{nedsatt sikt}
 är ett uttryck som ibland används i sammanhang som har med trafiksäkerhet eller meteorologi att göra. Termen används dock betydligt oftare i det sammanhang den från början var avsett för: diskussioner om hur lätt det är att se bandet på olika sorters spelningar. Sikten från piten är nämligen olika i olika musikgenrer. Den som av någon anledning uppskattar reggae bör till exempel antingen vara lång eller komma till spelningen tidigt. Många reggaefans har nämligen stora frisyrer och/eller mössor, vilket effektivt skymmer sikten för de som står bakom. Är man kort eller tidsoptimist är det därför lätt att glida in på Oi!-punk istället för karibisk folkmusik. Man skulle nämligen kunna tro att Oi!-punk är den bästa musiken för den som vill kunna se bandet, eftersom Oi!-entusiasten i regel är väldigt kortklippt. Men tyvärr är Oi! en genre vars fans tycker att det är tufft att göra som i England. Därför käkar de ohemula mängder friterad potatis och får således skyhögt BMI (body mass index). Därför hamnar den scenen på plus/minus noll när det gäller sikt och visibilitet. Av all faktorer som kan väga in i ett beslut att följa och engagera sig i en musikscen är hänsyn till sikten på spelningar troligtvis bland de minst vanliga.

\ditem[Neontetra]\label{neontetra}
 (\textit{Paracheirodon innesi}, plural neontetris) är den billigaste akvariefisken som man kan köpa på en djuraffär, men så är den också så liten att om den inte hade haft en så starkt färgad skrud hade man inte kunnat se den. Om man fixar en stor glasburk eller ett litet akvarium och skickar ner lite sand\ref{sand} och grus, vatten och ett gäng neontetror har man vips! ett akvarium som kan skänka mycket glädje åt det enklaste hushåll och som pappa kan vila blicken på efter en lång och strävsam arbetsdag.

\ditem[Nia]\label{nia}
 Titellösas tilltalsord innan Du-reformen och den del av kroppen som Pluton Svea vill skjuta Leif Loket Olsson i.

\ditem[Nicholas Cage-film]\label{nicholas cage-film}
 En Nicolas Cage-film är en spelfilm som på något vis är kopplad till skådespelaren Nicolas Cage - i de flesta fall genom att Nicolas Cage är med i filmen, men det är också möjligt att han varit inblandad i att göra filmer, och om så är fallet är även dessa Nicolas Cage-filmer. Vad som utmärker en Nicolas Cage-film är att den oftast är obegripligt fånig och dålig. Exempel som \textit{Gone in Sixty Seconds}, \textit{National Treasure}, \textit{Face Off}, \textit{Con Air}, \textit{Next} och \textit{Season of the Witch} talar för sig själva. Detta gör att många fascineras av Nicholas Cage och de filmer i vilka han onekligen fortsätter att figurera. Ingen gillar honom. Han är en bedrövlig skådespelare som med hjälp av några av samtidsfilmhistoriens största konstnärer (till exempel Charlie Kaufman) mirakulöst men temporärt rykts upp till en acceptabel nivå för att genast åter sjunka tillbaka ner till sin normala, absurt låga nivå av skådespeleri. Det sägs att John Travolta efter inspelningen av \textit{Face Off} blev så förvirrad av det faktum att Nicolas Cage forfarande erbjuds roller att han sökte sig till den Scientologiska kyrkan för att få svar på många av de frågor om livet och varat som dök upp under de grubblerier som Cages karriär föranledde hos Travolta, som ju som bekant själv uppvisar en högst tveksam konstnärlig förmåga.

\ditem[Nigeria]\label{nigeria}
 Land i västra Afrika. För att trygga världsfreden bildade man 1958 den så kallade \quotetext{Brev- och Frimärkesunionen} med Lichtenstein.

Fursten av Lichtenstein hade genom ett vänligt brev från Nigerias dåvarande president uppmärksammats på ett väldigt arv. Detta blev början på den djupa vänskapen mellan de två staterna. En gemensam valuta (Shilling Banco) infördes på 2000-talet. Ekonomer världen över ser detta som förutsättningen för dagens blomstrande ekonomi. Under 2010-talet har nya exportvaror tillkommit, många inom IT-sektorn. Bland annat så sköter Microsoft allt sitt säkerhetsarbete ifrån \quotetext{Afrikas Silicon Valley}.

\ditem[Nikolaj Valujev]\label{nikolaj valujev}
 är en rysk tungviktsboxare som i väst går under namnet \quotetext{the beast from the east.} Förutom sina tungviktstitlar och framförallt sin imponerande kroppshydda\ref{kroppshydda} på 150kg och 213cm över havet är Nikolaj Valujev känd i Europa genom att många föräldrar använder honom skrämma för att sina barn då de är olydiga, inte vill sova eller inte vill äta upp maten: \quotetext{Äter du inte genast upp maten kommer Nikolaj Valujev och boxar dig i huvudet\ref{huvud},} kan det heta. Idag har Valujev till många boxares lättnad\ref{laettnad} lagt handskarna på hyllan för att istället leta efter snömannen i Sibiriska grottor.

\ditem[Nissepedia]\label{nissepedia}
 60\% nonsens, 20\% lögner, 20\% übersmal trivia.

\ditem[Nitlott]\label{nitlott}
 Lotterispel arrangerat av Ägg tapes and records till förmån för träskpunkare\ref{traeskpunkare} vars bidrag tagit slut. Nitlotteriet arrangerades första gången på Augustibuller 2004 och första pris var Anti Cimex-Cliffs gamla skinnpaj. Varje lott kostar tio kronor och det går bra att betala kontant eller i folköl. Föreningen bakom nitlotteriet saknar i sann punkanda 90-konto hos postgirot och det mesta av de insamlade pengarna försnillas.

\ditem[Nollpresterare]\label{nollpresterare}
 är ett begrepp som används inom den karga och cyniska högskolepolitik som Jan Björklund\ref{jan bjoorklund} lagt grund för under sin tid som utbildningsminister. Begreppet används för att underlätta de godtyckliga försämringar som sagda minister oförtröttligt och mot större delen av den samlade universitetsvärldens vädjanden kämpar för att införa. I detta sammanhang är en nollpresterare någon som skriver in sig på en kurs men inte slutför den och därmed inte får några högskolepoäng, men ordet har också kommit att användas utanför detta sammanhang. Det används också om personer - framförallt unga män - som i sin tillbakalutade existens inte presterar mycket mer än unken luft och experience points i något dataspel förlagt i ett parallellt, mindre komplicerat universum. 

\ditem[Norberg]\label{norberg}
 Om man plattar ut jorden till en cirkel och sedan tittar precis där navet sitter så hittar man Norberg. I Norberg finns Myntgrillens gatukök och en anrik historia som till mångt och mycket följer arbetarrörelsens utveckling från födelse till storhetstid till stagnation.

\ditem[Nordisk kombination]\label{nordisk kombination}
 är en av de mer obskyra OS-sporterna. Grenen går ut på att man både hoppar backhoppning och åker längdskidor. Först drar man upp i backen och ser vem som hoppar längst och den som vinner får starta först i längden. Man kan skratta och tycka att det är en ganska fånig kombo, men om man tänker igen så känns det faktiskt ganska hårt. Bra mycket ballare än sprint, vattenskidor, puckelpist och andra grenar som är mer som lekar för vuxna.

\uline{Nationella varianter}

I Danmark finns det aldrig någon snö så där kör man istället \textit{dansk kombination}, som innebär att man kör säckhoppning\ref{saeckloopning} med rullskidor på fötterna. Grenen har inte OS-status.

\ditem[Tenzing Norgay]\label{tenzing norgay}
 var en nepalesisk bergsbestigare och en av de första människorna att sätta sina fötter på Mount Everests\ref{mount everest} topp tillsammans med Edmund Hillary\ref{edmund hillary}. Han säger själv att han var tvåa men påpekar att \quotetext{om det är en skam att vara den andre mannen på Mount Everest så är det en skam jag ska bära}. Detta sa han efter att ständigt ha blivit ställd frågan om vem som var först \textit{egentligen} av journalister som inte tyckte att svaret \quotetext{vi gjorde det tillsammans} dög. Detta är en viktig minnesbeta i vår åt helvete för individualistiska kultur: Mount Everest\ref{mount everest} bestegs inte av en enskild individ, utan \textit{tillsammans}. Glöm aldrig det.

\ditem[Norge]\label{norge}
 är ett land som ligger i närheten av det hos många nissepedialäsare betydligt mer välkända Sverige\ref{sverige}. Norge är bergigt, ligger nära havet och är väldigt smalt, vilket ofta leder till att folk misstar det för Chile. Vill man kontrollera vilket av dessa länder man är i kan man enkelt göra det genom att ställa en fråga till någon i lokalbefolkningen. Får man svaret på norska är det troligen Norge man är i.

\uline{Kultur}

Karl Ove Knausgård kommer från Norge, vilket tydligt framgår i den kritikerrosade romansviten \textit{Min Kamp}. Även Knut Hamsun och Edvard Munch ska visst komma från detta land.

\uline{Ekonomi}

Norge har jättebra ekonomi. Det har de fått genom att saluföra olja och tjocktröjor. Därför är alla norrmän och -kvinnor väldigt nöjda med sig själva och sitter mest på bron och dricker öl och lyssnar på true Norwegian black metal hela dagarna.

\uline{Religion}

Norge har länge varit kristet, men idag bekänner sig de flesta till satanismen.

\ditem[Norrbotten]\label{norrbotten}
 Även känt som \quotetext{paradiset} och omnämnt i första mosebok.
Är man från Norrbotten är man automatiskt lite bättre, lite snyggare och framförallt mycket tuffare än alla andra. Samtidigt löper man statistiskt sett 110\% större risk än Sveriges\ref{sverige} övriga befolkning att vara en arbetslös pornografikonsument med rovdjursskelett i garderoben.

\ditem[Norrtälje]\label{norrtaelje}
 eller Norrtelje som det ibland stavas, än en stad i Uppland\ref{uppland} och huvudort i Norrtälje kommun, vilket säger sig självt. Norrtälje ligger cirka sju mil från Stockholm\ref{stockholm} och ca åtta mil från Uppsala\ref{uppsala}. Norrtäljes stadsvapen är ett ankare, den traditionella symbolen för hopp, vänt upp-och-ned. År 2005 hade Norrtälje lite över 16200 invånare. Staden grundades redan 1622 av Gustav II Adolf men kring den finns fornlämningar från tidig järnålder och folkvandringstiden. Norrtälje kan stoltsera med en av världens högsta densiteter av stentrollsaffärer\ref{stentrollsaffaer} samt en för regionen rekordlåg utbildningsnivå bland befolkningen. Norrtälje är också scen för motorcykelsammankomsten Custom Bike Show som hålls första lördagen i juni och är Skandinaviens största event för hembyggda motorcyklar. Nedan följer ett slags litterär stadsvandring genom staden, häng med!

\uline{Stadsvandring genom Norrtälje stad}

Vi kliver iland på borgmästarholmen i Norrtäljes östra del och har gästhamnen framför oss där överklassens segelbåtar guppar. På andra sidan Norrtäljeviken har vi stadsdelen Grind\ref{grind} som blickar ut mot havet och vars skylt av naturliga skäl brukade bli snodd när det var punkspelningar i staden när det begav sig. Längre upp har vi Kvisthamra. Här bor det reklamare och företagare som har flyttat in från Stockholm. På andra sidan kärleksudden ligger adelsläkten Löwens gamla herrgård, Björnö. Vi går vidare!

Vi har nu passerat Kärleksudden där man kan bada eller köpa en öl, men det är ofta mycket barnfamiljer där på dagen. Vi går nu genom Socitetsparken - Norrtäljes svar på vad som i alla andra svenska städer kallas \textit{folk}park. Här har det utkämpats tallösa slagsmål och parken har i slutet av 90-talet stormats av piketsnut från Täby eftersom det hade kommit till deras vetskap att ungdomar var där och hade picknick. Nämen se där har vi S/S Norrtälje som guppar förnöjt och erbjuder lunch till landsbygdsyuppies fem dagar i veckan. Vi fortsätter.

Där har vi Socitetsbacken och här har vi caféer där ett oräkneligt antal 20-åriga flickor arbetar och säljer rattstora bullar till andra 20-åriga flickor. Längre bort skymtar Roslagsskolan där den övre medelklassens ungdomar går. Där finns musikklass och speciella klasser för \quotetext{Bäst i klassen}-tjejer som är sina föräldrars ögonstenar och som kommer att arbeta på dessa caféer. I ett av dessa spenderade 20-åriga flickor och pensionärer sina surt förvärvade pengar på handgjord konfektyr, vilket dock inte hindrade stället för att kånka. Vi går förbi Frälsningsarmébacken och på Stora torget hälsar vi på den acneanfräten tonåring som inte gått på Roslagsskolan men som står i \quotetext{Korvleones} grillvagn och tjänar tio och femti i timmen på att mata turister från Stockholm med korv. Vi kommer till Gröna ön där alkisarna hänger, hälsar på Arne Katten, Uffe Sotaren, Pinnen, Törstiga Törnan, Fylle-Gerd och de andra. På busstationen står det kickers som fortfarande finns kvar här och ett antal galna tanter som skriker och bråkar om en tom plastpåse. Vi passerar nu klassgränsen i Norrtälje och har kommit till underklassens halva av staden, den västra delen.

Vi går nu mot Sandkilen där det förut fanns ett bilgarage med en vägg av små glasrutor som man kunde kasta sten på. Här ligger också Norrtelje Tidning och bakom den ett industriområde där VPK-lokalen låg innan kommunen rev den och VPK-arna fick flytta in i ett garage vid norra industriområdet. Brevid ligger Rikets Sal. Ännu lite längre bort har vi kåken, där Hagamannen bland andra sitter.

Nu har vi kommit till Knutby torg, vars förut så exotiska namn nu klingar lite sämre. Här ligger Järnia och där kan vi handla spik och tapetväv på löpmeter. Här ligger Lidl, Preem, Willys och Rusta. Bakom dessa finns ännu ett industriområde, med bland annat Samhall, en skrot och bikerlokalen. Vi korsar Vätövägen och kommer till Lommarvägen. Här bor det socialfall, långtidssjukskrivna knegare, ensamma mammor, subotexpundare och nyalända invandrare. Parkerar man på Mekonomens parkering efter stängningstid får man böter inom några minuter eftersom någon i hyreshuset brevid ringer vaktbolaget. Här kan man köpa droger och billig sprit. Bensinmacken brevid har blivit rånad två gånger de senaste åren. Vi går till Vargheden, där man kan spela fotboll\ref{fotboll} eller smälla smällare och ställa till med jävelskap\ref{jaevelskap}. Lite längre bort hör vi stojet från Lommarbadet där det hittats vässade armeringsjärn i en klump cement nedsänkt vid bryggkanten, rakblad i vattenrutschkanan och en avhuggen hand i vassen. Bortom Lommabadet finns en vikingagrav, en hästskola och ett högstadium där arbetarklassen och invandrarnas barn går. Sen finns det inte mer.

\ditem[Norsjöblicken]\label{norsjooblicken}
 är ett fenomen som härstammar från den västerbottniska\ref{vaesterbotten} byn Norsjö. Det klart bristande intellektet hos byns invånare leder till en svårighet att svara på de lättaste av frågor. Frågar t.ex. någon utböling en Norsjöbo: \quotetext{Var ligger Frasses?\ref{frasses}} så svarar inte Norsjöbon \quotetext{Vid busstationen}, utan med en blick som liksom är fäst i fjärran. En blick lika tom som skallens innanmäte.

Norsjöbygden är även känd för sin höga koncentration av orgelbyggare samt att ha folkomröstat huruvida man vill ha ett Systembolag eller inte. Den oinsatte kan då lätt tro att Norsjöbon inte behöver bolaget då han kokar sin sprit själv, sålunda är inte fallet.

\ditem[Noshörningen Nelson]\label{noshoorningen nelson}
 var den första trubbnoshörningen att födas i fångenskap i Sverige. Nelson föddes 6 februari 1995 på Kolmårdens djurpark, men ödet hade försett honom med en obotlig hjärnsjukdom och han dog 20 februari samma år. Svenskarnas oförmåga att hålla sig sansade när det gäller känsliga frågor gjorde att Nelson aldrig fick vara ifred och det är troligt att mediedrevet gav honom stressfrakturer som skyndade på dödsbudet. Liket valsade runt i landet tills det år 2000 kremerades och fick sin sista vila i en glassbutt\ref{glassbutt} i en klippskreva på Lars Vilks konstverk Nimis.

\ditem[Nu går slakten på Bomans vind!]\label{nu gaar slakten paa bomans vind!}
 är ett kraftuttryck som används i Dala-Floda med omnejd för att markera att något är nära förestående, ungefär som en kombination av de lite populärare \quotetext{det var nära ögat} och \quotetext{snart går tåget}. Bakgrunden till visdomsorden är en grisbonde i Dala-Floda som hette Boman och tydligen brukade slakta på vinden.

\ditem[Nudist]\label{nudist}
 En nudist är någon som inte gillar att bära kläder av någon sund eller tvångsmässig anledning. Vissa nudister bär kläder utomhus men inte inomhus och vice versa. Andra vägrar konsekvent att bära kläder och ställer sig på så sätt utanför den sociala normen som de som bär kläder skapat. 

Det finns ingen lag i Svea rike om att man måste bära kläder, men om man inte för tillfället är Knug eller Drottning (och därmed är immun mot åtal och s.k. allmän praxis av allehanda slag) så kan man bli åtalad för förargelseväckande beteende, vilket sorteras under störande av allmän ordning, ifall man visar sig naken inför någon eller några som pliktskyldigt kan tänkas påkalla farbror blå.

Unga män som gillar att sitta\ref{sitta} hemma endast i kalsongerna på sin lediga tid kvalar in under kategorin semi-nudister.

\ditem[Nya Zeeland]\label{nya zeeland}
 Oceaniens Norge. Välmående, orimligt kristet, starka band till \textit{Sagan om ringen}. Hit flyttade alla de brittiska upptäcktsresande som var för fega för att bo i Australien\ref{australien}. Till skillnad från Australiens är Nya Zeelands fauna nämligen inte totalt livsfarlig utan istället gullig, mjuk och vänskaplig. Som alla civiliserade länder kör man bil på vänster sida och har Elizabeth den andra, \quotetext{av Guds nåde, Förenade konungariket Storbritannien och Nordirlands och hennes övriga riken och territoriers Drottning, Samväldets överhuvud, Troslärans försvarare} som statsöverhuvud. Till frukost äter man traditionellt kiwifrukt och till lunch kiwifågel. Middagen kan variera lite men utgörs ofta av en gröt baserad på dessa två ingredienser. Och till det så klart en kopp te\ref{te}. Vanliga namn: Edmund Hillary\ref{edmund hillary}.

\ditem[Nyliberalism]\label{nyliberalism}
 Ideologi som förr kallades nazism.

\ditem[Näbbmun]\label{naebbmun}
 är ett lika vanligt som populärt inslag i Hollywood och är för många biljetten in i filmbranschens glamorösa värld. Näbbmunnen består av organisk vävnad och silikon och ersätter den vanliga munnen. Näbbmunnen är som en kort näbb med vilken filmstjärnan kan plocka upp frön och korn från marken eller andra ytor. Världens idag kanske mest välkända näbbmun finns att återse i Angelina Jolies ansikte.

\ditem[När livet blir alldeles för mycket - Prof. Etiennes bästa gömställen, i urval]\label{naer livet blir alldeles foor mycket - prof. etiennes baesta goomstaellen, i urval}
 Efter ett stort antal publicerade titlar riktade till en mogen läsarskara beslöt sig Prof. Etienne\ref{prof. etienne} för att ge något till de som lånar ut jorden till oss - barnen\ref{barn}. Under sin karriär hade Professorn, på grund av sina samhällsomvälvande idéer, flera gånger behövt gömma sig på olika sätt på olika platser. I boken \quotetext{När livet blir alldeles för mycket - Prof. Etiennes bästa gömställen, i urval}, presenterar Etienne sina smartaste knep för att undvika att bli upptäckt.

\begin{itemize}
\item Raka munkfrisyr och slink in i en procession bestående av franciskanermunkar nästa gång en sådan drar förbi din by.
\item Under diskhon brukar fungera, om din familj inte har sin källsortering där. Om dina föräldrar källsorterar, be dem vänligt men bestämt att upphöra med det, då all separation av sopor från varandra baserat på människoskapade kategorier är en synd mot Gud.
\item Undvik ventilationstrummor. De är inte lika trevliga, rena och fria från smittobärande ohyra som på film, något jag fick lära mig den hårda vägen 1987, då jag ådrog mig en riktigt ihärdig släng fläcktyfus i samband med ett inbrott i livrustkammaren.
\item Den höfyllda oxkärran är en lika klassisk som effektiv utväg om du befinner dig i en belägrad rural medeltidsmiljö och vill undkomma den alltid lika ondskefulle dansken. Kryp helt sonika upp i höet och drag in samtliga lemmar. Hoppa triumferande ut först då du befinner dig ur räckhåll för dina plågoandar.
\item Förklä dig till uv\ref{uv} och kryp in i en svan\ref{svan}. Det lilla knepet räddade mig från en bunt manikeistpatrask i södra Schleisen för ett antal decennier sedan.
\item Lägg dig under en filt.
\item Vi har mycket att lära av djuren i allmänhet och de som lever i gryt i synnerhet. Underskatta aldrig värdet av ett riktigt djupt gryt. Ut och gräv ditt eget, bums!
\end{itemize}

\ditem[Nära vän till familjen]\label{naera vaen till familjen}
 (eng. friend of the family) är en global institution som finns för att främja omhändertagande och social inkludering. En nära vän till familjen återfinns ofta i anslutning till tämligen bofasta släkter som regelbundet firar tillställningar såsom födelsedagskalas och högtider. En nära vän till familjen bjuds alltid in till dessa tillställningar (även släktträffar) och ingen reflekterar nämnvärt över detta. En nära vän till familjen är ofta en mycket intressant och udda person utan egen familj, fast inte heller detta är det någon som nämnvärt reflekterar över.

Några exempel:

Sven är gammal och är kompis med din farfar. Svens fru gick bort för 22 år sedan och han har sedan dess varit en nära vän till familjen. Sven och farfar låg inkallade tillsammans under kriget (WWII, Charlottenberg, 1939-41). Han berättar rövarhistorier, paradnumret handlar om när de tappade ett kanonrör. De sa inget men en fanjunkare upptäckte en kritisk skada på kanonen. Det blev utskällningar, krigsrätt och 11 dagars potatisskalande men allt slutade lyckligt och ingen blev ju skadad.

Ylva är fritänkare och har svanmärkta kläder. Ylva gick gymnasiet med din mamma. Hennes föräldrar gick tragiskt bort för ett antal år sedan och hon har varit en nära vän till släkten sedan dess. Det blir alltid lite roligare när hon kommer på kalasen. Hon köper lite för dyra och vuxna presenter. Hon verkar lite ensam och hon träffar aldrig någon karl. Hon gillar dock att resa till Berlin och New York och har många spännande historier. Visst var morsan och hon i Grekland innan du var född?

Ardavan kommer ursprungligen från Iran. Ardavan var kollega med din morbror som jobbade på ett stort svenskt multinationellt företag (Asea, 1971-1995). De träffades som nyutexaminerade ingenjörer innan revolutionen i Iran. Ardavan bodde sedan i Sverige ett antal år från 1980. Han har varit en nära vän till släkten sedan 1980 (han var med på ett dop tredje dagen i Sverige). Han kunde tydligen Taekwondo och tog alltid med sig annorlunda konfekt och godsaker på kalasen. Under början av 1990-talet flyttande Ardavan till USA, din familj får dock fortfarande kort varje nyår. Din farsa brukar fortfarande säga att iranier är \quotetext{riktigt bra folk} och att \quotetext{alla rasister utan undantag är idioter}.

\ditem[Näs-flås]\label{naes-flaas}
 är ett samlingsbegrepp för de samtidigt subtila och påträngande pip- och pustljud som kommer ut ur självomedvetna människors näsor\ref{naesa}. Företeelsen är i Sverige\ref{sverige} framförallt förknippad med tidigare landsbygdsminister Eskil Erlandsson som uppvisar detta och mycket mer därtill.

\ditem[Näsa]\label{naesa}
 är en av andningssystemets två ventiler, kan man säga, och sticker ut på central plats i anletet på ett lite osvenskt och uppkäftigt vis. Likt en snorkel pyser den ut förbrukad luft och hämtar ner ny till lungorna. Näsan har för säkerhets skull två luftgångar, ifall en skulle blockeras av ett främmande föremål. De två gångarna mynnar ut i näsborrarna som på insidan är försedda med varsin ring av hårstrån. Näsan är formad lite som en takås, vilket gör att vatten och vind effektivt hindras från att tränga in i människan.

\ditem[Näverslips]\label{naeverslips}
 Slipsen, denna betecknare för maskulinitet och respektabilitet, har inte alltid sett ut som de släta och lite glansiga modellerna som vi är vana vid idag. I gamla tiders Sverige\ref{sverige} fanns inte lyxlirarmaterial\ref{storfraesare} som siden. Behovet av att piffa till sig innan man gick på dans fanns dock, och det var då någon handfallen hantverkare satte sig ner och flätade en näverslips. Idag är näverslipsen vanlig bland människor som vill vara \textit{vildmarkschic}.


%%%%%%%%%%%%%%
\newpage
\null
\\
\null
\\
\Huge
O
\normalsize
\\
\null
\\
\null
%%%%%%%%%%%%%%



\ditem[Oberoende olympiska deltagare]\label{oberoende olympiska deltagare}
 är sanna internationalister som vägrar att tävla i olympiska spelen under en enskild nations flagga. De tävlar därför som oberoende eller autonoma om man så vill. Oberoende har en stadig trupp och har ställt upp i flera OS.

\uline{Totalt medaljtilldelning}

\begin{itemize}
\item 1 silver (Jasna Šekarić, 10 m luftpistol)
\item 2 brons (Aranka Binder, 10 m luftgevär. Stevan Pletikosić, 50 m gevär)
\end{itemize}

\ditem[Odon]\label{odon}
 är ett bär som växer vilt lite varstans i Norden. Till utseendet påminner det om blåbär men smakmässigt är det betydligt mera trist och vattnigt. Odonbärets storhet ligger istället i dess lämplighet som bas för vinbryggning. Detta faktum i kombination med att namnet låter ganska mycket som Oden har lett till att bäret tidigt blev ett vida spritt afrodisiakum på svenska höskullar och kökssoffor\ref{kookssoffan}. Den gamle växtofilen Carl von Linné\ref{carl von linné} skrev i \textit{Flora Oeconomica} år 1749 att bären \quotetext{behaga barn och kalkone-ungar, men åstad komma ofta någon upphetsning}. Varför Linné valde en flock nykläckta kalkoner som referensgrupp vet vi inte, men plötsligt blir det inte lika märkligt att många trodde på troll och häxor\ref{trollpunk} vid denna tid.

\ditem[Ohemul]\label{ohemul}
 är ett adjektiv synonymt med exempelvis \quotetext{orimlig} och \quotetext{orättvis}. Vanligtvis talar man om ohemula priser men ordet fungerar lika bra till mycket annat. Varför inte ett ohemult utseende exempelvis? Skickligt brukat kan det också användas för att förstärka något positivt, ungefär som när man säger att något är \quotetext{djävulskt gott}. Svenska Akademien betalar ut fem kronor varje gång man använder ordet, i ett försök att vända dess tynande tillvaro. Skicka bara in bevis.

De lyckliga kompisarna sjunger i låten Egons Fest från skivan \textit{Tomat}:

\textit{Egon han var ohemult förmögen}
\textit{han hade spekulerat i en sjöbod på smögen}

\ditem[Oidipuskomplex]\label{oidipuskomplex}
 Det finns två typer av oidipuskomplex, den osunda där man vill mörda sin far för att lägra sin mor var populär i antikens Grekland. Den sundare varianten är den där man utåt sett inte vill bli som sin far men sakta i lönndom anammar dennes gubbighet och vanor.

\ditem[Old Black]\label{old black}
 är det första spåret på drone-bandet Earths fullängdare \quotetext{Angels of darkness, demons of light} och är en cover på Cacka Israelssons gamla dänga \quotetext{Gamle Svarten}.

\ditem[Old ox]\label{old ox}
 Ölet Old ox (eng. \textit{gammal oxe} Uttal. /gamoksä/) bryggs av Spendrups.

\uline{Copytext från lanseringen 1957:}

\quotetext{6.9 \% starkt. Bryggt för gubbar som sätter sig i fåtöljen efter en lång dag på kneget och tänker att de är förjävla slitna. Massiva. Tunga i kroppen. Långsamma, men stadiga som berget. Ingen slank sötnos direkt. Men vem fan vill vara det? Ingen smäcker kycklingfilé om man säger. Snarare en oxe. Fett kramar om massiv muskulatur. Det skaver i klövarna. Mulen är narig. Men man trampar på. Inget gnäll.}

\uline{Old ox och maskulinitet}

Den maskulinitet som uttrycks i Old ox beskriver till synes den hos någon som jobbat på fabrik i 40+ år av sitt liv. Föga överraskande säljer ölen bäst hos aspirerande akademiker i 20-års åldern med pinnsmala armar och flaskbottnade brillor.

\ditem[Oliver/Dawson Saxon]\label{oliverdawson saxon}
 är ett brittiskt metalband som består av tidigare medlemmar i metalbandet Saxon. Med tanke på att Saxon spelade i Smedjebacken och på finlandsfärja härom året kan man tänka sig att Oliver/Dawson Saxons karriär ligger på ungefär samma nivå som Skinned Alives, men utan den senare gruppens obestridliga status i undergroundkretsar.

\ditem[Olja]\label{olja}
 är en mörk trögflytande vätska som består av förmultnade dinosaurier (skräcködla) och gamla plankton. Den fanns länge som ett naturligt inslag i naturen och bidrog till skapande av social hållbarhet och kulturell utveckling. Numera utsätts oljan för en starkt kritisk lobbying, främst från arbetslösa nyutexaminerade arkeologer.

\ditem[Oljegark]\label{oljegark}
 En oljegark är en person i någon av de forna Sovjetstaterna som skor sig (eller har gjort det tillräckligt mycket för att numera kunna ägna sig åt att t.ex. köpa kända fotbollslag om dagarna) i gangsterstil på olje och energiaffärer. Det vanligaste är att oljegarken genom personliga kontakter, mutor eller mord och hot fått ett stort innehav av aktier eller en viktig post i något statligt eller privat energibolag strax efter att kapitalismen införts efter Sovjets kollaps. De nämnda taktikerna har många ojegarker lärt sig genom att tidigare i livet haft poster inom KGB eller FSB, eller så är de tillräckligt tjenis med någon inom FSB för att denne skall beordra grovjobbet åt dem. Liksom deras kollegor i maffian så håller sig oljegarkerna från att ha politiska poster och nöjer sig med att låta politrukerna vara knähundar åt dem. Även andra med informell makt förväntas att medvetet rätta sig in i ledet.

Ibland kommer oljegarkerna inte överens sinsemellan, vilket kan sluta med att någon får en lägerplats i Sibirien istället för att fortsätta sitt liv i materiellt överflöd.

\ditem[Olof Palme]\label{olof palme}
 är död.
Polis, media och säpo gick tidigt ut med att han blev skjuten och lurade därmed en hel nation att så var fallet. I själva verket dog han i en hjärtinfarkt efter att ha blivit jagad genom skogen av två fiskevårdare. Tjuvfiskare var han den jäveln!

\ditem[Opel kadett]\label{opel kadett}
 Vägarnas skräck, trafikpolisernas nemesis. Detta tyska fartvidunder kan skryta med 58 hästkrafter under huven (gäller högkompressionsmodellen 1.2S) och en prestanda du tidigare endast vågat drömma om. Med en modern utformning och snabba linjer blir du snabbt bygdens samtalsämne om du kör kadett.

Kör rätt! Kör kadett!

\ditem[Ornässtugans dass]\label{ornaesstugans dass}
 är lokalen där Sverige slutgiltigt kastade av sig det danska\ref{danmark} oket (Skåne ej inräknat) och började resan mot modern nationalstat. Det var här Gustav Eriksson genom en listig fint förde danska knektar bakom ljuset för att i ensam majestät ta sig till Stockholm\ref{stockholm} och tillträda tronen och avskaffa katolicismen. Christian Tyrann hade inte en sportmössa\ref{sportmoossa}.

Dasset står idag tillsammans med Haile Selassie I:s\ref{haile selassie} gravplats som enda avträden på FN:s världsarvslista.

\ditem[Orolighetskeps]\label{orolighetskeps}
 en är en röd keps med företagslogotyp\ref{kepsar med olika fooretagslogotyper}, närmare bestämt Gustavssons Åkeri och El AB, som finns på kontoret på Malå Sågverk\ref{saagverk}. När någon av de anställda bär denna keps vill de signalera till sina medarbetare att de är oroliga över något. Om de vill visa en medarbetare som inte är närvarande på kontoret skickas ett MMS innehållande ett porträtt där man är iklädd kepsen.

\ditem[Orrkammens isolering och gokart]\label{orrkammens isolering och gokart}
 är en sorts smutt kombinationsaffär\ref{kombinationsaffaer}, eller inte affär men företag i alla fall, som ligger i Orrkammen, Norrbotten\ref{norrbotten}. Ägaren driver under vinterhalvåret en firma som sysslar med isolering, medan sommarhalvåret ägnas åt dennes stora passion, gokart. Banan\ref{banan} är belägen där riksväg 95 korsar tvärbanan från Jörn, inte så långt ifrån Arvidsjaur. Här finns även ett café med våfflor till barnen och kaffe för de vuxna. Vuxna FÅR såklart äta våfflor också, men barn får fanimig inte dricka kaffe, hur skulle det se ut?

\ditem[Oscar Dronjak]\label{oscar dronjak}
 är en jättelång göteborgare som spelar gitarr i bandet Hammerfall. Han lyssnar inte på Mob 47\ref{mob 47}, för hade han gjort det hade han vetat att rustning är ett brott.

\ditem[Oscar Wilde]\label{oscar wilde}
 var en homosexuell irländare som skrev noveller och pjäser samt levererade snärtiga aforismer åt höger och vänster. Hör du ett lite ironiskt och underfundigt citat och avkrävs att gissa vem som sagt det är Oscar Wilde en mycket bra gissning. Är det inte han så är det Mark Twain som sagt det.

\ditem[osthyvel]\label{osthyvel}
 En slags kortare hyvel gjord för ost, och inte alls \textit{av} ost, som man annars kunnat tro.

Osthyveln uppfanns 1925 av den norske möbelsnickaren Thor Bjørklund som förövrigt också är upphovsmannen bakom den beryktade smörhyvel som dock inte fått samma genomslag. Världens största osthyvel står att finna i Ånäset\ref{aanaeset} alldeles vid E4.

\ditem[Ostmacka]\label{ostmacka}
 är en av våra mest vanliga och populära smörgåsar. Mackan består till 70\% av bröd, 20\% av ost och 10\% av smör (procentsatser approximativa). Överstiger brödhalten 75\% är ostmackan bröstark\ref{broostarkt}. Ostmackan är vanlig som inslag i frukostmåltiden samt som tillbehör till lättare måltider så som soppa och sallad.

\uline{Tillverkningsprocessen}

Bästa sättet för den som önskar lära sig själv tillverka en macka är att memorera följande steg:

\begin{enumerate}
\item Anskaffa brödskiva. Här måste man välja mellan hårt och mjukt bröd. Dessa olika sorter har olika fördelar. Väljs hårt bröd anskaffas detta genom att bryta en bit av en knäckebrödsskiva eller genom att fiska upp en skiva ur brödkartongen. Väljs mjukt bröd är det vanligt att det krävs ett visst mått av sågande medelst brödkniv i brödlimpa.
\item Smöra skivan. Skrapa upp en lagom mängd smör ut smörpaketet med en trubbig kniv. Tags för mycket smör i detta läge kan smöret återföras till paketet, men med lite övning och tålamod kommer detta så småningom kunna undvikas av den skicklige frukostätaren. För med jämna, lätta gester kniven över brödets ena yta så att smöret fördelas jämt.
\item Placera ost på mackan. Hyvla två (i vissa extrema fall tre) skivor ost från ostpjäsen med hjälp av osthyvel. Vissa rutinerade frukostätare använder också hyveln för att på ett mondänt vis placera ostskivan på brödet, men detta är knappast praktiskt nödvändigt - en mer vardaglig metod är att med den fria handen plocka upp ostskivorna från hyveln och placera dem bredvid varandra längsmed brödskivan.
\item Ej nödvändig garnityr. Den som så önskar kan placera en handfull gurkskivor, paprikabitar eller en späd basilikakvist på ostskivan för att på så vis addera lite färg till den. Tillskyndare av denna metod menar ofta att man njuter av en ostmacka inte bara med smaklökarna utan också med synapparaten, vilket antyder att det visuella inte är oviktigt när det gäller en mästerligt tillverkad ostmacka.
\end{enumerate}

\ditem[Ououou]\label{ououou}
 är det lite \quotetext{bluesiga} ljud som John \quotetext{The fog\ref{the fog}} Fogerty diggar fram efter första raden i Creedence Clearwater Revivals låt \textit{Long as I can see the light}:

\textit{Put a candle in the window Ououou.}


%%%%%%%%%%%%%%
\newpage
\null
\\
\null
\\
\Huge
P
\normalsize
\\
\null
\\
\null
%%%%%%%%%%%%%%


\ditem[Paddan i pannrummet]\label{paddan i pannrummet}
 är ett uttryck som används av den lite mindre bemedlade men mer livsnjutande delen av Stockholmsområdets\ref{stockholm} manliga befolkning. Paddan syftar på pedalen och i detta fall gaspedalen i en bil. Pannrummet syftar i sin tur på ett rum som ligger i källaren. Uttrycket kan alltså ungefärligt översättas till \quotetext{gasen i botten\ref{botte}} men används inte bara om hög fart utan även om fall då någon, till exempel Matt Pike, levererar till 110\%.

\ditem[Paj]\label{paj}
 är ett ord med, åtminstonde, fyra\ref{fyra} betydelser.

\uline{Inom modet}

Paj åsyftar här en jacka eller väst i skinn eller jeanstyg, så kallad skinnpaj eller jeanspaj. På de snyggaste plaggen står det \quotetext{Hawkwind} skrivet för hand på ryggen.

\uline{Inom nikotinismen}

Paj är här en bit bakat lössnus.

\uline{Paj inom mekaniken}

Inom mekaniken betyder paj att något är ställt ut bruksskick.

\uline{Inom gastronomin}

Inom gastronomin betyder paj ett slags rund macka av pajdeg och äggstanning.

\ditem[Paleontologi]\label{paleontologi}
 är läran om roliga stenar. Genom att studera en gammal sten kan forskaren (paleontologen) få fram information om vad som finns i stenen och om den ser rolig ut. Om forskaren bedömer att stenen ser tillräckligt rolig ut tar hen med den till sitt laboratorium där hen tittar ännu noggrannare på den innan den åker upp på en hylla i ett museum för att glädja andra. Paleontolog blir personer som saknar det estetiska handlaget för att bygga egna stentroll\ref{stentrollsaffaer} och istället måste ge sig ut i skogen för att hitta stenar som ser naturligt roliga ut.

\ditem[Palle Kuling]\label{palle kuling}
 eller kort och gott \quotetext{Palle,} är en påhittad figur som är ansiktet utåt för Fazers godisproduktserie med samma namn. Palle är påfallande lik Rasmus Nalle\ref{rasmus klump} polare Pelle, som likt Palle är en pelikan\ref{pelikan}. Detta kan få den misstänksamme att tro att Fazer utnyttjar Rasmus Nalles omåttliga populäritet hos den yngre publiken för att sälja sina produkter.

\ditem[Palltruck]\label{palltruck}
 eller pallbjörn som den även kallas (är den av fabrikat BT kan den även tituleras \quotetext{bolsjevik} då den är både röd och stark som tusan) är en nödvändighet på en normal arbetsplats. I efterblivna länder, typ Spanien\ref{spanien}, finns inte pallbjörnar, så dom får ingenting gjort.
Att kunna bruka en palltruck är en tydlig klassmarkör men kan även ses som ett intelligenstest för att se om någon är slug nog att få gå lös.
Pallbjörnen uppfanns 1947 av den sjukt underskattade Ivar Bryntse

\ditem[Panflöjt]\label{panfloojt}
 är ett förtrollande\ref{mani} blåsinstrument med både anor och attityd. Med sina sälla toner har panflöjten både upprört och berört allt sedan den första uttråkade sydamerikanen bröt av ett vasstrå och provade att tuta i. Dess hypnotiska egenskaper är omvittnade och det var ingen slump att gammelgubben Mozart lär Papageno samla frimurarna i \textit{Trollflöjten} med hjälp av en panflöjt istället för en australiensisk\ref{australien} didgeridoo, en tysk\ref{tyskland} dragbasun\ref{dragbasun} eller en dansk\ref{danmark} kazoo. Panflöjten återfinns över hela världen och tycks ha uppstått ungefär samtidigt på så olika platser som Kina, Rumänien och Peru. Materialet varierar efter naturresurserna men låter alltid lika vackert\ref{skensnygg}.

\ditem[Pangsionärerna]\label{pangsionaererna}
 är PROs väpnade gren.

\ditem[Panik]\label{panik}
 är en känsla som uppstår när stressen\ref{stress} har gått helt över styr. Individen som drabbas slutar ofta helt att tänka rationellt och logiskt och handlar antingen i blindo eller helt styrt av ryggmärgen. Detta medför sällan att individen löser problemet som lett till paniken utan bara försätter sig djupare in i Pans grepp. Här behöver man därför en medmänniska som kan skrika \quotetext{Ta dig samman!} och sedan hjälpa individen att åter ta kontroll över situationen. När man sedan är tillbaka i sitt lugna, metodiska jag förvånas man över hur konstigt man agerat när man hade panik, men så är det bara. En del människor är helt inkapabla att få panik, medan andra har det mest hela tiden.

\ditem[Parisare]\label{parisare}
 Är en parisare en person från Paris? I Sverige\ref{sverige} säger alla söder om Härnösand bestämt ja. I det egentliga Norrland (allt norr om Härnösand) blir svaret nej. Om man då frågar den egentlige norrlänningen vad en parisare är kommer den berätta om en korv. Parisarkorven.

Parisarkorven är ungefär som en falukorv, men inte alls lika krökt och smal och med mindre kötthalt och större mängd mjöl. Helst ska korven komma från Bastuträsk. För att parisarkorven ska bli en riktig parisare, måste den serveras på rätt sätt. En minst en centimeter tjock skiva korv placeras, efter att ha grillats lagom mycket, mellan två hamburgerbröd (puritaner kräver ofta att bröden ska ha värmts i våffeljärn, andra nöjer sig med att de ska ha grillats lätt, tills det blir ränder på dem från grillgallret). På parisaren har man ketchup, senap och bostongurka. Bostongurkan kan ersättas med gurksallad (eller gurkmajonnäs, om man tror att man är nåt).

Parisaren för en allt mer tynande tillvaro, även i det egentliga Norrland, då den kulturimperialistiska produkten hamburgare oupphörligen tvingas ned i halsen på de korvälskande norrlänningarna. Men de som är rädda att parisaren ska försvinna har fortfarande en korvens bastion att vända blicken och be mot. Skellefteå, guldstaden. I Skellefteå har enligt tidningen Norran en akademi för parisarens bevarande upprättats.

\ditem[Passa tider]\label{passa tider}
 Att passa tider är nära besläktat med att göra rätt för sig\ref{goora raett foor sig} och går ut på att inte vara sen. Säger man att man ska vara på plats 07.00 så är man där 06.55 så att man hinner finljuga\ref{finljuga} i fem minuter. Frågar man ärkeidioten Jimmie Åkesson så svarar han att det svenskaste\ref{sverige} som finns är att passa tider och att stå i kö.

\ditem[Patentkork]\label{patentkork}
 är en tysk uppfinning framtagen i slutet av 1800-talet. Vid denna tid hade så kallat \quotetext{sodavatten} börjar bli populärt inom Europa och tillhörande kolonier och denna dryck krävde en kork som kunde tåla tryck. Patentkorken löste med bravur detta problem med hjälp av en gummiring och återförslutningsmekanism. Den enkla konstruktionen gjorde patentkorken omåttligt populär och med större kärl kunde även sill konserveras på samma sätt. Teknikfantasten August Strindberg sägs ha blivit så fascinerad av konstruktionen att han gav order om att all mat i hushållet skulle konserveras på detta sätt. Det hela lär ha gått över styr när han åt ett tråg\ref{traag} konserverad julgröt från året innan och fick spendera resten av långhelgen på avträdet.

\ditem[PATSY award]\label{patsy award}
 var en slags djurens motsvarighet till Oscar i filmvärlden. PATSY står för \textbf{P}icture \textbf{A}nimal \textbf{T}op \textbf{S}tar of the \textbf{Y}ear och delades ut till särskilt framstående djur inom filmbranschen. Priset delades ut första gången 1951 och gick till den talande åsnan Francis. Bakgrunden till priset var att många kände ett behov av att höja djurens status i filmvärlden, som dittills varit på ungefär samma slit och släng-nivå som en papptallrik. Under inspelningen av \textit{Ben Hur} dog närmare 150 hästar\ref{haest}, och Ronald Reagan slog ihjäl tre schimpanser mellan tagningarna i \textit{Bedtime for Bonzo}. PATSY awards lades ned 1986 på grund av bristande finansiering. Men det gjorde inte så mycket för då hade nästan alla roliga djur hunnit bli utrotningshotade\ref{utrotningshotade djur} ändå.

\ditem[Paul du Chaillu]\label{paul du chaillu}
 blev 1859 den förste vite mannen att se en levande gorilla, och kort därefter den förste vite mannen att skjuta en.

\ditem[Paxa]\label{paxa}
 Att paxa är att rättmätigt (eller orättmätigt) tilldela sig något, t.ex. den sista varma mackan eller framsätet i bilen.

\uline{Historiska paxningar}

\begin{itemize}
\item England, Frankrike och Holland paxar Nordamerika till många indianers förtret
\item De flesta länder i Europa kapp-paxar Afrika, förutom Liberia och Etiopien.
\item Kapitalisterna paxar alla produktionsmedel och naturresurser i hela världen
\item Sionisterna paxar Palestina
\item Kyrkan paxar tolkningsföreträdet vad gäller universum och allt däri
\item Den tyska mustigheten\ref{den tyska mustigheten} paxar Polen
\item Män paxar merparten av alla resurser och maktpositioner
\item Uvarna\ref{uv} paxar platsen som det mäktigaste djuret, långt före människorna kommer på villfarelsen att de är det.
\item Stockholmskapitalet paxar utdelningen från gruv- och skogsindustrin annorstädes i landet.
\item Nyliberalismen\ref{nyliberalism} paxar plats som chefsideologi i de flesta media
\end{itemize}

\ditem[Pelikan]\label{pelikan}
 en är ett slags fågel som har ett slags pung under näbben.

\ditem[Pelle Fosshaug]\label{pelle fosshaug}
 (född 1965) är en föredetta svensk bandyspelare som enligt egen utsago lider av \quotetext{klockeren} ADHD. Fosshaugs ADHD har lett till elitseriens snabbaste mål någonsin (fem sekunder från avslag) samt ett och annat övertramp, som när han bröt sig in på Zinken och lärde ett iranskt barn åka griller, eller när han sopade till en finländare i magen med klubban och sedan krosscheckade nästa lirare som kom till finnens undsättning. Fosshaug lyssnar på Slayer och är bandyns stor grabb\ref{stora grabbars och tjejers maerke} nummer 197.

\ditem[Pelle Karlsson]\label{pelle karlsson}
 Alla ställen som säljer begagnad vinyl har minst en lp av Pelle Karlsson.
Vem är denne mytiske man och varför hamnade i princip hela upplagan på loppis?

\ditem[Pelle Svensson]\label{pelle svensson}
 född 6 februari 1943 i Nylandsån, är en svensk brottare\ref{blomkaalsoora} och rättskämpe.

Pelle Svensson har två VM-guld i brottning, grekisk-romersk stil, och ett OS-silver. För dessa framgångar gavs han smeknamnet \textit{Pelle Sving}. Framgångarna till trots är det ändå rättvisa som alltid varit Pelles huvudsakliga intresse. Under 80-talet försvarade Pelle offentligt bland annat den man som planlagt en kidnappning av Peter Wallenberg. Genom åren har Pelle kommit att hjälpa många andra människor som anklagats för sådant som av staten ansetts vara \quotetext{brott}.

Som den renässansman han är ägnar sig Pelle ibland också åt poesi. Så här skaldade han 16 maj 2010 på sin blogg: \textit{\quotetext{Att twittra är som att kvittra. Vad rör sig inom mig just nu? Gör som fåglarna sjung ut, för annars tar livet plötsligt slut}}.

Pelle har nu mera statlig sjukersättning efter att ha lyckats få förföljelse av massmedia klassat som arbetsskada.

\ditem[Personlig assistans]\label{personlig assistans}
Alla människor här i världen har inte fysisk och/eller psykisk förmåga av att klara av vardagen, och behöver därför hjälp av en medmänniska. I Sveriges ändå relativt välutvecklade välfärdssystem beviljas i många fall personlig assistans, som går ut på att någon får betalt för att bistå den assistansberättigade i dennes hem. Det här är det absolut inget konstigt med. Vad som däremot är väldigt konstigt är att regeringen Reinfeldt slog fast att ett hektiskt medelklassliv började klassas som ett funktionshinder och fullt friska människor fick rätt till en begränsad form av assistans i form av barnpassning, städning och matlagning. Det här blir visserligen inte gratis, men subventionerat med skattepengar. Regeringen motiverar detta med att folk annars ordnar sin assistans på svarta marknaden och således bryter mot lagen. Men om den välbärgade medelklassen bryter mot lagen, ja då ändrar man lagen.

\ditem[Personligt varumärke]\label{personligt varumaerke}
 Idag måste man jobba hårt för att stå ut från mängden. Det är ett obestridligt faktum att människorna på vår planet bara blir flera och flera. På antiken levde bara en handfull giktdrabbade skäggiga gubbar i toga, som alla kände varandra och var bisarra på helt unika sätt. Då var det lätt att vara unik. Man kunde till exempel raka sitt skägg eller vara kvinna.

Men det råder en inflation på marknaden för unika människor idag. Visst, varje småstad har sin enda trallpunkare. Men så fort blicken riktas utanför den egna hemorten, mot till exempel Borås, inser man snart att den lokala trallpunkaren bara är en i massan av andra personer som klär sig likadant, lyssnar på exakt samma musik och har samma politiska åsikter som det lokala exemplaret.

Att inte vara unik kan orsaka stor stress hos den genomsnittlige västerlänningen, då upplevelsen av sig själv som extremt speciell är det bränsle som får denne att gå upp varje morgon. Om man är karriärist är det också i hög grad en fråga om att försörja sig att bli unik. På arbetsmarknaden måste man vara en produkt som utmärker sig för att arbetsköparen ska konsumera en.

Men hur är man unik i en värld där allt redan finns? Svaret är allt som oftast att ligga steget före i modetrender. Och som alla vet går modetrender ut på att blicka bakåt i tiden eller till avlägsna geografiska platser. Så fundera på vilket årtionde som ännu inte repriserats och vilken kultur som ännu inte härmats.

År 2014 är en spaning vi gärna delar med oss av att 1400-talets Inkaindian-stuk kommer göra stor comeback i Sverige under 2015. Köp rikligt med fjädrar, bladguld, läderarmband med turkosa detaljer, nedteckna information på föreläsningar genom att tvinna snören (quipu) och börja framställa keramik formad som knubbsälar för att trygga din känsla av att vara speciell och cutting edge till våren.

\ditem[Perspektiv]\label{perspektiv}
 beror på vad man jämför med, eller vem man frågar. Vanligtvis kontrolleras perspektiv av institut och tankesmedjor\ref{institut och tankesmedjor}.

\ditem[Peru]\label{peru}
 Enligt uppgift sydamerikas Finland\ref{finland}.

\ditem[Perversa elektriker]\label{perversa elektriker}
 är som elektriker är mest, fast mycket, mycket sämre. De är moderater\ref{moderat} och tycker att alla som vill betala skatt är dumma i huvudet\ref{huvud}. De pratar inte om nåt annat än jobb på fikat, och då bara om hur dumma i huvudet f.d. arbetskamrater är, eller nåt annat än jakt när man arbetar. Varför är dessa elektriker perversa, undrar kanske ni? Jo det är för att de tycker att det är \quotetext{kul} att MMS:a porr\ref{poorr} till sina arbetskamrater. Dessa bilder är påfallande ofta bögporr och ofta så pass extrem karaktär att mottagaren får fundera på om den begår ett brott genom att bara titta på dom. Det enda sympatiska draget hos den perverse elektrikern är att han (för det är alltid, alltid, alltid en man) föder upp strävhåriga taxar, och det är inte så jävla sympatiskt egentligen. Perversa elektriker är inte heller med i med i Elektrikerförbundet, LO:s stridbaraste fack.

\ditem[Petrus de Dacia]\label{petrus de dacia}
 (1230-tal - 1289) omnämns ofta som Sveriges\ref{sverige} första författare, det faktum att han endast skrev på latin, ej svenska, till trots. Hans mest kända verk är nedtecknandet av Kristina av Stommelns uppenbarelser.

\uline{Tidigt liv}

Petrus de Dacia föddes på 1230-talets Gotland. En på den tiden hemsk, karg plats där måsar stora som dagens albatrosser härskade i skyarna, svävandes ovan de gråa lavafälten som utgör Gotlands blasfemiska grund, spanandes efter deras huvudsakliga föda; människokött. Raukarna hade inte hunnit bli eroderade i lustiga skepnader ännu, utan var än så länge bara tråkiga stenar. I denna miljö växte Petrus de Dacia upp. Han kom sedemera att gå med i dominikanerordern, i och med ett dominikanerkonvent i Visby, där hans passion för det skrivna ordet och hans lust att berätta\ref{beraetta} kom att växa sig än starkare än innan. Han drevs av sin kunskapstörst ut ur Sverige\ref{sverige}, till kontinenten, Europa. Där kom han att stöta på flera av dåtidens stora tänkare, däribland Thomas av Aquino, som var hans lärare i Paris.

\uline{Paris}

Petrus de Dacia skrev själv i sina memoarer om skoltiden i Paris: 

\quotetext{\textit{Paris är så bedårande. Ack det skälver uti min lekamen när jag vandrar längst med Notre Dame och känner Guds kärlek skaka min existens. Om det inte hade varit för skolan hade allt varit perfekt. Min magister Thomas av Aquino är en sådan rese. Han tror på fullaste allvar att det går att kombinera Aristoteles idé om en Primo movens och universums Evighet med kristen teologi. Löjeväckande är bara förnamnet!}}

Petrus de Dacia och Thomas av Aquino hamnade ofta i väldiga dispyter med varandra, vilka allt som oftast slutade med att Thomas av Aquino, i egenskap av magister, beordrade Petrus de Dacia att ta på sig en Chapeau de paysan\ref{chapeau de paysan} inför hela klassen, varpå all diskussion omedelbart upphörde.

\uline{Konflikt}

Petrus de Dacia reste efter en längre tids studier i Paris runt på fastlandet. Hans resväg är tämligen lätt att följa då han ofta dyker upp i diverse straffregister. Lösdriveri och offentlig berusning var de domar han oftast ådrog sig. I Bologna sägs han ha träffat på Halbera Snorresdotter, dotter till Snorre Sturlasson, på en nedgången taverna. De två ska ha hamnat i bråk med varandra i och med att Petrus de Dacia anklagade hennes far för att vara en ooriginell klåpare som stal stora delar av sina historier från den höviska litteraturtraditionens mest kända verk. Halbera Snorresdotter tillbakavisade detta å det grövsta och kontrade med att kalla Petrus de Dacia för ett skitsvin\ref{vin}. Hur konflikten löste sig förblir oklart.

\uline{Kristina av Stommeln}

Biografin över Kristina av Stommeln kom till i och med att Petrus de Dacia anlände till Köln i studiesyfte. Han hade bränt alla sina broar i Paris och var nu på desperat jakt efter ett break. I Köln träffade han, på en ost och vin-kväll\ref{vin} arrangerad av ett beginkloster, Kristina av Stommeln. Kristina av Stommeln var dotter till en grisfarmare och småfifflare tillika, Gerhard av Brömmeln. Sin fars skojargener hade gått i arv till dottern som under kvällen insåg att Petrus de Dacia var en kille som \quotetext{sällan var omöjlig}, som hon själv uttryckte det i ett brev till sin väninna Katryna. Petrus de Dacia och Kristina av Stommeln kokade så ihop en historia som skulle göra dem båda kända. Efter ett utdraget parti whist avgjordes det att lotten föll på Kristina av Stommeln att fejka vansinne för att de båda skulle bli kända. Resten är historia. Petrus de Dacia nedtecknade Kristina av Stommelns påhittade visioner och skrev vitt och brett om hennes stigma. Kristina av Stommelns anfall av stigma upphörde i och med Petrus de Dacias död 1289.

\uline{Skänninge och hem till byn}

Petrus de Dacia bodde under sju år i Skänninge där han, förutom att skriva brev till Kristina av Stommeln, drev ett kvinnligt dominikanerkonvent med ett flertal fromma och förmögna kvinnor. Dominikanerkonventet hette S:ta Ingrids systrakonvent. Stämningen på konventet beskrevs av en syster som \quotetext{bizarr} och \quotetext{skandalös}. Händelserna som utspelade sig i Skänninge kom sju hundra år senare att influera Olle Hellbom när han skrev manus till sin och Lasse Hallströms dunderhit Tuppen.

På sin ålders höst flyttade Petrus de Dacia hem till Visby. Måsarna hade krympt i och med att inget byte fanns kvar. Gotlands avbefolkning hade inletts och raukarna hade börjat lakas ur på grund av försurningen av Östersjön. Petrus de Dacia dog utan att lämna någon avkomma efter sig.

\ditem[Petter]\label{petter}
 är en kille från Stockholm\ref{stockholm} som bor på Söder, går på kändisfester och driver ett skivbolag. Han har det otroligt svårt, och hans sätt att hantera livets ofrånkomliga tragedier och orättvisor är att pratsjunga om dem i en mikrofon, liksom att åka till sommarstugan på västkusten som han pratar om i en av sina många låtar. Ibland är han dock glad och då pratar han om att gå på krogen och vifta med armarna.

\ditem[Philibert Humla]\label{philibert humla}
 (1814-1891) var en svensk jurist från Karlskrona. 1862 skrev han boken \textit{Inledning till läran om stöld och snatteri}, och med facit i hand kan man se att den har influerat många människor.

\ditem[Picknickbog]\label{picknickbog}
 (från hönsens \textit{picka} ungf. \quotetext{hacka fram}, amerikanskans \textit{nickel} ungf. \quotetext{liten valör}, skånskans \textit{bog} ungf. \quotetext{grisarsle}) är formpressat kött som tryckts ner i en plåtburk. Innan det hamnar i burken har köttet kokats i saltlag så när man öppnar är det färdigt att äta. Massan har en konsistens i gränslandet mellan paté och blodpudding och faller lätt sönder. Det hela ser ganska vidrigt ut så det är inte så många som köper picknickbog. Såvida du inte känner en veteran från första världskriget är det troligt att du inte vet någon som käkat picknickbog mer än en gång på skoj. 

\ditem[Piteå]\label{piteaa}
 är en stad i Norrbotten\ref{norrbotten} som inte luktar speciellt gott. I Piteå är det, och kommer alltid att vara, herrens år 1992.

\ditem[Pizzaracer]\label{pizzaracer}
 Folklig benämning på Toyota Celica, BMW M3 och liknande bilar med \quotetext{sportig} approach. Köres företrädesvis av entrepenörer i pizzabranchen.

\ditem[Pizzarulle]\label{pizzarulle}
 En rulle oskattade hundralappar. Företrädesvis förvarade i fickan på ett par säckiga kockbyxor.

\ditem[Pjäxfett]\label{pjaexfett}
 är ett svenskt adjektiv som syftar på något starkt positivt. Vinner man 25 kronor på en trisslott är det fett, men hittar man en spelbutik som säljer en rulle fulsnus för under 200 kronor är det pjäxfett.

\ditem[PK]\label{pk}
 är en förkortning av Politiskt Korrekt. Avser ofta smygrasister som tycker illa om sverigedemokraterna för att man borde, i stället för av naturliga skäl. Men eftersom sverigedemokrater av princip aldrig ser sig själva i spegeln så tror de att alla som tycker illa om dem egentligen fejkar för att plocka poäng. Att som rimlig människa själv använda uttrycket PK är inte tillrådigt, eftersom ett lillfinger åt det hållet lätt leder till att man helt plötsligt har hela handen i det bruna.

\ditem[Place Jourdan]\label{place jourdan}
 är ett torg i Bryssel, beläget i närheten av parlamentskvarteren. I mitten av torget finns ett slags stuga där man friterar pommes frittes i grisfett. Här kan den förbipasserande 24 timmar om dygnet se lokalbefolkningen girigt köa för att köpa sig en strut av de eftertraktade frittorna. Ivriga bävrar har öppnat barer kring denna stuga där man får sitta och äta sina avlånga bintje-stavar\ref{bintje} och samtidigt dricka sig redlös. Gläd dig medan du kan, tycks budskapet vara, för imorgon kan du dö av en maffig propp i aorta.

\ditem[Plocka päron]\label{plocka paeron}
 är en sång av Sveriges proggflaggskepp Philemon Arthur \& the Dung. Låten släpptes för första gången år 1992 på bandets skiva \textit{Musikens Historia Del 1 o 2} (Silence records), och blev en omedelbar hit bland landets batikklädda befolkning. Sången premierades framförallt för sitt allmogebudskap och sin enkla refräng som var lätt att komma ihåg och sjunga med i. För att ge extra kraft åt sången plockades alla verser bort och texten består enbart av refräng. Hoola Bandoola Bands träpinneslagare Håkan Skytte\ref{vansinnets historia} var från början kritisk och menade att päron inte är ett proletärt bär\ref{proletaera baer}. Han fick dock svar på tal av Thomas Mera Gartz i Träd, Gräs och Stenar\ref{traed, graes och stenar}, som alltid bjöd sina vänner på päronhalva\ref{paeronhalva}. Hösten 1992 svämmade den svenska matpressen över av delikata recept på päronsoppa, spagetti med päronsås, fyra små päronrätter och päronburgare. Succén blev dock för mycket, och Philemon Arthur \& the Dung har aldrig sjungit om någon annan frukt sedan dess.

\uline{Text}

\textit{Jag plockar, plockar, plockar päron}
\textit{Plockar päron, plockar päron}
\textit{Plocka, plocka, plocka päron!}
\textit{Plocka päron! Plocka päron!}

\ditem[Pluta]\label{pluta}
 Att pluta kan vara två saker. Den ena är att med läpparna göra en pussmun. Den andra är att nedgradera något. T.ex. om någon säger att lunchen var jättegod, för att sedan ändra sig och säga att den var \quotetext{Helt OK}, så har således lunchen blivit plutad.

\uline{Etmyologi}

Verbet härstammar från sommaren 2006 då Pluto blev nedgraderad från planet till en vanlig himlakropp.

\ditem[Pneumatiska rör]\label{pneumatiska roor}
 är ledningar i vilka information eller saker (så som pengar) skickas i små kapslar med hjälp av lufttryck. Idag används pneumatiska rör i första hand för att skicka pengar från kassa till bankvalv i större varuhus och banker, men då det begav sig, i viktorianska England, hade man planer på att försöka blåsa folk och allt möjligt genom tunnlar under marken. Parisarna\ref{parisare} var så taggade på uppfinningen att ett 43 mil stort nätverk byggdes under jord i centrala Paris.
Jänkarna\ref{united states of america} som aldrig varit rädda för att misslyckas byggde flera system, det första i Philadelphia 1893 och det kändaste i New York fyra år senare. New Yorks system sträckte ut sig totalt sig nästan 4.5 mil på Manhattan. I rören skickades även smörgåsar, s.k. \quotetext{subs}. Därav dess cylinderform.
Amerikanarna tog det pnematiska brevsystemet ur funktion 1953 medan fransmännen häll ut ända tills 1987 innan de pensionerade sitt.

\ditem[Poetisk rättvisa]\label{poetisk raettvisa}
 Drabbar ofta onda människor såsom:

\begin{itemize}
\item Ayn Rand
\item Jörg Haider
\end{itemize}

\ditem[Polis]\label{polis}
 är en yrkesgrupp vars uppgifter är varierande men till största del finns de till för att förhindra att man har roligt. Tycker du om att köra bil så har polisen alltid laglig rätt att stoppa dig för att sedan ge dig böter, har du inte gjort något fel så kan du få böter för det också.
Polisen har två typer av hot som dom gärna svänger sig med, dessa är:

\begin{itemize}
\item Skall jag ge dig böter eller?
\item Ska vi ta med dig till stationen kanske?
\end{itemize}

Är du en \quotetext{före-detta buse} så kommer du alltid att få höra spydiga kommentarer och få ett taskigt bemötande av en polis, det är polisens huvudsyfte; att minnas om du varit dum vid några tillfällen för 20 år sedan.

Det finns poliser som tycker om att bryta av armarna på dem de griper och slå batongen mellan benen på en överförfriskad gammal man som kissar mot fel byggnad i staden och så finns det trevliga snälla poliser, ca 25 st i hela landet, 22 av dom har gått i pension. De som är kvar är generellt mer kriminella än genomsnittsmedborgaren; vilket i mångt och mycket förtar poängen med en poliskår.

\uline{Polis i populärkultur}

Polisen är ett tacksamt objekt att besjunga. Företrädesvis av råpunkare. Därtill finns flera tv-kanaler som enbart förhärligar dessa ruttnande själars värv.

\ditem[Polska helgdagar]\label{polska helgdagar}
 Polen är ett land som gillar att fira saker. Och vad är väl ett bättre sätt att fira än att vara ledig från kneget? För att alla ska få vara lediga och fira samtidigt ibland har Polens riksdag infört en rad nationella helgdagar. Vissa har man lånat in från andra håll i världen och vissa har man hittat på helt själv. Nissepedia kan som första uppslagsverk presentera en redogörelse för samtliga av Polens helgdagar på svenska.

\begin{itemize}
\item \textbf{1 Januari} En klassisk helgdag hos alla länder som kör med den gregorianska kalendern. Eftersom nyåret inträffar mitt i natten är det klart att man vill ha lite sovmorgon dagen efter. Inget konstigt med det.
\item \textbf{6 Januari} Denna dag firar man i Polen att dom tre vise männen hittade Jesus. Redan här blir man lite misstänksam mot polackernas benägenhet att vara lediga från jobbet. Är det verkligen något att fira att tre snubbar kommer och grattar Jesus på födelsedagen flera dagar för sent?
\item \textbf{Påskdagen} Jesusrelaterat igen. För de som bryr sig var det ju himla tråkigt att Jesus blev uppspikad på ett kors. Eftersom Polen är ett katolskt land förstår man att polackerna vill vara hemma och tänka lite extra på Jesus då.
\item \textbf{Annandag påsk} Eftersom påskdagen alltid infaller på en söndag kan man förstå om polackerna känner sig lite blåsta på konfekten eftersom de flesta ändå redan är lediga på söndagar. Då är det klart man vill kompenseras med en ledig måndag också.
\item \textbf{1 Maj} Officiellt är denna dag inte arbetarrörelserelaterad i Polen. Det kan tänkas hänga ihop med att en del polacker är rätt lacka på kommunismen men fortfarande gillar att ta ledigt för att fira. Man har dock inte hittat på någon annan anledning heller så lite konstigt är det.
\item \textbf{3 Maj} Här börjar man ana ett mönster. Helgdagarna ligger ofta två och två i tät följd, vad är grejen med det? Troligtvis är polackerna lite deppiga efter den första helgdagen, det var ju så fint att vara hemma liksom. För att muntra upp dom sätter man in en helgdag till, och ångesten känns genast mindre tung.
\item \textbf{7:e söndagen efter påsk} Okej, det låter lite konstigt, men spelar inte så stor roll eftersom det ändå är på en söndag.
\item \textbf{9:e torsdagen efter påsk} Här börjar det kännas som att polackerna inte är helt seriösa när dom instiftar sina helgdagar. Är inte det här mest en ganska illa formulerad ursäkt för att kompensera för att många var osmarta nog att ta ut all semester i början av sommaren innan det hunnit bli riktigt varmt?
\item \textbf{15 Augusti} Jungfru Marias himlafärd. Firar man att Jesus dör i två dagar kan man väl fira att hans morsa dog i en dag, tänkte antagligen polackerna.
\item \textbf{1 November} Alla helgons dag. Alla andra helgdagar i Polen som har med firande av döda att göra handlar ju om folk man inte känner personligen. Någon gång kan det ju vara nice att fira dom man faktiskt träffat också.
\item \textbf{11 November} Påvens födelsedag. Enligt Polsk-katolsk tro fyller alla påvar år denna dag. Fast egentligen handlar det nog mest om att polackerna tycker det är rätt spejsat med ett datum som bara består av 1:or. Och så är det ju lätt att komma ihåg därför också, vilket säkert var viktigt förr i tiden när man inte gick så länge i skolan och samtidigt ofta var undernärd.
\item \textbf{25 December} Den stora finalen på alla andra Jesusrelaterade dagar man firat under året. Och så får man ju dricka glögg också.
\item \textbf{26 December} Då är det klart att man vill fira lite extra och lyxa på med två lediga dagar på rad.
\end{itemize}

\ditem[Pop-rock]\label{pop-rock}
 är en blandning av dels pop och dels rock. Ofta är det svårt för en artist att definiera sin personliga stil. \quotetext{Jag spelar både pop och rock,} kan man höra svenska musiker säga. Ofta går artisten sin egen väg, trött på att föras in i olika \quotetext{fack}. Musiken får tala för sig själv, utan en massa regler och förväntningar. Vi har denna otvungna attityd att tacka för odödliga låtar som Tomas Ledins \quotetext{Sommaren är kort}.

\ditem[Porträtt av det postmoderna renässansgeniet som ung]\label{portraett av det postmoderna renaessansgeniet som ung}
 \textit{Porträtt av det postmoderna renässansgeniet som ung} är en självbiografisk roman av den svenske belletristen, smugglaren och skriftställaren Prof. Etienne\ref{prof. etienne}. Romanen handlar om en ung man, Stefan Dahlås, med konstnärliga ambitioner och dennes spirituella utveckling. Till yttermera visso handlar den om Dahlås' intellektuella utveckling. I de första tjugo kapitlen får vi följa bokens hjälte då denna får kontakt med intellektuella musiker. Denna del av romanen baserar sig på den period i Etiennes liv som han så förtjänstfullt beskrivit i Självbiografi, del 2 - De förlösande thinneråren. Nästa utvecklingsfas tar vid då huvudpersonen kommer i kontakt med Margit Sandemos\ref{margit sandemo} böcker om Isfolket, vilket utlöser en period av kvasi-religiöst grubbel hos Dahlås. Detta grubbel återges som en öppen medvetandeström, en litterär teknik som Prof. Etienne enligt egen utsago ska ha uppfunnit, men också i form av dialog mellan huvudperson och bikaraktärer som representerar olika institutioner och filosofier och livsval. Boken slutar dramatiskt med att Dahlås accepterar sin lott i livet och kommer till insikten att det ska levas till fullo efter att ha sett Ivar Bryntse beställa en korv i smörpapper\ref{korv i smoorpapper} utanför Zinken, vilket för honom representerar ren livsglädje och -intensitet. Bokens epigraf är hämtad från Linnés\ref{carl von linné} \textit{Systema Naturae} och lyder: \quotetext{Människans fysiologi, både den yttre och den inre, visar att korv och rullpizza utgör hennes naturliga föda.}

\ditem[Post-coitus]\label{post-coitus}
 Latin för \textit{efter samlag}. Uttrycket används för att beskriva skedet efter ett fullbordat könsumgänge mellan två eller flera parter. Inom begreppet finns flera dimensioner, då det post-coitala tillståndet kan variera från gång till gång. Om man är lyckligt lottad innebär det att röka cigaretter, småtjafsa och skoja och kanske dricka ett glas vin i sängen med den man haft sex\ref{sexa} med. Men det kan också innebära inträde i det psykologiska tillståndet \textit{post-coital tristesse}. Med det menas en lätt melankoli som sprider sig i kroppen efter könslek. Och visst går det att känna igen sig i ett milt missnöje som kittlar i en, när man ligger i en säng efter fullbordad akt och känner ejakulat och andra könsvätskor kallna och stelna, fastklibbat mot ens bäcken.

\ditem[Posten]\label{posten}
 Den förut så stolta institutionen posten för i vår digitaliserade och privatiserade värld en tynande tillvaro. Därför är det skönt att veta att en av våra mest namnkunniga radiopersonligheter, Jonas Hallberg, är en ivrig tillskyndare när det gäller just posten och postverket. De i politiken som ägnar sig åt att privatisera\ref{privatisering} och att göra nedskärningar i välfärden, till exempel Folkpartiet\ref{folkpartiet}, Kristdemokraterna\ref{kristdemokraterna}, Moderaterna\ref{moderat}, Centerpartiet\ref{centerpartiet} och Socialdemokraterna, har gått så till väga att man först ersatte alla postkontor med något märkligt som kallades för Svensk Kassaservice, för att skapa förvirring. Sedan en dag togs dessa kontor bort och alla postångare\ref{postaangare} sänktes och alla postlådor\ref{postlaada} flyttades samt fick en ny design. Därför krävs det idag noggrann planering och lite jävlar-anamma för att skicka iväg ett brev.

\ditem[Postiljon]\label{postiljon}
 1000000000000000000

\ditem[Postkolonialism]\label{postkolonialism}
 är en vetenskaplig teoribildning som i korthet går ut på att se hur den tidigare kolonialiserade delen av världen påverkats av detta faktum. Framträdande teoretiker är Edward Said, Homi K. Bhabha\ref{homi k. bhabhas son} d.ä, Gayatri Chakravorty Spivak, Trinh T. Minh-ha och Chandra Talpade Mohanty. Det vanligaste postkoloniala inslaget i vardagen hos 6-åringar, som med all rätt hatar fiskbullar i dillsås är det berömda maximet \quotetext{Men tänk på barnen i Afrika!}.

\ditem[Postlåda]\label{postlaada}
 En postlåda är en behållare som används av privatpersoner och företag för att ta emot information och objekt som sänts från jordens alla hörn. Det är så genialiskt att det nästan låter löjligt.

\ditem[Postmodern morförälder]\label{postmodern morfooraelder}
 Den postmoderna morföräldern är en morförälder som förhåller sig till sitt barnbarn på ett fragmentariskt, motsägelsefullt och liberalt vis. Hos morföräldern finns inga rätt och fel. Morföräldern vill bara barnbarnets väl, men samtidigt finns där alltid ouppnåliga krav som liksom hänger outtalade i luften. Men exakt vad är det som förväntas av det lilla barnbarnet? Morföräldern har lagt svångrämmen på hyllan och levererar julklappar som är könsneutrala. Detta betyder inte att barnbarnet har det särskilt mycket bättre hemma hos morfar och/eller mormor. Istället för aga används subtila ideologiska mind-games mot barnbarnet. Istället för tvång används skuldbeläggande. Istället för bibliska\ref{bibeln} fabler med solklara budskap inpräntas motsägelesfulla värden medelst förvirrande filmer med hysteriskt dansande djur och animerade gröna troll. Istället för det nedbrytande tickandet från marmoruret stimuleras barnbarnet med ett mummel av olika röster, åsikter, åskådningar, krav och förväntningar. Barnbarnet blir av allt detta mycket förvirrat och vet just inte vad det ska ta sig till. Barnbarnet vill bara sitta och läsa sin morsas gamla serietidningar, men vissa sidor saknas och av berättelserna återstår bara fragment. De obesvarade frågorna hopar sig. Fikat är ekologiskt, men å andra sidan kommer det från Brasilien.

\ditem[Postpostrock]\label{postpostrock}
 (jmf. postpostmetal, postposthardcore) är en benämning för rock producerad efter att postrocken slutade kännas OK, det vill säga någon gång tidigt 00-tal. Termen är inte applicerbar i vissa Europeiska länder, så som Island, där man ännu inte tagit steget in i postpostrockens era utan även i fortsättningen sitter och sjunger i falsett och plingar i nån jävla klocka man hittat i nån gammal låda.

\ditem[Postseminarium]\label{postseminarium}
 syftar på det som händer efter ett akademiskt så kallat högre seminarium (post betyder \quotetext{efter} på hebreiska). Medan det högre seminariet består i att forskare och doktorander diskuterar och ventilerar manus till vetenskapliga artiklar och avhandlingar går postseminariet ut på att samma doktorander och forskare går till en krog och äter pubmat, dricker sig berusade och antingen idkar ohämmad älskog eller hamnar i luven på varandra.

\ditem[Postångare]\label{postaangare}
 är båtar som främst transporterar post och helt eller delvis drivs av ånga från ångmaskiner. Tidigare hade den som varit sugen på att skicka ett vykort från en kontinent till en annan varit begränsad till att be en salt sjöbuse eller en välvillig men ack så naiv ballongfarare transportera försändelsen. Vilket i 99 fall av 100 resulterade att brevet försvann, och i det sista fallet blev kvarglömt hemma. Världens postiljoner\ref{postiljon} följde med rödgråtna ögon denna veritabla misshandel av skråets själva grundsten. Men eftersom postens\ref{posten} medarbetare alltid varit det offentliga samhällets stoltaste yrkesgrupp tog man saker i egna händer och började skeppa över försändelserna för egen maskin. Från början gick det åt en hel del brev för att hålla ångan uppe i pannorna, men sedan telefonkatalogen uppfanns har över hälften av all skickad post till en transoceanisk adress tagit sig fram.
Postångare är ett viktigt inslag i nästan alla Jules Vernes böcker.

\ditem[Potatisbar]\label{potatisbar}
 En potatisbar är ett matställe där allting har anknytning till den delikata rotfrukt dansken\ref{danmark} känner som kartoffel. Till förrätt kan man till exempel få en tallrik mos garnerat med pommes och till huvudrätt King Edward-rakor\ref{king edward} vända i potatismjöl av finaste Blå Kongo. Detta sköljs ned med ett glas bärs\ref{baersfylla} som hällts upp i en urgröpt Bintje\ref{bintje}. Till efterrätt blir det självklart potatisschwish med lika delar, Jansson, gräddpytt, Hasselback\ref{hasselbackspotatis} och en Magnum Bonum på toppen.

På barnmenyn återfinns traditionella rätter som förgyllts med en potatistouch. Det kan exempelvis vara spagetti och potatis, fattiga potatisar, quatro potato eller potatisar med sylt\ref{sylt} och grädde.

\ditem[Potatistryck]\label{potatistryck}
 är en enkel metod för att uttrycka sig.
En potatis, fördelaktligen bintje\ref{bintje}, delas i två delar och sedan karvas det ut ett motiv som sedan doppas i färg. Det fungerar bra med vattenfärg men som vanligt är det roligare med en typ som aldrig går bort. Den enkla grundregeln är att det som sticker upp är det som syns.
Men spegelvänt förekommer också, eftersom en del grafiker älskar att se när motivet de skapat plötsligt förändras och blir något annat - det blir speglat och nästan främmande, som att röra sig själv med en handske. Potatistryck används ofta i terapeutiska syften för att få barn\ref{barn} och gamla att reproducera sin talang.
En del utövare av så kallad gatukonst använder sig av de små charmiga knölarna för att trycka sina budskap på stolpar och i hörn.

\ditem[Praktarsle (negativ)]\label{praktarsle (negativ)}
 Praktarsle kan man kalla en person som man tycker gör dåliga saker. Om en person till exempel ringer till Ticnet samma dag som det släpps biljetter till en konsert med Bruce \quotetext{Bosse Sprängsten} Springsteen, och kommer fram, men istället bokar plåtar till ett Status Quo-coverband, ja då har man att göra med ett riktigt praktarsle.

\ditem[Praktarsle (positiv)]\label{praktarsle (positiv)}
 En på alla sätt iögonfallande ändalykt.

\ditem[Prins Charles]\label{prins charles}
 är den person i världen som varit tronföljare längst tid. Så länge han får ha hälsan ska ingen kunna hota Charles rekord för hans morsa tänker i alla fall aldrig dö.

\ditem[Prinskorv]\label{prinskorv}
ersätter morötter i recept där detta föreskrivs.

\ditem[Privatisering]\label{privatisering}
 Stöld från folket.

\ditem[Privatspanare]\label{privatspanare}
 En privatspanare är en privatperson som hjälper polismyndigheten att lösa brott. Många poliser tycker det är mycket roligare att slå sönder saker och arrestera istället för att lägga ihop ett och ett, så därför sköts istället mycket av det jobbet av privatspanande eldsjälar. Likt bibelns berättelse om den barmhärtige samariten handlar privatspanare helt utan egen vinning. En rättvisare värld är istället vad som driver dessa välvilliga förebilder. Generellt kan man säga att ju svårare brottet är, desto fler privatspanare behövs för att lösa det. Alla privatspanare tar fram sin egen teori och dessa ställs sedan mot varandra tills den bästa vunnit och brottet blivit löst. Poängsystemet är tyvärr för komplicerat för att redogöras för här men ibland kan det ta flera år innan man kommer fram till vilken teori som fått flest poäng. Likt seriespelet i bandyallsvenskan kan en teori toppa länge men sedan tappa mark och se sig omsprungen av andra. Det kan låta orättvist men hänger ihop med att sanningen alltid måste segra. Sveriges skickligaste privatspanare är Pelle Svensson\ref{pelle svensson} som bland annat löst palmemordet, friat Tomas Quick och tagit OS-silver i brottning.

\ditem[Problematiskt]\label{problematiskt}
 Att använda ordet \textit{problematiskt} för att beskriva en företeelse är i nio fall av tio helt menings- och ryggradslöst. Bruket av ordet ska tillskriva den som använder det en överlägsen intelligens, som att den förstod något på ett flertal nivåer, med en flugas mångfacetterade blick, och som att personen hade en ständigt rasande dialog inom sig, \quotetext{är detta rätt eller fel?}. Läsaren/lyssnaren ska ges intrycket att personens hjärna är fylld av kolliderande dialektiska åskviggar, en urladdning som skulle kunna lysa opp hela Roslagens Hallsberg (det vill säga Rimbo), hade bara blixtarna manifesterats i fysisk form, inte psykisk.

Egentligen signalerar användandet av \textit{problematiskt} bara att en person inte törs säga om den tycker att någonting är dåligt eller bra.

\ditem[Processa mot länsstyrelsen]\label{processa mot laensstyrelsen}
 Att processa mot länsstyrelsen är en gammal hederlig svensk tradition som syftar till att manifestera medborgarnas autonomi mot Kronan. I ödsligare delar av landet betraktas en persons första process mot länsstyrelsen ofta som ett tydligare steg in i vuxenvärlden än konfirmationen eller det första hemmakokade brännvinet\ref{braennvin}. De abrahamitiska religionerna snappade tidigt upp traditionens betydelse och lät föra in ett avsnitt i Gamla Testamentet som behandlade temat under rubriken \quotetext{David och Goliat}. Tanken med detta var ungefär densamma som när man lät döpa om midvinterblot till julafton.

\uline{Populära frågor att processa om}

\begin{itemize}
\item Rätten att ha hur många bilvrak man behagar på gården.
\item Att vargen redan var nedgrävd när man kom till platsen.
\item Att man bygger vad fan man vill på sin egen tomt.
\item Att hastighetesgränsen inte gäller på de vägar man själv använder mycket.
\item Att det inte är någon som använder naturreservatet, så det är visst en bra idé att köra enduro där.
\end{itemize}

\uline{Legendrarer inom aktivitetens historia}

\begin{itemize}
\item Folke Pudas\ref{pudaslaada}
\item Bengt Sändh
\end{itemize}

\ditem[Produktionsknull]\label{produktionsknull}
 Ett produktionsknull är ett samlag vars enda syfte att producera en avkomma. Det är ett kallt, själlöst arbete. Det kan utföras så väl under lysrörsbelysning som framför en sprakande brasa, utan att det gör någon skillnad. Njutning är underkastat. Allt som spelar roll är att ättenamnet fortlever. Man kan även produktionsknulla för att säkra sin pension. Barn kan vara bra att ha om man vill bli omhändertagen på ålderns höst.

\ditem[Prof. Etienne]\label{prof. etienne}
 småskurk, evighetsakademiker, entreprenör, belletrist, smugglare och skriftställare. En äkta renässansman med andra ord.

\uline{Privatliv}

I Times magazine kunde man i majnummret 2010 läsa att Prof. Etienne utöver sin akademiska karriär, tillsammans med sin fru Beatrix har startat en firma vars huvudsakliga syssla är att framställa och exportera taurin. Firman går under namnet \quotetext{Yeee haw!} - en interjektion som Prof. Etienne också ska ha myntat.

\uline{Bibliografi (urval)}

\begin{itemize}
\item \textit{P.S Dra åt helvete! - Brev i urval} Federativ förlag, Stockholm 1992
\item \textit{Dävertspotting}\ref{daevertspotting} Natur och Kultur, Stockholm 1999. 
\item \textit{Om arternas uppkomst}. Bonnier Fakta, Svinesund 1993. 
\item \textit{Hundra sätt att få ligga !{}Hundra sätt att få ligga}. Allers förlag, Kalmar 2005. 
\item \textit{Morsan! Farsan! Digga kniven! - en stridsskrift för barns rätt till att plåga djur utan att bli stigmatiserade av samhället.} Daidalos, Göteborg 1998 
\item \textit{Barnagans förträffliga pedagogik}.\ref{barnagans foortraeffliga pedagogik} Rabén\&Sjögren, Uppsala 1998.
\item \textit{Hur man blir en vinnare utan att behöva vara nykter i onödan}. Alla barns bokklubb, Säffle 2003. 
\item \textit{101 saker man kan lösa med våld}. Bonnier Fakta, Bälinge 2001.
\item \textit{Usch och fy! - Världens 100 äckligaste dofter}. Studentlitteratur, Gagnef 1995.
\item \textit{På spaning efter den bov som flytt}. Piratpiratförlaget, Bollebygd, 1997
\item \textit{Den övre magmunnen i litteratur och mytologi}. Nordiska rådets förlag, Stockholm, 1997.
\item \textit{Självbiografi, del 1 - Snälla mamma, mata mig som vore du en fågel}. Timbro förlag, Finland 1999.
\item \textit{Självbiografi, del 2 - De förlösande thinneråren}. Timbro förlag, Finland 2002.
\item \textit{När livet blir alldeles för mycket - Prof. Etiennes bästa gömställen, i urval }\ref{naer livet blir alldeles foor mycket - prof. etiennes baesta goomstaellen, i urval}. Bengt Alsterlinds förlag, Skattkärr 2002.
\item \textit{Självbiografi, del 3 - En bärs, en bärs, min järndanksamling för en bärs}. Timbro förlag, Finland 2004.
\item \textit{Självbiografi, del 4 - När jag sköt Elefantmannen}. Timbro förlag, Finland 2007.
\end{itemize}

\uline{Filmografi}

\begin{itemize}
\item \textit{Mazurka på sengekanten} (1970) i rollen som Dr. Preben Jurgens.
\item \textit{Discografi}
\item \textit{Fistula brothers - Hallå herr hålkortsoperatör!}, keytar och maracas.
\end{itemize}

\ditem[Professor skytteanus]\label{professor skytteanus}
 är en yrkestitel som innehas av den manlige akademiker som erhåller Johan Skyttes\ref{johan skytte} professur i retorik och statsvetenskap vid Uppsala universitet. Idag innehas den av Li Bennich-Björkman, men hennes titel är professor skytteana för att hon är brud. Det är världens äldsta professur, till råga på allt.

\ditem[Proggig inre frid]\label{proggig inre frid}
 Ett motstånd mot den vulgära materialistiska hybris som breder sig ut över världen. Ett tillstånd av ekvilibrium, där du och Gaia återförenas och kramas på ett mentalt plan, vid ditt inre Yggdrasils rötter.

Du kan gå längst med gatorna i vilket radhusghetto som helst, oberörd av SUV:ernas stön och ångestmotionerande 45-åringar. Du bara tänker på hur vackert det är att björkarna knoppar, och hur skönt det är att igelkotten du såg leta efter ett hem förra hösten verkar ha grävt ett gryt under Johanssons stjärtlapp som legat framme sen i november och nu börjar tina fram igen.

Innan du och de andra i gänget la ner bandet efter att släppt en LP hade ni förhoppningar om att bli de nya Träd, Gräs och Stenar, men så blev det inte. Och visst, det var faktiskt jobbigt att se hur new wave-tönten Per Gessle blev helt enorm, trots att han inte fattar någonting om musik. Men de få gånger de jobbiga tankarna tränger sig på idag, så täljer du bara en slev och sen är allt skönt igen. Häromdagen kom en ung kille fram och frågade om det verkligen var du som hade spelat vattenskinn i det gamla bandet. Du snackade med honom ett tag och stod sen paralyserad utanför konsumbutiken i en halvtimme och kontemplerade hur flummigt det är att musikaliska vågor kan sträcka sig inte bara genom rum, utan även genom tid. När du kom hem hade smöret i konsumkassen smält, men det smakade gott ändå.

\textit{Du lever nämligen i ett tillstånd av proggig inre frid.}

Varning: Att uppnå proggig inre frid kan ibland kräva att man ger sig hän åt aktiviteter som också kan leda till storhetsvansinne. Var aktsam när du: sover utomhus i orimligt låga temperaturer, börjar diarieföra din avförings konsistens, skaffar skäggrifsyr med hjälp av ett rakblad du smitt själv och när du trakterar obskyra blåsinstrument på bergstoppar.

\ditem[Prokrastrinering]\label{prokrastrinering}
 är ett slags beteendemönster som är speciellt vanligt hos unga män från övre arbetar- eller lägre medelklass och tar sig uttryck i att den unge mannen skjuter upp att göra saker genom att sysselsätta sig med en annan, mindre produktiv men minst lika tråkig aktivitet. I det protestantiska Norden ser många på prokrastrinering som något ont, vilket är en tämligen onyanserad hållning, medan det i själva verket har medfört många positiva och (för andra människor) uppbyggliga resultat under årens lopp. 

\uline{Prokrastrinering som psykologiskt fenomen}

Inom psykologin förklaras prokrastrinering som ett slags uppsättning av mekanismer genom vilka ett företagande som i sig är minst lika oattraktivt för individen\ref{individ} som det man förelagts att utföra framstår som mer önskvärt. Istället för att skriva uppsats samlar man med händerna damråttor från golvet under sängen. Istället för att klippa gräset ser man hur många stenar man kan pricka ett visst träd med i följd. En skicklig prokrastriatör kan i princip hitta en aktivitet att sysselsätta sig med i vilken given situation som helst och i avsaknad av extern input eller redskap.

\uline{Prokrastrinering som kampmetod}

Inom arbetslivet liksom den moderna familjen används prokrastrinering som ett slags kampmetod i lågintensiva motsättningar mellan kapital och arbetare (sommarjobbare) eller familjeöverhuvud och lägre rankade familjemedlemmar. Kampmetoden går ut på att in i det sista, när arbete inte längre kan undvikas, spela \textit{Angry Birds} och på så vis försinka och fördröja den utdelning som den ekonomiska eller geneaologiska makten förväntar sig av prokrastriatören.

\uline{Lista på populära aktiviteter inom prokrastrineringens ädla konst}

\begin{itemize}
\item Betapet
\item Kasta macka
\item Tugga tuggummi och lyssna på rock'n'roll
\item Tetris
\item Tälja
\item Röka gräs
\item Vilskita\ref{vilskita}
\item Bygga en fågelholk som rymmer en solpanelsdriven bandspelare med en loopkassett med \textit{Surfin' bird\ref{surfin bird}}
\end{itemize}

\ditem[Proletära bär]\label{proletaera baer}
 är bär som spisas i stora mängder av arbetarklassen. Gemensamt för dessa bär är att de växer vilt i stora mängder, så vem som helst med lite tid över kan plocka dem och konsumera. Har man inte tid kan man också köpa dem relativt billigt i sin närmsta konsumbutik\ref{konsumbutik}. De två mest klassiska proletära bären är blåbär (till arbetarens morgonfil) och lingon (till arbetarens blodpudding på lunchen). Även smultron och hjortron räknas faktiskt till de proletära bären, men hör till den speciella undergruppen \textit{proletära festbär}. En annan undergrupp är \textit{trasproletära bär} och dit räknas rönnbär och enbär.

De proletära bärens antites är de borgerliga bären med jordgubben i spetsen. Denna måste planteras och vattnas och skyddas mot fåglar; inte så proletärs direkt. Även körsbäret hör hemma här eftersom det växer på ett träd som tar många år att växa till sig.

Bananen\ref{banan} står som vanligt utanför den gängse klassificeringen. Å ena sidan måste det fraktas från andra sidan jordklotet (skapligt borgerligt), men å andra är den ofta smårutten när den väl hamnar i affären (ganska proletärt).

\ditem[Propeller]\label{propeller}
 En propeller är en maskindel som underlättar maskinell drift av marina farkoster. Den sitter vanligtvis baktill på en båt och påminner till formen vagt om Storbritanniens före detta premiärminister, Tony Blair.

\ditem[Propellerkeps]\label{propellerkeps}
 Danmarks\ref{danmark} nationalhuvudbonad. Lanserades först som en del i opinionsbildningen för vindkraft. Vid denna tidpunkt fruktade många danskar att minskad kolkraft kunde medföra bristande tillgång på kol så att de inte kunde rena sitt hemmakokade brännvin. Resultatet blev dock det omvända och det är därför alla danskar älskar vindkraft nu.

\ditem[Prunka]\label{prunka}
 Att prutta och runka samtidigt. Prutten kan komma antingen från anus eller vara pruttliknande ljud från prunkarens könsorgan. Ibland är prunkning avsiktlig. Ibland är den inte det. Vid ofrivillig prunkning övergår ofta onanisessionen i att man antingen skrunkar\ref{skrunka} eller grunkar\ref{grunka}.
Prunka betyder även att vara grann och att pryda sig, på gränsen till högmod.

\ditem[Prutta högljutt]\label{prutta hoogljutt}
 Att prutta högljutt är en av humor-historiens klassiker. Skämtet har idag en viss rustik atmosfär kring sig, men förekommer då och då även i de mest metropolitiska miljöer. En av de saker som gör skämtet så uppskattat är att det är förenat med en viss risk för den som utför det, eftersom det kan resultera i oönskade olägenheter för denne. Lyckas skämtet blir det å andra sidan många gånger en riktig glädjespridare och framkallar skratt och munterhet.

\uline{Högljutt pruttande i den danska kulturindustrin}

Det sägs att Lars Krogh kan prutta så våldsamt att det är som att en vindil drar genom trädtopparna. Om han bara vill, men han vill inte. Han vill bara ge ut sjutummare\ref{sjua} med olika garageband.

\ditem[Psoriasis]\label{psoriasis}
 Ungefär som ärr fast inte lika tufft. Faktum är att Englands dermatologiska förening, i samarbete med Esrange, har bevisat att Mars yta i själva verket består av sedimentär kontinental psoriasis. Tesen framfördes av Syd Barrett redan 1965, men som så många andra genier var han före sin tid och fick ordentligt gehör först när ecstasy slog igenom på dansgolven.

\ditem[Psykedelisk morförälder]\label{psykedelisk morfooraelder}
 En psykedelisk morförälder är en person som har en dotter som i sin tur har ett barn\ref{barn}. Till yttermera visso omger sig den psykedleiska morföräldern av en kalejdoskopisk tillvaro där tonerna av Baby Woodroses \textit{Third Eye Surgery} och Blue Cheers \textit{Vincebus Eruptum} liksom omsluter barnbarnet likt en kokong av mjukhet och föränderliga proportioner. När barnbarnet frågar sin psykedeliska morförälder hur det var förr i tiden sätter denne sin avkommas avkomma i knät, lägger på en \quotetext{lakritspizza} med \textit{Lamp of the Universe} på skivspelaren och rabblar upp en svårbegriplig massa ord vars första bokstäver tillsammans bildar orden \textit{Pluteus salicinus}. Sedan cyklas det Christianiacykel\ref{christianiacykel} till sjön där den psykedeliska morföräldern gör solhälsning, letar svamp och till sist somnar in, som ett barn, under en gran, medan barnbarnet på uppdrag åt sin mors förälder pliktskyldigt bygger ett sandslott åt Sun Ra.

\ditem[Pudaslåda]\label{pudaslaada}
 En Pudaslåda är en låda, stor nog att rymma en fullvuxen människa, som används som ett medel att protestera mot något. Den ursprungliga Pudaslådan användes av den norrbottniske\ref{norrbotten} taxichauffören Folke Pudas\ref{folke pudas} när denne skulle processa mot länsstyrelsen\ref{processa mot laensstyrelsen} (och i förlängningen regeringen) genom att ligga i sin låda på Sergelstorg och vara sur.

När ni läser detta framstår kanske Pudas som en gammeldags rättshaverist, men i slutändan stod det Pudas 1 - Svenska staten 0. Denna process resulterade i lagen lex Pudas. Pudaslådan är således ett underutnyttjat medel i kampen mot byråkratin. Att kalla Pudas för en folkhjälte\ref{folkhjaelte} är inte en överdrift.

\uline{Pudaslådan i kulturen}

Pudaslådan besjungs i en sonett av en viss välkänd tidningsredaktör.

\quotetext{Sonett\ref{sonett (engelsk)} till Folke Pudas}

\textit{Högt i skyn flyger en vit pelikan }\ref{pelikan}\textit{.}
\textit{Säg, var är vår pungprydda vän på väg?}
\textit{Han sjunker nu ner mot mitten av stan}
\textit{för att på en liten kub slå sig ned.}
\textit{Vad är detta för gåtfullt litet skåp,}
\textit{med sina många och arga plakat?}
\textit{Inifrån hörs dunsar och ylande gråt}
\textit{och snyftningar som är fulla av hat.}
\textit{I lådan ligger en norrbottnisk man}
\textit{som processar mot länstyrelsens }\ref{processa mot laensstyrelsen}\textit{ dom.}
\textit{Vi får troligtvis veta vem som vann}
\textit{då denna märkliga låda står tom.}
\textit{För kampen mot trams och byråkrati}
\textit{fortgår så länge nån ligger däri.}

\ditem[Punk]\label{punk}
 är en musikstil som kom till i USA kring mitten av 70-talet men fick sitt stora genomslag i och med att britterna i Sex Pistols släppte en riktigt, riktigt kass skiva. Förr i tiden var det provocerande och lite farligt att vara punkare. Idag är det lika farligt och provocerande att vara punkare som jazzpjatt.

\ditem[Punkgryta]\label{punkgryta}
 Här följer ett recept på alla malätna crustares favoritmat, punkgrytan.

\begin{enumerate}
\item Bli lite full, gärna bärsfull\ref{baersfylla}.
\item Kolla om det finns lite mat i diverse containrar.\ref{sopletare}
\item Köp det som fattas, nä jag skojade bara.
\item Gör valfri mat bestående av olika grönsaker, tillsätt krossade tomater.
\item Koka ris tills det ser ut som mannagrynsgröt.
\item Upptäck att grytan bränt fast, rör om, låt bränna fast igen.
\item Servera, maten får under som inga som helst omständigheter mätta sällskapet.
\end{enumerate}

\ditem[Punkscenshumor]\label{punkscenshumor}
 är ett slags humor som är besläktad med PK-humor och liknande grenar på humorns vidsträckta släktträd. Punkscenshumorn består i skämt som på något vis har att göra med punkscenen och som delvis har som uppgift att cementera den gemenskap som punkaren känner då han eller hon umgås med sina punkkompiser. Rolighetsnivån på skämten är i genomsnitt strax ovanför att förlora en närstående till kräfta. Ett ofta förekommande skämt är att mellan låtar ropa \quotetext{Spela snabbare!} eller \quotetext{Mangla!} Detta skämt har förekommit på samtliga svenska punkspelningar åtminstone sedan mitten av nittiotalet och har alltså traderats mellan generationer. Andra punkskämt har anekdotform och är uppdiktade historier om inom scenen namnkunniga strulputtar med en svaghet för starkvaror som skämtaren i lönndom ser upp till, så som Jonsson i Anti-Cimex. Ännu en gren av skämt går ut på att antingen häckla äldre band som inte spelar punk eller att skämtaren påstår sig lyssna på ett band eller skiva som anses vara dåligt eller töntigt, som exempelvis Charta 77 eller Moderat Likvidations tolva \textit{Mammutation}. Det som gör denna humor speciell är att ingen faktiskt tycker att den är rolig. Liksom ett pidgin-språk är den heller inte någon individs faktiska \quotetext{modersmålshumor,} utan används enbart inom \quotetext{scenen}.

\ditem[Putsbilar]\label{putsbilar}
 En putsbil är ett äldre motorfordon som ägs av en svårt neurotisk man i medelåldern, vilken ägnar all sin lediga tid åt att polera eller tänka på att polera sin bil. Bilarna står oftast still, trots renovering till nyskick. Dock får de en aning motion på sommarhalvåret då de ska visas upp på diverse helylletillställningar. Det närmsta en putsbil kommer hederligt arbete är när ägaren (då hans egna företag går lite svajigt) hyr ut sig själv och sin bil till ordnade tillställnigar, exempelvis bröllop.

\ditem[Pysselbyxa]\label{pysselbyxa}
 är den mer vanliga och konventionella arbetsbyxans syskon och är avsedd att användas i samband med hobbyn pyssel. Brukaren bär denna förtjänstfulla byxa när hen dekorerar ljusstakar med mossa, målar ägg, limmar samman flirtkulor eller tillverkar en ivrig bäver\ref{ivriga smaa baevrar} av \textit{papier maché}. Periodvis har pysselbyxan också används som mode av medvetna stockholmsprofiler, men eftersom pysselbyxan inte har något speciellt utseende utan är ett par byxor som användaren valt att avsätta för pysseländamål är det svårt för gemene man att förstå detta fashion-statement.

\uline{Historia}

Redan Leonardo da Vinci ska ha ägt ett par pysselbyxor, vilka han använde då han målade Mona Lisa. Vissa konsthistoriker har spekulerat i att detta kan vara anledningen till Mona Lisas lite hånfulla leende som Da Vinci undermedvetet skulle ha målat in i sin berömda tavla på grund av den osäkerhet som oundvikligen smyger sig på en fullvuxen skäggig man som iklädd pysselstrumpbyxor arbetar inför en okänd kvinna i ett otal timmar.

\ditem[På fat]\label{paa fat}
 Kaffe, denna svarta och vederkvickande dryck, inmundigas med fördel på fat.
Detta kan göras på en hel del olika sätt, men enligt den vindpinade, skäggige och LO-anslutna elektrikern Anders \quotetext{Skäggu-Anders} Johansson vid Malå\ref{malaa} Sågverk\ref{saagverk} görs det enligt följande. Kaffet hälls upp på fatet genom att låta det rinna längs koppen på fatet. Sedan lyfts fatet medelst tre fingrar (nybörjare tillåts använda fyra för att undvika söl) och en sockerbit placeras mellan läpparna, framför tänderna. Sedan sörplas kaffet graciöst genom sockerbiten och får således en perfekt sötma. Ljuvligt.

Om det däremot är helg rekommenderas detta att följas upp med en liten bit choklad, kanske en After Eight, en stadig konjak (gärna Grönstedts tre-stjärniga) och en liten cigarill. Här började kamrat Johansson drömma sig bort från det grådassiga fikarummet till Greklands kritvita stränder och fantastiska bistros, varpå skifteslaget fick påminna om att det var flera månader kvar till semester.

\ditem[Påsförslutare]\label{paasfoorslutare}
 En påsförslutare är en lite mackapär som används för att försluta en plastpåse och på så vis förhindra att dess innehåll går förlorat vid transport. Påsförslutaren kan vara gjord av plast och likna en hästsko till en mycket liten ponny. Den kan också bestå av en plastingjuten ståltråd som liksom viras runt påsen. Dessa sorters påsförslutare får man på köpet då man inhandlar fabriksbröd, men det finns också andra, bättre och mer beständiga påsförslutare som kan köpas enkom och som då levereras i storpack om minst ett tjog. Dessa är avsedda för storfräsare\ref{storfraesare} som inte ids att varsamt samla på sig påsförslutare, en i taget, i försänkningen bredvid den som är avsedd för teskedar i besticklådan. Som med så mycket annat måste storfräsaren ha omedelbar utdelning och belöning, och värderar således inte sina påsförslutare lika högt som gemene man, som med all rätt prioriterar två Tingsryd 2,8or att avnjuta frampå fredagskvällen\ref{fredag}.

\ditem[Påsk]\label{paask}
 är en högtid som firas varje år för att uppmärksamma att Jesus\ref{jesus} fortfarande är död. Den är väldigt populär att fira för man får äta hur många ägg\ref{aegg} man vill utan att behöva tänka på kolesterolen.

\ditem[Påsmygande själv-alienation]\label{paasmygande sjaelv-alienation}
 Tänk dig att du medan skymningen faller varit på väg hem från en avlägsen tätort. Din Saab 900 har kokat och du har tvingats söka husrum, och glatt välkomnats, hos en pensionerad man som tidigare drivit lanthandel med sin nu bortgångna hustru, Hagar. Det bjuds på portvin och ni spelar canasta. Sent på kvällen befinner du dig smygandes på tå, med långa, försikta steg längs gårdstomten, med regelbundna blickar över axeln och famnen full av uppstoppade fåglar som du för ditt liv inte kan förstå att du snott från din givmilde värd, som nu möter dig med ett artigt leende på väg tillbaka från utedasset.

\ditem[Päronhalva]\label{paeronhalva}
 är en i arbetarklassen omåttligt populär efterrätt och är något av det finaste man kan bli bjuden på när man är på besök hos någon. Päronhalvan köps alltid i konservburk och serveras med fördel tillsammans med vispgrädde eller After Eight, vilket är den förnämaste formen av päronhalvsservering. Den som bjuds på päronhalva med After Eight står traditionellt i oupplöslig tacksamhetsskuld till värden. När någon har ätit så många päronhalvor att den snart nog inte kan få ner en enda liten päronhalvstugga till ska värden säga \quotetext{ta den sista du - annars står det bara och blir gammalt} och då måste man kliva fram och verkligen göra sitt bästa.

\ditem[Pölsa]\label{poolsa}
är en smet av inälvor som åts i Sverige förr i tiden på grund av nöd, och under senare år för att bestraffa barn.

\uline{Pölsa i litteraturen}

Pölsa omnämns ofta i svensk arbetarlitteratur, men den mest minnesvärda litterära skildringen av pölsa är utan tvekan Torgny Lindgrens \textit{Pölsan} (2002). Boken kretsar kring två män som åker runt i Skellefteåtrakten och smakar olika pölsor, och bara älskar det så mycket. Lindgren är känd för att använda sina sagor som metaforer för skeenden i verkligheten. Det har spekulerats mycket om vad pölsan ska representera, och de flesta gissningarna har varit helt idiotiska. Efter en blöt utekväll med prästen som katoliken Lindgren biktar sig för, har Nissepediaredaktionen lyckats reda ut begreppen. Varje pölsa representerar ett av Clay records första fem släpp, där den mest överlägsna pölsan är en symbol för Discharge - \textit{Realities of war} 7" (Clay 1: 1980), en skiva som Torgny Lindgren älskat ända sedan han bytte till sig den från en grisfull träskpunkare utanför Trästockfestivalen 1991 mot ett sexpack ägg och en halvdrucken Bjørnebryg 7.7 \%.

\ditem[Pörr]\label{poorr}
 Riktigt äckligt snusk. Sånt din mor skulle göra dig arvlös för om hon fick reda på att du tittade på't. Sånt som gör fullvuxna karlar helt likbleka av skam och självförakt. Sånt småungar hittar under en mossig sten i skogen när de letar patroner. Sånt riktiga vidron stoppar in mellan \textit{Fievel i vilda västern }och \textit{Lilla sjöjungfrun} VHS-erna på Videoline. Sånt kyrkor startas för att protestera mot. Sånt innebandykillar och militärer kollar på i grupp. Sånt som har titlar man skrattar åt när man hör, men blir mörkrädd och lite ledsen av när man tänker efter.


%%%%%%%%%%%%%%
\newpage
\null
\\
\null
\\
\Huge
Q
\normalsize
\\
\null
\\
\null
%%%%%%%%%%%%%%


\ditem[Queequeg]\label{queequeg}
 är både harpunisten i Moby Dick och agent Dana Scullys hund i Arkiv X. Den senare dog när den blev uppäten av en krokodil eller en dinosaurie, det hela är något oklart.

%%%%%%%%%%%%%%
\newpage
\null
\\
\null
\\
\Huge
R
\normalsize
\\
\null
\\
\null
%%%%%%%%%%%%%%

\ditem[RAC]\label{rac}
 Förkortning för \quotetext{rock against communism}, är en musikstil smalare än tantsång\ref{tantsaang}. Likt black metal karaktäriseras RAC mer av en inställning till livet och teman i texterna än av hur musiken låter. Det finns förvisso flera genomgående musikaliska drag, såsom att det låter sämre än både belgisk\ref{belgien} grishardcore och irländsk stadiumrock, och att sångaren stönar fram texterna på grund av sin övervikt. Men det är mer märkliga sammanträffanden\ref{maerkliga sammantraeffanden}. Något annat som kan te sig lite märkligt i sammanhanget är att det mest är politiskt vänsterorianterade människor som lyssnar på musiken. Att på detta sätt trotta\ref{trotta} en musikstil som man egentligen inte gillar har inte skådats sedan kristna började lira rock. Det vanligaste att sjunga om inom RAC är så klart att alla rödingar ska dö. \quotetext{Hellre död än röd} är en vanlig paroll som den överviktiga sångaren i mjukisbyxor grymtar fram. För att konceptet inte ska bli allt för förutsägbart brukar man stoppa in en eller två låtar som handlar om något annat på varje skiva. Det kan till exempel vara att vikingarna är deras förfäder eller att pitbulls är världens gulligaste hundar.

\ditem[Radioreklam]\label{radioreklam}
 är ett fenomen som skapats för att vokalgrupper som The Real Group och Viba Femba ska ha något att göra nu när resten av världen äntligen börjat avfärda acapellamusik som ren skit. I radioreklamens värld har acapellan nämligen fått en fristad eftersom hippiesarna i Stockholms innerstad fått för sig att det bästa sättet att skapa uppståndelse hos pöbeln är medelst en rejäl skopa skit. På den yttersta dagen ska reklamarna få betala för detta genom att Aleksandr Karelin\ref{aleksandr karelin} och Pelle Svensson\ref{pelle svensson} bryter av deras ben från varsitt håll samtidigt. På den yttersta dagen har alla regler nämligen upphört och det är okej att bruka kampsport utanför sin dojo\ref{dojo}.

\ditem[Rage-quita]\label{rage-quita}
 Du är 26 år och gamer. Du ser dig själv som \quotetext{OG}. En äldre kusin hade ju en Amiga 600 och du spelade mycket Quake för väldigt många år sedan (2009, online-versionen). Nu spelar du nya Call of Duty på din nya spel-PC. Singelplayer var ok, men det är online som gäller. Fast när du spelat online går det inte så bra. De senaste matcherna har du k/d på 0.1. Det beror dock mycket på de andra i ditt lag. I denna matchen har du nu hamnat 1 mot 1 med samma spelare 6 gånger (TitanGamerBoy2002). Det är någon 12-åring som är helt sjukt säker. Hur fan gör han, jävla hacker! Han är ohyfsad i chatten också. Nu jäklar skall du visa punk-ungen vad erfarenhet betyder. Nu, nu har du läget, 1 mot 1 igen. Denna gång hinner han skriva \quotetext{lol u suck…} i chatten innan han skjuter dig med ett headshot. Det blir för mycket, med en nyfunnen kraft knäcker du det aluminiumförstärkta gamer-tangentbordet över knät. Sedan drar du en haymaker, som Floyd Mayweather (far och son) skulle uppskattat, mot Mariodockan du köpt på Game. Sedan drar du ut nätverkssladden.

\ditem[Raison d'être]\label{raison dêtre}
 Kombination av de franska orden \textit{raison} [russin] och \textit{d'être} [ätbart]. Utöver att syfta på en torkad vindruva som är hanterbar för det mänskliga matspjälkningssystemet, har uttrycket en vidare filosofisk innebörd, då det stundvis används som en metafor för att förklara drivkraften i en människas existens. Exempelvis, vad är Göran Greiders raison d'être? Vilket är det tilltalande skrynkliga russin som får honom att resa sig ur sängen och sätta på sig gummistövlar varje dag? Jo, det är drömmen om ett socialistiskt samhälle, gulliga katter och soliga sensommardagar.

\ditem[Randall Finefield]\label{randall finefield}
 är Sveriges f.d statsminister Fredrik Reinfeldts\ref{fredrik reinfeldt} täcknamn när han vill röra sig mer fritt. Iförd en sportig blazer, en jordgubbsblond tupé och ett par kolsvarta Ray bans reser Randall Finefield jorden runt för att förlusta sig. Han introducerar sig som miljardär, filantrop och älskare av kvinnor. Från kamelpolo i Dubai till exklusiva konstauktioner i New York; Randall Finefield är överallt, hela tiden. Eftersom det egentligen kan vara ett rätt uppslukande jobb att vara politiker har Fredrik Reinfeldt sett till att klona sig själv, så att han kan tillbringa mer tid i rollen som sitt supersexuella jetsetande alter-ego, Randall Finefield. Samtidigt sitter hans klon hemma och snyter den egentlige Reinfeldts barn och slaskar upp en alkoläskfylla\ref{alkolaeskfylla} med Anders Borgs alter-ego, Robert Wells.

\ditem[Ranta Runtiringen]\label{ranta runtiringen}
 Ranta \quotetext{Jarmo} Runtiringen (1928-1958) är Finlands\ref{finland} sämsta boxare någonsin. Han dog på sin 30-årsdag i en alkohol-, hundspann-, jojk-, kniv- och lädervästrelaterad olycka. I hemstaden Oulu i norra Finland finns än idag ett monument över stadens största kändis kvar. Utanför det magnifika paradisbadet står en boxarhandske greppandes en morakniv, uthuggen i massiv granit, med inskriptionen \textit{\quotetext{Yksi, Kaksi, Kuolema}} (Ett, Två, Död). Inskriptionen är ett citat av Runtiringen, från innan han 1935 gick in i ringen för att möta sin främste konkurrent om bottenplaceringen, Jagger \quotetext{Ballista} Keffiringen.

\ditem[Rasism]\label{rasism}
är en psykisk sjukdom som bottnar i en irrationell rädsla för att julafton ska avskaffas. Symptom innefattar förföljelsemani, vredesutbrott och psykotiska episoder där den sjukes verklighetsuppfattning inte är samstämmig med det övriga samhällets.

\ditem[Rasmus Klump]\label{rasmus klump}
 är det danska originalnamnet på den tecknade brunbjörn i toppluva och prickiga hängselbyxor svenskar känner som Rasmus Nalle. Med tanke på seriernas bristfälliga manus får man för en gångs skull faktiskt ge danskarna rätt – Klump passar mycket bättre. Helt uppenbart är Rasmus Klump egentligen bara en taskig rip-off av den sedelärande Bamse, men utan den senares känsla för solidaritet och civil olydnad. Som sidekicks har Rasmus Klump den nervösa pingvinen Pingo (Lille skutt), pelikanen\ref{pelikan} Pelle som alltid har något användbart i sin näbbpung (Skalman) och den ständigt piprökande sälen Sälle (Burres\ref{burre} pappa). Det är minst sagt oklart varför, men hela 37 album om Rasmus Klump finns faktiskt publicerade. Det är i och för sig lätt att dra till med en hög siffra eftersom den som får för sig att kontrolläsa ändå skulle tröttna efter 3-4 album.

Varje år delas \textit{Rasmus Klump-priset\ref{danska hedersbetygelser}} ut i Danmark\ref{danmark} till en dansk som \quotetext{uppvisat de egenskaper som serien förespråkar}. Vad fan nu det är. Pristagare är i alla fall bland andra kronprins Fredrik och Michael Laudrup.

\ditem[Raw justice]\label{raw justice}

 var ett svenskt punkband från Fagersta. Själva skulle de nog subkategorisera sig som råpunk, men de flesta lyssnare brukar anse att det snarare rör sig om jävligt dålig punk. Från början kallades bandet Cruel Maniax, och under denna etikett släppte man en splitsjua\ref{sjua} med eskilstunaorkestern No Security och var med på samlingsskivan \textit{The vikings are coming}. Sen bytte man namn, oklart varför, och gick tillbaka till att släppa kassettdemon. I ett nummer av fanzinet Chaos is King utsågs Raw Justice till världens sämsta band. I ett samtal mellan en nissepediamedarbetare och en medlem i bandet deklarerade dock bandmedlemen att Raw Justice var världens bästa band.

\ditem[Ray Jones IV]\label{ray jones iv}

 (född 7 december 1967 som \textit{Roy Ray Jones}) är en svensk actionskådis som spelat in milstolpar såsom \textit{Sökarna}, \textit{30:e november} och \textit{Sökarna 2 - Rebelz}. Dessutom har Ray Jones IV gjort flera bejublade inhopp i TV-serien \textit{Rederiet}. Att Ray Jones IV har romerska siffror i sitt namn gör att man lätt misstänker att han inte bara är kung på bioduken utan också i verkliga livet. Nissepedias medarbetare har ännu inte hittat några konkreta bevis för detta men sökandet fortsätter. Helt klart är man speciell om man lyckas byta ut \textit{ }Roy i förnamn mot \textit{ }IV i efternamn.

\ditem[Realister]\label{realister}
 är svenska män som, ofta utan byxor\ref{sans pants}, kommenterar nyheter i dags-, kvälls- så väl som lokaltidningar. Där uttrycker de sin oro inför mångkultur och genusdebatt, vänsterpolitik och bensinpriser. Realisten inser att progressiv debatt, en modern invandrings- och asylpolitik och andra fenomen som karaktäriserar eller borde karaktärisera samtidsklimatet är mycket skadliga och hotar att leda till samhällets kollaps. Med \quotetext{samhällets kollaps} menar realisten skapandet av ett jämlikt samhälle och den desperata situation där han och andra män utan byxor inte längre utgör en speciallt priviligerad grupp inom väljarkåren och får lika mycket eller litet att säga till om som alla andra människor i det demokratiska Sverige\ref{sverige}. Realisten är därför en varm tillskyndare av etnisk rensning, mordhot, anti-intellektuell häxjakt och högerextremism, som realisten tycker är realistiska politiska krafttag i kampen mot invandrare, kvinnor, homosexuella, vänsterintellektuella (sk. \quotetext{kulturmarxister}), journalister, damfotboll, muslimer, konstnärer och författare (dock inte Lars Wilks), miljörörelsen, fredsrörelsen och pride-paradens deltagare.

\ditem[Repet]\label{repet}
 är det tredje scoutmärket och föregås av Nyingen och Scouten. Märket sys fast väl synligt på scoutskjortan. För att få repet måste man kunna:

\begin{itemize}
\item Knopar; Råband, Skotstek, Överhandsknop, Pålstek.
\item Raksurrning, vinkelsurrning, trefot.
\item Miniorknoparna samt timmerstek,tältlineknop, dubbelt halvslag.
\end{itemize}

För att få utmärkelsen är det inte nödvändigt att tillverka ett vapen av en socka med en tvål i och spöa upp en tjock unge med det. Att göra så kan till och med försvåra processen med att få utmärkelsen avsevärt eftersom det anses gå emot scoutrörelsens värderingar.

\ditem[Richard Dybeck]\label{richard dybeck}
 Född Mutapuro (Rickardo), eventuellt Mutapiki. Stamm' aus Finnland, echt Schwede.

\ditem[Riddare]\label{riddare}
 är ett yrke som funnits i hundratals år och går ut på att försvara heder och ära. Ofta verkar det mest vara sin egen heder riddaren försvarar, så det är lite oklart var finansieringen kommer ifrån. Börsanalytiker har på grund av detta länge fruktat en riddarbubbla, men då höjer riddaren bara sin mäktiga lans och alla blir genast lugna. Vanliga arbetsredskap för att klara av ridderiet är, förutom lans, sköld och häst, vilka riddaren tar med sig till ett tornerspel. På tornerspelet använder riddaren all sin styrka och list till att besegra andra riddare, vilket ger den omåttlig ära och berömelse. Tack vare sin mystiska aura har riddare gett upphov till det engelska ordet för gåta, riddle, mejeriprodukten riddargrädde, och maträtten fattiga riddare.

\ditem[Rikemanssidan]\label{rikemanssidan}

 Som de flesta spörsmål har även livsmedlet bröd två sidor. En del brödsorter som bakas på häll, exempelvis tunnbröd och polarkaka, har en bullig sida och en slät sida. Dessa sidor är som de mesta annat här i världen präglat av klassamhället och utgörs av en fattigmanssida och en rikemanssida. Rikemanssidan är den bulliga sidan av mackan och är det för att den rymmer helt kopiösa mängder smör. Då fattiga bönder och industriarbetare vänder på sitt tunnbröd för att inte slösa på smöret så öser storfräsaren\ref{storfraesare} på som om det inte fanns någon morgondag. Vill du använda rikemanssidan men saknar resurserna kan du sitta och pilla ut smör ur hålen men det har du fanimig inte tid med!

\ditem[Rikskuponger]\label{rikskuponger}
 är en smart uppfinning som används som valuta\ref{valuta} men till skillnad från vanliga pengar kan man bara köpa ett fåtal sorters varor med den, oftast pizza. Rikskupongerna får man i ett behändigt litet häfte som man kan ha i bakfickan\ref{bakficka} och bara slänga fram när man ska betala för sin Hawaii-pizza\ref{hawaii-pizza} eller vad det nu kan vara. Häftet innehåller kuponger med valörerna 2, 5, 10, 20, 40, 50, 60, 70. Den som har rikskuponger som intresse och hobby kan ta för vana att besöka Rikskupongers hemsida med jämna mellanrum för att där lära sig mer om denna populära kupong.

\ditem[Riksregalier]\label{riksregalier}
 är föremål som bärs av ett lands statschef som en symbol för makt. Generellt brukar riksregalier för det mesta ligga inlåsta i skattkammare och bara åka fram vid viktiga ceremonier. Ordet kommer från latinets \textit{regalis}, kunglig. De mest kända riksregalierna är de brittiska där man lyxat på med att trycka in världens största slipade diamant i spiran och världens näst största slipade diamant i kronan. Sverige har inte lika pråliga riksregalier, men väl två ganska tuffa i form av Gustav Vasas gamla svärd. Danmark\ref{danmark} går som vanligt mot strömmen och har bland annat gett riksregalestatus åt en gammal slokhatt som tillhört Kim Larsen, en oöppnad Tuborg\ref{tuborg} Grøn med feltryckt etikett och en slarvigt sydd dannebroge\ref{dannebrogen} som Christian Tyrann gjorde i slöjden.

\ditem[Rikssamtal]\label{rikssamtal}
 Att ringa någon som bor så långt bort att det tar minst två växeltelefonister att koppla samtalet. Desto längre bort man vill ringa, desto kortare bör man fatta sig. Dels eftersom det blir dyrare och dyrare för varje involverad station, och dels eftersom det kan finnas andra som också behöver ringa som man inte vill störa genom att blockera ledningarna. Den sparsamme använder sig istället av det stolta postverket\ref{posten} som tar lika mycket betalt oavsett vart i riket du än ämnar skicka ditt vykort. Ett vykort erbjuder också möjligheten att inkludera ett vackert lokalmotiv i fyrfärg alldeles gratis; en egenskap som inte ens telefaxen\ref{fax} kan mäta sig med. Det enda som är dyrare att ringa än rikssamtal är utomlands eftersom det även involverar en gränspolis, och till mobiltelefon eftersom det involverar strålskyddsinstitutet.

\ditem[Rimbo]\label{rimbo}
 är en tätort i Norrtälje kommun och har cirka 4600 invånader, give or take. Här finns ett äventyrsbad\ref{aeventyrsbad} som dessvärre är nerlagt efter att ha tömts på pengar av yuppien som köpte det, ett vattentorn, Britts mode\ref{britts mode}, två bensinmackar, kanske sju snabbmatsställen, nynazister så att det räcker och blir över, samt en idrottsanläggning. På skylten vid den östra infarten till tätorten har någon skämtsamt sprayat ett a av i:et i \quotetext{Rimbo} så att det står \quotetext{Rambo}.

\uline{Lista över vad Rimboborna gör en vanlig onsdagkväll}

\begin{enumerate}
\item Spelar bilbingo
\item Hyr Congo på VHS.
\item Går ut och röker
\item Går ut med hunden
\item 2, 3 och 4 samtidigt
\item 1 och 3 samtidigt.
\item Spelar i Oi!-band.
\end{enumerate}

\ditem[Rolf]\label{rolf}
 är från början ett tyskt namn och betyder \quotetext{någon som skrattandes rullar omkring på golvet.} Rold Lassgård, bland många andra svenskar, bär detta anrika förnamn.

\ditem[Roliga timmen]\label{roliga timmen}
 är den sista timmen på framförallt låg- och mellanstadielevers skolvecka. Den fungerar som ett slags smidig övergång mellan realskolans alla bördor och krav och helgens alla förnöjelser med 24-karat på Sveriges\ref{sverige} television, äventyrsbad\ref{aeventyrsbad} och glad lek med rullande tunnband. Det äts bullar. Det dricks festis. Storfräsarens ungar äter munkar och snickers och dricker coca-cola. Några spelar upp en pjäs som man repeterat i förväg. Den är inledningsvis i bästa fall diffus och snart blir skådespelarna för ivriga och nervösa och allt faller samman i skrik och spring. Men det är OK. Någon har tagit med sin minikanin och visar upp den för klassen och den tjocka killen med jobbiga hemförhållanden skrämmer den och åker ut ur klassrummet: \textit{Roliga timmen}.

\ditem[Romantik]\label{romantik}

 kan vara att sitta på en pittoresk restaurang och stirra djupt in i ögonen på den du älskar. Det kan också vara att sätta sig i skogen med en flaska sprit, börja supa den och bara titta på svampar och mossor, ståtliga tallar och burriga granar. Om man i det tillståndet väljer att måla av det man ser, mynnar det hela ofta ut i tysk mustighet\ref{den tyska mustigheten}.

\ditem[Roy Andersson-väder]\label{roy andersson-vaeder}
 är en i meteorologiska kretsar ofta använt begrepp som syftar på sådant trist, grådaskigt, mulet väder som får allt att se smutsigt och fult ut (inklusive människor).

\ditem[Rugga]\label{rugga}
 En rugga, på norrtäljeslang, är något som är mycket generöst tilltaget i proportion eller motsvarande. I andra delar av landet används ordet \textit{rackabajsare} för att tala om samma sak. Exempel på ruggor är Tolstojs \textit{Krig och Fred} och Gerard Depardieus näsa.

\ditem[Rulla hatt]\label{rulla hatt}
 betyder samma sak som att gå på lokal, det vill säga att gå ut i senkvällen och dricka alkohol. För att man ska kunna sägas rulla hatt måste man dricka en viss mängd - man kan inte bara ta en öl och sedan ursäkta sig med att man har tvättid. Uttrycket kommer sig av att det det förr var vanligt att överförfriskade herrar tog av sig hatten och som ett litet spratt rullade den nedför gatan.

\ditem[Ryckepungvägen]\label{ryckepungvaegen}
 är en gata i Falun där Mekonomen ligger.

\ditem[Rygg]\label{rygg}
 Ryggen är den del som finns bakpå kroppen, ovanför stjärten, men under halsen. Ryggen har i den västerländska kulturhistorien åtnjutit en aura av mystik, vilken tros ha uppkommit genom det faktum att ryggen endast med stor möda kan ses av den som äger den. Alla människor och djur har en rygg, men på vissa djur, till exempel maneter\ref{manet}, är det svårt att avgöra exakt var på djurets kropp ryggen finns. På ryggen placerar man de fyrkantiga tygstycken som det står \quotetext{Saxon} på och det är också där man tatuerar in omslaget till den Leonard Cohen-skiva där denna legendariske artist fångats på bild mosandes in banan in i munnen\ref{mun}.

\ditem[Rygga]\label{rygga}
 Att rygga något är att snatta genom att placera varan, t.ex. ett paket kaffe eller en påse Palle Kuling\ref{palle kuling}, mellan ryggsäcken och ryggen och puta ut med stjärten för att hålla den tillskansade varan på plats.

\ditem[Ryggtryck]\label{ryggtryck}
 Sverige har Carl von Linné\ref{carl von linné}, Norge har sin olja, Danmark\ref{danmark} har avsagt sig alla sociala normer samt har lego\ref{lego} - och Finland\ref{finland} har ryggtryck. Ryggtryck är en modeterm som benämner bilder som föreställer antingen exotiska djur, turnéscheman eller en klatchig anti-social slogan som tryckts på ryggen av en t-shirt eller en munkis. Denna oväntade vändning i modehistorien gör det möjligt för en finländare att signalera för en person som står framför henom att hen lyssnar på thrash metal samtidigt som alla som råkar passera bakom uppmanas att fara dit pepparn växer. Finlands framstående position i rockens universum har gjort att ryggtrycket spritt sig till andra delar av den industrialiserade världen, men utanför Finland har det inte riktigt tagit sig in i finrummen, än.

\ditem[Räkmacka]\label{raekmacka}
 är en typ av underlag med väldigt bra glid. Dess glatta yta har visat sig ytterst lämpligt som smörjmedel vid de flesta tillfällen, men används främst av bekväma personer.

\ditem[Räksallad]\label{raeksallad}
 var ursprungligen ett postmodernt provokationsprojekt utfört av den franske tankeakrobaten Jean-François Lyotard. Lyotard inkluderade i sitt magnum opus \textit{Det postmoderna tillståndet} ett recept på räksallad. Fransosen menade däri att tre deciliter majonnäs med lite skaldjursskulor i, hade lika mycket rätt kallas för en sallad som den mer traditionella grekiska eller caesarianska modellen. Detta var en uppenbar provokation, riktad mot det totalitära tolkningsföreträde som dittills rått i de gastronomiska finrummen. Till mångas förvåning (säkerligen också Lyotards) blev snart räksalladen en folkkär favorit, särskilt på Sveriges västkust, med viss spridning även i norra Småland och Värmland. Till dags dato har räksalladen krävt lika många liv som Jonestownmassakern, Francesco Schettino och Zodiacmördaren sammanslaget.

\ditem[Räksmörgås]\label{raeksmoorgaas}
 är en typisk rätt som serveras fredagsmysande vuxna och består av en macka med räkor på. Räksmörgåsen blev populär i Sverige\ref{sverige} på sjuttiotalet och har egentligen inte försvunnit från topplistan över smutta saker att äta framför På Spåret eller Melodifestivalen sedan dess, trots hård konkurrens från bland annat fondue-grytan, popcornmaskinen och ostbrickan. Ja, och så innehåller ordet \textit{räksmörgås} - till skillnad från kusinen\ref{kusin}, räkmacka\ref{raekmacka} - de svenska diakritiska bokstäverna \textit{å, ä} och \textit{ö}.

\ditem[Råååål]\label{raaaaaaaal}
 är den ål som har flest \quotetext{å} i namnet. Det är den självklart jävligt kaxig över. Den lever i Rååån som rinner genom Råå.

\ditem[Rögad ål]\label{roogad aal}
En maträtt som stammar från vår södra landsända och kännetecknas av att den alltid, oavsett vem som yttrar dess namn, uttalas på grötig, mörk Asmundtorps-skånska. Om du vill göra din egen rögade ål så bör du först hitta en ål. Förr i världen slängde folk ner ålar i brunnar för att hålla brunnen ren från mask så där kan du slänga ner en krok och se vad du får upp. Har du ingen brunn i ditt närområde så kan du prova i första bästa surhål, ju äckligare vatten, desto bättre. När du fångat din ormliknande fisk knäcker du nacken på den och lägger ett slarvigt snitt längs den alldeles för långa buken. Gräv ur allt som inte verkar gå att äta och ge det till katten. Ta nu din ål och lägg den i ett nät som du fäster ovanför köksbordet och sätt dig och kedjerök under. Tobaken som gäller är Hamiltons blandning, men skulle du vara för töntig för pipa går det bra med röd Marlboro eller gula Blend. Efter ett par dagar kan du avnjuta denna maträtt som är så fet att du kommer att kräkas, men så god att du kommer backa om.

\ditem[Röksignaler]\label{rooksignaler}
 är en interaktiv teknik för spridande och inhämtande av information över stora avstånd. Tekniken bygger på samma princip som transistorn (ström/inte ström, rökpuff/inte rökpuff) och uppfanns av en gammal indian, Hövding Gamnacke, som tröttnat på att sitta\ref{sitta} och svettas i sin malätna poncho. Istället bytte han om till höftskynke och slängde den gamla trasan på lägerelden. En annan indian, Lilla Coyotedräparn', ryckte dock ut och räddade plagget eftersom han insåg hur festligt och bizarrt det skulle kännas att vara naken under den. Båda insåg genast vilket revolutionerande fenomen de snubblat över, så dom rökte en fredspipa och resten är historia.

\ditem[Rökå]\label{rookaa}
 är en ort i Malå\ref{malaa} kommun. Byns slogan är \quotetext{Hä ska du vetta och hä väit du} som på rikssvenska betyder \quotetext{Det ska du veta och det vet du}. Rökåbygden såg till att hålla Umedalens mentalsjukhus igång. I Rökå bor 50 personer uppdelade på tre efternamn, Hedström, Bjuhr och Hultmar. Här bor också det äkta paret Atmar och Lucia.

\ditem[Rött]\label{roott}
 är ett etablissemang i stadsdelen Berghem i Umeå. Här säljs olika alkoholhaltiga drycker till den törstige \textit{flanûren} och enklare mat till den hungrige. På grund av krogens läge invid Umeå Universitets campus går affärerna relativt bra eftersom postseminarium\ref{postseminarium} och arbetsrelaterade middagar ofta går av stapeln just på Rött och på grund av den utbredda alkoholism som många akademiker har utvecklat som ett sätt att hantera år av förödmjukande nederlag, aggressiva studenter och föraktfulla medarbetare. Man serverar också gladeligen fotbollsspelare som tagit ut sig på någon av de närbelägna fotbollsplanerna\ref{fotboll}.

\ditem[Rövgitarr]\label{roovgitarr}
 En rövgitarr är ett strängförsett musikinstrument. Den uppfanns i Danmark\ref{danmark} och skiljer sig mot andra gitarrer i det att den \textit{alltid} låter rent förjävligt (på danska: røvigt.), därav namnet. Att spela rövgitarr är inte särskilt svårt, det räcker i princip att slå på strängarna med vad som helst för det låter ändå lika dant. Länge trodde konspirationsteoretiker att Lemmy spelade på en kamouflerad rövgitarr. Men det visade sig att det var en helt vanlig Lemmy-bas\ref{lemmy-bas}. Den kändaste brukaren av rövgitarr är istället Kim Larsen, som drar ett långt solo på singelversionen av sin hit \textit{ Væd gør vi nü lillæ dü?}. Danskättlingen, Machiavelli-inkarnationen tillika Metallicatrummisen, Lærs Uølrich försökte rida på hajpen genom att införa rövtrumsettet. Men det är han fortfarande ganska ensam om.

\ditem[Rød pølse]\label{roood pooolse}
 är korv som görs på överblivna delar från andra maträtter inom det danska köket. Alla rester (exempelvis överdelen på Ballerinakex, sallad, fläsksvålar med kort datum) samlas först i en kompostliknande behållare som har ett grovmaskigt nät över sig för att förhindra allt för stora objekt såsom pilsnerflaskor och klövar följer med. Därefter lyfts nätet bort och man går loss på det som trillat igenom med en högaffel eller golfklubba för att få det mera finkornigt. Därefter späds massan med tapetklister och tappas upp i formar innan den hinner stelna. Den karaktäristiska röda färgen uppstår i slutfasen av framställningen genom att korvarna kokas i falu rödfärg.


%%%%%%%%%%%%%%
\newpage
\null
\\
\null
\\
\Huge
S
\normalsize
\\
\null
\\
\null
%%%%%%%%%%%%%%


\ditem[Saltbas]\label{saltbas}
 är en speciell teknik att spela bas där man typ duttar och slår på strängarna i stället för att spela som vanligt. Namnet kommer sig av att det lättaste sättet att få rätt feeling när man spelar är att tänka sig att basen är en stor hummer som man måste salta ordentligt innan förtäring. Med detta i minnet är det bara att låta fingrarna börja studsa runt över instrumentkroppen. Källa: Överhört samtal på ett dassigt ölhak.

\ditem[Saltsjöbadsavtalet]\label{saltsjoobadsavtalet}
 tecknades i överklassområdet Saltsjöbaden i Stockholm mellan facket och kapitalisterna. Kortfattat gick det ut på att facket sa: \quotetext{Vi lovar att inte strejka och ställa till med jävelskap\ref{jaevelskap} för er.} Kapitalisterna sa \quotetext{tack} med munnen full av oxfilé och vaktelpaté.

\ditem[Samtida nordisk undergroundmusik]\label{samtida nordisk undergroundmusik}
 I de nordiska länderna, till skillnad från Frankrike och Belgien\ref{belgien} till exempel, skapas förhållandevis mycket undergroundmusik. De olika länderna kom efter unionsupplösningen mellan Sverige\ref{sverige} och Norge den 26e oktober 1905 överens om att fokusera på olika genrer. Nedan följer en utförlig lista på de olika ländernas inriktningar.

\uline{Sverige}

Som gammalt storvälde är Sverige även när det gäller samtida nordisk undergroundmusik liksom spindeln i nätet. Här finns de flesta genrer representerade. Trollpunk\ref{trollpunk} och NYHC från föredetta bruksorter är exempel på livskraftiga genrer.

\uline{Danmark}

Som alla begriper har man en ganska avslappnad inställning till undergroundmusik i Danmark\ref{danmark}. Det mesta som ser dagens ljus gör det genom pilsnergubben Lars Kroghs skivbolag Bad Afro och oftast rör det sig om nån slags THC-inspirerad gladporrmusik. Ett exempel på detta är Baby Woodrose.

\uline{Norge}

I Norge är det lajvande morsgrisar som står för musikskapandet, till 99\% i form av \quotetext{rå} black metal. Man går på fjället. Man skaldar om getter. Man har såna där kängor med en plåt på framsidan.

\uline{Finland}

I Finland\ref{finland} är det finsk pappersbruksarbetarkraut\ref{finsk pappersbruksarbetarkraut}, tango och kängpunk som gäller för hela slanten.

\uline{Island}

Islänningar i stickad tröja som plingar i små klockor och sjunger i falsett. Denna genre har ännu inte fått något namn eftersom det är olagligt i övriga språkområden att låna in isländska ord.

\ditem[Samurajernas hederskodex]\label{samurajernas hederskodex}
 (även kallat \textit{Bushidō}, \begin{CJK}{UTF8}{min}武士道\end{CJK}; \quotetext{krigarens väg}) är det moraliska rättesnöre som styr en samurajs levnad. Den vilar framförallt på sju principer: 
\begin{itemize}
\item Rättrådighet. Basera dina beslut på sanningen. Varför köpa Tuborg\ref{tuborg} när du kan få en Sofiero med samma alkoholhalt två kronor billigare?
\item Mod. Ducka inte för det obekväma. Ett bredställ i kurvan känns lika fint i kroppen oavsett om länsman råkar ligga i backspegeln\ref{inre backspegel}.
\item Universell kärlek. Alltings rätt att existera. Skrota inte den gamla pärlan\ref{volvo 740} utan låt den vila i frid på gården.
\item Respekt att göra det rätta. Visa artighet och vördnad. Acceptera aldrig den nya sångaren\ref{den nya saangaren}.
\item Uppriktighet. Lögnen brukas enbart av den fege. Ring till P1 varje morgon och låt världen veta vad du tycker om grannens trädgårdstomtar.
\item Heder. Visa att du förtjänar din respekt. Blir du omkörd så hytta med näven\ref{hytta med naeven}.
\item Lojalitet och hängivenhet. Överge ingen i svåra tider. Köp alla skivor med Discharge, även de sugiga från 90-talet.
\end{itemize}

\ditem[Sand]\label{sand}
 är ett grundämne med beteckningen Tb i det periodiska systemet (atomnummer: 86, grupp: alkaliska jordartsmetaller). Sand återfinns ytligt på alla kontinenter i mitten av de tektoniska plattorna utom Antarktis. Tidigare fanns stora mängder sand även där men den förbrukades för ungefär 100.000 år sedan då stora populationer av polarkatter drog runt och använde den i sina kattlådor. I Sahara finns det mest sand. Faktiskt lika mycket som det finns smör i Småland och te\ref{te} i Kina. Förutom att vara huvudråvara i de flesta former av kattsand är sand även ett populärt material för byggande av slott och ökenslätter\ref{slaett}. En ordentlig stig\ref{stig} består normalt också av en stor del sand. I populärkulturen är sand ingen jättevanlig referens även om undantag finns, till exempel Metallicas \textit{ Enter sandman} och Peter Jöbacks \textit{Guldet blev till sand}. Dessutom finns det så klart ingen actionrulle som är riktigt komplett utan minst en scen med kvicksand.

\ditem[Sandpapper]\label{sandpapper}
 är ett ovärderligt hjälpmedel för varje snickare. Det har nämligen den obetalbara egenskapen att det får saker att se noggranna ut. Innan sandpapperet uppfanns såg allting ut som skit, ungefär som att ett dagisbarn hade slängt ihop det. Titta på en gammal trärelief från medeltiden\ref{medeltiden} till exempel. Den är full som fan med en massa stickor och utbuktningar. Hade man haft sandpapper att tillgå hade det fått en len och jämn yta som till och med kungen skulle kunna tänka sig att äta på. Som allt annat som har med noggrannhet att göra är sandpapper astråkigt att använda. Det finns typ hundra sorter med olika grovhet så när du är klar med det första måste du byta till nästa, och sen nästa. Den kvicktänkte\ref{kvicktaenkt} med bättre saker för sig tar därför fram burken med klarlack direkt och penslar ordentligt över hela skiten. Resultatet blir en tjock och glänsande yta som nästan påminner om bärnsten. Kanske inte lika snyggt som om du använt sandpapper men good enough. Lycka till!

\ditem[Sanningssägande bloggar]\label{sanningssaegande bloggar}
 2000-talets fortsättning på arga insändare i lokalpressen. Präglas ofta av överdrivet användande av versaler, grava syftningsfel och förvirrade resonemang. Trots detta läser påfallande många fullt friska detta elände och låter sina sinnen förgiftas.

\uline{Några handfasta exempel}

\begin{itemize}
\item När Annika\ref{annika} upptäcker att hon inte kommer kunna vara hemma tills barnen börjar skolan på statens bekostnad startar hon en sanningssägande blogg.
\item När Sture misstänker att Hizbollah tagit över kiosken för att snubben i kassan odlat ett värmande vinterskägg, och Sture som en konsekvens av detta inte vågar lämna in tipset startar han en sanningssägande blogg.
\item När marginaliserade kvinnliga akademiker opponerar sig mot att könsroller tvingas på små barn på dagis startar Göran\ref{gooran}, själv förvånansvärt nog barnlös, en sanningssägande blogg.
\end{itemize}

\ditem[Sans pants]\label{sans pants}
 \textit{Sans pants} (fr. Fr. \textit{sans} ung. \quotetext{utan} och Eng. \textit{pants} ung. \quotetext{byxor}) är en lite finare term för att vara utan byxor. I en exempelmening kan den användas som följer: (Statusuppdatering på facebook) \quotetext{Sitter framför datorn och kollar på gamla foton, \textit{sans pants}.}

\ditem[Sarin]\label{sarin}
 är en dödlig nervgas och mycket populärt kemiskt stridsmedel bland fattiga länder. Medan välmående industrinationer ofta utvecklar olika typer av prestigefulla kärnvapen är fattiga länder ofta hänvisade till denna betydligt billigare variant av massförstörelsevapen när man vill skapa oreda. Fast det är fortfarande dom rika länderna som framställer gasen, för de måste ju tjäna pengar på de fattiga på nåt sätt. Basen till sarin kan utvinnas ur en billig form av klorlösning som används till att bleka andrahandssorterad pappersmassa för kiosklitteratur. Detta innebar länge att Bonnier-koncernen i princip hade monopol på framställning av gasen. Detta luckrades dock upp i och med ett EU-direktiv i kölvattnet av den så kallade \quotetext{Uti våg hage-affären} där det framkom att Bonnier sålt stora mängder jultidningar till Israel. 

\ditem[Semikolon]\label{semikolon}
 Ortografiskt skiljetecken uppfunnet av den emancipatoriske lingvisten Carl von Linné för att rättfärdiga tveksamma bisatser.

\ditem[Sengekantsfilm]\label{sengekantsfilm}
 är en filmserie i åtta delar, producerade i kungadömet Danmark\ref{danmark}. Vad som kännetecknar filmserien är också vad som kännetecknar Danmark - en eklektisk mix av \quotetext{lystspil, folkekomedie og porno\ref{poorr}}. Alla åtta filmer producerades mellan 1970 och 1976 och ses idag som otroligt inflytelserika inom genren \textit{filmer som innehåller riktigt sex}. Vad som skiljer detta från mer klassisk pörr\ref{poorr} är att det sällan rör sig om råbarkat knullande utan mer folkligt bussex, inramat av konstnärlig ambition. Idag kan man köpa både de ocensurerade danska versionerna och den censurerade svenska versionen, där hardcore-sexet klippts bort ur de två filmer som innehåll sådant (observera att censuren kännetecknar Sverige\ref{sverige} på samma sätt som blandingen mellan buskis och porr kännetecknar Danmark).

\ditem[Senilsnöre]\label{senilsnoore}
 Ett senilsnöre är en uppfinning som används för att förhindra att senildementa militärhistoriker tappar bort sina glasögon. Senilsnöret består av en bit snöre eller ett band med öglor i ändarna. Dessa fästs i glasögonens skalmar. Snöret läggs sedan mot den senildemente militärhistorikerns nacke och glasögonen placeras som vanligt på näsryggen. Nu kan den senildemente militärhistorikern lugnt skjuta upp glasögonen på pannan eller låta dem dingla framför bringan utan att glasögonen spårlöst försvinner (de ligger vanligtvis någonstans på skriv- eller nattduksbordet). Detta fantastiska implement ska ha uppfunnits av Jan Guillou, enligt honom själv, och finns nu på marknaden i alla upptänkliga färger och utföranden, så att den senildemente militärhistorikern kan känna att just hans (för det rör sig om en man, tro mig) exemplar reflekterar hans personlighet. Mest populär och därför svårast att hitta är Synoptiks klassiker \quotetext{Operation Torch, 8 November 1942}
i sandfärgad nylon. 

\ditem[Sexa]\label{sexa}
 En sexa är det högsta man kan slå med den vanligaste typen av tärning.

\ditem[Sextant]\label{sextant}
 En sextant är ett redskap för att bestämma vinklar. Troligtvis är det en brittisk uppfinning eftersom dom gillar att göra saker onödigt krångliga och att anspela på misogyna föreställningar. Varför inte bara nöja sig med ett lättbegripligt och neutralt vattenpass liksom?

\ditem[Shizo Kanaguri]\label{shizo kanaguri}
 (1891-1984) var en japansk\ref{japan} geografilärare som gått till historien som tidernas sämsta maratonlöpare. Hans första lopp var sommar-OS i Stockholm\ref{stockholm} 1912 och hur fan han lyckades kvalificera sig dit är egentligen en större gåta än hans sluttid. Kanaguri startade hur som helst loppet men fick ungefär halvvägs så förjävla tråkigt att han stannade till hos en familj som satt i trädgården och fikade. Alla som någon gång sprungit vet förmodligen precis hur uttråkad han måste ha känt sig, och på den tiden hade man inte ens uppfunnit hörlurar så han kunde inte ta med sin mp3-spelare fullmatad med Gism och Melt Banana. Den enda portabla musik som fanns att tillgå då var att låta någon springa bredvid med en plåttratt och skrika Jussi Björlinglåtar. Nåväl, åter till trädgården där Kanaguri bjöd in sig själv och njutningsfullt klämde i sig några bullar och ett glas saft. \quotetext{-Fan så mycket roligare än att springa}, tänkte han antagligen (OBS. källa saknas). När han fikat klart hade alla andra hunnit så långt före att Kanaguri tyckte det kändes ovärt att springa klart så han drog hem till Japan istället. Men arrangörerna, som alla var tjurskalligt nitiska Socialdemokrater, vägrade frångå sin livsfilosofi om att rätt ska vara rätt och lät klockan gå eftersom ingen anmälan om avbrutet lopp inkommit. 1962 lyckades man lokalisera Kanaguri och informera honom om situationen. Det visade sig att han fortfarande levde sitt liv \textit{mañana} men efter ytterligare några år pallrade han sig faktiskt åter till Stockholm och gick i mål. Kanaguris tid blev 54 år, åtta månader, sex dagar, åtta timmar, 32 minuter och 20,3 sekunder. Portugisen Francisco Lazaro som också deltog i loppet har fortfarande en tjänstemannateoretisk\ref{tjaenstemannateoretisk} möjlighet att få en sämre tid då han ännu inte gått i mål eller anmält att han brutit. Han dog dock av uttorkning efter att ha sprungit mindre än två mil så alla rationella förståsigpåare\ref{foorstaasigpaaare} är överens om att han aldrig kommer kunna hota Kanaguri. 

\ditem[Shockrockare]\label{shockrockare}
 Artist som gör det oväntade och lite skrämmande för sin publik. Typiska saker shockrockare sjunger om är läskiga varelser och otäcka fenomen. Alice Cooper är den mest kända shockrockaren och han sjunger till exempel om gift i låten \textit{Poison} och om monster i låten \textit{Feed my Frankenstein}. Ganska läskigt, eller hur? Det krävs dock mer än att bara sjunga om läskiga saker för att man ska bli en riktig shockrockare; man måste kunna förmedla en shockande stämning också. Dia Psalma sjunger till exempel om näcken och djupa skogar men är bara helt vanlig trollpunk\ref{trollpunk} - inte särskilt shockerande. Sveriges första shockrockare var bandyspelaren Gösta \quotetext{Snoddas} Nordgren\ref{goosta snoddas nordgren}, som spred kaos i folkparkerna med sin syndiga och djävulsdyrkande refräng \quotetext{haderian hadera} i låten \textit{Flottarkärlek}.

\ditem[Sidensvans]\label{sidensvans}
 är ett djur som tillhör familjen fåglar i naturens prunkande\ref{prunka} släktträd. Sidensvansen är stor som en 33cl. starköl ungefär och har ett slags näbbmun\ref{naebbmun} som kan jämföras i storlek med en cigarettfimp, som den använder för att äta och skrika med.

\ditem[Sign of the hammer]\label{sign of the hammer}
 den stiliserade Torshammaren, Manowars\ref{manowar} fjärde skiva från 1984. Efter samarbetet med Orson Wells på skivan Battle hymns återkom bandet storstilat, siktandes på nordens olymp: Valhalla. Skivans första spår \quotetext{All men play on ten} är åskgudens hammare rakt i ansiktet på föräldragenerationen: \quotetext{Be like us and get a sound that’s real thin. Wear a polyester suit, act happy look cute. Get a haircut and buy small gear.} Verserna och refrängens brygga lever upp till textens löften. Refrängen är dock ett maffigt coitus interruptus. Låten bjuder ändå på några sköna glidande falsetter som i mångt å mycket har blivit Eric Adams credo. Som åskgudens vigg dyker andra spåret \quotetext{Animals} ner på förfesten. Här laddas det för könsligt umgänge. Klassiskt tuggande guitarr-riff leder oss till klimax - \quotetext{I'm gonna give all you can take all night. And leave you in the morning feeling right}. Låten saknar komplex dynamik, vilket antagligen är gott. Suget i magen som i så fall skulle uppstått vid tredje refrängen hade antagligen bildat ett tarmfientligt vakuum. Ross the Boss’ guitarrsolo är oklanderligt. Skivans tredje spår \quotetext{Thor (The Powerhead)} är antagligen skivans bästa. Nu inser man att Wagner var den förste amerikanen. Liksom den tyske själsfränden bygger Manowar sitt mythos på en olycklig sammanblandning mellan Ragnarök och Götterdämmerung. Låten startar på Opelns fjärde växel. Varje falsett är väl avvägd: Odin, High osv. Låtens guitarrsolo lämnar en del i övrigt att önska. Ross the boss gör sig bättre när han excellerar i attityd, inte i teknisk snabbhet. Text och musik är så väl sammanflätade, musiken är lika grandios som lyriken - precis lika fantastisk. Härnäst slungas vi med på en hisnande färd. Tematiskt ordnad, från Anschluss till det totala nederlaget. I \quotetext{Mountains} luras vi in, bergtagna så att säga, på färden. Som sirener lockar Joey deMaios basguitarr-intro. Det rustas för det totala kriget. Som Operation Barbarossa stormar Manowars heavy metal över trumhinnornas bördiga jordar. Titelspåret \quotetext{Sign of the Hammer} lämnar ingen oberörd.

\textit{Onward pounding Into Glory Ride}
\textit{Sign of the Hammer be my guide}
\textit{Final warning all stand aside}
\textit{Sign of the Hammer it's my time}

Fortsatt mäktigt tills bandet når sitt hjärtas Stalingrad: \quotetext{The Oath}. Bergfast står de i den amerikanska heavy metalens Festung Europa. Någonstans vet man ändå att det är förlorat.

\textit{only courage and heroism linger after death}
\textit{So, hold fast thy sword, rejecting pain, feel the dragons breath}
\textit{- I've sworn the oath}

Med låten \quotetext{Thunderpick} tolkas bombingen av Dresden på basguitarr. Avslutningsvis - undergången, Jim Jones Nero-order i ett sydamerikanskt träsk. \quotetext{Guyana (and the cult of the damned}. Ovanligt friskt vågat val av tema för Manowar. Det är annars sällan de kommer längre i Amerikanan än Errol Flynn. Mondo Bizarro. Det är en besynnerlig värld.

\ditem[Sigvard Thurneman]\label{sigvard thurneman}
 Arkivassistent och biträde i herrekipering i Sala. Ett genuint intresse för hippieflum ledde Thurneman till en del tveksamheter. Det var en del mord, rån och så. Dessutom tallade han på kompisar efter att ha hypnotiserat dem. I Västmanland (bortsett Västerås) ses Thurneman idag som en frihetskämpe av samma grad som Engelbrekt Engelbrektsson.

\ditem[Simhud]\label{simhud}
 är en tunt lager skinn mellan fingrar och/eller tår. Dess praktiska användning varierar, men de flesta djur använder simhuden till att simma med medan exempelvis fladdermusen istället använder den till att flyga. Andra djur som har simhud är vattensvin (\textit{Hydrochaeris hydrochaeris}), myskanka (\textit{Cairina moschata}) och italiens långbensgroda (\textit{Rana latastei}).

\ditem[Singelsnurra]\label{singelsnurra}
 är ett städredskap som vid användning kan liknas vid att köra tryckluftsborr och rida på en vildhäst\ref{haest} - samtidigt. De två aktiviteterna gör ofta att dess utövare framställs som sexig och vild, men singelsnurran har ungefär lika mycket sex appeal som en säck ruttna mandariner. Att bemästra singelsnurran är ett inträdesprov för städare i hela Sverige\ref{sverige}.

\ditem[Sinkadus]\label{sinkadus}
 är en klassisk kötträtt som hör juletiden till. Rätten består av älgkött som saltats och pepprats innan det steks i grädde. \quotetext{Det låter ju inte vidare krångligt att laga till}, tänker ni säkert. Teoretiskt stämmer detta, men sedan den globala kolesterolkonspirationen tagit kontroll över världens medier är det bara åldrade Allersläsare som klarar att hälla på tillräckligt med grädde i stekpannan utan att få arga blickar av sin hjärntvättade omgivning. Namnet kommer från franskans \quotetext{femma och tvåa} (cinq et deux), vilket egentligen kanske är det största mysteriet med det hela. 

\ditem[Sitta]\label{sitta}
 Att sitta är att inta en populär ställning som passar till en stor rad olika aktiviteter, så som att skriva, surfa på World Wide Web\ref{world wide web}, se på TV, äta, köra flisbil\ref{flisbil} samt att spela brädspel. Den mest vanliga och populära sitsen intags på följande vis:

\uline{Grundsitsen}

Steg 1: Finn ett lämpligt föremål att sitta på. En tumregel är att föremålet bör ha en platt horisontell yta som inte orsakar smärta eller smutsar ner dina eventuella kläder. Ytan bör inte vara högre eller lägre än ditt knä, men hittar man ingen sådan yta får man improvisera. Här fungerar naturligtvis sådana fär ändamålet framtagna föremål som pallar, stolar och fåtöljer bäst, men mer rudimentära objekt så som stubbar och stenar kan fungera nästan lika bra.
Steg 2: Ställ dig med ryggen mot föremålet. Kontrollera att dina vader har kontakt med föremålet.
Steg 3: Böj sakta dina knän så att din bak närmar sig ytan och slutligen får kontakt med den.
Steg 4: Fördela nu kroppsvikten så att den koncentreras på den del av dig som befinner sig på föremålet du utsett.
Steg 5: \textit{Voila!} - du sitter.

Öva dessa steg tills du behärskar dem, så ska du se att du snart inte alls har några svårigheter att sätta dig ned. Prova gärna att variera underlag och hastighet. När du behärskar denna teknik kan du prova följande varianter:

\uline{Benen i kors}

Utför ovanstående steg, men när du befinner dig i sittande läge lägger du det ena benet över det andra. Detta sätt att sitta är populärt bland kvinnor som bär kjol. Dessa individer brukar vanligtvis lägga det ena låret över det andra, medan män i cowboyboots gärna lägger det ena benets vrist över det andra benets knä.

\uline{Bredbent}

Åter igen, upprepa ovanstående steg. När du befinner dig sittande viker du ut benen åt varsitt håll. Detta sätt att sitta är vanligt bland män som har problem med sin sexualitet och hos manhaftiga kvinnor.

\uline{Framåtlutad}

Upprepa ännu en gång ovanstående steg. Luta dig sedan framåt och vila armbågarna på knäna. Denna sits är vanlig hos avbytare i lagsporter som hockey\ref{hockey} och fotboll\ref{fotboll}.

\ditem[Sittsova]\label{sittsova}
 Sittsömn är för sovare vad struphuvudskrossen är för sithlords, ett bevis på att man är jävligt bra på det man gör. Istället för att på konventionellt sätt ligga ner för att sova, kan sittsovaren i upprätt läge, rak i ryggen ladda batterierna och göra sig redo för en andra andning. På bussen till Vännäs, på bilsemestern till Iggesund, på efterfest hos sitt ex sen tre år tillbaka och på färjan Dover - Calais kan sittsovaren tänja på det rimligas gräns och fullfölja sitt mystiska värv. Enligt hinduerna är sittsömnen (i dessas religion refererad till som yoga) en väg till Nirvana, frihet från lidande.

\ditem[Siv-Berit]\label{siv-berit}
 är ett gammalt svenskt kvinnonamn. Siv betyder, som alla vet, sil på engelska. Berit är en reduktion av orden \quotetext{ber om det} (\textit{ber om det} \textbf{ -> } \textit{ber om \'et} \textbf{ -> } \textit{berom\'t} \textbf{ -> } \textit{berit}). Siv-Berit betyder således \quotetext{den som vill ha knark}. Den maskulina formen av Siv-Berit är Syd Barrett.

\ditem[Sjua]\label{sjua}
 En sjua, även kallad \textit{7\"}, är discofilens beteckning för en vinylsingel. Det är också en populär siffra. 

\ditem[Sjundedagsadventistisk skola]\label{sjundedagsadventistisk skola}
 På den sjundedagsadventistiska skolan får eleverna lära sig att förhålla sig till vuxenvärldens regler. Här är det rektor, tillförordnad av den helige ande, som sätter agendan och inte popmusik och ungdomsfilm och datorspel och smart drugs och allt vad det heter nu för tin. Här får eleverna lära sig att klä sig som små tanter och farbröder redan i tonåren och därmed kan man redan där rationalisera bort alla livets faser mellan oskuldsfull barndom och fruktlös pensionsålder. Puberteten är en styggelse som man i nuläget tvingas ha överseende med, och man har lärt sig rutiner för hur man bäst skapar förvirring och osäkerhet kring kropp och samliv och på så vis kraftigt reducerat riskerna för att eleven lägger sig till med en fallenhet åt hedonistiskt leverne. Alla föräldrar kan alltså vara lugna. Här är flickor flickor och pojkar är pojkar!

\uline{Glädjes åt skapelsen!}

Lärarlaget består av ariska män med mustasch och kvinnor med långkjol som har det gemensamt att de icke tvivla på herrens ord och att de kräva militärisk disciplin från sina lärjungar. Men även dessa personer har väl varit unga en gång, och ibland - inte för ofta, men ibland - går man ungdomen halvvägs till mötes. Då tar man fram gitarren\ref{gitarr} och går ut till en solig plats och leder hela klassen i allsång, bara för att det är viktigt att glädjas ibland också och inte bara plugga tyska böjningsformer. Och visst är det lätt att se all det där vackra när en tjugohövdad kör låter Ted Gärdestads \quotetext{Sol, Vind och Vatten} eka mellan kapellet och gymnastiksalen, där en grupp flickor tränar uppvisningsgymnastik med fladdrande band och små färglada bollar.

\ditem[Sjungande trummis]\label{sjungande trummis}
 Sjungande trummisar är nästan lika ovanligt som trummisar som skriver sina egna låtar, det vill säga väldigt ovanligt. Det finns två trummisar som sjunger som är värda att nämna och det är Phil Collins i Genesis som sen gick solo, och Reemu Altonen i Hurrgianes som märkligt nog lever fortfarande. Den ene är alltså ett brittiskt pretto som gjort en bra låt och den andre en alkoholiserad och kriminell finländare som stal sitt första trumset och skapade en ganska solid backkatalog. Satanic Surfers hade också sjungande trummis men av hänsyn till bandets medlemmar nämns hen här inte med namn. På scenen kommer det inte dröja länge innan all den mystik och cred som bandet jobbat upp rämnar likt SJs tidtabeller när den sjungande trummisen tar till orda. Publiken skruvar på sig generat och undrar om det är playback eller vad fan det är frågan om. Till slut skymtar de svaret mellan bastrummor och  pukor - en besserwisser som inte räds att visa upp att hen minsann kan traktera både rytm och melodi samtidigt. Scenkanten gapar tom och spotlightsen letar febrilt efter något intressant att fokusera på. Detta rör inte den sjungande trummisen i ryggen, då hen inte förstått att musik upplevs lika mycket med ögon som öron.

\ditem[Självförtroendeplagg]\label{sjaelvfoortroendeplagg}
 är ett plagg man bara kan ha när man känner sig riktigt ball, tell exempel en t-shirt som är snygg som fan, men på gränsen till för liten. När man vaknar på morgonen och känner sig som Steve McQueen, då kan man ha den. De morgnar man vaknar och känner sig som Sven-Otto Littorin är det helt omöjligt. Eller kanske ett par jeans som man inte riktigt vet om de är snygga eller ej egentligen, men ibland bara förstår att man kan bära upp, precis som Wendy O Williams skulle göra.

\uline{Falskt situationsbundet självförtroende}

Vid vissa situationer kan en person få för sig att den kan bära upp ett plagg på grund av situationen den befinner sig i. Ett exempel på detta är när en person är full. Då är det lätt att gå bananas\ref{bananas} och dra på sig en skinnpaj med fransar på ärmarna, eller skoja upp en chapeau de paysan\ref{chapeau de paysan} på hjässan och tro att man kommer undan med det. Det gör man sällan, vilket gör skammen än större när man vaknar dagen efter och inser vilken \textit{schmuck} man var. Ett annat klassiskt tillfälle då allt omdöme fallerar är utlandssemester, särskilt till sydostasien. På plats kan det vara ok att ha på sig flip flops, snäckhalsband - kanske till och med kläder av märket billabong. Men när personen kommer hem till det västerländska, senmoderna samhället och försöker dra runt på stan med samma billabongshorts som var helt ok i Phuket, kommer den oundvikligen att drabbas av enormt stigma\ref{stigma}. Det inneboende självförtroende som kommer med en utlandssemester visar sig vara en chimär. Väl tillbaka i Uppsala\ref{uppsala} är man inte lika cool som Heath Ledger i Point Break, utan bara ännu en tönt som ska falla in i det chinosbeiga ledet.

\ditem[Självmordsspåret]\label{sjaelvmordsspaaret}
 är förmodligen den teori privatspanare\ref{privatspanare} ägnat minst tid åt kring mordet på Olof Palme\ref{olof palme}. Enligt förespråkare för detta spår var mördaren ingen minder är Palme själv, i ett försök att lansera sig som den nye Che Guevara. Palme var omvittnat förtjust i standar och flaggor med politiska ledare och när Ches porträtt började spridas på T-shirts och nyckelringar blev han grön av avund. Spåret belystes första gången i förordet till Prof. Etiennes bok \textit{På spaning efter den bov som flytt}. I vanligt ordning presenterade Prof. Etienne inga konkreta bevis för sitt påstående utan litade helt till sin gubbsäkerhet\ref{gubbsaeker}.

\ditem[Skagen]\label{skagen}
 kallas det område i norra Danmark\ref{danmark} där ljuset är så fantastiskt att konstnärer från hela norra delen av landet vallfärdar dit på dagarna för att måla av det. På kvällarna vallfärdar dom hem igen för det är som bekant ganska kallt vid havet. I ärlighetens namn är det sällan någon som orkar måla så mycket eftersom fantastiskt ljus även inbjuder till att ha picnic på en filt och lyssna på \textit{Dark side of the moon}. Många konstnärer började ta med sig burkar med majonäs i stället för målarfärg, som de på förvirrat konstnärsmanér glömde skruva tillbaka locket på efter avslutad picnic. Insekter kröp ned i burkarna och fastnade i majonäsen och det var så skagenröra uppfanns.

\ditem[Skam]\label{skam}
 Skäms gör man rätt ofta. Ibland för sin egen skull och ibland för andras. Skam har av en psykiater på Umeå universitetssjukhus beskrivits för denna artikels författare som en av de hemskaste känslor en människa kan känna. Ett gott exempel på när man skäms är när man suttit på en uteservering i en sydsvensk stad, limmat på en mäktig tjej bara för att se henne dra med sin polare. Besvikelse\ref{besvikelse} är den första känslan man känner i anknytning till en sådan händelse. Men sen kryper skammen på, när man börjar fundera över hur uppenbart allt var hela tiden och vilken total sopa man måste ha framstått som. Om man dricker alkohol är det lätt att förskjuta skammen, då människan efter ett litet intag av tidigare nämnda drog förlorar precis all skam i hela kroppen. När skammen kryper sig på morgonen efter slår den ofta till med dubbel styrka. Det är inte ovanligt att man ligger och spelar upp sekvenser från kvällen innan i sitt huvud och ba Neeeeeiiiin!!!!

\ditem[Skedstork]\label{skedstork}
 Skedstorken är en fågel som dragit en nitlott i evolutionens lotteri. Istället för en näbb fick den salladsbestick i bakelit i ansiktet. Den är antagligen utrotningshotad\ref{utrotningshotade djur} då den enda mat den kan äta är caesarsallad som det är sparsmakat med i dess habitat 

\ditem[Skellefteå]\label{skellefteaa}
 är en stad som ligger norr om Umeå\ref{umeaa} och söder om Piteå. Det är svårt att sätta fingret på vad grejen är med Skellefteå. Det är väl egentligen en vanlig håla, som ett halverat Västerås. Men det ligger nåt i luften. Är det guldet som förfinas av Boliden AB, strax utanför staden, som lägger ett ömsom euforiskt ömsom ångestladdat midasskimmer över skylinen? Eller är det de mystiska influenser som glider in i på ångare från orienten till Skelleftehamn som sprider ett fördunklande dis över Möjligheternas torg? Är det Torgny Lindgrens mytologiska sagor som tunnat ut gränsen mellan verklighetens bas och fantasins överbyggnad mellan Kåge i norr och Bureå i söder, mellan Boliden i väst och Ursviken i öst? Eller är det inaveln, den primitiva blodsreligion som förbjuder att blanda ut den starka Marklundska genpoolen med undermåligt Anderssonskt piss, som driver på den pulserande rödlila nerv som slår an takten för livet vid Skellefteälvens mynning?

\ditem[Skensnygg]\label{skensnygg}
 Förgängligt utseende som avslöjas först på nära håll. Beskrevs träffsäkert av David Lynch i Twin Peaks med repliken \quotetext{the owls\ref{uv} are not what they seem}.

\ditem[Skiftnyckel]\label{skiftnyckel}
 Skiftnyckelns främsta användningsområde är att knacka och slå på saker när en hammare inte finns, eller helt enkelt inte är placerad inom bekvämt avstånd. De längre varianterna kan användas som brytspett. Ska man dra muttrar ska man ha fasta nycklar\ref{fasta nycklar}, hör ni det!?

\ditem[Skillnaden mellan ånga och dimma]\label{skillnaden mellan aanga och dimma}
 Ibland när man befinner sig i ett vitt moln\ref{moln} kan det vara svårt att fastställa om fenomenet är dimma\ref{dimma} eller bara vanlig ånga. Som tur är finns några enkla tummregler att tillgå: Ånga uppstår ur en enskild källa. Till exempel från bastun som lokförarens fackförbund förhandlat fram ska finnas i alla lok, eller från en het källa som stenåldersmänniskor använde för att koka ägg innan de lärt sig bemästra elden. Dimma däremot uppstår överallt där det är vatten i luften som är för lite för att regna och för mycket för att dunsta. Överallt där det är så blir det dimma. Om du befinner dig i ett vitt moln och snabbt måste veta om det är dimma eller ånga räcker det alltså med att du ser dig omkring och tittar om det finns en enskild källa eller inte. Ta det lite försiktigt bara, bastuaggregat och varma källor är väldigt varma.

\ditem[Skinhead]\label{skinhead}
 Klä sig som en pensionär och supa med nazister, jättekul om du har \quotetext{kraftig benstomme} och bor i nån jordbruksort\ref{jordbruksort}.

\ditem[Skinnskatteberg]\label{skinnskatteberg}
 (tidigare Skinnsäckeberg, OBS! Sant!) är en kommun i norra Västmanland. Skinnskatteberg är födelseort för välkända figurer såsom Fakta\ref{fakta}, Andreas Kleerup, Slisken och Sober-Jimmy\ref{sober-jimmy}. Även Jan Myrdal har fattat tycke för ortens blomstrande stillhet och bosatt sig där på äldre dagar. Skinskatteberg har i likhet med Malå\ref{malaa} ett sågverk. 

\ditem[Skinnslips]\label{skinnslips}
 Skinnslipsen kan ses som näverslipsens\ref{naeverslips} efterträdare. En individ i skinnslips kommer med största sannolikhet att sno din partner på dans, eller din bättre hälft på ert bröllop. Skulle handgemäng utbryta klarar sig personen i skinnslips alltid undan med hjälp av sitt garvade munläder.

\ditem[Skita i den korvbröda kökssoffan]\label{skita i den korvbrooda kookssoffan}
 är den proletära varianten av att skita i det blå skåpet\ref{skita i det blaa skaapet}.

\ditem[Skita i det blå skåpet]\label{skita i det blaa skaapet}
 Kraftuttryck för att markera att någonting gått riktigt snett. Ursprungligen tros uttrycket härstamma från 1800-talet, när färgämnet \quotetext{berlinerblått} började massframställas så att också vanliga knegare kunde börja blåmåla sina möbler. Tidigare var blått främst förunnat borgerskapet, och allmogen fick istället nöja sig med gammal hederlig faluröd och ockra. Associationen mellan blått och fisförnämhet levde dock vidare, vilket vi ju kan se än idag, så man blåmålade först och främst de finare möblerna i hushållet. Färgen är således egentligen inte det primära i talesättet, utan det faktum att någon varit dum nog att tömma sin tarm i finskåpet istället för exempelvis den mer vardagliga kökssoffan\ref{kookssoffan}. Man kan jämföra det med att spy i handfatet, där det vanligtvis finns en för ändamålet betydligt lämpligare klosett bara någon meter därifrån. Uttrycket populariserades i modern tid av Janne \quotetext{Loffe} Carlsson i filmen \textit{Göta kanal}\ref{sveriges sju underverk} från 1981. I ett trängt läge, där någon avlägsnat en propeller\ref{propeller} från Carlssons båt, utbrister denne: \quotetext{Nu är det krig! Nu har dom skitit i det blå skåpet!}. I citatboken \textit{Bevingat} säger Carlsson att det var hans far som lärde honom uttrycket, och det finns väl ingen anledning att ifrågasätta det.

\ditem[Skjortponcho]\label{skjortponcho}
 En skjortponcho är ett klädesplagg som rent hypotetiskt används av en färgstark individ i underhållningssyfte. Det har varit svårt att finna några konkreta bevis på dess existens, då den endast skymtats på skivomslaget till Charles Bradleys singel Heart of Gold. Detta är dock ej tillförlitlig information i dagens tider med Photoshop och dylikt. Experterna hävdar att omslaget är manipulerat, men entusiaster världen över lever på hoppet.
Det sägs att plagget består av peruanskt vikunjaull. Något som låter rimligt eftersom belgisk polis enligt inofficiella uppgifter fann peruansk valuta på Bradley i samband med en rutinprocedur som uppbar vissa likheter med konventionell kroppsvisitation. Om uppgifterna stämmer beräknar statistiker en kraftig tillbakagång för lamabeståndet i världen, då den ökade efterfrågan uppmuntrar till tjuvjakt och annat ofog.

\ditem[Sko]\label{sko}
 Någonting man gör sig på andra, eller möjligen en häst\ref{haest}. Detta är grundtanken i kapitalism.

\ditem[Skogsrave]\label{skogsrave}
 Ett rave som tar plats i skogen. Skogsravet är en relativt ny företeelse som går ut på att ett gäng käcka ungdomar drar med sig ljudutrustning som drivs av ett dieselaggregat eller nåt ut i skogen och spelar högljudd musik, dansar samt tar kemiskt framställda droger. Vildlivs- och musikvetaren Mulva Mossrock har yttrat sig om fenomenet i följande ordalag: \textit{\quotetext{I teorin är det en rätt juste idé att dra ut i skogen, lyssna på vild musik och kanske ligga lite i en murrig backe, men just fenomenet skogsrave omgärdas ofta av den högtravande idén att det ska vara så jävla häftigt och speciellt att vara i skogen, vilket pajar det fina med hela grejen. Inte sällan kommenterar nån av deltagarna att det ändå är rätt black metal att vara ute i skogen, vilket inte är helt osant. Problemet är bara att deltagarna själva inte ens gillar black metal eller friluftsliv utan har något sorts post-ironiskt förhållningssätt till hela den hedniska naturdyrkargrejen och vill egentligen avnjuta sin blip blop på lokal. Således förlorar inte bara deltagarna i ravet på dess existens, då de blir blöta, kalla och inte sällan ådrar sig urinvägsinfektioner. Även vi genuina naturfestdiggare förlorar på att skogsravet finns, då popkidsens dieseldrivna dunka dunka-maskiner dränker ut rocknojset från våra dassiga bandare och skrämmer bort alla fina djur som eventuellt hade kommit förbi och ätit ur våra händer.}}

\ditem[Skolbespisningsmat]\label{skolbespisningsmat}
 Gemensamt för all skolbespisningsmat är att den serveras i tråg\ref{traag}.

\begin{itemize}
\item Fläskpannkaka (finns utan fläsk)
\item Frikadellsoppa
\item Grillburgare (max tre (3) per elev)
\item Mexicanasoppa
\item Korvslantssoppa
\item Leverbiff
\item Potatisplättar
\item Kokt potatis
\item Fiskbullar
\item Torsk
\item Ryssröra
\item Vad huset förmår
\item Raggmunk (oätlig)
\item Lappskojs\ref{lappskojs}
\end{itemize}

\ditem[Skottar]\label{skottar}
 Hämndlystna och giriga på onödiga saker. Stora fans av dronemusik.

\ditem[Skotte]\label{skotte}
 är en stycksak av choklad som produceras av Marabou. Skotte finns att tillgå i dubbelutförande med två bitar på 30g vardera. Skotte består av nougat och, för ovanlighetens skull i konfektyrvärlden, russin. Detta gör naturligtvis att skotte är avskyvärt äcklig\ref{aeckligt godis}, men någon, oklart vem, köper den ändå och medan vi ser självklara klassiker försvinna från godishyllorna levererar godisbilen oförtröttligt skottar\ref{skottar} till landets alla gotteaffärer. Namnet har den fått för att den ger ett liknande helhetsintryck som skotsk mat.

\ditem[Skottfint]\label{skottfint}
 En skottfint är en manöver som framförallt används i fotboll\ref{fotboll} och som är speciellt vanlig i matcher mot det italienska landslaget. Finten går ut på att bollinnehavaren låtsas ladda för att skjuta ett hårt skott. Den mötande italienaren kastar sig då till marken med händerna för ansiktet, föranledd till detta av den italienska folksjälens narcissistiska svaghet. Skrikandes från marken anmodar italienaren bollinnehavaren att inte skjuta. Den senare kan nu obehindrat runda den liggande italienaren, vars sikt är avsevärt begränsad på grund av ohejdad gråt.

\ditem[Skrapade skraplotter med vinst]\label{skrapade skraplotter med vinst}
 är små papperslappar som kan växlas in mot en viss summa pengar, vilken anges på själva lappen. Summan varierar mellan ungefär 25:- och åtskilliga miljoner. Till skillnad från sin kusin, den skrapade skraplotten utan vinst, kan denna mer förnäma lott användas som betalningsmedel, det vill säga som valuta\ref{valuta}. Detta är dock ovanligt, vilket följer av att skrapade skraplotter med vinst är mycket mindre vanliga om man jämför med vanliga pengar eller skraplotter utan vinst. Bellman\ref{bellman}, skald och för en tid förståndare för det kungliga lotteriet, sedemera Bellmanlotteriet, ska enligt trovärdiga källor ha sagt om den skrapade skraplotten med vinst att: \quotetext{denna avklädda dam äro förvisso mer ärhevördighet än sin fruktlösa syster,} i ett för honom karaktäristiskt anfall av sexism.

\ditem[Skrattfnatt]\label{skrattfnatt}
 är ett stående inslag i Kalle Anka\ref{kalle anka} \& Co. Det består i roliga historier som läsaren hittar längst ned på varje sida i tidskriften. Dessa historier kan memoreras för att sedan hivas fram när man står kring kontorets water cooler eller är på disco med sin mellanstadieklass.

\ditem[Skrunka]\label{skrunka}
 (verb, obestämd form singular; bestämd form \textit{skrunk}) är en multipel handling där utövaren skrattar högljutt samtidigt som den onanerar. Fenomenet har blivit mindre vanligt på senare tid i och med att man sällan har något riktigt roligt att skratta åt. Skrattet måste vara ärligt.

\ditem[Skruvkapsylöl]\label{skruvkapsylool}
 Vissa, främst utländska, ölmärken på bolaget har skruvkapsyl som standard (t.ex. Gösser, Miller) och andra har det periodvis (t.ex. San Miguel, Victoria Bitter). Det råder delade meningar bland öldrickare huruvida en skruvkapsyl är något att hänga i julgranen eller inte. Vissa förordar att skruvkapsylerna är av godo då de gör kapsylöppnare överflödiga och det på så vis går marginellt snabbare för den törstige att komma åt sin dryck. Andra menar att skruvkapsylen förtar det roliga i att smidigast öppna en flaska på det sätt man själv föredrar, såsom med en snusdosa, tändare eller helt enkelt en kapsylöppnare som var och varannan bär i sin nyckelring. Den vanliga kapsylen har som bekant skänkt oss ett av de vanligaste partytricken - att öppna sin öl med något ovanligare objekt man inte tänkt på att använda i nyktert tillstånd (såsom ögat eller kanske en bordskant där det garanterat uppstår ett eller flera permanenta märken efteråt). Skruvkapsylsöl kan ge upphov till olika lite mer sällsynta skador. Ifall konsumenten grundat en helkväll rejält på ett märke med skruvkapsyl och sedan byter till ett märke utan så finns risken att konsumenten under rus tror att även det andra märket går att öppna med händerna och skärsår uppstår tills motsatsen bevisats. Ölkonsumenter med bristande erfarenhet av att öppna ölflaskor med ena ögat ger sig gärna i kast med tricket då skruvkapsylen inte sitter lika hårt som en vanlig kapsyl, vilket långt från alla gånger faller väl ut. Detta skulle nog den argentinske poeten och essäisten Jorge Luis Borges vara den förste att hålla med om, ifall han hade levt idag. Trots att skruvkapsylen började användas redan år 1875 i England så har de svenska bryggarna inte tagit den till sig än. Endast några få - däribland norrländska Werde från Zeunerts - har gett den en chans. Perlenbacher, som saluförs av den tämligen suspekta livsmedelskedjan Lidl, är sannolikt den enda folkölen i Sverige som har skruvkapsyl. Om man vill fördjupa sitt intresse för skruvkapsylens historia kan man besöka Flaskmuseumet i Sonkajärvi, Finland.

\ditem[Skrymmande]\label{skrymmande}
 är postens definition av brev eller paket med mått som överstiger vissa gränser och därför måste ha fler frimärken för att kvala in i en högre kategori. Den mest klassiska typen av skrymmande försändelser är vinylskivor, men likt Einsteins relativitetsteori kan allt större en ett vykort i fel sammanhang vara det. Vill du till exempel skicka en palltruck\ref{palltruck} i födelsedagspresent till din bästa kompis på åttaårsdagen är det troligt att den kommer klassas som skrymmande även om du köper frimärken för alla pengar du har i bössan. Men även något så lätt och litet som en vanlig bruksgök\ref{bruksgook} kan visa sig vara skrymmande om du vill skicka den med vanlig A-post. Ska man skicka något och känner sig osäker på om det är skrymmande kan man alltid höra med någon av postens\ref{posten} rekorderliga medarbetare, för de vet allt om post. Var dock noga med att kontrollera att det inte är en vanlig ICA-personal i förklädnad för de kan ingenting om post och paket.

\ditem[Skräckväldet]\label{skraeckvaeldet}
 När historiker brukar prata om Skräckväldet handlar det påfallande ofta om den period då den spritt språngande galne Robespierre hade makten i Frankrike i slutet på 1700-talet. Mer sällan handlar det om den period i Malå\ref{malaa} kommuns historia mellan 1974 och 83 när den var hopslagen med Norsjö\ref{norsjooblicken} kommun. Detta drevs igenom av Socialdemokraterna. Det ska dock sägas att Malås kommunalråd Karl-Ymer Berglund (s) var mot en sammanslagning, men höll sig partiet trogen. Inget ont om n'Karl-Ymer. Malåborna protesterade under parollen \quotetext{Nej till diktatur} i 9 år tills Malå kunde bryta sig ut och bli det municipalsamhälle vi alla känner och älskar idag.

\ditem[Skräp]\label{skraep}
 kan vara lite vad som helst och förvaras vanligtvis i uthus och fäbodar. På vinden går också bra. Många samlar på skräp för att det kan vara bra att ha. Skräp som fastnar på tröjan när du legat på rygg i skogen kallas för bös, boss eller bysche - beroende på vilken del av Bergslagen du stammar från. 

\uline{Inom bilindustrin}

Franska bilar är oftast skräp.

\uline{Inom musiken}

Fransk musik är oftast nån slags fusion-skräp.

\ditem[Skröna]\label{skroona}
 En skröna är en historia som hittas på i stundens hetta när två personer sitter och finljuger\ref{finljuga}. Ofta har historierna till en början en viss sanningshalt, men med tiden brukar denna tänjas på allt mer. Skrönor är inte samma sak som andra berättelser, som t.ex. tv-serier, filmer och böcker utan dessa är ofta historier antingen helt och hållet påhittade eller sanna. Gamla människor är bäst på att dra skrönor då senilitet och demensanlag gör det svårt att berätta någonting överhuvudtaget utan att rucka på vad som faktiskt hänt och inte.

\uline{Exempel från verkligheten}

Min morfar är som alla andra gamla människor en jävel på skrönor. Senast idag berättade han om hur han under en tågresa ledsnat på att personen i slafen under honom snarkade så förtvivlat. Morfar sträcker då ner handen och luggar den snarkande karln' i skägget, som slutar snarka. På morgonen visar det sig att den snarkande mannen är två meter lång och bitig som fan. Än senare visar det sig att samme snarkande monster har en biroll i filmen \quotetext{Vi hade i alla fall tur med vädret!}. I filmen har han polisonger och osedvanligt oljig t-shirt. Han har morfar alltså luggat i skägget. Inget gäck.

\ditem[Skuggan]\label{skuggan}
Begrepp med olika betydelser inom olika områden i det postmoderna samhället:

\uline{Inom filosofin}

Skuggan är inom CG Jungs psykoanalaytiska teori ett begrepp som representerar det onda och irrationella inom människan.

\uline{Inom plantagenäringen}

Det är också något man kan sitta och dricka Southern comfort i.

\uline{Inom idrotten}

Hockeyspelare i Lule Hockey,flest elitseriassist!

\uline{På gatan}

En civilare vid namn Roger som följer efter en väldigt odiskret.

\ditem[Skäpparkrans]\label{skaepparkrans}
 (\textit{Podiceps cristatus}) är ett slags ansiktsbehåring som skeppare och vissa andra har valt att gå, eller segla, omkring med. Skäpparkransen är som en krage men går längs skäpparens käke - från det ena örat till det andra. Den liknar för det otränade ögat förvillande mycket ett klassiskt skägg, men kombineras inte som det klassiska skägget med en mustasch, för det tycker skäpparen är onödigt och opraktiskt. Skäpparkransens funktion, förutom att vara en identitetsskapande symbol, är att hindra mat och vätska att sippra ut ur skepparens mun\ref{mun} och drälla ner över hela bröstet. Istället fastnar överskottsmaten och -vätskan i skäpparkransen från vilken den kan avlägsnas så snart skäpparen kommit iland eller fått lite tid över för rekreation och nöjen. Alla som har skepparkrans är inte pederaster men alla pederaster har skepparkrans.

\ditem[Skåne]\label{skaane}
 är Sveriges\ref{sverige} sydligaste landskap. En vacker plats med böljande åkrar, monumentala kustvyer, wundersköna lövskogar och Turning torso. Folk som bor i Skåne tycker om att se sitt landskap som motsvarande den amerikanska södern, fast i Sverige\ref{sverige}, och skriver gärna att de kommer från \quotetext{The dirty south} på sina facebook-profiler. Liknelsen är inte helt orimlig då Skåne, likt sydstaterna i USA, utgör en bastion för grisfarmande och främlingsfientlighet. Antagligen vill skåningarna själva mena att likheten består i den fräcka laglösheten, rebellandan och den generella go against the rules-attityd som är förhärskande i den romantiserade bilden av den amerikanska djupa södern. I Sverige\ref{sverige} står dock den andan främst att finna vid landets norra gräns mot Finland\ref{finland}, där laglöshet råder i ordets sanna bemärkelse då snutar inte existerar norr om Boden\ref{boden}.

\uline{Folk}

Det finns tre sorters skåningar, nämligen:

Alfaskåningen, som är en fryntlig filur som gärna skrattar och pratar (eller \quotetext{blurrar} som infödingarna kallar det) på sitt underfundiga, diftongerade skorrande vis med vemhelst de stöter på. De är gästvänliga och väl belästa inom alla artes liberales, med särskild fokus på quadrivium. Alfaskåningarna är inte sällan duktiga på musik! Hos Alfaskåningarna är det inte heller ovanligt att studera många olika språk för att kunna vara utåt och sociala över så många nationella barriärer som möjligt.

Den andra sortens skåning kallas Betaskåningen. Den sortens skåning är fet, lat, ful, snål, okunnig, vräkig, högljudd och dum. Till skillnad från Alfaskåningen, vars dialekt är extremt gullig och tilltalande, låter det som om nån bronkithostar en i örat när man hör Betaskåningen babbla på om hur nästa års svinskörd hotas av den ökade invandringen och Sveriges\ref{sverige} alldeles för låga straffsatser.

En tredje kategori man kan finna i framförallt Malmö är inte skåningar i ordets egentliga bemärkelse då de sällan eller aldrig är födda och uppvuxna i Skåne, de talar i regel ej heller skånska.

\ditem[Slagg]\label{slagg}
 är ursprungligen de ämnen som inte är metaller som blir över i en metallurgisk process. Förutom att bygga husgrunder av är det tämligen värdelöst. Begreppet används idag för att beskriva restprodukter i alla former av processer. Vegemite är exempelvis en slaggprodukt som bildas vid framställning av ölet Foster’s\ref{australien}. De mest kända slaggprodukterna är förmodligen brugden\ref{brugd}, som uppstod ur de överblivna delar evolutionen inte behövde ha för att skapa alla råa hajar, samt landet Belgien\ref{belgien}.

\ditem[Slan]\label{slan}
 kan betyda en massa olika saker, men är ungefär synonymt med dåligt eller slarvigt.
Som adjektiv eller adverb (slanigt) kan det känneteckna ett ting eller företeelse som inte är så jävla smutt.

\uline{Exempel}

Som adjektiv
\quotetext{Det var ganska slanigt att inte städa på tre år}
\quotetext{Den här danska lättölen\ref{dansk laettool} smakar så jävla slanigt}

Som animat substantiv
\quotetext{Anderssons grabb har gått och blivit ett riktigt slan}

Som verb
\quotetext{Nu har jag slanat i soffan hela dagen.}

Rent geografiskt är användandet av ordet centrerat till Närke och Bergslagen\ref{bergslagen}, men sprids med diasporan. En kulturell referens till ordet är Krigshot - Slanig snut, och innan de spelade låten på Augustibuller 2006 sa Jallo: \quotetext{Anders åkte på en fortkörningsbot på vägen hit, två tunkor, så nu kör vi Slanig snut}.

\ditem[Slayerklass]\label{slayerklass}
 är ett slags kvalitetscertifikat på musik som kan utfärdas av vem som helst. Utfärdaren garantrar därmed att musiken i fråga är så bra att den lika gärna skulle kunna varit skriven av Slayer.

\ditem[Slentrian]\label{slentrian}
 Lianer som växer i nerförsbackar. Ofta i klasar och ganska slarvigt.

\ditem[Släktträffsberusning]\label{slaekttraeffsberusning}
 Ett stadie där du ligger med vem som helst utan hänsyn till fysiska/psykiska defekter, konsekvenser och eventuella släktskap. I vissa kretsar är tillståndet permanent.

\ditem[Slätt]\label{slaett}
 En slätt är ett platt landskap utan större urbana inslag. Riktiga slätter bör domineras av endast en typ av vegetation, med fördel gräs, sand\ref{sand} eller grus. Vid utformande av sin egen slätt bör man välja material utefter vilka aktiviteter man planerar på den. För cricketspelande rekomenderas grässlätter medan ökenrallyn är bäst lämpade på sandunderlag. Exempel på personer som ridit på slätter är ryska rövare och Evert Taube.

\ditem[Smidesstäd]\label{smidesstaed}
är en typ av förbannat tung arbetsbänk för smidesarbeten.
Precis som järnspettet\ref{jaernspett} så är städet homogent, dvs tillverkat av 100\% solitt stål.

\uline{Tillverkning}

Man tillverkar egentligen inte smidesstäd utan letar upp ett järnvägsspår, kapar ur en lämplig bit (ca 80Cm) med en vinkelslip och monterar sedan rälsbiten på en stubbe i lämplig höjd.
Det finns 1000-tals kilometer järnvägsräls i Sverige\ref{sverige} så ingen kommer att märka om det fattas en liten bit.

\ditem[Smuldegspappor]\label{smuldegspappor}
 är den lägre klassen i samhällsgruppen \quotetext{Moderna pappor}. En smuldegspappa tvingas leva på smulorna av surdegspappornas frodigare liv.

\ditem[Smygfascist]\label{smygfascist}
 är den bästa låten på KSMBs debutplatta Aktion. Ett tag såg den ut att kunna bli Sverigedemokraternas\ref{sverigedemokraterna} officiella val-låt, men så kom någon på att det ju var det här man ville dölja. Smygfascismen alltså. Den fick således se sig besegrad av låten \quotetext{Jimmie Åkesson - tjalalalala} som vi alla känner till så väl idag.

\ditem[Smygsexist]\label{smygsexist}
 En smygsexist är en person som gömmer sig bland buskar och ropar \quotetext{VICKA BRÖN HÖRRÖ} till förbipasserande kvinnor. 

\ditem[Småbyxa]\label{smaabyxa}
 är ett plagg som är populärt bland små barn\ref{barn} (som ju som bekant inte har nåt som helst folkvett alls) och kvinnliga skådespelare och sångerskor från the United States of America\ref{united states of america}. Småbyxan ingår i kategorin hygienplagg som avdelar sig från kategorin värme- och komfortplagg på så vis att småbyxan är där mer eller mindre enbart för att förhindra nakenhet.

\ditem[Småskurk]\label{smaaskurk}
 En småskurk är en brottsling som håller sina lagöverträdelser inom acceptabla ramar. Tar man exempelvis påtår på kaffet fast det egentligen inte ingår så är det okej. Lika så är det lugnt att man snattar billigare saker i mataffären så länge det inte är en konsumbutik\ref{konsumbutik}. En småskurk kan även fiffla med deklarationen så länge den inte är rik. Rika personer kan aldrig vara annat än storskurkar. Det är allmänt accepterat att vara småskurk. Vissa ser det till och med som lite charmigt.

\uline{Exempel på kända småskurkar}

\begin{itemize}
\item George Best
\item Lemmy
\item Petrus de Dacia\ref{petrus de dacia}
\item Leila K
\item Cockney Rejects
\item Prof. Etienne\ref{prof. etienne}
\end{itemize}

\ditem[Småstadsalternativ]\label{smaastadsalternativ}
 Ett lapptäcke av influenser utmärker den småstadsalternativa stilen. En bombarjacka med Marilyn Manson-backpatch, kängor med neonlila skosnören och ravebrallor kombineras utan förbehåll. Åt fanders med de strikta regler som vanligtvis styr subkulturer. I småstäder som Bollnäs, Härnösand och Mariestad finns ingen respekt för stilmässiga regelryttare. På ett sätt kan man bara berömma den totala skit-i-allt attityd som präglar småstadens rebeller. Samtidigt ser det ju för taskigt ut.

\ditem[Sneakers]\label{sneakers}
 är ett hipsterord lånat från engelskan och betyder jumpaskor.

\ditem[Snefotad ultrapelikan]\label{snefotad ultrapelikan}
 (\textit{Pelecanus conspiratoris}) är en art inom familjen pelikaner\ref{pelikan}. Den lever ett rätt stillsamt liv i Mogadonien och några småöar runtikring på fisk och vatten, ungefär som fattigpensionärer. I jämförelse med andra pelikanarter är den snedfotade ultrapelikanen ganska stor och kan väga upp till 15 kilo. Dess maffiga vikt i kombination med dålig flygförmåga och att den har så sneda fötter att den inte kan gå så fort har medfört att den inte klarar sig speciellt bra i trafiken. Så nu är den rätt utrotningshotad\ref{utrotningshotade djur}. Men självklart har den en stor pungliknande säck under näbben. Den har fullt med ohyra i fjäderdräkten så man bör inte klappa den, men den är väldigt social så att sällskapa går bra.

\ditem[Snesegla]\label{snesegla}
 Att snesegla är det samma som att hamna på glid och är något som strulputtar ofta gör. Början på en sneseglats är ofta att den unga människan börjar skolka från skolan och dricker folköl i någon park hela dagarna. Ett exempel på en sneseglare är Christer Petterson. Andra vanliga namn på sneseglare är Ronny, Conny och Johnny.

\ditem[Snowjoggers]\label{snowjoggers}
 kallas de vinterskor som användes utbrett bland de lägre samhällsklasserna i stora delar av västvärlden under slutet av 1900-talet. Förespråkarna menade att de var varma, bekväma och lätta att ta på medan motståndarna menade att de var bedrövligt fula. Utbredd mobbning mot dess användare ledde slutligen till att skon försvann från marknaden.
Snowjoggers är numera förbjudna inom EU.

\ditem[Snus]\label{snus}
 är ett slags kräm gjord av malda löv och används för att fingera tillhörighet till arbetarklassen. Snus anses felaktigt vara ett prestationshöjande preparat hos isolerade grupper så som människor som känner kopplingar till Piteå, skinheadscenen\ref{skinhead}, innebandykultur och polisutbildningen. Men det är det inte.

\uline{Logistik}

Snus levereras till konsumenten i form av puck-formade dosor som i sin tur förpackas av mottagaren i ett par jeans.

\ditem[Snusbrist]\label{snusbrist}
 är ett livs- och själhotande tillstånd som infinner sig hos svenskar och orsakar obeskrivligt lidande. I vissa fall kan det indirekt orsaka dödsfall genom att individen\ref{individ} drivs till vansinne och tar sitt liv genom helt sonika fylla\ref{baersfylla} snusgropen med en kärve hagel. 

\ditem[Snutkaffe]\label{snutkaffe}
 Kaffe utan socker, mjölk, grädde, kask, smör, honung, nånting. Bara kaffe. Ibland med en ryssfemma\ref{femma} i. Helst i pappmugg.

\ditem[Snutnamn]\label{snutnamn}
 Ett Snutnamn är ett \quotetext{nytaget} efternamn som gärna låter just nytaget. De kallas Snutnamn för att de är vanligt förekommande bland just poliser, men återfinns även i många andra sammanhang där efternamnet tillmäts stor betydelse.

\uline{Exempel på snutnamn}

\begin{itemize}
\item Carnestedt
\item Fuglesang
\item Gärdestad
\item Orustfjord
\item Tigercrona
\item Ärlestål
\end{itemize}

\ditem[Snutröv]\label{snutroov}
 En snutröv är lika bred som en hockeyröv\ref{hockeyroov} men kännetecknas av att den är extremt ihopknipen.

\ditem[Snälla killar som aldrig får ligga]\label{snaella killar som aldrig faar ligga}
 borde egentligen heta \quotetext{Killar som själva tycker att de är snälla men aldrig får ligga}, men det är i längsta laget, även för Nissepedia\ref{nissepedia}. Dessa killar har lite olika utgångspunkt, en del är bittra och säger att tjejer bara vill ha snubbar som behandlar dem som skit. Andra är mer oförstående och förvirrade. En del av dem skriver krönikor som heter \quotetext{Vi som aldrig sa hora}. Snälla killar som aldrig får ligga borde egentligen bara hålla käften.

\ditem[Snälla mamma, mata mig som vore du en fågel]\label{snaella mamma, mata mig som vore du en faagel}
 är den första delen i Prof. Etiennes\ref{prof. etienne} självbiografiska verk och avhandlar författarens första stapplande steg, från födsel till det att han fyller åtta.

\uline{Synopsis}

Författaren föds i Ödeshög. Tidigt i sitt liv märker den unge Etienne att han är mer benägen att identifiera sig med fåglar än människor. Han vill flyga långt bort, obunden av sina fysiska begränsningar (han var ett korpulent barn). För att bli en fågel vill den unge Etienne bli behandlad som en sådan och får sin mamma att mellan 4-7 års ålder mata honom som fågelmammor matar sina ungar, det vill säga genom att modern först tuggar maten för att sedan vomera den i sin avkommas mun\ref{mun}. Om Etienne inte fick som han ville drabbades han av extrema raseriutbrott som kunde pågå i flera timmar. Hans fixering vid fåglar släpper när han blir attackerad av en kråka som han slår ihjäl i självförsvar. Efter att ha dödat sin nemesis öppnar Etienne bröstkorgen på fågeln med sin schweiziska armékniv, tar ut kråkans hjärta och äter det på plats, rått. Efter denna ritual anser Etienne att han tillförskansat sig fågelns makt och kan återgå till att bara vara en plufsig människopojke.

\ditem[Snärt]\label{snaert}
 En snärt är en måttenhet som betecknar det som i vardagligt tal uttrycks som \quotetext{lite grann}. En centimetertjock skiva falukorv är en korvsnärt.

\ditem[Snåltarmen]\label{snaaltarmen}
 är det organ i människokroppen som ser till att pensionärer alltid vill ha kvitto i affären trots att de har för dålig syn för att se vad som står. Snåltarmen är också skälet till att man blir grisfull\ref{grisfull} så fort det är bjudsprit\ref{bjudsprit}.

\ditem[Snöskor]\label{snooskor}
 är ett fenomen som nästan alla känner till men som få har sett i verkligheten. Har man någon gång besökt ett badrum\ref{historiska haendelser i badrum} i ett hushåll där det vistas barn har man med all säkerhet även läst \textit{Kalle Anka \& Co}. Om tidningsexemplaret är utgivet under vinterhalvåret är sannolikheten att ett par snöskor figurerar i någon av serierna ungefär lika stor som att Björnligan äter sviskonpaj\ref{sviskonpaj} till middag. Snöskorna utgörs av två stycken tennisracketar med varsin läderrem genom nätet\ref{world wide web} så att de kan fästas utanpå ett par vanliga skor. Varför Kalle Anka\ref{kalle anka} måste ha snöskor när han redan har simhud mellan tårna kan man verkligen fråga sig, men eftersom denna artikel mest handlar om sannolikhetslära och inte biologi lämnar vi den frågan därhän. Sannolikheten att dina föräldrar kommer berömma ditt initiativ att tillverka ett par egna snöskor av deras tennisracketar är närmare noll. Så bry inte din lilla hjärna med sådana projekt utan sitt istället kvar på muggen och läs en sida skrattfnatt\ref{skrattfnatt} istället.

\ditem[Sober-Jimmy]\label{sober-jimmy}
 är en välkänd musikprofil från Skinnskatteberg\ref{skinnskatteberg}. Han grundade No Fun At All tillsammans med Fakta\ref{fakta} och spelar på bandets största studiostund \textit{Vision}. Därefter kände Sober-Jimmy att No Fun tog upp för mycket tid och hoppade av för att helt fokusera på sitt andra skötebarn Sober. Där gick Sober-Jimmy en lysande framtid till mötes och släppte bland annat legendariska EP:s så som \textit{Blowjob} och \textit{Snowbored}. I nuläget ligger bandet tyvärr på is, och Sober-Jimmy har skaffat sig ett riktigt jobb.

\ditem[Socialdemokrati]\label{socialdemokrati}
är den politiska ideologi som styrt Sverige längst, monarkistisk diktatur ej räknat. Man skulle kunna skriva långt och rikt om vad socialdemokrati egentligen innebär. I tider som dessa är det många som undrar, nämligen. Tills vidare kan vi konstatera att hela Nissepedia i sig fungerar som en sorts terapeutisk ventil där vi bearbetar socialdemokratins inverkan på våra tidiga liv.

Sen kan vi också slå fast att den tydligaste kvarlevan av den gamla skolans socialdemokrati som fortfarande gör sig påmind i vår vardag, är fenomenet att det i Sverige är god sedvänja att främlingar säger hej! till varandra när de möts på glest trafikerade gågator och i elljusspår, samt känslan av tomhet som infinner sig om man inte får ett hej! tillbaka.

\ditem[Solidaritet]\label{solidaritet}
 är det enda som kan rädda oss från varandra och oss själva.

\ditem[Sommar]\label{sommar}
 är den årstid som är mest Bruce Springsteen. Våren och hösten är inte alls lika mycket Bruce Springsteen och vintern är inte särskilt Bruce Springsteen alls. På sommaren har tjejerna sommarkläder, och det tycker Bruce Springsteen om. Då brukar han åka ner till floden och ta sig ett dopp, rejsa längs gatorna i sin bil och uppvakta tjejer från New York. Det är som att Springsteen liksom brinner inuti när det är sommar. 

\uline{Kännetecken}

Sommaren kännetecknas av att det är betydligt varmare då än under resten av året. Många har likt Bruce Springsteen jeans-shorts på sommaren.

\uline{Sommar i Australien}

Som med mycket annat har Australien\ref{australien} valt att göra tvärt om så de har sommar på vintern istället.

\ditem[Sommarplågsmusiker]\label{sommarplaagsmusiker}
 Personer som livnär sig på att förstöra människors liv genom att komponera, framföra eller sjunga sommarplågor, medvetet. Låtar handlar t.ex. om:att få ligga, att befinna sig på Stureplan, att lyssna på reggae, att dansa Macarena och ryta. Vi på Nissepedia\ref{nissepedia} förespråkar såklart alla människors lika värde, men om denna samhällsgrupp skulle försvinna under mystiska omständigheter skulle vi inte ställa några knepiga frågor utan bara gå vidare med våra liv i en lite, lite bättre värld.

\ditem[Sonett (engelsk)]\label{sonett (engelsk)}
 Den engelska sonetten är en diktform med ursprung i medeltiden\ref{medeltiden}, men som användes utbrett först under renäsanssen eftersom den inte räknas som en sakral diktform (folk diggade Gud\ref{gud} nåt så in i vassen på medeltiden). Den behöll sin populäritet genom den engelska litteraturhistorien enda fram till modernismens formexperiment och krav på en mindre bunden poesi. Sonetten har ett fjortonradigt versmått och består av tre strofer om fyra rader plus den omisskänneliga tvåradiga coupletten. Varje rad har oftast tio stavelser, och metern är ofta iambisk pentameter. Raderna är slutrimmande och rimschemat är som fäljer: a-b-a-b-c-d-c-d-e-f-e-f-g-g. För att förklara vad detta schema innebär följer här en sonett författad av den kände skalden Prof. Etienne\ref{prof. etienne} som exemplifierar hur raderna rimmar:

\begin{itemize}
\item Hade det inte varit så att far (a)
\item i vredesmod och med vidgad anal (b)
\item alltid vrålade \quotetext{Se till att bli klar!!} (a)
\item hade jag nog kissat som en normal (b)
\item Hade det inte varit för farmor (c)
\item som alltid fick mig att känna mig ful (d)
\item där jag stod i mina nya steppskor (c)
\item hade jag kanske ibland haft det kul. (d)
\item Hade jag bara gått min egen väg (e)
\item när bror, som skulle gadda kroppen (f)
\item övertalade mig att följa med (e)
\item hade jag inget hakkors på snoppen (f)
\item Men för mig blev det inte riktigt rätt (g)
\item och så slutar här min lilla sonett (g)
\end{itemize}

\ditem[Sonny Listons son]\label{sonny listons son}
 Sonny Liston var en ball boxare från USA som var stor och snäll men som också tyckte om att supa och knarka ganska mycket. Han dog ung och det finns misstankar om att hans död var \textit{foul play}. Det har det pratats ganska mycket om sen hans död. Vad som inte pratats om lika mycket är att Sonny Liston och hans fru adopterade en son när de var på besök i Sverige i början av 1960-talet. Pojken hette Daniel. Det finns inga bilder på Daniel eller någon information om vad han gör idag. Om Daniel Liston läser det här kan du väl skicka oss på Nissepedia ett e-mail? Vi är nämligen alla jättenyfikna på hur i helvete ditt liv kan ha sett ut. Ha det bra! MVH, Det glada gänget på NP.

\ditem[Sopletare]\label{sopletare}
 I fattiga delar av världen finns det för många inget annat sätt att finna föda än att rota i andra människors sopor. I rikare delar av världen visar därför medelklassungar sin sympati med dessa desperata och hungrande människor genom att själva rota i soporna utanför matvarubutiker så som ICA och konsum\ref{konsumbutik}. Dessa människor kallas dock vanligtvis inte sopletare, utan går under mer snajdiga namn på utrikiska, så som \quotetext{freegans} eller \quotetext{dumpster divers.} En av Nissepedias medarbetare vet speciellt mycket om detta ämne och brukar inte sällan bjuda på bananer\ref{banan} och godissnören som någon tagit en tugga av och ba \quotetext{Blä!} och sedan slängt i ett dike.

\ditem[Sorbet]\label{sorbet}
 är ett annat ord för mosad calippo. En sorts förklädd isglass, alltså. Om man glömmer en öl i frysen lite för länge får man en smaklig sorbet späckad med B-vitaminer.

\ditem[Spanien]\label{spanien}
 är ett ganska fyrkantigt land på den iberiska halvön längst ned i Europa. Till och med oseriösa länder som Italien och Frankrike ligger högre upp. Närheten till det varma Medelhavet gör att spanjacken gärna vill svalka sig och det gör hen helst med en petflaska sangria eller ett glas spanskt lättvin\ref{spanskt laettvin}. Framåt kvällen grillas en hel spädgris och för underhållningen står en trubadur som spelar spansk gitarr och sjunger om hur vacker hens mamma är.

\uline{Spansk kultur}

Vara otrogen och kasta getter från kyrktorn. Förtrycka nationella minoriteter.

\uline{Spaniens historia}

Morerna försökte styra upp detta kaos men elefanterna trivdes inte så dom åkte hem och det var ju synd. Hur man lyckades kolonisera typ halva världen är helt obegripligt. 

\uline{Kända personer från Spanien}

\begin{itemize}
\item Manuel\ref{manuel}
\item Cato Fong
\item Don Quixote
\item Enrique Iglesias
\item Alberto Contador
\end{itemize}

\uline{Spaniens framtid}

En sten som faller i vakum.

\ditem[Spansk haloumi]\label{spansk haloumi}
 Riven hushållsost.

\ditem[Spanskt lättvin]\label{spanskt laettvin}
 tillverkas genom att slå lite vatten i ett glas vin\ref{vin}. Det har många likheter med danskt lättöl\ref{dansk laettool}, men en stor skillnad och det är att när dansken bara dricker ett glas så drar spanjacken i sig en hel pava. Det här är fortfarande inget hinder för att dra ut och köra bil\ref{bil} eller backa med släp\ref{backa med slaep} för den delen.

\ditem[SPAP]\label{spap}
 är en förkortning för Supa På Annan Plats, ett klassiskt knep som kan förgylla vilken unken lördag som helst. När man konsumerar alkohol görs det ofta i en begränsad fysisk sfär (typ i bersån på Hallonvägen 2, Scharinska villan eller Lottas krog), vilket efter ett tag kan börja kännas rätt gammalt. Tricket består i att komma på en ny miljö att inmundiga alkohol i och således ge längre livslängd till sysslan att supa. Nedan följer några förslag på andra platser att supa på: i en sommarstuga, på den grekiska restaurangen Costas i Umeå, en idyllisk klipphäll, i Valhall - de fallna kämparnas sal, vid återvinningsstationen på Berghem, Skellefteå\ref{skellefteaa}, i en biosalong, ensam vid Umeälven med Sabbath i lurarna, ensam på sitt rum - lyssnandes till Floyd, i morsan \& farsans garagekällare, i Libanon\ref{libanon}, runtomkring Axtorpet\ref{axtorpet}, domarstolen på en tennisbana, i en övergiven borgruin, på Berlin Alexanderplatz.

\ditem[Speedos]\label{speedos}
 är ett slags småbyxa\ref{smaabyxa} för män och används i första hand på stranden och andra vattenrelaterade platser. Ibland kallas småbyxan därför också badkalsong. Speedon inger enligt vissa en känsla av frihet\ref{frihet} och ledighet, medan andra emotsätter sig dess ärliga framhävande av kroppens eventuella skavanker. I Europa är speedon mest populär i Tyskland och Grekland, men har också haft en storhetsperiod här i Norden under 70- och 80-talen, innan den blev utkonkurrerad av dess storebror, badshortsen.

\uline{Tillvägagångssätt vid påklädning}

För att ta på dig ett par speedos lägger du dessa framför dig på marken eller golvet, sedan kliver du med fötterna ner i de för detta ändamål ägnade hålen i plagget med vardera foten. OBS! Akta så att ingen står bakom dig och ser på! Därefter tar du tag i kalsongens överkant och drar den över knäna och så småningom upp i skrevet. Nu är det dags att justera dragskon\ref{dragsko} och knyta en rosett av snörena samt kontrollera att allt som ska vara inne i speedon är inne i speedon. Om allt är som det ska har du lyckats med att ta på dig badplagget och kan nu sakta jogga fram och tillbaka längs stranden, skutta omkring i vattnet eller vad du nu vill göra.

\ditem[Spegel]\label{spegel}
 är ett slags gest som används för att värja sig när någon pekar finger åt en. Gesten består i att handflatan riktas i den riktning som den obscena gesten kommer från. På så vis signalerar man att den som pekar finger åt en misslyckas med sitt tilltag och i själva verket pekar finger åt sig själv. Då får de något att tänka på.

\ditem[Spegelmöte]\label{spegelmoote}
 Det fenomen som uppstår när två människor möts och håller på att krocka, och försöker kliva åt sidan fast åt samma håll. De krockar igen och en humoristisk situation uppstår. Detta spekuleras av glädjevetenskapliga\ref{glaedjevetenskaper} arkeologer vara världens äldsta skämt.

\uline{Skämtet på stenåldern}

Det första registrerade fallet av detta är en grottmålning från en grotta i Frankrike som har daterats till att vara skitgammal. En grottmänniska var på väg hem till grottan med en sabeltandad tiger på ryggen, varpå en annan grottmänniska ser det och tänker \quotetext{Fan va pjäxfett\ref{pjaexfett}, undrar om det finns mer sabeltandad tiger?} och ger sig av mot grottmänniska A. De möter varandra på stigen och håller på att krocka men viker av åt samma håll och båda grottmänniskorna håller på att garva läppen av sig.

\uline{Skämtet på medeltiden}

Två arméer är ute och krigar och möter varandra ute på slätten\ref{slaett}. Just de här arméerna hade inga konkreta planer på att döda och lemlästa varandra så de tar alla ett kliv åt sidan, men tror du inte att de går åt samma håll! Ett masspsykotiskt skratt bryter ut i de båda härarna och sen slår de ihjäl varandra i alla fall, för medeltiden\ref{medeltiden} var en jävlig tid.

\ditem[Sportmössa]\label{sportmoossa}
 är en matematisk term som används vid beräkningar av sannolikhet. Om något helt saknar sportmössa är det närmast otroligt att det ska inträffa, och då skriver man i sin matematiska beräkning (eller rejält tilltagna höftning) att: \quotetext{det finns inte en sportmössa att det händer}. Etymologiskt härstammar termen från den amerikanske brottaren Rulon Gardner som lyckades med det osannolika att vinna en match mot Aleksandr Karelin\ref{aleksandr karelin} och blev så glad att han åt upp sin hatt. 

\ditem[Spritfylla]\label{spritfylla}
 En brinnande dimma, ett skenande lok, en röst som skriker NEJNEJNEJNEJ!!!!!!!
Vakna med ett halvbrunnet stearinljus i handen och en avsliten handklov kring handleden, en trasa kloroform i fickan, tjugo meddelanden i röstbrevlådan som alla ber för ditt liv, bakgrunden på mobilen bytt från din glada golden retriever med sina valpar, till ditt uppkåtade kön. Karatesparka en jävel på käften, försök smälta en kundvagn med telekinesi, grilla korv med tändsticka, bär underkläder som hatt, skrik allt vad du orkar och kanske 110 \% till. Ge allt, lite till maxa maxa mer. 

\textit{Spritfylla}

\ditem[Spritvev]\label{spritvev}
 Handgemäng under påverkan av rusdrycker. Vanligt i byarna runt Avesta.

\ditem[Spunka]\label{spunka}
 (verb, obestämd form singular; bestämd form spunk) är en multipel handling där utövaren spyr över sitt egen könsorgan i kombination med ejackulering. Fenomenet har blivit mindre vanligt på senare tid i och med utedassens dalande popularitet och Onkel Kånkels tragiska bortgång.

\ditem[Sputnik]\label{sputnik}
 är inte bara Sovjetunionens triumf över den fascistoida västvärlden. Det är även en grusåkare och countrysångare från Telemark. Han är mest känd för att ha spelat på Röda Torget strax innan imperiets sammanbrott, och för att ha släppt över 20 kassetter där han på de flesta har en fimp i mungipan. Rökning kan inte vara skadligt då Sputnik är sådär 70 år.
Sputnik säger själv \quotetext{Imorrn är det kanske ingen som gillar Sputnik, men grus och sten - det ska dom ha!}.

\ditem[Spärrballong]\label{spaerrballong}
 En spärrballong är ett gammalt beprövat försvar mot ovälkomna fiender som attackerar från luften på låg höjd. En stor ballong i slitstarkt material fylls med gas och förankras i ett kraftigt rep\ref{repet}. Den lågflygande fienden vill förhoppningsvis inte riskera att skada sin maskin på repen, och lyfter därför mot högre höjd där den får svårare att sikta och samtidigt blir ett lättare mål för luftvärn. Ponera att du strövar omkring på ditt gods och tittar till att bokskogen växer som den ska, med saltbössan i hand. Plötsligt hör du ett surrande ljud\ref{surrande ljud}, och upptäcker att en av dina edsvurna fiender från grevskapet bredvid är på väg rakt mot ditt slott i en fullrustad Messerschmitt. För hundra år sedan hade du varit rökt, och antagligen samma sak \textit{om} hundra år. Men är du rätt rustad kan du nu tack och lov möta det stundande hotet genom att hissa dina i förväg gasfyllda, och för en förmögenhet inköpta, spärrballonger. Därefter har du bara att bege dig till ditt luftvärn och skjuta ner den edsvurna fiendens flygmaskin. I Sverige fanns länge bara en enda reparationsverkstad för spärrballonger, en sektion på Degerfors järnverk. Idag finns ingen alls.

OBS: \textit{Nissepedia uppmuntrar på intet sätt till användning av attack från luften på låg höjd som ett sätt att vedergälla edsvurna fiender. Den som ämnar genomföra en sådan aktion får absolut inte använda sig av artikeln för att undkomma eventuella spärrballonger. Överträdelser kommer anmälas till luftfartsverket}.

\ditem[Stark mat]\label{stark mat}
 är motsatsen till svag mat\ref{svag mat}. Stark mat kan för en del människor förknippas med resor till exotiska länder och spännande upplevelser. Händer som möts och oväntade möten med trummor och sång. Det finns undersökningar som visar att stark mat har en välgörande effekt på hälsan och själen. Den store naturfilosofen och rabulisten Linné anmärkte varnande: \quotetext{Vid wal av föhdoämnen skall du eftersträäwa sådana som i och för sig hava smak eller angenäm lukt; de starkt luktande och de retande skall du undewika}.
Exempel på stark mat är \quotetext{Chili con carne}.

\ditem[Steglits]\label{steglits}
 (Carduelis carduelis) är en fågel i familjen finkar i naturens stora släkt. Den är färgrann och fin och stor som två snusdosor ungefär. Den har också en näbb som den använder för att äta med, vilket den måste för annars dör den av svält.

\ditem[Stenad]\label{stenad}
 På en klippa vid en svensk insjö. Lite ont i lungorna just nu men trettio sekunder senare har det förbytts mot en torr mun\ref{mun}. En klunk folköl och allt är bra. Ett samtal om en björn i manchesterbyxor. .... Vad är det läte du lyssnat på de sista minuterna? En spillkråka? Kanske. Om rymden är oändlig och dess innehåll är en finit massa så borde det betyda att det inte finns något utanför de himlakroppar som är längst ut i universum. Och om resten är tom är det bara potentiellt rum och då är väl universum, för fan, inte oändligt. Eller...? Hur det än är med det så borde man inte bry sig så mycket om sånt där, för det viktigaste är väl hur man har det här och nu. Fan en snus hade ju inte vart fel. Jävlar vad jag vill lyssna på Hawkwind just nu. Fast trummorna i den här låten är ju helt sjukt feta. Fast inte lika bra som i den där låten som går typ så här ba: dump ba-dump ba-du-u-dump kwrääääng! Tänk om jag skulle kunna välja att inte finnas i typ ett år och sen komma tillbaka och bara vara som vanligt.: undrar hur folk skulle förhålla sig till mig då? Antingen skulle de bara vara som vanligt eller så skulle de på något vis vara mer distanserade - även om vi bara talade om världslig skit som tvättider och vem som spelade gitarr i vilket band och gud\ref{gud} vad jobbigt det är med relationer!!!! Vad fan är det han heter nu igen? Fan jag har det fan på tungan. Eller i bakhuvet\ref{huvud}...eller vad man nu brukar säga. Säger man så: \quotetext{i bakhuvet}? Varför det? Typ som att man hade ett lagerskåp med olika tankar som var ordnade i mappar i bakhuvudet. Just det jävlar! Rasmus Nalle\ref{rasmus klump} heter den lilla jäveln! Björnen i manchesterbyxor. Vad skulle hända om månen plötsligt träffades av en meteorit och exploderade? Skulle jorden då sugas in i sin närmsta grannplanet och gå under som i Motörheads \textit{Metropolis}? Vad fan är det han snackar om nu igen? Just det, väderkvarnar! \textit{\quotetext{Väd-er-kvar-nar}}. Vilket jävla sjukt ord egentligen. Hahaha! \textit{\quotetext{Egen-kli-en}}. Hahaha! Kan man bada så här års? Mjaaaäääääj.

\textit{Stenad}

\ditem[Stenlapp]\label{stenlapp}
 En stenlapp är en bit sten som man karvat in något i för att minnas det. Idag är det inte supervanligt att folk använder denna fiffiga minnesteknik då vi har filofaxer, men i det gamla Mesopotamien var det ohyggligt vanligt förekommande. Där ristade glatt alla hemmafruar in mjölk och ägg\ref{aegg} på en bit kvarts och Babylonska borgmästarns talksrivare log ofta i godan ro när han hamrade in ca 40 000 miljarder\ref{fyrtiotusen miljarder} tecken i stenlappar gjorda av utsökt granit. De mest kända stenlapparna är de tio budorden, vilka Gud\ref{gud} gav till människan för att de skulle komma ihåg några grejer han funderat på ett tag. Mindre kända stenlappar med mer obskyrt innehåll, typ böner tillägnade några halvluriga babylonska gudar, finns för allmän beskådan på Pergamon Museum i Berlin.

\ditem[Stentrollsaffär]\label{stentrollsaffaer}
 En stentrollsaffär är en affär där man kan köpa stentroll, trädgårdstomtar, vargar av keramik, en och annan träbjörn\ref{traebjoorn} och andra sådana grejer. Stentrollsaffärer både drivs av och riktar sig till kvinnor i medelåldern. Stentrollsaffärer är ofta kombinationsaffärer. Trots att stentrollsaffärer sällan är väldigt lukrativa finns det ett mycket stort antal sådana, speciellt i Norrtälje\ref{norrtaelje}. Anledningen till detta är att det är många tanters dröm att driva en stentrollsaffär och få göra precis som de vill där inne. Vill de ta in en speciell keramiktomte som skjuter en liten skottkärra som man kan placera en hysaint i, till exempel, så tar de helt enkelt in ett gäng tomtar som skjuter små skottkärror. Ingen kan säga emot. Tanten är chef och det är hon som bestämmer. Likaså, om hon inte vill ta in tomten så tar hon inte en den. Så enkelt är det.

\ditem[Stia]\label{stia}
 Stior är utrymmen som brukas av två typer av varelser; präster och grisar. Prästens stia kallas sakristia och fungerar som det man i dagligt tal benämner omklädningsrum. Grisens stia kallas svinstia och fungerar som det man i dagligt tal benämner crust as fuck existence. Hur detta hänger ihop är ett riktigt trepipsproblem\ref{trepipsproblem}.

\ditem[Stig]\label{stig}
 En stig är den klart mest demokratiska formen av väg. Ingen övermäktig stat som bestämmer vars man får gå och inte, inte heller något företag som tar ockerpriser för anläggningen av underlag som asfalt och grus. Nej, en stig är summan av olika människors vilja att ta sig fram på egen hand, precis där de vill gå. Stigar är sällan till problem, om man inte är stighatare.

\ditem[Stigma]\label{stigma}
 är ett begrepp som uppfanns av Jesus\ref{jesus} men sedan återuppfanns av sociologen och teaterentusiasten Erving Goffman. Ett stigma är något som försvårar det sociala livet avsevärt.

\uline{Exempel på stigman}

Stigma kan anbringas av sådant som armsvett, synlig stjärtskåra, finnar, högt uppdragna byxor, vara med i Agnostic Front, veganism och/eller tjockisflås.

\ditem[Stjärtlapp]\label{stjaertlapp}
 är den sämsta av alla utförsåkningsredskap. Medan storfräsarens\ref{storfraesare} unge bränner ner i hissnande hastighet på sin sprillans snowracer med fotbroms och tuta sparkar du frenetiskt med fötterna för att ta dig över första guppet, men det går segare än Sober-Jimmys\ref{sober-jimmy} gitarrsolon. Du ägde en gång en klassisk orange pulka men plasten i den var så torr att farsan klev rakt igenom den i somras när han var lite ouppmärksam. Du tänkte inte så mycket på det då, eftersom det var strax efter midsommar och mest tyckte det var drygt att få skäll för att du inte tagit undan den än. Men nu sitter du där, med pissblöta tummvantar i ylle och Galne Gunnars kopia av en bävernylonoverall, och förbannar den värdelösa jävel som uppfann stjärtlappen. Det faktum att den också låg ute på grusgången hela sommaren\ref{sommar} har inte direkt sänkt friktionsgraden. Möjligtvis fick den ett tunt vallaliknande lager när farsan använde den för att skotta undan hundskit, men det förutsätter att du vänt den sidan nedåt vilket du naturligtvis glömt. Så där sitter du nu, med hundskit i röven och svett och snor i ansiktet, och funderar på om stjärtlappsjäveln åtminstone är hård nog för att du ska kunna halshugga stekarungen med den.

\ditem[Stockholm]\label{stockholm}
 är ett ställe som ligger söder om Norrtälje\ref{norrtaelje}. Här finns \quotetext{allt}. Det vill säga moderater\ref{moderat}, microbryggerier och kaknästornet\ref{kaknaestornet}.

\ditem[Stockholmare]\label{stockholmare}
kör obegagnat och lämnar in på märkesverkstad, kremerar katten hos veterinär, anammar yngre generationers mode, känner bara till en måttenhet och det är kvadratmetern samt tänker inte på sig själv, utan hundralappar, när det är val. 

\ditem[Stonerskin]\label{stonerskin}
 Ett stonerskin är en person som deltar i en subkultur som är en blandning mellan skinhead-\ref{skinhead} och stonerrockkulturen. Personen har kort hår och polisonger, bär ofta cammoshorts eller uppvikta jeans, Dr Martens, Fu Manchu-munkis och under denna en Goatsnake- eller Cockney Rejects-tshirt. De flesta stonerskins läser en hel del böcker, målar tavlor, doktorerar, bygger choppers, dricker öl, lyssnar på stoner och Oi!, åker på litteraturkonferens i Sydafrika och Australien och har djur, växter och psykadeliska torn tatuerade på sina armar.

\uline{Musik}

Stonerskin lyssnar ofta på sådana band som The Oppressed, Kyuss, Cock Sparrer, Earthless, Last Resort, Farflung, 4skins, Bongzilla, Blitz, Sleep, The Templars, God Grows his Own, Oxblood, Super Joint Ritual, Bonecrusher, Jajayra, Gundog, 500 ft. of Pipe, Sham 69, Weedeater, Angelic Upstarts och Black Pyramid. 

\ditem[Stor-Anders]\label{stor-anders}
 har fått sitt namn inte bara för att han är stor utan också på grund av hans kollega Lill-Anders som för övrigt är gift med Anki. Stor-Anders är vaktmästare på Leksands Folkhögskola där alla ser upp till honom för den karlakarl han är. Han kan lyfta stora saker som ingen annan på skolan skulle kunna rubba. Dessutom fixar han allt som behövs. T.ex, om skolan skulle behöva bygga ut så skulle han bygga ut skolan på egen hand, utan att de skulle behöva anlita ett helt byggföretag. Stor-Anders håller så hårt på sitt hockeylag\ref{hockey}, Leksands IF, att han tar ledigt från jobbet för att se dem spela kvalserien oavsett var i landet de spelar. Det sägs att han skall vara över 190 cm lång. 

\ditem[Stora Grabbars och Tjejers Märke]\label{stora grabbars och tjejers maerke}
 är en utmärkelse som delas ut till särskilt framstående idrottsutövare i Sverige. Utmärkelsen har funnits sedan 1928, och för att bli stor grabb/tjej krävs att man samlar på sig ett antal poäng under sin aktiva karriär. Poängräkningen ser lite olika ut inom olika sporter men vanligt är att landskamper och internationella medaljer ger poäng. Varje ny medlem får också ett unikt nummer, och Pelle Fosshaug\ref{pelle fosshaug} är exempelvis \quotetext{stor grabb nummer 197} i bandy. Nummer ett innehas av Sven \quotetext{Sleven} Säfwenberg, och nummer noll av Sune Almkvist. Det är oklart varför ett så anrikt pris är så pass okänt som det är bland de breda folklagren. Kanske är medlemmarna rädda att någon ska stjäla deras nummer.

\ditem[Storbossnörd]\label{storbossnoord}
 En storbossnörd är en nörd som lagt ned alla anspråk på att passa in i samhället i stort. Vanliga människor är så jobbiga och fattar inget så han (det finns kanske tre hon i hela världen också) har lagt ned det där och kör sitt eget race fullt ut istället med mjukisbyxor i skolan och cykelhjälm. En vanlig nörd kan normalt förstå varför vissa människor till exempel föredrar att köra bil framför att åka tåg. Storbossnörden har noll förståelse för detta eftersom tåget erbjuder så mycket större möjligheter att använda restiden till att spela Gameboy eller läsa regelböcker. Ett sådant kompromisslöst beteende borde rimligtvis försätta storbossnörden i en rad onödigt besvärliga\ref{kineseri} situationer. Men storbossnörden har för länge sedan funnit bot på detta besvär och rör sig helt enkelt bara i sammanhang där han inte ifrågasätts. Den enda risk han löper är i princip att stöta ihop med en annan storbossnörd och då kan det gå desto hetare till. Om två storbossnördar råkar i muntlig dispyt är det högst troligt att de blir ärkefiender och att oenigheten kommer följa dem i graven. En storbossnörd har nämligen aldrig fel.

\ditem[Storfräsare]\label{storfraesare}
 En storfräsare är en person som inte tvekar att ta på sig spenderarbyxorna när chansen finns att göra sig lite märkvärdigare än alla andra. Det är Levis 501or istället för Konsum-jeans, och riktiga sommardäck istället för att rycka dubben ur dom slitna vinterdäcken som gäller. På barnkalas ska man inte bli förvånad om storfräsarens unge dyker upp med en magnumflaska Pommac, medan knegarungarna får hålla sig till varsin 33 cl Champis\ref{champis}. Självklart har ungen ett sånt där sugrör som är format som ett par glasögon också. Men egentligen är det inget att vara avundsjuk på, för innerst inne mår storfräsaren dåligt över att aldrig kunna uppnå den obetalbara njutning en arbetare kan känna när hen unnar sig en kubb till kaffet på en tisdag eller bränner fram på en solbelyst landsväg med Bruce Springsteen i kassettdäcket.

\uline{Exempel på storfräsare}

\begin{itemize}
\item Thomas Jisander
\item Jan Guillou
\item Modebloggare
\item Drottningen av Storbritannien
\item Wall Enberg
\item Alexandra Pascalidou
\end{itemize}

\ditem[Storhetsvansinne]\label{storhetsvansinne}
 ska man inte skoja om. Det är en åkomma som ofta drabbar skäggiga män i Tyskland\ref{tyskland} (på grund av den tyska mustigheten)\ref{den tyska mustigheten} men har också påträffats annorstädes. Storhetsvansinnet utbryter hos individen i en rad olika steg, som redogörs för nedan:

\begin{enumerate}
\item Om han är man skaffar individen\ref{individ} skägg (och kanske fax)\ref{fax}
\item Ut med det gamla skrället och in med en SUV\ref{uv-ljus}
\item Individen köper en Chapeau de paysan\ref{chapeau de paysan}, med fjäder längst opp
\item Nu vill individen lära sig spela ett blåsinstrument, kanske basklarinett
\item Individen livnär sig till 87\% på mörkt bröd, typ pumpernickel
\item Olika slags medaljer och märken dyker upp på individens rock- eller kappslag
\item Individens hus har fått ett torn. Han/hon bär särk\ref{saerk} och talar till \quotetext{Davids folk}
\item Individen är inte längre individen, utan \quotetext{Ezekiel}
\item \quotetext{Ezekiel} står länge på höga kullar och ser ner på världen
\item Så, plötsligt en dag, självantänder han/hon och återföds som en uv\ref{uv}
\end{enumerate}

\uline{Vanliga utlösande faktorer}

Ofta utlöses vansinnet av att man

\begin{itemize}
\item Har för vana att vakna varje morgon till tonerna av Strauss' \textit{Also spracht Zarathustra.}
\item Livnär sig på en kost som till stor del består av viltkött man jagat själv.
\item Skaffar fler än tre barn, som brukligt.
\item I sin fågelskåderi-gärning börja närma sig slutet av listan på Sveriges\ref{sverige} alla fågelarter.
\item Vid upprepade tillfällen ligger med kulturstockholm\ref{ligga med kulturstockholm}.
\item Sätter alla rätt på stryktipset.
\item I all välmening tar syra och råkar hitta ett subliminalt budskap på Hawkwinds gamla klassiker \textit{In Search of Space}.
\item På fyllan dissar Torbjörn Tännsjö genom att ropa okvädesord och flaxa med armarna.
\item Blir övervakad av FRA.
\item Av tankspriddhet läser klart \textit{På spaning efter den tid som flytt} av Proust.
\item Får blankt protokoll på bilbesiktningen.
\end{itemize}

\ditem[Storsien]\label{storsien}

 är en mindre ort i Kalix kommun i Norrbotten\ref{norrbotten}. Där fanns ett arbetsläger dit staten antagligen hade skickat samtliga Nissepediamedarbetare\ref{nissepedia} om det velat sig riktigt illa.

\ditem[Storspov]\label{storspov}
 (Fornsv. \textit{stor} ung. \quotetext{ansenlig}, Forndan. \textit{spov} ung. \quotetext{mjödhorn}) är en vadarfågel (världens största!!!) i familjen snäppor. Storspoven är brungråspräcklig med en smal vit triangel från stjärtens bas upp på ryggen, stor som en bärsback ungefär och har en lång, nedåtböjd näbb som den använder för att äta med. Näbben ser ungefär ut som en sån där sked man äter sniglar med.

Storspovens anseende upprätthålls av FFSSB\ref{ffssb}.

\ditem[Storspren]\label{storspren}
 en är ett djur som återfinns sporadiskt i Sveriges\ref{sverige} två nordligaste län. Den är nära släkt med fabeldjuret grip, men istället för en korsning mellan ett lejon och en örn är det en korsning mellan en storspov och en ren.

\uline{Föda}

Storspren äter mycket sällan, och då den väl gör det är den väldigt kräsen och äter bara lappskojs\ref{lappskojs}. Resten av tiden livnär den sig på upplevelser, som till exempel att flyga omkring och titta på saker.

\uline{Fortplantning}

Storsprenen är könlös, eller tvekönad, beroende på perspektiv\ref{perspektiv} och lägger ägg. Äggen är stora som ensilage, men skickligt kamoflerade så att de liknar gråsten. Efter en veckas ruvning kommer Storsprenkalven till världen, dock utan vingar. Ungarna föds upp genom att få höra skrönor\ref{skroona} av de äldre storsprenarna. Sen när vingarna kommit skickas de ut i världen för att själva uppleva saker. T.ex. att bli dumpade, besöka Sagrada Familia i Barcelona, ligga med klasskompisar på en folkhögskola, se solen gå upp över Sarek och allt annat man kan uppleva här i världen.

\uline{Storsprenen i kulturen}

I den förkristna samiska kulturen fyllde storsprenen jultomtens funktion, fast istället för presenter i materiell form så gav den bort, ni gissade det, upplevelser. Dessa kan te sig svåra att slå in, så de samiska barnens jular var extremt tråkiga.

\ditem[Storswänsk]\label{storswaensk}
 De flesta Storswänskar bor i storstäder och den högsta populationen har uppmätts i Stockholm. En storswänsk är en som anser att det finns en svensk nationalidentitet, givetvis utformad av akademiker i Uppsala\ref{uppsala}, Stockholm\ref{stockholm} men till viss del även i Göteborg\ref{gooteborg}. Enligt Storswänsken är Sverige\ref{sverige} ett land och inte som det egentligen är en samling av stater för de har vapenmakten att hävda det. De förespråkar en ekonomisk politik som handlar om att suga ut de delar av landet som de ockuperar. Lögner är även frekvent i deras propaganda, som att Norrbotten\ref{norrbotten} kostar mer än det generar eller när och hur Svea har befolkats.

\ditem[Streiff]\label{streiff}
 var den häst\ref{haest} Gustav II Adolf red åt helvete på i Lützen. Pollen klarade slaget men dog några dagar senare på väg tillbaka till Sverige\ref{sverige}. Precis som med kadavret efter kungen bestämde sig soldaterna för att släpa hem den döda hästen till slottet. Kungen var så illa åtgången så honom kunde man inte göra något åt, men hästen var ännu så pass fräsch att det gick att stoppa upp den.

Streiff var en betydligt större och snabbare häst än vad som var vanligt i den svenska armén, så kanske (förmodligen) var det dennes fel att kungen vart kanonmat eftersom den sprang före alla andra. Hade det inte varit för detta hästskrälle hade kanske krigarkonungen levat än idag. Det får vi aldrig veta.

Den som vill visa sitt hån mot denna landsförrädare kan gå till Livrustkammaren där den står än idag och skäms.

\ditem[Strelka]\label{strelka}
 var den tredje hunden i omloppsbana i rymden.

\ditem[Streptokocker]\label{streptokocker}
 är en bakteriefamilj som växer i par eller i kedjor. Streptokocker kan exempelvis leda till kvarka. Fast egentligen är dom inte så farliga bara man ser upp.

\ditem[Stress]\label{stress}
 är en känsla och ett tillstånd som infinner sig till exempel när man är i slutskedet av sitt avhandlingsprojekt eller om man sitter instängd i en 1x1m stor bur och någon står och tutar mot en med en sån där hockeytuta. Stressen tar sig olika uttryck. Beroende på om det ses som ett tillstånd eller en känsla leder detta tillstånd antingen till att vissa känslor infinner sig eller så leder stresskänslan till ett pärlband av andra emotionella hemskheter. Man blir arg. Man blir aggressiv. Man blir förvirrad och disträ. Man blir känslig. Kombinationen av att vara stressad, arg, aggressiv och förvirrad kan leda till många olika tragedier och dråpligheter, som normalt i sin tur leder till skam\ref{skam}. Men frukta icke! Det finns vissa mått man kan tilltaga för att gardera sig då stressens mörka moln samlas över en:

Kräma på lite sjysst rymdig doom och öppna kylen. Knäck en folle och känn hur stressen sjunker undan samtidigt som vissa delar av ljudbilden liksom träder fram. Tänk på att det faktiskt är något utfattigt jävla geni som skapat denna fantastiska effekt som tidigare helt undgått dig. Byt så småningom till nån mer mellow rymdrock eller kraut. Drick té. Läs en bok.

\ditem[Strulputte]\label{strulputte}
 En strulputte är någon som har strulat till det för sig eller är i färd med att göra det. Kanske har man varit hemma hos en polare och rökt gräs ur bong och lyssnat på Bob Marleys \textit{Uprising!} Senare, på vägen hem, har man kanske fulkopplat TV8-profilen Lars Adaktussons Golf GTi 2004, blivit stoppad av länsman\ref{polis} och oförhappandes hamnat i handgemäng. Exempel på kända strulputtar är:

\begin{itemize}
\item Burre\ref{burre}
\item La Camilla
\item Kungen
\item Christer Pettersson
\item The Hof
\item Mona Sahlin
\item Kurt Cobain
\item Tomas de Quincey
\item Hunter S. Thompson
\item Melvin Gibson
\end{itemize}

\ditem[Strumpor]\label{strumpor}
 är ett slags cylindrar av tyg som sytts samman längst ned och används för att beklä fötter. Många människor i västvärlden byter strumpor dagligen. Bland de som inte gör det finns en viss överrepresentation av unga killar utan karriärsplaner. Storfräsaren\ref{storfraesare} bär golfstrumpor med diskret mönster och betalar 150 spänn paret. För vanligt hederligt folk duger det bra med tubsockor med logga från nåt lokalt åkeri som man fått som tröstpris på pimplingstävling eller jaktstig.

\ditem[Stryknin]\label{stryknin}
 är ett gift som heter som det gör för att man \quotetext{stryker med} av det.

\ditem[Stråtrövare]\label{straatroovare}
 Ett anrikt yrke för den som inte har något att förlora, och går ut på att ta den ekonomiska omfördelning som staten misslyckas med i egna händer. Stråtröveriets ädla konst är lika gammalt som äganderätten\ref{aeganderaett} och går ut på att med våld avtvinga någon pengar eller annat som man behöver. Till skillnad mot det närliggande yrket tjyv så går stråtrövaren inte in i folks hem för att tillskanska sig ägodelar, utan låter istället berget komma till Mohammed. Stråtrövaren arbetar istället längs rikets landsvägar, beväpnad med förslagsvis ett armborst eller en knölpåk, och väntar på att en vagn ska passera. I bästa fall är det självaste kungen och stråtrövaren kommer därifrån med en skattkista full med guld, diamanter och smaragder, i sämsta fall är det en bonde på väg till marknaden och bytet en tunna sättpotatis. I dagsläget ägnar sig inte särskilt många åt stråtröveri utan de flesta, påverkade av förbränningsmotorns upptäckt i seklets början, tog steget att bli vägpirater.

\ditem[Strömavbrott]\label{stroomavbrott}
Lampan slocknar utan att du tryckt på knappen. Ditt WiFi är stendött. Makaronvattnet slutar koka på plattan. Du är drabbad av ett strömavbrott.

Vi är fortfarande till viss del slavar under naturens regler, och i Sverige blir det sällan så tydligt som under strömavbrott. Vårt moderna samhälle bygger på förmågan att behärska en fientlig omvärld. Det är vad som ger oss tid till att titta på Paradise hotel och skriva putslustiga men ändå intetsägande blogginlägg om att borgerlig politik är dålig. Men utan tillgången till allt det, vad ska vi då ta oss till?

Vi kan prata med varandra, uppfinna spel som ingen tänkt på förr och i det flackande sken som Coops värmeljus kastar över ett rum, avslöja våra mörkaste hemligheter. Men en konsekvens av att samhället är modernt är att vi är medvetna om att strömavbrottet är en parentes i vår vardag. Det är denna vetskap vad som hindrar oss från att gå fullkomligt bananas och plundra skivbutiken på hörnet så snart en glödlampa pajar. Snart kommer elen tillbaka, och med den all anständighet och respektabilitet som det moderna samhället avkräver oss. Och då måste vi sluta prata med varandra och kasta bort våra nyuppfunna spel (som i stort sett alltid inbegriper experimenterande med kroppsöppningar). Sist släcks värmeljusen och vi försöker desperat glömma bort att vår bästa vän just avslöjade att den ibland på fyllan skriver krönikor i Neo under pseudonym. I gengäld förväntar vi oss att vännen också glömmer vad vi just avslöjade, kanske att vi i hemlighet uppskattar dansk reggaeton.

Med elen kommer den bekväma men också lite kvävande filt av lögner som låter oss leva förhållandevis städade liv. Strömavbrottet är ett bryskt men frigörande uppvaknande, en påminnelse om tider som varit och djuret som lever i oss alla. Ett djur som ska respekteras, men måste tyglas.

\ditem[Strövtåg]\label{stroovtaag}
 SJs skyddshelgon.

\ditem[Stöcksjö sunny resorts]\label{stoocksjoo sunny resorts}
 SSR, är ett turistmål i en by belägen en mil söder om Umeå centrum. Bland utbudet som erbjuds så utlovas något slags safari med bl.a. myskoxar. Det sorgliga med denna fattiga turistattraktion är dock att det inte alls är myskoxar utan två nordsvenska hästar som ägaren kastat två bruna trasmattor över ryggen på. För att ge hästarna horn har slanet\ref{slan} försett deras huvuden med cykelstyren och färgat dessa en färg snarlik hornfärg.

Alla som besöker gården erbjuds att ta med sig sin egen vikt i metallskrot på vägen därifrån. De har helgöppet.

\ditem[Stödkorv]\label{stoodkorv}
 När man beställer ett skrovmål på sin lokala grill och inte vill lida hungerns kval ända tills det är klart kan beställningen med fördel kompletteras med en korv, en s.k. stödkorv. Korvarna tar inte lika lång tid att tillaga som gängse hamburgertallrik och tillfredsställer därför kunden mer eller mindre omedelbart.

\ditem[Sugmästare]\label{sugmaestare}
 En sugmästare är en person som förstör för andra, medvetet eller omedvetet\ref{det omedvetna}. Det tydligaste exemplet är Metallicas nye basist Robert Trujillo som är jätteduktig på sitt instrument men ändå gjort att man vill spy när man hör gruppen och fått Cliff Burtons lik att ruttna lite snabbare.

Sugmästare förhåller sig antonymt till ägmästare\ref{aegmaestare}.

\ditem[Supa ensam]\label{supa ensam}
 Att sätta sig ner med några starköl eller en flaska sprit och bara låta det ske. Finns det något finare? Många ser ner på folk som super ensamma, men belackarna är i nio fall av tio tråkiga töntar som är för fega för att sätta sig på en pinnstol, vrida på Pink floyd på helgvolym\ref{helgvolym} och med några stadiga järn som färdmedel företa sig en resa till sitt eget inre.

Men allt ensamsupande innebär inte spirituella resor. Ibland kan det vara schysst att bara kolla på larviga djurklipp på youtube, höhöhö:a lite och bli så full att man glider av stolen.

Kändisar som är bra på att supa ensamma:

\begin{itemize}
\item Ulf Brunnberg
\item Burres\ref{burre} farsa i Bamse
\item Ika i rutan (även om hon småfuskar, då hon krökar tillsammans med sin skelettpolare Åke)
\end{itemize}

\ditem[Surfin' bird]\label{surfin bird}
 är ett fågeldjur inom familjen gäddoppingar. Den förväxlas lätt med sin nära släkting hackspetten i Kalle Ankas julafton men känns igen på sitt karaktäristiska läte \quotetext{papa-oom-mow-mow} i stället för \quotetext{ara-pa-pa-pa-pa-pa-pa-pa-pa-pia}, som den sjunger högt och ofta. Arten återfinns vanligtvis på nedlagda soptippar i Danmark där den lever på cigarrstumpar och tuggummin. På power meet-helgen flyger eller liftar den till Västerås för häckning och flipperspelande.

Andra arter inom familjen gäddoppingar:

\begin{itemize}
\item Hjulben
\item Palle Kuling\ref{palle kuling}
\item Duffy duck
\end{itemize}

\ditem[Surrande ljud]\label{surrande ljud}
 är ett ofta förekommande miljöproblem i vår allt mer maskinella omvärld. Surrande ljud är inte bara irriterande utan kan också vara skadliga då de kan orsaka hörselskador och skapa stressyndrom så som huvudvärk, sömnproblem och fetor ex ore\ref{fetor ex ore}. Därför är det viktigt att kunna lokalisera orsaken till surrande ljud så att de om möjligt kan avlägsnas ut den direkta ljudmiljö i vilken man framlever sina dagar.

Hör du ett surrande ljud bör du kontrollera om det kommer från de ljudkällor som listas nedan. Om ingen av följande källor verkar vara orsaken till ljudet bör du inkalla en ljudexpert\ref{hoogtalartips}.

\begin{itemize}
\item Grannen har startat sitt Messerschmitt och har satt kurs mot ditt gods\ref{spaerrballong}
\item Alexander Bard har dykt upp i TV-rutan och talar om sin liberalism
\item Gammalt kylskåp
\item Dålig datafläkt
\item CD-skiva som sakta slipar ner laserhuvudet\ref{huvud} i din CD-spelare
\item Fluga mellan fönsterglasen
\item Någon står bakom dig och talar till dig med väldigt låg röst
\item En liten robot har tagit sig in i byggnaden och åker omkring och rekognoserar terrängen.
\end{itemize}

\ditem[Svag mat]\label{svag mat}
 är motsatsen till stark mat\ref{stark mat}.
Svag mat kan för en del människor förknippas med pensionärsmiddagar med ett stilla tickande från moraklockan som soundtrack. Blickar som möts och väntade möten med byte av stomipåse och tantsång\ref{tantsaang}.

Det finns inga undersökningar som visar att svag mat har en välgörande effekt på hälsan och själen. Exempel på svag mat är \quotetext{Ängamat} och \quotetext{Krösamos}.

\ditem[Svan]\label{svan}
 är ett djur som under evolutionens snirkliga resa försetts med vingar och näbb, vilket kvalificerar den till gruppen fåglar. Den har vit kropp och svart huvud, och påminner därför lite om sin nära släkting späckhuggaren. En fullvuxen svankropp är stor som en bag-in-box ungefär och halsen är lång som en Uzi. Den är en flitig utlandsresenär under vinterhalvåret. Som många vet bildar svanar par och håller sedan ihop resten av livet. Vad många inte vet är att dom hatar varandra.

\uline{Svan som råvara}

Svan är en av ingredienserna i uvsvane\ref{uvsvane} och svanskrove och huvudråvara vid tillverkningen av svanväskor.

\uline{Trivia}

Svanen är det djur som har förärats störst antal fjädrar och är därmed jobbigast att plocka. Detta leder ofta till att man (i likhet med potatis) stuntar i att \quotetext{skala} den innan förtäring.

\ditem[Svart alibi]\label{svart alibi}
 Fordom tida personifierades det svarta alibit av Onkel Tom.
Numera av Nyamko Sabuni som dessutom dubblar som folkpartiets kvinnliga alibi\ref{kvinnligt alibi}.
Ett svart alibi behöver varje organisation som vill framstå som fri från fördomar trots att man driver en politik i klass med Sydafrika före 1994.

\ditem[Svarta tavlan]\label{svarta tavlan}
 är ett uråldrigt pedagogiskt hjälpmedel, avsedd att skriva mattetal, tyska verbböjningar och lärarens för- och efternamn på. Man använde en vit bit krita för att skriva på den svarta tavlan och i enlighet med 1900-talets fäbless för barbari och blodsutgjutelse användes avhuggna harlemmar för att sudda på den. Redan under dess storhetstid var alla, både lärare och elever, införstådda med att det måste finnas ett bättre system för lärande än svarta tavlans dammiga gnisslande och den skolkritslunga som många ådrog sig av inandning av kritdamm. Under 90-talet kom svarta tavlan att ersättas med den glassiga whiteboarden och sedemera den interaktiva tavlan.

\ditem[Svensk bilsemester]\label{svensk bilsemester}
 är en riktig klassiker i de breda folklagren, men hos storfräsare\ref{storfraesare} duger det naturligtvis inte att tvinga in familjen i 740:n, bränna ner till Kolmården och sedan slå upp ett tält och supa sig full. Hos storfräsaren är det Plaja del sol och komplicerade transaktioner på utrikiska som gäller. Svensk bilsemester är trots allt precis det storfräsaren skyggar för: enkelhet och frihet och en fantastisk chans att samtala med andra bilister om det alltid för höga bensinpriset.

\uline{Tips inför bilsemestern}

\begin{itemize}
\item Föreställ dig att du tagits som fånge av en framtida postapokalyptisk civilisation vars enda kvarvarande nöje består i att anordna death races. Vägen är en tävlingsbana och alla andra bilister är dina fiender. Arbeta upp en alarmerande hög stressnivå och ge dig sedan på att komma först i alla lägen som kan tänkas dyka upp.
\item Packa tills bilen är full. Finns det plats kvar är det bara att fylla på med grillkol.
\item Låt påskina att du menar allt du säger på största allvar genom att vända huvudet lite åt sidan och skrika allt du orkar åt barnen i baksätet. Redan vid första fikapausen på en skräpig parkeringsficka vid Sveriges\ref{sverige} stolta del av Europaväg 4 ska barnen helst vara rejält rädda för dig.
\item Peka ut betonganläggningar som skymtar från motorvägen och påstå att du varit där när du låg i lumpen. På så vis ger du de vettskrämda barnen ett stycke familjehistoria samtidigt som de lär sig något om vikten av rikets försvar.
\item Efter att nogsamt ha studerat GBs priskarta, inför utan förklaring regeln att man bara får välja en glass för en summa som gör det omöjligt att ta en som är god. På så vis blir semestern en pedagogisk lektion som illustrerar att även om man kan ha det bra och roa sig ibland så är det för det mesta jävligt torftigt.
\end{itemize}

\ditem[Svenska jägareförbundet]\label{svenska jaegarefoorbundet}
 Tidigare namn på Sverigedemokraternas seniorförbund. Ändrades på anonym gruppbegäran när en pedofilring tog samma namn och man ville undvika risken att medlemmar glömde betala medlemsavgift till båda.

\ditem[Svenska Kennelklubben]\label{svenska kennelklubben}
 Intresseförening för män och kvinnor som heter \quotetext{Kennel}. Föreningen har idag inga medlemmar, men märkligt nog finns det 5 personer som heter Kennel.

\ditem[Svenska Kennetklubben]\label{svenska kennetklubben}
 Intresseförening för män vilka bär namnet Kennet.

\ditem[Svenskt näringsliv]\label{svenskt naeringsliv}
 är en nationalsocialistisk organisation som tidigare gick under det mer tydliga namnet Svenska Motståndsrörelsen.

\ditem[Sverige]\label{sverige}
 Avlångt land i närheten av Östersjön. Lever på ryktet om sin fantastiska välfärd, eftersom nyheten om att den nyliberala regimen har avskaffat densamma ännu inte nått ut till utländska nyhetsbyråer.

Man kan vara nazist utan att älska Sverige, men man kan inte älska Sverige utan att vara nazist.

\uline{Lagar och förordningar}

I Sverige kör man bil på höger sida av vägen och det har svenskarna gjort sedan 1967. Det råder också tyst trafik sedan 1935, det vill säga att man inte ska tuta i onödan, utan använda tutan till dess ursprungliga funktion; hälsa på folk man känner.

\uline{Ekonomi}

Sverige har sålt alla sina stora industrier till Polen, Kina osv. och de som inte blev sålda sattes i konkurs eller flyttades till Pajala.

\ditem[Sverigedemokraterna]\label{sverigedemokraterna}
 (skämts. sammansättn. a. platsnamnet \quotetext{Sverige\ref{sverige}} och gr. \quotetext{demokrati}) är en partibildning som i huvudsak samlar sydsvenska grisbönder, innebandykillar och samtliga anställda inom norrortspolisen. Ideologiskt står man nära Sveriges andra högerpartier och röstar i nio fall av tio med den borgliga alliansen. SD har under sin första mandatperiod haft två prioriterade politiska arbetsområden.

\begin{itemize}
\item Att arbeta oförtröttligt med att genom parlamentariska beslut nedrusta välfärden och försämra arbetstagarens ställning på arbetsmarknaden.
\item Att kasta ut alla invandrare för att det är deras fel att välfärden nedrustats och arbetstagarens ställning på arbetsmarknaden försämrats.
\end{itemize}

\uline{Kulturpolitik}

Som symbol har man en \textit{chic} blå-gul blomma, för det finaste som finns är ju Sverige, och det näst finaste är ju påhittade blommor.

Partiets officiella sång heter \quotetext{tjalalalala}.

\uline{Kritik}

SD har sedan urminnes tider av ingen anledning motarbetats av ett osynligt imperium bestående av vanvettiga ockultister som arbetar som/förkläder sig till frilansande journalister. Dessa hackar inte sällan sverigedemokraters facebook-\ref{facebook} och twitterkonton och skriver korkade rasistiska och homofoba saker.

\ditem[Sveriges sju underverk]\label{sveriges sju underverk}
 utsågs i två omröstningar som, oberoende av varandra, hölls av Aftonbladet och P1s Vetenskapsradion Historia under nådens år 2007.

I Aftonbladet blev följande byggnadsverk utsedda till underverk:

\begin{itemize}
\item Göta Kanal (ej att beblanda med filmen \quotetext{Göta Kanal} med bla. Janne \quotetext{Loffe} Carlsson)
\item Visby Ringmur
\item Regalskeppet Vasa
\item Ishotellet i Jukkasjärvi
\item Turning Torso
\item Öresundsbron
\item Globen\ref{globen}
\end{itemize}

I P1s Vetenskapsradion Historia:

\begin{itemize}
\item Göta Kanal (ej att beblanda med filmen \quotetext{Göta Kanal} med bla. Janne \quotetext{Loffe} Carlsson)
\item Ales Stenar\ref{ales stenar}
\item Malmö Moské
\item Lunds Domkyrka
\item Karlskrona (?)
\item Visby Ringmur
\item Hällristningsområdet i Tanumshede
\end{itemize}

Den kritiske historiekännaren kan påpeka att följande imponerande bedrifter skamlöst glömdes bort vid omröstningen:

\begin{itemize}
\item Pudaslådan\ref{pudaslaada}
\item Kaknästornet\ref{kaknaestornet}
\item Ornässtugans dass\ref{ornaesstugans dass}
\item Polhemshjulet
\item Kebabnekaise
\item Göta Kanal (filmen med bla. Janne \quotetext{Loffe} Carlsson)
\end{itemize}

\ditem[Svinpäls]\label{svinpaels}
 Negativt laddad synonym för extra gemena\ref{jaevelskap} storfräsare\ref{storfraesare}. Tänk dig en ulv i fårakläder, men som i brist på äkta fårskinn istället tagit en gammal dammig grishud och limmat på lite svinto med billigt epoxylim. Det är skitigt, det är oanständigt, det är riktigt jävla lågt. Bara en äkta svinpäls skulle göra något sådant.

\uline{Kända svinpälsar i populärkulturen}

\begin{itemize}
\item Karaktären Svinpäls i Disneys \textit{Räddningspatrullen}
\item Saruman i J.R.R Tolkiens\ref{j.r.r tolkien} böcker
\item Gandalf i Åke Ohlmarks\ref{aake ohlmarks} \textit{Tolkien och den svarta magin\ref{tolkien och den svarta magin}}
\item Den onda kvinnan i Black Sabbaths låt \textit{Evil woman}
\item Tjorven i \textit{Vi på Saltkråkan}
\end{itemize}

\uline{Kända svinpälsar i finkulturen}

\begin{itemize}
\item Hela familjen Bonnier
\end{itemize}

\uline{Kända svinpälsar i fulkulturen}

\begin{itemize}
\item Charlie Sheen
\item Johnny Takter\ref{johnny takter}
\item Fernando Torres
\end{itemize}

\ditem[Sviskonpaj]\label{sviskonpaj}
 Björnligans favoritpaj.

\ditem[Svotto]\label{svotto}
 Fornnordiskt namn som var vanligt åtminstone järnåldern ut. Namnet har under modern tid återigen blivit populärt, främst genom den episka diktsviten \quotetext{Badgirl eller par}, skriven av den länge anonyme författaren, politikern, och välrenommerade akademikern Svotto Littorin.

\textit{Här: erfaren välutrustad}
\textit{bestämd social och seriös}
\textit{Man i Sthlm.}

\textit{Där: söt nyfiken tjej}
\textit{eller par som behöver}
\textit{och vågar.}

\textit{Är du nyfiken på bdsm,}
\textit{ageplay, hårda tag}
\textit{mm så hör av dig.}

(Utdrag ur \quotetext{Badgirl eller par})

Den ursprungliga betydelsen av namnet tros ha varit \quotetext{farlig plats} eller kanske \quotetext{han som känner bra könsumgänge}.

\ditem[Svälta räv]\label{svaelta raev}
 är världens i särklass tråkigaste kortspel. Till och med att förvara leken i asken och gissa vilket kort som ligger överst är roligare. Svälta räv går ut på att \quotetext{svälta} motståndaren genom att komma över alla dennes kort. I början av spelet delas leken upp jämt mellan spelarna som alla förvarar sin del i en hög på bordet med ryggen\ref{rygg} uppåt. Alla spelare drar det översta kortet från sin hög. Om korten har samma färg vinner spelaren med den högsta siffran korten. Om färgerna är olika drar man nya kort tills de överensstämmer och så där håller det på. Det enda positiva med svälta räv är att det faktiskt inte tar så lång tid att avsluta ett parti som man minns från sin barndom, och att det är ett bra sätt att lösa konflikter på.

91:an spelar mycket svälta räv vilket speglar seriens kvalité väl.

\ditem[Svåra saker]\label{svaara saker}
 blir lätt fel.

\ditem[Schwarzwald Larsson]\label{schwarzwald larsson}
 En filmsnut i de populära \textit{Beck}-filmerna. Han är farfar till den mer framträdande karaktären Gunvald, men förutom det har de inte mycket gemensamt. Om man tycker barnbarnet är hårt är det ingenting mot vad Schwarzwald är. Det är därför han inte syns i filmerna, han är så rå och veril att folk blir förskräckta och stänger av när han dyker upp i rutan.

\ditem[Syfilis]\label{syfilis}
 Upptäcktes till skillnad från Amerika inte av Columbus utan av hans matroser. Vanligtvis har sjukdomen fått sitt namn efter staden, orten eller landet varifrån folk kom hem med \quotetext{morbus gallicus}. En uppsjö av namn således: franska sjukan eller franzosen, dock ej att förväxla med en parisare\ref{parisare}. Även Neapel, Kina och Tyskland\ref{tyskland} var länge populärt. Till Sverige och Norden kom syfilis med det saxiska gardet under Junker Schlentz. Roskilde begåvades ett nådens år 1497 med den \quotetext{frandzoske siuge oc kranchedt}. 1508 kom sjukdomen till Finland\ref{finland}, hädanefter alltsomoftast kallad \quotetext{syffe}. Den olycklige libertinen, riddaren Åke Jöransson, såg på sin dödsbädd ljuset, grundade franciskanerorden.

\ditem[Sylt]\label{sylt}
 är en ö i Nordsjön som tillhör förbundslandet Schleswig-Holstein inom Förbundsrepubliken Tyskland\ref{tyskland}. Geografiskt ligger ön på samma latitud som södra Alaska.

Den största kommunen på Sylt heter Sylt-Ost. Kommunen består av flera mindre byar och har sammanlagt ungefär 5.500 invånare. På hela Sylt bor det ca. 21.000 människor, dubbelt så många som i hela Härjedalen. Det är mycket.

\ditem[Syndikalism]\label{syndikalism}
 bygger på idén, hopkokad av hålögda, frihetshatande gangstrar, att man ska skänka värdet av det man skapar varken till storkapitalet eller partiet. Istället har man fräckheten att tänka sig att man arbetar för varandra, utan chef, men med en tydlig demokratisk och kooperativ struktur där besluten tas gemensamt, arbetet görs gemensamt och det värde man skapat delas rättvist. Denna fascistoida idé har som tur är gjorts till gemensam måltavla för Svenskt näringsliv, Socialdemokraterna, Alliansen, Timbro, större delen av den svenska journalistkåren, diverse motorcykelgäng och nynazister och andra frihet-\ref{frihet} och rättviseivrande aktörer. I morgondagens fräscha och avreglerade samhälle kan vi därför förhoppningsvis framleva våra liv utan denna styggelse till ideologi.

\ditem[Synonymer för anus]\label{synonymer foor anus}
Svenska breda vokabulär innehåller en rad synonymer för anus.
\begin{itemize}
\item Knuten
\item Pruppen
\item Skitkiken
\item Ana
\item Dajmkrysset
\item Pussmunnen
\item Grötbössan
\item Rossen
\item Dajman
\item Fisgluggen
\item Brunan
\item Hästögat
\item Tvåan
\end{itemize}

\ditem[Syo]\label{syo}
 står för Studie och Yrkesvägledning och syftar till att leda ungdomar rakt ner i fördärvet. Tjänsten innehas ofta av en människofientlig och kraftigt närsynt tant. Det kan rentav vara så att detta är en en del av kravspecifikationen för att få förleda.

\ditem[Säcklöpning]\label{saeckloopning}
 är en idrottsgren som går ut på att förflytta sig framåt i en jutesäck\ref{saerk} på kortast möjliga tid. Sporten uppfanns av den senile indianen Trötta Sommarkatten som råkade ta på sig sin poncho upp och ned. Säcklöpningstävlingar är mycket publikvänliga tack vare att deltagarna ofta snubblar på sig själva;\ref{semikolon} försök komma på något som är roligare att titta på liksom.

Grovt räknat finns två olika taktiker att välja mellan. Antingen kan man hoppa jämfota eller så tar man snabba myrsteg inne i säcken. Det första är jobbigare och det andra är vingligare. Deltagaren bör därför börja med att fråga sig vad som känns värst: att bli svettig eller dratta på arslet. Båda alternativen är ju ganska tråkiga så kanske gör du istället bäst i att sitta kvar på gräsmattan och ta en bärs till.

\ditem[Sälar]\label{saelar}
 (Pinnipedia) finns i tre familjer; öronsälar, öronlösa sälar och valrossar. Valrossen är den häftigaste av dessa med sina balla sabeltänder. Sälen spenderar större delen av sin vakna tid med att ligga och jäsa eller ta det lugnt. Den lever främst på fisk men vissa arter har även utvecklat en febläss för krill.

I modern tid är sälen kanske mest känd för att ha gett upphov till det smygborgerliga Miljöpartiet\ref{miljoopartiet} som utnyttjade sälens efterblivna utseende till att fiska röster. Sälen är dock av naturen på intet sätt knuten till besvikna proggare. Punkbandet Happy Farm tog exempelvis ställning i sälfrågan i en låt med textraden: \quotetext{Vill du döda en säl? Nej, tack!}.

Att kalla någon för en säl betyder att man anser personen i fråga ha en aningen trind kroppshydda\ref{kroppshydda}.

\ditem[Sällskapsresan]\label{saellskapsresan}
 \textit{Sällskapsresan, eller Finns det svenskt kaffe på grisfesten?} (1980) är den mest framgångsrika filmen i riket Sveriges\ref{sverige} historia om man räknar till antalet gånger den spelats ombord på bussar som går till skidorter, äventyrsbad\ref{aeventyrsbad} och scoutläger. \textit{Sällskapsresan} handlar om en lång, smal, så kallad tönt (spelad av Musse Pigg-fetishisten Lasse Åberg) och en norrman (spelad av småskurken\ref{smaaskurk} Jon Skolmen) som blir polare under en resa till medelhavet. Väl framme i semesterorten råkar de ut för en massa motgångar, men får till sist ligga med två lapplisor. Filmen gick direkt till den svenska publikens hjärtan, delvis på grund av det inte finns så mycket annat att se på, förutom Colin Nutleys\ref{colin nutley} \quotetext{filmer}.

\ditem[Sämskskinn]\label{saemskskinn}
 är en speciell form av mjukbehandlat skinn med väldigt hög uppsugningsförmåga och användbart till väldigt mycket. Till exempel att putsa dragbasuner\ref{dragbasun} och bilar\ref{bil}. Det kan också användas till att filtrera bort vatten i bensin om det skulle behövas. Skinnet framställs ifrån djurhudar genom fettgarvning och kommer idag oftast ifrån get, men ibland även ifrån ko eller älg. Nuförtiden tillverkas det, enligt Flashback, även på syntetisk väg. Huden skrapas ren ifrån hinnor på köttsidan och hår på narvsidan. Sedan smörjer man in hudens köttsida i fett, ofta använder man djurets hjärna. Hjärnan måste dock kokas tills den blir vit och sedan svalna om denna skall användas. Karl XIIs drabanter hade sämskat älgskinn under sina bröstharnesk. Det bästa är finskt sämskskinn.

\ditem[Särk]\label{saerk}
 var en av två sorters klädesplagg som existerade på medeltiden\ref{medeltiden} (det andra var byxor men var inte lika vanligt). Särken är vanligtvis brun eller grå och ser mycket eländig ut. Den slutar en bit ovanför knäna och var man så lyckligt lottad att man hade att par byxor kunde man ha den instoppad så såg man istället mer fånig än bedrövlig ut. I princip vad som helst kan bli en särk men vanliga material är linnetyg och potatissäckar. Särken var ett sätt för staten att se till att medborgare föraktade sig själva och andra och stöptes i samma form, ungefär som skolan fungerar i Jan Björklunds\ref{jan bjoorklund} Sverige\ref{sverige}.

På senare tid har särken fått en renässans och ingår som klädkod hos bland annat Lidl, Tåg i Bergslagen, Ryan air, Ica och Överskottsbolaget\ref{te}.

Turistbyrån i Norberg har som tradition att påtvinga den minst gillade arbetskamraten detta plagg.

\ditem[Särske]\label{saerske}
 är ett adjektivt som betecknar att någonting är undermåligt, dumt eller konstigt på ett dåligt sätt.

\uline{Exempel}

\quotetext{Vad fan gör du, är du särske eller?}
\quotetext{Jag är så trött på den här jävla särske-cykeln\ref{christianiacykel}.}

\ditem[Sågverk]\label{saagverk}
 Har du någonsin hållit i en två tum fyra och funderat \quotetext{Hur är det möjligt?}. Då kan du mycket snart sluta undra. Här följer nämligen en redogörelse för hur ett modernt sågverk fungerar. Empirin bygger på 6 års erfarenheter av sågverket i Malå\ref{malaa}.

\uline{Från skog till sågverk}

En tall tornar upp sig majestätiskt över en myrhed. Den svajar lätt i vinden och omgivningen är väldigt fridfull. Men vad nu? Ljudet av motorbuller skär genom landskapet och vid horisonten uppenbarar sig en skördare. Denna skogsmaskin har bara en uppgift: att göra om den frodiga skogen till karga kalhyggen. Tallen kapas just ovanför roten och avkvistas på en gång, mycket fiffigt. Tallen, som nu blivit en timmerstock, läggs åt sidan och skördaren fortsätter sitt korståg genom skogen. Sen kommer skotaren, vars uppgift är att plocka upp stockarna och köra dem till närmsta väg. Där väntar en lastbil som kör stockarna i ilfart till närmsta (i idealfallet, på grund av kapitalismens irrationella verkningar blir det dock sällan så) sågverk.

\uline{Mätning och avbarkning}

Tallen ligger nu på ett lastbilsflak tillsammans med sina artfränder. Lastbilen parkerar på sågverket efter vägning och en stor traktor med en ännu större grip lyfter av ett knippe stockar och placerar dessa på mätstationens timmerbord. Här skakas stockarna en och en upp på ett rullband och mäts med hjälp av laser eller nåt annat högteknologiskt och sorteras sedan i olika fack. Samma traktor lyfter sedan upp stockarna och placerar dessa på ännu ett timmerbord, det som leder in i barkmaskinen. Samma princip som i mätstationen, men istället för att mätas åker de genom en så kallad barkmaskin som, ni gissade det, skalar av barken med hjälp av en osthyvelsliknande grej. Stocken fortsätter sedan sin resa mot själva sågen.

\uline{Sågning och paketering}

Stocken åker med en våldsam fart genom en maskin utrustad med två lodräta klingor, sen en maskin med två horisontella. Vips är stocken inte längre karaktäristiskt rund, utan fyrkantig. Sedan klyvs stocken av en tredje maskin till brädor eller plankor - skillnaden här emellan är så komplex att den kommer avhandlas för sig senare. Plankan och brädan sorteras sedan var för sig och åker genom ytterligare en station där de mätas och sorteras i olika fack för att bli till virkespaket när ett tillräckligt antal plankor eller brädor uppnåtts. Virkespaketen skickas nu till den så kallade läggaren som ser till att virket ligger i snygga rader med torkströn emellan, men torkströna sköts av en annan maskin som har en tendens att jävlas väldigt mycket, till allas förtret. När paketet staplats till önskvärd höjd åker den ut ur sågen och får en lapp häftad på sig där det står vilket mått som virket har, samt ett serienummer för att signalera precis hur unikt varje paket är.

\uline{Torkning}

Trä, i egenskap av att vara ett levande material, innehåller vatten. Det här vattnet ska inte vara kvar i den slutliga produkten för då blir den inte särskilt rejäl, som de flesta som köper trä vill att det ska va. Därför torkas virket i de fantasilöst döpta torkarna. Hade jag bestämt hade det hetat virkesbastu. I alla fall, virket bärs till torken av en truck. Det finns två sorters torkar, vanliga och kanaltorkar. Vanliga torkar ser ut som jättelika garderober där paketet ställs in för att sedan tas ut från samma håll. Kanaltorkarna har portar i två ändar och virkespaketen åker på en räls genom torken. Det spekuleras i att virkespaketen tycker att det här är jätteroligt, men detta har inte kunnat styrkas empiriskt. I torkarna blir det väldigt varmt, säkert 100 grader, och allt vatten i trät försvinner. Vill man att det ska vara lite vatten kvar går det säkert också att ordna. Man tar inte reda på vattnet utan låter det försvinna ut i atmosfären så att det blir regn.

\uline{Justering och/eller hyvling}

Virket har när det kommer ut från sågen samma mått på höjden och bredden, men inte på längden. En del som köper virke vill att det ska vara samma längd på alla brädor eller plankor och det här sköts på justerverket. Processen liknar den på sågen, efter det att stocken kluvits. En truck kör virkespaketen hit och virket blir sedan mätt, sorterat, kvalitetsbedömt, kapat och paketerat så att kundens alla önskemål uppfylls och tillråga på allt får paketen en lapp med specifikationer och serienummer.

Om kunden inte önskar få en massa stickor av sitt virke måste det dock hyvlas. Det här sköts på det återigen fantasilöst döpta hyvleriet. Här skrapas det yttre lagret trä (där stickorna sitter) bort med hjälp av en jättelik elektrisk rubank. Virket blir slätt och fint och lämpligt att använda till en uppsjö saker, t.ex. sommarstugor, dansbanor och Trojanska hästar. Ska träet vara utomhus så måste nästa steg genomföras.

\uline{Impregnering}

Virket doppas i en bassäng fylld med en massa mer och mindre farliga kemikalier så att inget vatten kan tränga in mellan träets fibrer. Sen åker det ut och är redo att möta alla moder jords påfrestningar, utom eldsvådor. Det är något av träets akilleshäl.

\uline{Den eviga debatten: planka eller bräda?}

En bräda blir en planka när den överstiger måttet 38 X 100 mm. Den minsta brädan som skickas ut från Malå Sågverk är 19 X 100. Den största plankan är 63 X 200. När det sågas 19 X 100 så är tempot på sågverket väldigt lågt och alla är glada och hälsar glatt. När det sågas 63 X 200 är det vansinnigt hektiskt och dålig stämning är kutym. Så nästa gång ni ska använda detta monster till planka tycker jag att ni ska tänka på hur mycket slit som ligger bakom.

\uline{Sågverkets fauna}

Det dominerande inslaget i faunan är knäppskorven, eller snytbaggen som Carl von Linné\ref{carl von linné} döpte den till. Den älskar trä och på sågverk finns denna vara som bekant i överflöd. De flesta knäppskorvar kommer till den sista vilan genom att åka med virkespaket i torkarna. Här samlas döda knäppskorvar i stora högar och sopas på somrarna ut av ortens ungdomar. Timmermannen är en annan insekt som trivs på sågverk. Dennas främsta egenskap är dess långa horn, varav hanen har längst. Ibland landar hanarna i intet ont anandes sommarjobbares skägg och panik\ref{panik} utbryter, men de gör inte så mycket mer än att bara sitta och idla. Honorna däremot bits ganska hårt men gör det bara i självförsvar. Vidare kan i alla fall Malå Sågverk stoltsera med ett visst bestånd av ren. Dessa söker sig till området i jakt efter skugga och ligger ibland och softar i spånhögar. Tack vare trucktrafiken lever dessa renar farligt och blir därför ibland avhysta, men det är för deras eget bästa. Det har påträffats benrester av ren på området, så det är ingen säker plats för våra beklövade vänner. Inne i sågen brukar svalor bygga bo under semesteruppehållet, men när maskinerna startar upp igen antar jag att de drar någonstans där bullernivån är betydligt lägre. Det finns också ett rikt bestånd av människor. Dessa spenderar dagarna med olika former av hårt arbete och livnär sig på kaffe och matlådor. Dessa för också in ytterligare en art i biotopen, nämligen hunden. Dessa sitter ofta i arbetarnas bilar och gnyr, men ibland sitter de fast med koppel i bilens kula och skäller på förbipasserande truckar. Traktorförarna som har små hundar brukar skjutsa runt dem i traktorerna.

\uline{Klasskamp på sågverk}

Den överlägset vanligaste formen av klasskamp som bedrivs är att \quotetext{Livsfarlig Ledning}-skyltar sätts upp på chefers dörrar. En anekdot berättar om hur en arbetare en gång gått in på kontoret på Malå Sågverk och sagt \quotetext{Om vi int få en krona till i timmen, då stann vi av!}. Chefen hade inget annat val än att höja lönen, och den kronan, den finns kvar än idag! Även en bild på Setras (som äger sågverket) VD Bengt Börjesson har hittas på en anslagstavla på Malå Sågverks justerverks anslagstavla, med ett häftstift rätt i pannan! Obehagligt.

\ditem[Sårrengöringsvätska]\label{saarrengooringsvaetska}
 Vätska för rengöring av öppna sår, ögon, näs- och munhåla. Kiss är den renaste som finns. Kejsaren Vespasianus blev till exempel hårt åtgången av sina samtida då han lyckades med att både strypa katten å ha den kvar. Ergo - han lät ta skatt på urinoarerna samtidigt som urinet samlades in för de romerska legionernas behov av sårtvätt. Han lär då ha sagt \quotetext{Non olet}, det luktar inte. I historieböcker beskrivs det ofta som om att det är pengarna som inte luktar, hans skamliga profit. Vad vi nu vet är att han syftade på kisset - det är det renaste som finns.

\ditem[Söndag]\label{soondag}
 en är den sista, och enligt många den keffaste, dagen i veckan. Andra, däribland Micke Alonzo\ref{micke alonzo}, menar att måndagen\ref{maandag} är sämst. Söndagen är i kristendomen vilodagen och kallas i Bibeln\ref{bibeln} den sjunde dagen. På söndagen är nästan inget öppet, ingen vill göra något och det finns inget att göra.

\uline{Regionala seder}

I Västerbotten\ref{vaesterbotten} är det vanligt att man ber och skäms på söndagen.


%%%%%%%%%%%%%%
\newpage
\null
\\
\null
\\
\Huge
T
\normalsize
\\
\null
\\
\null
%%%%%%%%%%%%%%


\ditem[T-rexarmar]\label{t-rexarmar}
 har man enligt personer bosatta i Roslagens famn om man har svårt att nå och därför till exempel tvingas använda krattan när man spelar biljard eller måste stå på en stol för att kunna ta ner glas eller tallrikar från skåpet ovanför diskbänken.

\ditem[Ta för sig]\label{ta foor sig}
 Att ta för sig kan antingen vara något väldigt fult eller något väldigt fint, beroende på perspektiv\ref{perspektiv}. Att ta för sig, oberoende av hur man ser på det, går ut på att man inte väntar på sin tur eller att man inte är beredd att dela på kakan med andra, utan att man helt sonika kliver fram och tar det man vill ha. Är man tokliberal\ref{tokliberal} är det väldigt fint att ta för sig: så fint, faktiskt, att tokliberalerna ofta får något sentimentalt över sig och de genomgår en känslostorm som inte sällan gör att en tår sakta rullar nedför deras rödlätta kinder, likt hos inbitna RKU:are när de ser en montage-bild av vietnametiska bönder som broderligt delar på en riskaka. Att ta för sig är nämligen det som nyliberalism går ut på. Nyliberaler ser det som ett stort problem att de inte får ta för sig av allt omkring dem och tycker att detta är det värsta av det offentliga samhällets många övergrepp på individen\ref{individ}. Man har till och med utvecklat en egen, mycket framgångsrik, form av feminism som går ut på att kvinnor ska lära sig att ta för sig. Om det i något sammanhang uttrycks missnöje med att någon tagit för sig eller att människor uppmanats att ta för sig brukar tokliberalerna himla med ögonen om tala om jantelagen och att det är \quotetext{typsikt svenskt} att inte ta för sig. Med \quotetext{typsikt svenskt} menar de att det är ett arv från arbetarrörelsens formativa årtionden att det bland gemene man anses ofint att stjäla andras eller kollektiv egendom för eget bruk och nyttjande.

\ditem[Taggen]\label{taggen}
 är en fiktiv karaktär i den svenska dramaserien Tre Kronor. Han har utländskt påbrå och vänstersympatier. Han står med ena foten utanför samhällets normer och regler men följer ett slags inre moral, så han är ganska snäll trots allt. Han har en syrra också, Alma, vilket är sydlänska för \quotetext{träd} - så här har manusförfattarna fört in en viss röd tråd.

\ditem[Talgoxe]\label{talgoxe}
 (\textit{Parus major}) tillhör liksom kopparormen\ref{kopparorm} och uven\ref{uv} de djur som heter en sak men i själva verket är något helt annat. Den är nämligen paradoxalt nog en fågel. Den är väldigt vanlig i Sverige\ref{sverige}, där den bor och lever av att äta frön och talgbollar. Den är gul, vit lite här och var och svart längst upp. Talgoxen tycker om att flyga, sitta i träd och buskage och att äta talgbollar. Enligt ett visst konkurrerande internetlexikon låter fågeln \quotetext{tjitt, tjitt,} \quotetext{ping, ping}. Ibland låter den dock \quotetext{pitt, spick}, vilket ju kan upplevas som lite pubertalt.

\ditem[Tantkläder]\label{tantklaeder}
 Krymplèn, allvädersstövlar, stora halsband.

\ditem[Tantnöjd]\label{tantnoojd}
 Att vara tantnöjd är ett sätt att stävja ångesten inför åldrande och ett ljummet parförhållande genom att finna glädje i de små tingen runt oss. Att ta en promenad en vacker söndag i mars och plocka ett knippe blomster eller få besök av ett litet barn\ref{barn} som säger så många klokheter som vi vuxna borde ha vett att ta till oss. Kanske finner glädje i ett klokt ord med tillhörande naturbild som postats på facebook\ref{facebook} eller i en TV-dokumentär om någon som gått igenom en svår sjukdomsperiod utan att för den delen tappa livsgnistan. Det fina med tantnöjdheten är att den i sig genererar mer material för tantnöjdhet i form av tänkvärda ord som sprids medelst broderier eller hemgjorda vykort, facebook-uppdateringar och kylskåpsmagneter. En dag kommer kanske tantnöjdheten att sprida sig över vår blåa planet och göra krig och elände till ett avlägset minne. Man kan bara hoppas. Och tro.

\ditem[Tantsång]\label{tantsaang}
 är en sub-genre inom musiken som utmärker sig på så vis att den är en av få genrer som har gått från att vara mainstream till att bli sjukt underground. En gång i tiden var tansången en av de mest livsstarka genrerna på musikhimlen och folk stod i kö med hatten i hand för att få höra sånger om \quotetext{tunga fjät} och folk som kommer \quotetext{över mon}. Idag är fans av genren tvugna att leta med ljus och lykta efter skivor med tantsång eftersom de flesta skivor med tantsång kasserades i och med att sekuläriseringen av Sverige. När piratradiostationerna Radio Syd och Nord tvingade fram etablerandet av radiokanalen P3 var slaget om skivlistorna förlorat för tantsångsgenren. Idag vet inga tantsångsfans hur man spelar CD-skivor och mp3-filer, vilket omöjliggör tantsångens återintåg på de kommersiella arenorna.

\uline{Tantsång idag}

Idag hålls tantsångspelningar på mindre spelställen som ofta är anslutna till kyrkor, hembygdsgårdar och äldreboenden. I likhet med den centripetala rörelsen inom Norges black metal-scen i början av nittiotalet, då sub-genren och -kulturen vände sig inåt och medvetet gjorde sig otillgänglig för den breda massan, har tantsången under senare år sett till att det är i stort sett omöjligt för det genomsnittlige fanset att veta var och när tantsångsspelningar äger rum. Detta ska dock inte ses som att tantsången är på utdöende - tvärt om: tantsången är mer vital än någonsin, om man tillåts uttrycka det så. I och med att 40-talisterna går i pension och börjar komma till åren förväntas tantsången explodera i aktivitet, om ej försäljningsmässigt. En annan stor förändring som kan skönjas är att influenser idag tas inte bara från den standardiserade psalmboken och ett och annat skillingtryck utan även från sådana genrer som arbetarrörelsens musikarv, dragspelande smilfinkar från \textit{Allsång på Skansen} och ett och annat örhänge från svensktoppens tidiga dagar.

\ditem[Taxichaufför]\label{taxichauffoor}
 En taxichaufför är oftast en person som, utan möjligheten att få betalt för att skjutsa runt folk som har en redig bärsfylla\ref{baersfylla}, skulle gå under i dagens samhälle. Taxichauffören är mer ofta än inte en blekfet antisocial 20-nånting som gillar nihilistisk black metal, en gubbjävel som misslyckats med allt han företagit sig och nu hatar världen, eller en aggressiv stigmatiserad biratant som tycker om att rulla runt i en bil mellan (alt. under) fyllorna. Att dessa förtappade själar skulle krossas av det cyniska marknadssamhället om det inte vore för det sociala skyddsnät som taxiyrket utgör, förhindrar dem inte från att vara tokliberaler\ref{tokliberal} och hylla övermänniskoidealet. Sen finns även de taxichaufförer som utgör den altruistiska motvikten. Tanter, gossar, tjejer, gubbar och iranier som gärna säger att de är en \quotetext{glad skit} och lyssnar på ZZ Top på helgvolym\ref{helgvolym} i bilen. Men de är få, allt för få.

\ditem[Te]\label{te}
 Man blir skitpigg av att dricka te, och så är det med de(t).

\uline{Tillvägagångssätt för att brygga te}

För att brygga en kopp te gör du följande. Koka upp vatten på valfritt vis (kastrull, tekokare, blåslampa, kittel över eld etc). När vattnet kokat upp lägger du en tepåse i en kopp med en volym mellan en och tre deciliter. Kontrollera noggrant att den lilla lappen som är förtöjd i påsen med ett litet snöre dinglar på utsidan av kruset\ref{krus}. Nu häller du sakta på önskad mängd vatten i koppen. På tepaketet kan det stå att tepåsen ska ligga i vattnet i ca 3 minuter. Detta är lögn. Istället tar du genast tag i den redan nämnda lappen och snabbdoppar påsen i vattnet några gånger, sedan slänger du den över gärdesgården och ignorerar den. När temperaturen har sjunkit i koppen och är som du önskar den kan du dricka ditt té.

\uline{Te och samhället}

Arbetarklassen föredrar kaffe, medelklassen dricker te med förtjusning och gärna specialare som \quotetext{lapsang} eller \quotetext{darjeeling} och överklassen är som vanligt helt verklighetsfrånvänd.

\uline{Te i världen}

I Storbrittanien dricks det mycket te (det är, förutom simhallsdoftande kranvatten, den enda icke-alkoholhaltiga dryck som förtärs), och det här anses av de flesta etnologer bero på vita anglosaxernas\ref{anglosax} kollektiv-psykologiska svårigheter att handskas med sitt koloniala arv. I Kina lär det finnas hur mycket te som helst, samma mängd som det finns smör i Småland faktiskt.

\ditem[Teenage Mutant Ninja Turtles]\label{teenage mutant ninja turtles}
 är en USA-berättelse om fyra sköldpaddor som medelst retomutagen och fysisk kontakt med ungdomar (?) blir transformerade till humanoida ninjasköldpaddor som utnyttjas av en likaledes humanoid råtta vid namn Splinter till att slåss mot noshörningar och hjärnor från andra dimensioner. 

Teenage Mutant Ninja Turtles kan inte, som vår egen svenska motsvarighet till dem, Skalman, dra in huvudet i skalet eller ens gömma saker i det. Däremot kan de röra sig mycket snabbare än Skalman någonsin skulle vilja göra. Skalman kan uppfinna saker, men det kan en av Teenage Mutant Ninja Turtles också göra, om än inte lika finurliga grejer. Skalman är smart och pacifist, medan Teenage mutant Ninja Turtles inte är lika smarta och inte skyr våld i kampen för rättvisa. Skalman bor i ett fett hus, medan Teenage mutant Ninja Turtles bor i kloakerna i New York. Skalman har tagit sig ur ett omfattande drogmissbruk, vilket skildras i en serie från 1988, medan Teenage mutant Ninja Turtles uteslutande äter pizza och inte har några betänkligheter med det. Så till \textit{syvende og sisdt} får man nog säga: Skalman - Teenage mutant Ninja Turtles, 1-0.

\ditem[Tegare]\label{tegare}
 En tegare är i Umeå en korv i bröd med allt på. Tegaren är döpt efter den rika stadsdelen Teg i Umeå\ref{umeaa}. Således passar tegaren in i kategorin fet och grisig mat döpt efter lyxiga ställen/personer\ref{fet och grisig mat doopt efter lyxiga staellenpersoner}.

\ditem[Tegsnäsare]\label{tegsnaesare}
 är ett begrepp för de träskidor som produceras i Granö, Västerbotten\ref{vaesterbotten}. De är det överlägset bästa transportmedlet på snö. Jämfört med en skoter kan de inte köras fast, skära, få slut bensin och annat näsligt. De har inte heller någon variatorrem som måste bytas ute i kylan. Jämfört med snöskor\ref{snooskor} får man skjuts i nedförsbackar och de långa skidorna har ofta bättre bärighet i lössnö. Skulle det vara så att de ändå sjunker igenom så hittar laxstjärten längst fram på skidan alltid upp i det fria igen. Drivmedlet består av palt. Den i många sammanhang fantastiska människokroppen omsätter detta i kraft framåt. Framåt mot nya upptäckter.

\ditem[Televerket]\label{televerket}
 På den gamla goda tiden när det fanns nån att ställa till svars för saker och ting fanns ett verk som såg till att medborgarna kunde kommunicera medelst telefoni. Detta fungerade år efter år utan fåniga reklamkampanjer eller tonåringar som \quotetext{brinner för att sälja}\ref{brinner foor att saelja} i köpcentrum. Televerket skapade en helt ny färg, televerksorange som det holländska\ref{holland} fotbollslanslaget sedan plankade. Verkets personal hade uniformer och kunde om kriget kommer beväpnas för att försvara medborgarens rätt att ringa hem. Den medborgare som visat sig pålitlig och dygdig kunde få tillåtelse att ha telefon - notera att medborgaren inte själv ägde telefonen, den var Televerkets egendom och bara till låns. \quotetext{Det här är ju för bra för att vara sant} tänkte Carl Bildt när nazisterna kom till makten 1991 och så krossade han detta hedervärda verk.

\ditem[Television]\label{television}
 är en form av informationsteknik som tillhör etermedia på informationsteknikens vittförgrenade släktträd. Den kommer konsumenten till godo via en digital-teknologisk apparat som till det yttre påminner lite om en mikrovågsugn vari Lars Adaktussons huvud fram till för några år sedan talade om något på ett ganska borgligt och sövande vis. I likhet med facebook\ref{facebook} utgör televisionen något som medlemmar av den lite finare delen av det svenska folket påstår att de vägrar befatta sig med, eftersom den genomsnittliga människan har för vana att ibland titta på TV för att inhämta information om dagsaktuella händelser, Frida Kahlos liv, hur man talar Serbokratiska, vilken av fyra olika djurliknande pappfigurer som ska bort eller helt enkelt för avkoppling efter en tung arbetsdag på sågverket eller Institutionen för språk och litteratur.

\ditem[Telverksorange]\label{telverksorange}
 Borgarna kan ta vårt televerk\ref{televerket} men de kan aldrig ta vår färgnyans. När revolutionen kommer ska de doppas i tjära färgad R253 G139 B52 och rullas i fjädrar från Hedemorahöns uppfödda på Blåvitts\ref{blaavitt} allfoder. Se även: färgskala.

\ditem[Tengah]\label{tengah}
 är den ö i Stilla havet som förmodligen har störst plats av alla Stilla havets öar i det svenska folkets hjärtan. Det var nämligen här den första upplagan av Expedition Robinson spelades in. Miljontals svenskar har förälskat sig i den lilla korallön med soliga sandstränder och svalkande djungel. Vem har inte någon gång kommit på sig själv med att dagdrömma om hur det vore att byta ut sin urbana tillvaro mot Tengahs inbjudande och okomplicerade famn. Mumsandes på ett palmhjälta vandrar man på sälsamma stigar där en redan då åldrad Harad Treutiger kanske smet ifrån TV-teamet för att hitta en lämplig stock att slå sig ned på med neddragna byxor och kräma ur sig tre dagars konservmat. Var det bakom den här stenen Jochem prövade lyckan med Dr. Åsa? Var Zübeyde en hemsk människa redan innan hon kom hit? Livet, vad är det egentligen? Tengah har svaren.

\ditem[The fat Spanish waiter]\label{the fat spanish waiter}
 Supportrarnas smeknamn på Rafael Benitez efter att denne blivit klar som tränare för Chelsea FC. Vid första matcherna hälsades han välkommen med burop och fyndiga, men inte särskilt välvilliga, banderoller. The fat Spanish waiter varade i klubben en halv säsong.

\ditem[The Fog]\label{the fog}
 John \quotetext{The Fog} Fogerty, född 28 maj 1945 i Berkeley, Kalifornien, är en amerikansk sångare, gitarrist och levnadsförebild. Hans karaktäristiska sång- och spelstil har lett till att hans musik betecknas som swamp rock. Tillsammans med Bruce Springsteen tävlar han om titeln \quotetext{Världens mest heterosexuella man}. The Fog började sin musikaliska karriär i Creedence Clearwater Revival där han skrev de flesta låtarna. Gruppen spelade för enkelhets skull bara in hits och vid utgivning av samlingsskivor är det kutym att man slumpar låtar från back-katalogen. Karriären varade dock bara i fem år och splittringen föregicks av vilda debatter i media om huruvida medlemmarna verkligen sett ett riktigt träsk någon gång. The Fog fortsatte därefter som soloartist och skrev bland annat \textit{Rocking all over the world} som senare blev en hit med Status Quo.

\ditem[Thomas Wassberg]\label{thomas wassberg}
 Thomas \quotetext{Säcken} Wassberg åkte ofta skidor, men hade framförallt skägg.

\ditem[Thorstenkram]\label{thorstenkram}
 En thorstenkram är en kram under vilken en långtidssalongsberusad man i sina övre medelår för sin ena hand ogenerat ned mot sin betydligt yngre, kvinnliga motparts rumpa. En thorstenkram är ofta, från givarens sida, ett resultat av behagsjuka eller lojalitetskänsla.

\ditem[Thrashzan]\label{thrashzan}
 är en fiktiv superhjälte som beskyddar de hårda bandens universum. Hans superkraft är att kunna röja sanslöst på spelningar längre än någon annan. Han kan alla klassiska moves och blandar friskt mellan genrer så man ska inte bli förvånad om man ser honom köra kängnäven\ref{veva med kaengnaeven} och djävulstecknet samtidigt med varsin hand. Tyvärr är Thrashzan ibland lite väl hård på spriten, vilket gör att han tillfälligt tappar sina krafter. Klassiska strider han förlorat är mot Scary Guy på Load och Reload, och mot Vic Rattlehead på Risk.

\ditem[Tia]\label{tia}
 Mässingsmynt som berättigar till två stycken fika a 5 kronor. En tia kan  teoretiskt sätt vara en vinylskiva av sagda tumsmått, men sådana är sällsynta och ofta ofattbart dåliga.

\ditem[Tisdag]\label{tisdag}
 är den dag då tvådagars-bakfyllan efter helgen släppt och det är dags att bygga upp kroppen igen. Med fördel spelas fotboll\ref{fotboll} under de varmare månaderna och under de kallare månaderna ägnar man sig åt längdskidor eller TV-spel.

\ditem[Titta på ord]\label{titta paa ord}
 En person som tittar på ord sitter med en bok (oftast kurslitteratur) i ett bibliotek (oftast universitetsbibliotek), hukad under en lampa och tittar på orden i den tidigare nämnda boken. Personen kan lätt misstas för en ambitiös student\ref{den ambitioosa studenten}, då den ofta sitter länge med boken och bläddrar ganska ofta. I verkligheten har personens blodsocker sedan länge sjunkit till en nivå som omöjliggör intagande av information. Således tittar bara personen på ord, snarare än att läsa. Distinktionen mellan dessa två (läsa/titta på ord) är ytterst viktig.

\ditem[Tivoli]\label{tivoli}
 Dansken besitter som bekant en no-nonsens-attityd och är som lika bekant en hedonist av rang. Därför är det inte förvånande att ett av världens äldsta tivolin, med det träffande namnet Tivoli, finns just i Danmarks huvudstad Köbenhavn. Hit går dansken för att ta öldrickandet till en högre nivå genom att dricka Tuborg ombord på sinnrikt konstruerade åkattraktioner utan säkerhetsbälte. Mellan åken mosar dansken i sig rød pølse och wienerbröd som funnes det ingen morgondag samtidigt som hen står och skjuter lite halvhjärtat på plåtburkar i hopp om att vinna en stor nalle att gå runt och göra sig lustig med.

\uline{Historik}

Tivoli byggdes redan på 1400-talet, alltså på den tiden då man låg i fejd med de gamla grekerna\ref{de gamla grekerna} och hunnerna. Man behövde en fästning som alla kunde springa in i om det kom nån armé som ville hålla på. Man påbörjade således arbetet med vad som planerades att bli en stor borg och som skulle kallas Borg. Arbetet gick dock sakta och det var inte så många som kom dit och hjälpte till. När Christian Tyrann sa \quotetext{amen nu fo di skaerp er før faen! Kom igjen nå. Hvem kan hva med och bygge?} så sparkades det mest i gruset och mumlades bortförklaringar om inbokade att vara hundvakt och besök från släktingar. Kungen dog sedermera av vällevnad och i glappet mellan honom och hans arvinge, Hamlet, kom nån på idén att bygga ett tivoli istället. Då blev det ett fasligt liv och allt stod klart innan någon hunnit säga flasklock. På den vägen är det.

\ditem[Tjack]\label{tjack}
 Vit substans i pulverform med ursprungs i Östeuropa. Orsakar vid ymning intagande att barndomens spring i benen för en stund återvänder. Perfekt för den som behöver städa ur garaget för att få fram klotgrillen till valborgsfirandet eller har punktering på cykeln och således måste gå till dansbanan i grannsocknen. Uppstår meningsskiljaktigheter har du överhanden, eftersom du har:

\begin{itemize}
\item energi.
\item ingen känsel.
\item ett icke obetydligt mindrevärdeskomplex (gäller endast män).
\end{itemize}

\textit{Tjackad}

\ditem[Tjamstan]\label{tjamstan}
 Ett berg som med sina 400 m.ö.h tornar upp över Malå\ref{malaa} samhälle. På dess västra och norra sida finns skidbackar, på dessa södra ett stup. Detta stup inhyser i sin tur Ättestupan\ref{aettestupa}. Här skickades gamla och sjuka mot sin död för att inte vara sina släktingar till last. Detta kommer antagligen bli aktuellt igen med tanke på de nedskärningar man försöker få igenom på landstingsnivå.

\ditem[Tjena Roger!]\label{tjena roger!}
 är en till användningen mycket bred och mångsidig interjektion som är Sveriges svar på spanskans \textit{Ay Caramba}. Ordet är av okänd härkomst men används huvudsakligen i Norrtälje\ref{norrtaelje}, Skandinaviens Mexico, och inom den alltid växande norrtäljska diasporan. Tjena Roger! kan användas i situationer som uppfattas av talaren och dennes vänner som positiva såväl som i sådana som uppfattas som helt igenom negativa. En person som hittar en stor mängd pengar på gatan och avser att använda dessa för att köpa sig och sina vänner förfriskningar utan att behöva göra rätt för sig\ref{goora raett foor sig} kan av förtjusning däröver utropa Tjena Roger! Lika så kan någon som plötsligt får se att den egna huden har angripits av en aggressiv form av psoriasis eller någon annan oönskad hudåkomma i bestörtning utropa detta användbara uttryck.

\ditem[Tjock-TV]\label{tjock-tv}
 är den historiskt sett mest populära formen av TV-apparat, men har på sistone förlorat en hel del terräng till förmån för borgerlighetens motsvarighet, platt-TVn. Framför denna samlas familjer i de svenska hemmen i allt högre utsträckning. Tjock-TV:n utmärks av att vara ungefär lika tjock som den är bred, det vill säga kubformad, och att den ger ifrån sig ett högt genomträngande tjut. Den är också vanligtvis extremt tung. Tjock-TV kan, liksom den konkurrerande TV-formen, levereras tillsammans med en fjärrkontroll, eller som det tidigare kallades, en dosa. Med denna kan den som manövrerar tjock-TVn byta kanal och finjustera ljudnivå och bildkvalité utan att behöva resa sig och gå fram till apparaten. En stor bonus som köparen av en tjock-tv får vid införskaffandet av en apparat är text-TV, där man kan läsa om sportresultat, väder och aktuella världshändelser. Text-TV kan ses som en alternativ, och föregående, form av internet\ref{internet}.

\uline{Platt- eller tjock-TV?}

Många är de som har svårt att välja mellan att köpa en tjock-tv eller en platt-TV. Här kan Nissepedia\ref{nissepedia} hjälpa konsumenten att göra ett informerat val, men valet måste konsumenten själv göra. Det viktigaste att tänka på är hur apparaten gör sig i rummet, det vill säga inredningsnivån av frågan. Har du ett smalt rum och saknar plats för alla de föremål du vill ha i det? Tja, då kan kanske en platt-TV vara din grej. Har du istället för mycket plats och för få ägodelar? Då är tjock-TVn det val som det mesta talar för. Den som vill avgöra om en viss TV-apparat är en tjock- eller platt-TV går så till väga att hen ställer sig vid sidan av TVn och kontrollerar hur bred apparatens sida är. Är den cirka en decimeter tjock rör det sig om en platt-TV, men om dess tjocklek är i gränslandet till Trafikmagasinets framlidne Christer Glenning har man att göra med en gammal hederlig tjock-TV. Ett ytterligare sätt att kontrollera detsamma är att försöka lyfta TVn. Kan man göra detta med begränsad ansträngning är det man har i sin famn antagligen en platt-TV, men om ens rygg slutligen verkar vara nära att nå sin bristningspunkt och svett tränger fram i armhålor och tinningar är det nog en tjock-TV.

\uline{Klass och TV}

Vanliga pantade knegare nöjer sig med tjock-tv medan storfräsare\ref{storfraesare}, glidarjobbare och andra kälkborgare\ref{kaelkborgare} suktar efter senaste modellen av platt-tv trots att man inte kan göra ett skvatt mer än med en tjock-tv. Således en pålitlig klassmarkör, men även en skiljelinje såväl etniskt som socio-ekonomiskt.

\ditem[Tjänstemannateoretisk]\label{tjaenstemannateoretisk}
 Föreställningar av typen: Tvingas söka jobb som VD på Volvo för att erhålla försörjningsstöd, när din högsta arbetsmerit är att ha sommarjobbat åt kommunen. Det \textit{kan} ju faktiskt gå.

\ditem[Toalettpapper]\label{toalettpapper}
 är ett oftast mjukt papper upprullat på en hylsa av papp. Vanligtvis ligger det och dräller eller är prydligt packat i ett plastemballage i badrum samt under eller bredvid sängar och datorer där grabbar i puberteten lever och sover. När man som gäst nyttjar ett badrum kan man pga toapapprets kvalité bedöma vilken ekonomi ägaren till badrummet har. Låginkomsttagare som skiter i att läsa på förpackningen och bara kollar på kilopriset tenderar att köpa budegetmärken såsom Euroshopper, Eldorado (som för övrigt är förvånansvärt vanligt på allehanda ovanliga bensinkedjor såsom Q-Star, St1 och Uno X) och Coop X-tra. Det tragiska för den omedvetne konsumenten i sammanhanget är att de märken som ligger några kronor över det billigaste ofta är mer ekonomiska sett till hur många meter du får per rulle. T.ex. är Änglamark kompakt fyrtiotvå meter per rulle i lager om tre, i kontrast till det billigare Coop X-tra som är strax under de trettio i lager om två. I samband med att folk vill tro att man kan konsumera sig till ett gott miljösamvete har toalettpappersidustrin börjat lansera toapapper för den miljömedvetne konsumenten som är gjort av returpapper. Detta är oftast mycket dyrare än de andra men också mer kompakt då hylsorna är mindre för att minska transportvolymen. Toalettpapper som är kritvita och har någon form av gulligt mönster är i regel dyrare än papper som ser ut att vara gjorda av innehållet i din lokala papperscontainer.

\uline{Toapapper i levd så väl som medierad folkkultur}

En dekorativ funktion som toalettpapper kan fylla är på barnkalas och Halloween-fester där någon lustigkurre klär ut sig till mumie genom att linda toalettpapper kring större delen av sin kropp. Ett klassiskt busstreck i västvärlden där toalettpapper utgör ammunitionen är när barn eller ungdomar går till ett hus under en sen kväll eller natt och kastar rullarna över taket på målet för att jävlas\ref{jaevelskap} med de inneboende. Vill man krydda till det ytterligare kan man satsa på att även kasta rullar över ev. träd i anslutning till huset. 

\ditem[Tofu]\label{tofu}
 är en vit massa med oklart ursprung, antagligen nån form av könsvätska från delfinen\ref{delfin}. Men det är gott.

\ditem[Tokliberal]\label{tokliberal}
tokliberaler kan en massa fina ord, såsom incitament\ref{incitament}, frihet\ref{frihet}, laissez-faire\ref{laissez-faire} och individ\ref{individ}.

\ditem[Tolkien och den svarta magin]\label{tolkien och den svarta magin}
 är en reportagebok från 1982 av den svenske akademikern Åke Ohlmarks\ref{aake ohlmarks} (1911-1984). Bokens syfte är att avslöja vad vi alla länge misstänkt; familjen Tolkien och de tolkiensällskap som bildats över hela världen är i själva verket en maffialiknande sammanslutning som ägnar sig åt sexorgier, nazistiska ritualmord och organiserad brottslighet. Allt detta sker i hemlighet och kontrolleras av Tolkiens efterlevande från residenset i Oxford. Försäljningen av alvöron och gandalfhattar är i själva verket bara en täckmantel för att finansiera betydligt värre saker. För att bevisa sin tes använder sig Ohlmarks av den, inom humaniora, mycket beprövade metoden \textit{guilt by association}. Exempelvis belyser Ohlmarks det klockrena sambandet mellan pappan till Svenska Tolkiensällskapets ordförande, som är tandläkare, och grundaren av Ku Klux Klan, som också är tandläkare. Han drar också paralleller från tolkiensällskap till \quotetext{\textit{... satansprästen Allister Prowley}\quotetext{ [sic] och drogkarteller }\textit{... där röks troligen hasch; antagligen har man också börjat eller kan när som helst börja också med tyngre droger}} Bland de mer allvarliga anklagelserna finns också mordbrand. Det framgår till slut att mordbranden handlade om att Ohlmarks fru sängrökte.

\ditem[Tolva]\label{tolva}
 är ett annat ord för LP- eller grammofonskiva. Tolvor kan aldrig bli bättre än första sjuan\ref{foorsta sjuan}. 

\ditem[Tomte]\label{tomte}
 är ett prefix som läggs till ett adjektiv eller verb för att ge ordet motsatt betydelse. i roslagsslang. \textit{Tomtebra} betyder således \quotetext{dåligt} och \textit{tomtesnyggt} betyder \quotetext{fult}. Suffixet uppfanns i slutet av nittiotalet av en fjortonårig norrtäljekille. Det används framförallt i konversation som nedan:

\uline{Exempel}

-\quotetext{Tyckte du att Waterworld med Kevin Costner var bra?}
-\quotetext{Ja, den var oemotsägligen tomtebra. Tack för tipset!}

\ditem[Tomten]\label{tomten}
 är ett spretigt fenomen vars minsta gemensamma nämnare är att alla har toppluva. Förutom denna uppmuntrande huvudbonad är de olika typerna av tomtar dock väldigt olika.

\uline{Gårdstomten}

Två äpplen och en brakskit lång smyger denna lilla gynnare runt din bostad och ser till att skogsrået, bäckahästen och annat oknytt håller sig borta från ladugården. Tyvärr kan gårdstomten vara lite lynnig på samma sätt som en träskpunkare\ref{traeskpunkare} och plötsligt få för sig att göra något rådumt som att låta brunnen sina eller höet mögla. Varför detta sker är något oklart och de enda råd som finns för hur man får in gårdstomten på den fromma vägen igen är tyvärr muntligt traderade skrönor fulla av absurda inslag.

\uline{Trädgårdstomten}

I indistrialialismens tidevarv kom fler och fler människor att överge den agrara ekonomin och därmed minskade behovet av att hålla sig väl med den stundtals väldigt slaniga\ref{slan} gårdstomten. Trädsgårdstomten - gjuten som han är i betong eller porslin, i vissa undantagsfall brons - har på många sätt kommit att bli den stora symbolen för kapitalismen då den låter sig köpas för pengar och därefter aldrig sätter sig upp mot sin herre.

\uline{Haschtomten}

Lika oberäknelig som gårdstomten fast på ett annat sätt. Medan gårdstomten stjäl mjölken från korna drar haschtomten på mackshopping\ref{mackshopping} och försnillar sina egna pengar på strandleksaker, en finsk\ref{finland} fasadflagga och gasol till fingrillen som hen inte äger. Många av de skrönor som berättar hur man kommer till rätta med gårdstomten tros härröra från haschtomtens expanderade fantasivärld när denna lyssnat för länge på Pink Floyd eller \textit{Dopesmoker }\ref{dopesmoker}. Haschtomtens bidrag till allmogens liv är att besitta utbredd kunskap om djuphavsfiskar, botanik och rymden (kanske inte bråddjupa kunskaper om det sistnämnda utan mer en förmåga att beskriva hur ball den är).

\ditem[Torbjörn]\label{torbjoorn}
 är en punkare från Umeå som älskar rollspel, korta shorts, Bathory, kir\ref{kir} och grisig mat. Alla andra som heter Torbjörn är döpta efter honom, som en eloge.

\ditem[Torgny Mogren]\label{torgny mogren}
 är en svensk skidåkare från Hällefors\ref{haellefors}. Han har bland annat ett OS-guld och fyra VM-guld hemma i prisskåpet, varav ett från Oberstdorf. Under ett träningsåk 7 februari 1995 bröt han ryggen och sitter sedan dess i rullstol. Idag är han verksam inom VVS-branschen.

\ditem[Tornedalslåset]\label{tornedalslaaset}
 Består i att luta en kvast mot dörren för att signalera att man inte är hemma.
Tornedalslåset är lätt att forcera och har därför ännu inte slagit igenom på internationellta mässor inom home security.

\ditem[Torsdagar]\label{torsdagar}

 Torsdag är den fjärde dagen i veckan. På grund av sin mittplacering är torsdagen den första dag som slumpas vid undantagstillstånd.

\uline{Historiska händelser på torsdagar.}

\begin{itemize}
\item Robinson Kruse och Fredag kör en flotte på grund och döper händelsen till Skärtorsdag.
\item Benny Bus\ref{benny bus} skolvecka börjar.
\item Sommartiden infaller år 1972.
\item Sagoman har namnsdag.
\end{itemize}

\ditem[Torsten Bengtsson]\label{torsten bengtsson}
 (1914-1998) var en centerpartistisk politiker. De två politiska frågor han kände mest passion för var den osannolika kombinationen att alla burfåglar skulle släppas fria (varför han kallades \quotetext{Pippi}) och strikt nykterhet. Bengtsson propagerade bland annat hårt för den drakoniska alkohollagsskärpning som i och med sitt instiftande 1977 förbjöd matvarubutiker att sälja mellanöl i Sverige. Trots att hela grejen med fåglarna är helt vansinnig minns eftervärlden honom mest tack vare hans hårda avståndstagande från alkohol. Detta i och med att den dionysiske och gravt efterblivne skalden Eddie Meduza besjungit Bengtsson i flera av sina mest älskade låtar, däribland \textit{Torsten hällde brännvin i ett glas åt Karin Söder}, \textit{Ursäkter} och \textit{Mera brännvin}, som ska vara tillägnad Torsten Bengtsson.

\ditem[Traditionell finsk medicin]\label{traditionell finsk medicin}

 Läran om att dricka läkarsprit, bada bastu och amputera med morakniv. \quotetext{Där varken bastu, brännvin\ref{braennvin} eller tjära hjälper finns ingen bot}

\ditem[Traditionell kinesisk medicin]\label{traditionell kinesisk medicin}
 Läran om att göra potensmedel av udda delar på utrotningshotade djur\ref{utrotningshotade djur}. Till exempel myrkottsfjäll, noshörningshorn, tigergalla och hajfena.

\ditem[Trans]\label{trans}
 (1982) är en skiva av den kanadensisk-amerikanske rockartisten Neil Young. Skivan har ett väldigt digitalt, cyborg-aktigt ljud och på fem av de nio låtarna används en vocoder, eller röstlåda som vi säger på Nissepedias\ref{nissepedia} redaktion, som transformerar Youngs röst till ett slags robotröst. Detta ska ha föranletts av två saker:

\begin{itemize}
\item Youngs son led av en CP-skada och var oförmögen att tala, varför Young som terapi valde att experimentera med förvrängt tal.
\item Young tyckte att det var ball.
\end{itemize}

Skivomslaget föreställer naturligtvis ett slags science-fictionartat urbant landskap, komplett med framtidsbilar\ref{bil} och flygande farkoster. Det har spekulerats i varför skivan är så hemskt dålig. Skivbolaget Geffen påstod i en rättstvist att Young medvetet skulle ha producerat osäljbar musik, medan Youngs tillskyndare har påstått att \textit{Trans} innebar ett slags ironisk kommentar till samtida musik och alla dess många brister.

\ditem[Trasmattans dag]\label{trasmattans dag}
 är en tillställning som firas varje år i samband med Kristi himmelsfärd. I Fagersta\ref{fagersta} firas högtiden som mest på hembygdsgården med utställning av trasmattor och inte så mycket annat. Det är oklart om det finns något historiskt samband mellan trasmattor och Kristi himmelsfärd, vilket man lätt kan tro eftersom de firas samtidigt. Kanske är det bara så att kristendomen i vanlig ordning förlagt sitt firande samtidigt för att parasitera på en redan etablerad högtid.

\ditem[Trea]\label{trea}
 En trea är en bostad med tre rum och kök.

Se även: Nolla, Etta\ref{etta}, Tvåa\ref{tvaaa}, Fyra\ref{fyra}, Femma\ref{femma}, Sexa\ref{sexa}, Sjua\ref{sjua}, Åtta\ref{aatta}, Nia\ref{nia}, Tia\ref{tia}, Elva\ref{elva}, Tolva\ref{tolva}.

\ditem[Trepipsproblem]\label{trepipsproblem}
 är problem som kräver sådan djupgående och vidsträckt tankeverksamhet att problemlösaren hinner röka hela tre välstoppade pipor innan en lämplig åtgärd kan skönjas.

\ditem[Trevlig]\label{trevlig}
 Urban medelklasslang för halvtråkig. Nedan följer exempel på bruk av ordet i vardagssamanhang.

\begin{itemize}
\item - ''Trevlig grillfest Lothar!\ref{lothar} Att servera ljummen bärs var ju en riktig home run.
\item - Du, Gunborg\ref{gunborg}, din nya karl, Petter\ref{petter}, han verkade ju riktigt trevlig! Intresserad av bildäcksproduktion? Hade jag aldrig anat.
\item - Jörgen, det här kanadensiska rockbandet du hade på kassett var ju väldigt trevligt. Groovy, liksom.
\item - Din konstsmak är ju genial, Hillevi\ref{hillevi}. Se på den här väldigt dramatiska porslinsdalmatinen till exempel. Jättetrevlig!
\end{itemize}

\ditem[Trilobit]\label{trilobit}
 är en allmän benämning på pusselbitar som ramlat ner från bordet eller tappats bort på annat sätt.

\ditem[Trivselskrot]\label{trivselskrot}
 är skrot som man samlat på gården för tivselns skull, i första hand, och för att det kan komma till användning i andra hand. I tredje hand har man samlat det där för att öka möjligheterna för att någon förbipasserande tvillingskäl ska knacka på och fråga om ens rostiga bromsok eller kofångare till ens skrotade Saab 99 är till salu. Då går man ut på gården, står framför skroten och försöker komma fram till om man har någon gemensam bekant. Detta är en viktig komponent i affären så det är viktigt att inte ha bråttom.

\uline{Exempel på saker som kan ingå i ens samling av trivselskrot}

\begin{itemize}
\item Nedslitna bil- och traktordäck
\item Pusch Dakota utan hjul
\item Snöplog
\item Brädstumpar och lastpallar att palla upp saker med
\item Cykelkedjor och kättingar
\item Defekt utombordare
\item Spillolja i plåttunna
\item Styre till en BMX
\end{itemize}

\ditem[Trocadero]\label{trocadero}
 Denna gyllene brygd är de nordligaste länens livsblod. Dess smak är svår att beskriva i ord, men enligt innehållsförteckningen borde den smaka både äpple och apelsin, samtidigt. Whoa! Den görs av en uppsjö av tillverkare och de som är bäst att köpa av är:

\begin{itemize}
\item Nyckelbryggerier
\item Vasa bryggeri
\item Zeunerts
\end{itemize}

\ditem[Trollkull]\label{trollkull}
 är en idrott som ännu inte fått OS-status, men förhoppningsvis blir det ändring på det till OS 2018. Idrotten går ut på att ett tjog förpubertala ungdomar rusar omkring och skriker i falsett i en jumpasal. Alla har ett tygband nerstoppat bak på sina byxor. En av deltagarna försöker desperat jaga klassens tjockis in i ett hörn för att där rycka åt sig tjockisens band och på så vis kulla honom/henne. Nu är det två\ref{tvaaa} som kullar och ju fler troll de kullar desto mer spännande blir det, för sisten kvar med bandet bestämt nerkört, eller i extrema fall fastknutet, i underkläderna vinner nämligen.

\ditem[Trollprutt]\label{trollprutt}
 är ett ord som kommer att ersätta det uttjänta och lite rustika begreppet \quotetext{brakskit}. Även om orden syftar på samma sak, för det tidigare till skillnad från det senare tankarna till folktro och mytologi. Det tillåter gamla människor att berätta\ref{beraetta} för yngre släktingar att man förr i världen brukade säga att \quotetext{det går troll i området}, då det karaktäristiska ljudet hörts och en vägg av rötstank slår emot ynglingen. Likaså kan man roa utlänningar som kommer på besök med att redogöra för betydelsen av detta begrepp och på så vis föra in samtalet på lokala seder och bruk och på nordbornas traditionella föreställningsvärld istället för att be om ursäkt för det internationellt likvärdiga snedsteg man just gjort sig skyldig till.

\ditem[Trollpunk]\label{trollpunk}
 På det gamla hederliga 90-talet visste inte så många i Sverige\ref{sverige} att det fanns undergroundmusik. Därför kallades allt som inte var Nordman antingen för punk eller synth. När så pop-rockbandet\ref{pop-rock} Dia Psalma kapade åt sig marknadsandelar genom att sjunga om troll och näcken som rövar bort jäntor eller att klota i en bäck och drunkna, lades grunden för musikgenren trollpunk. Reaktionerna lät inte vänta på sig. Dia Psalmas debutskiva \textit{I Gryningstid} såldes i fyrtiotusen miljoner\ref{fyrtiotusen miljarder} exemplar, och blev en ekonomisk succé som direkt ledde till att Per på Birdnest tog skivbolagets kassaskrin och satte iväg över Östersjöns is med den erfarne Prof. Etienne som guide och mentor. De två vännerna tog sig ner till skatteparadiset San Marino via Baltikum och Östeuropa och har sällan setts till sedan dess. Detta slag mot trollpunkenscenen gjorde den dock bara starkare, eftersom attityden hos kidsen alltid varit att man inte deltar i den för pengarnas skull, utan för att uttrycka sig i sång och musik om troll och andra oknytt.

\ditem[Trotta]\label{trotta}
 Att trotta är ett annat ord för att infiltrera en organisation med en ganska slarvigt dold agenda. Trotte = trotskist (skällsord i vissa kretsar sent 60- och tidigt 70-tal)

\uline{Exempel}

Trotten söker medlemskap i en hemslöjdsförening, schweiziskt institut, arbetarkommun eller, varför inte, ett politiskt ungdomsförbund. Trotten är till en början en helt vanlig medlem och är inte så högljudd utan bara flyter med. Sen plötsligt en dag utbrister denne något i stil med \quotetext{Hur kan vi få den här slöjdföreningen att jobba för införandet av socialismen i hela världen?} och går vidare med att svamla om massmöten och namnlistor och frontorganisationer i Kamerun och fan och hans moster. Efterspelet kan resultera i:

\begin{itemize}
\item Trotten lyckas med sin kupp och får organisationen att bli lite mer vänster.
\item Någon handlingskraftig knegare greppar trotten i svångremmen och skickar denne med huvudet före över tröskeln.
\item De övriga lämnar organisationen med trotten kvar som trots detta tjuras med att det är en massorganisation.
\end{itemize}

\uline{Att motverka trotteri}

Enligt säkra uppgifter är en gammal hederlig ishacka det bästa redskapet. Senare amerikanska presidenter har på olika sätt försökt minimera förekomster av trottar, ibland har man kanske gått lite långt eller över gränsen till andra länder både bokstavligt och fysiskt. Oftast med en större mängd vapen och diverse svepskäl för att få vistas i dessa land och skjuta på alla med T-shirt och trotteglasögon.

\ditem[Träbjörn]\label{traebjoorn}
 En träbjörn är en björn byggd utav trä. För att en björn ska klassas som en träbjörn måste dess beståndsdelar främst bestå av trä. Till yttermera visso ska dessa träbitar vara ihopförda på ett sätt som får dem att tillsammans likna en björn. Världens största träbjörn återfinns på torget i Sveg.

\ditem[Träd, Gräs och Stenar]\label{traed, graes och stenar}
 är ett svenskt rockband bildat 1969. Bandet bestod då av BoAnders Persson (gitarr), Arne Ericsson (elcello), Torbjörn Abelli (bas) och Thomas Mera Gartz (trummor). Gruppen räknades till proggrörelsen, men saknade den tydliga politiska inriktning som kännetecknade delar av denna rörelse. Progressiv rock med lite politisk ton var det gruppen sysslade mest med under 1970-talet. År 2012 gjorde bandet ett musikaliskt lappkast och började införa sång i alla sina låtar. Alla true proggare gråter över detta.

\ditem[Träskpunkare]\label{traeskpunkare}
 (även kallad containerpunkare, pisspunkare, sumppunkare och grispunkare) är en subgenre inom punkkulturen. Träskpunkaren lever i möjligaste mån på bidrag och sopor och vistas gärna i flock. Den duschar sällan och sköter istället sin hygien genom att bada naken eller vandra runt Kalle anka\ref{kalle anka} på festival. Den bor i en svart skinnpaj dekorerad med musikgrupper, som alla börjar på prefixet dis-, påmålat med tipp-ex. Den karaktäriseras i övrigt av sina hastiga växlingar mellan svintrevlig och outhärdlig. Vill man lära känna en träskpunkare bör man mata den med folköl eller mäsk. De frodas som bäst i åldern 16-27 och blir sällan mycket äldre än så då den lever efter devisen: live fast - die. Vissa exemplar i fångenskap har i undantagsfall blivit över 50 år.

\uline{Fortplantning och övervintring}

Under sommarhalvåret söker sig träskpunkaren ut på öppna fält där den plankar in på festivalcampingar för att finna en partner och para sig till allmän beskådan. De brunstiga individerna lockar på varandra genom att spela Anti Cimex genom en dekorerad kassettbandspelare eller sjunga Onkel Kånkel för full hals. Fortplantningen försvåras dock av träskpunkarens obenägenhet att göra skilland på kön och parar sig med lite vad som helst. När vintern kommer söker den sig till socialen eller övervintrar på ett squat i Tyskland.

\uline{Kända träskpunkare}

\begin{itemize}
\item Benny Bus\ref{benny bus}
\item Nasse
\item Ramen
\item Jonsson
\item GG Allin
\end{itemize}

\ditem[Tråg]\label{traag}
 Ett tråg är ett föremål med försänkning i mitten som man har mat i. Vad som skiljer tråget från andra liknande föremål såsom badkaret, baljan, hon och den igenproppade stuprännan är just att det är avsett att äta ur. Vad som skiljer tråget från andra föremål avsedda att äta ur såsom djuptallriken, bunken, grytan och papptallriken är dess storlek. Ett tråg är nämligen så stort att flera kan äta ur det samtidigt. Vanligtvis är det konstruerat i någon form av trä och har en avlång form. Det behöver inte rengöras så noga utan det räcker att man spolar av det någon gång ibland. På medeltiden\ref{medeltiden} åt alla människor ur tråg men nu för tiden är det mest grisar och kor som gör det. Om man idag ser en människa inta sin förplägnad ur ett tråg är de troligt att denna drabbats av den tyska mustigheten\ref{den tyska mustigheten}.

\ditem[Tsygan]\label{tsygan}
 (\textit{\begin{otherlanguage*}{russian}Цыган\end{otherlanguage*}}, ryska för \quotetext{zigenare}) var den första alkisschäfern i rymden. Många uppger felaktigt att Laika var den första men hennes bedrift låg istället i att vara den första att flyga i omloppsbana. Tsygan överlevde sin flygning och landade säkert på sovjetisk mark 29 januari 1951 efter en kortare åktur.

\ditem[Tuborg]\label{tuborg}
 (uttalas ) är danska för prosit.

\ditem[Tuff-frysa]\label{tuff-frysa}
 Att tuff-frysa är ett väldigt framkomligt sätt att vinna respekt på skolgården och busshållsplatsen och går i korthet ut på att man genom den svenska sub-arktiska vintern envist vägrar ha vantar och mössa samt att man har jackan oknäppt. Knepet är vanligt i åldrarna 7-15 år och bland samhällsgruppen hockeykillar\ref{kukenkillar} som röker  cigg\ref{cigg} utanför sportbarer. Ibland kan det, om man är ett barn\ref{barn}, vara nödvändigt att använda sig av en hel del list och uttänkta strategier för att lura sina föräldrar som, som vanligt, ska hålla på och tjata om att man måste ha mössa och vantar \textit{et cetera}, men det är det ofta värt för när man väl börjat närma sig ögonblicket då Viktoria i 5b tycker att man är tuff finns det egentligen ingen återvändo. Just på det viset, och också för att det suger att frysa som fan dag ut och dag in, är situationen lite som för de själar Dante möter i inferno i \textit{Den gudomliga komedin}.

\ditem[Tunntarmen]\label{tunntarmen}
 Tunntarmen kommer innan magsäcken i människans matspjälkningssystem. Sedan kommer tjocktarmen följt av ändtarmen. Någonstans bland alla dessa olika sorters tarmar finner man den övre magmunnen\ref{oovre magmunnen}.

\ditem[Tura]\label{tura}
 Att tura är att åka fram och tillbaka över Öresund mellan Helsingborg och Helsingør\ref{danmark} i syfte att i första hand bli full, i andra hand att äta mat. Det är inte mycket att se längs denna farled, förutom Hamlets\ref{hamlet} slott.

\ditem[Turtlestestet]\label{turtlestestet}
 är ett psykologiskt diagnosverktyg för att fastställa mustigheten i en persons karaktär. Testet går i korthet ut på att man monterar in en bärbar hydrograf i munhålan på patienten för att sedan visa denne valda utdrag ur en vanlig svensk standardpizzameny varpå man mäter flödet av snålvatten som rinner till för de olika pizzasorterna. Tvingas man använda salivsug redan vid margherita så kan man avluta testet och direkt utesluta all form av mustighet. Stannar man på Vesuvio eller en klassisk Capricciosa finns det fortfarande hopp om försöksobjektet i fråga bara är ett barn. Självfallet kan patienten ofrivilligt räkna in förbättrande omständigheter så som bearnaise, vitlökssås eller andra smakförstärkare vilket kan ge s.k. metodfel. Högre mustighet finner man hos de som går i spinn på t.ex. en calskrove\ref{calskrove}, hawaii-\ref{hawaii-pizza} eller någon av de lokala specialpizzorna. För att kvantifiera patientens mustighet använder man förutom sunt förnuft även en rad olika parametrar som sätts in i en hyperkub för att beräkna mustighetskoefficienten. 

\uline{Parametrar}

\begin{itemize}
\item Antal ingredienser
\item Antal ingredienser som endast återfinns i just denna pizza på menyn.
\item Sannolikheten att en eller flera ingredienser endast återfinns i just denna rätt.
\item Transportsträcka för ingrediensen från skördeområde.
\item Om pizzan innehåller ingredienser från både växt-, djur- och mineralriket.
\end{itemize}

\uline{Bakgrund}

Det är sedan länge känt att Teenage Mutant Ninja Turtles\ref{teenage mutant ninja turtles} var föregångare i kategorin postavantgardistisk gastronomi och då framförallt Michelangelo (Den orangea med nunchucks). Michelangelo var så framstående att han aldrig tog en pizza rakt från menyn utan komponerade själv sin pizza på plats i ett stream of consciousness (en. inre monolog). 

\ditem[Tvåa]\label{tvaaa}
 Slangord för anus.

\ditem[Tysk toalett]\label{tysk toalett}
 En vanlig toalett har, som bekant för de flesta på 10-talet, ett vanligt hål som ens träck ramlar ner i och ofta döljs bitvis av. I Tyskland\ref{tyskland} däremot är det standard praxis att ens avfall hamnar på en liten porslinstallrik i toaletten så att man kan undersöka sin orenhet i lugn och ro. För den med en gnutta insikt i tysk kultur är det fullkomligt logiskt att man som följd av den tyska mustigheten\ref{den tyska mustigheten} finner ett enormt nöje i att examinera och dagbokföra sitt exkrements karaktär. Immanuel Kant skrev exempelvis långa tirader om sitt brunas kvaliteter från dag till dag (publicerades i bokform som \textit{Ding an sich} och översattes till svenska med ett underbart själfullt förord av underhållaren, perukbäraren tillika laktosallergikern Carl von Linné)\ref{carl von linné}.

\ditem[Tysk tårttant]\label{tysk taarttant}
 En tysk tårttant är en tant från Tyskland\ref{tyskland} som ofta bakar tårtor. Kanske är detta hennes levebröd? Jo, det är det. Den tyska tårttanten är en motsvarighet till den svenska fluortanten och kommer till klassrummen för att ge alla barn Schwarzwaldtårta\ref{schwarzwald larsson} så att de blir riktigt mustiga\ref{den tyska mustigheten}.

\ditem[Tyskland]\label{tyskland}
 är en förbundsrepublik som oblygt breder ut sig mellan Polen och Belgien\ref{belgien}.

\uline{Kultur}

Tysk kultur består i huvudsak av freikörperkultur\ref{freikoorperkultur} och pampiga bensinmackar. Germanofilen Edward Blom brukar sammanfatta det som är bra med Tyskland med treenigheten \quotetext{öl, korv och blåsmusik}. Tysken avnjuter dessa tre tillsammans och känner att de inte kommer till sin fulla rätt om de konsumeras var för sig. Därför finns i landet ett oräknerligt antal \textit{bierschtubes} där den långväga besökaren möts av det varmt skenet från den öppna härden, ompa-ompamusik, klanget av ölsejdlar och en solid vägg av korv-andedräkt\ref{fetor ex ore}.

\uline{Litteratur}

Tysken läser en hel del Goethe och Schiller, men oftast läser tysken Henning Mankell.

\uline{Natur}

Ruhrområdet sägs vara så gastkramande vackert att många gripits av plötsligt och oåterkalleligt storhetsvansinne\ref{storhetsvansinne} redan vid avfarten vid Essen. Det finns en endast obetydligt djur i denna del av Tyskland, vilket beror på att man malt ner allt som lever och gjort korv av det. Alperna i söder fungerar som den tyska nationalsjälens andliga skafferi. Här vandrar skäggiga män med fjädrar i hatten längs bergsryggar och bygger upp metafysiska system som de efter hemkomsten skriver ner i tjocka böcker. Här har även några väldigt traumatiserade djur lyckats gömma sig undan den tyska matlagningsglädjen, men de är få. Alltför få.

\uline{Idrott}

För att kunna genomföra sina själastärkande vandringar i Ruhrområdet, och för att orka äta all korv\ref{mangel}, är det viktigt för tysken att ha en god fysisk kondition. Idrott är därför ett utbrett fenomen och tyskar återfinns i elitskiktet i alla sporter. De flesta håller på med minst en lagsport, för där kan man agera som en \quotetext{maskin} vilket skänker tysken stor tillfredställelse. 

\ditem[Tåga]\label{taaga}
 är fackspråk bland friluftsmänniskor för att bära en kanot eller kajak på land längst med ett vattendrag. Varför man skulle vilja göra det kan rimligtvis inte bero på något annat än att det är hål i kanoten, men att man vill spara den av sentimentala skäl. Konceptet är troligtvis danskt, för vem skulle annars komma på idéen att släpa runt på en trasig kanot i skogen bara för att man söp ihop med Kim Larsen i den en gång för 10 år sedan? Man, just let go...

\ditem[Törley gala]\label{toorley gala}
 Promenera hem från jobbet på en tisdag. Du kan redan känna smaken av Ica basics fiskpinnar mot tungan. Det plaskar om dina konsum-Coverse mot den novemberslaskiga asfalten. För all del inte mer suicidal än någon annan tisdag i november. Men det är inte en fredag i juli, direkt. Plötsligt minns du! En liten rest från helgen... Kvarglömd av en träskpunkare\ref{traeskpunkare} som firat utbetalning av bidrag. Inslagen i en systemkasse med hexadecimal färgkod \#177F75, gömd längst in i kylen. Tanken om fiskpinnarnas brödiga yta mot tungan byts ut mot den om en porlande, bubblande tungkyss. Den bekanta vägen hem kryddas med danssteg, som Fred Astaire. Du mummlar oskrivna sonetter i trapphuset. Att dra upp kylskåpsdörren är som att svepa en älskare av sina fötter. Att öppna påsen är som att hjälpa den du mest åtrår ur sitt bomullsfängelse. Att dra ur korken - \textit{közösülés\ref{koozoosülés}}. Varje efterföljande klunk av den klara drycken är ett post-coitalt\ref{post-coitus} cigarettbloss, på den mjukast smakande cigarett du någonsin slutit dina läppar kring. Du kommer aldrig älska igen som när du älskat \textit{Törley gala}.


%%%%%%%%%%%%%%
\newpage
\null
\\
\null
\\
\Huge
U
\normalsize
\\
\null
\\
\null
%%%%%%%%%%%%%%



\ditem[Ugglekonst]\label{ugglekonst}
 är konst föreställande ugglor. Tekniken kan variera men garntavlor är överrepresenterade. En viss 70-talsanda bör finnas med i verket för att det ska få betecknas ugglekonst. Uttrycket ugglekonst kan lite slarvigt användas för att beteckna föreställande konst med andra motiv än ugglor. Det kan vara andra fågelarter, djur eller barn\ref{barn}, allt med ett stänk av romantisk realism.

\ditem[Umeå]\label{umeaa}
 är ett slags tätort som ligger i Västerbotten\ref{vaesterbotten}, lustigt nog utmed Umeälven. Befolkningen utgörs av urbaniserade samer, studenter, akademiker, vanliga människor och folk som lyssnar på hardcore, samt Martin Emtenäs, trumslagare och pappaledig programledare för Mitt i Naturen.

\uline{Politik}

I Umeå bestämmer en lite läderartad, grodliknande man som lystrar till namnet Lennart Holmlund\ref{lennart holmlund}.

\ditem[United States of America]\label{united states of america}
 är världens största nöjespark och det globala nyliberala\ref{tokliberal} imperiets centrum. Här är i princip alla moderater\ref{moderat} och de som inte är moderater är tokkonservativa, homofoba sociopater vars gräsrötter arbetar oförtrutet i lackande svett för att skicka ut den egna arbetarklassen, speciellt svarta, i anfallskrig runtom i världen. Man har också många olika sorters fantasifullt utformade jumpaskor\ref{sneakers}, som görs av barn i Asien, till försäljning. På grund av detta fascineras många av denna plats. Martin \quotetext{Rocky} Kellerman brukar till exempel ofta rita serier om hur han och hans killkompisar besöker the United States of America för att köpa jumpaskor.

\ditem[Universitet]\label{universitet}
 En samling hus på slät mark vilken påminner om en prärie. Där drar stora horder av overallstudenter runt likt boskap medan förvirrade utbytesstudenter letar efter busshållplatsen.

\ditem[Uppland]\label{uppland}
 Sveriges\ref{sverige} Washington. Här samlas den politiska makten - det vill säga tillresta fascistiska grisbönder från skåne\ref{skaane} och Thatcher-nostalgiska liberaler från Täby som av rördhet snyftandes ser tillbaka på fornstora dagars massavrättningar på latinamerikanska fotbollsarenor. Att Uppland är Sveriges\ref{sverige} politiska centrum märks inte minst på alla de runstenar och historiska lämningar som vittnar om den tusenåriga historia av demokrati och fredsälskande brödraskap som förtryckta kommentarfältherrar sammanbitet beskyddar mot feministisk och mångkulturalistisk revisionism. I Uppland finns natursköna områden så som Roslagens skärgård och Sightunas vackra omnejd där resliga ariska män och rejäla blonda fruntimmer vandrar hand i hand genom lummiga ekskogar. I Uppsala\ref{uppsala}, Sveriges\ref{sverige} fjärde stad, är borgerligheten som sig bör fortfarande vid makten - vilket är en anmärkningsvärd skillnad från Sveriges\ref{sverige} mer förfallna kommuner där hjärntvättade kulturmarxister, uppfyllda av stalinistisk ondska, smider ränker för att lavinartat höja antalet dagismammor och -pappor (!) i den kommunalt drivna barnomsorgen. Nej, i Uppland råder än så länge den ordning som cementerades under femtiotalets ekonomiska storhetsperiod i USA, då husmödrar knaprade amfetamin som vore det Salta katten och husbönderna rökte pipa bakom uppslagna dagblad, även om detaljer har förändrats och amfetaminet bytts mot cafe latte och dagbladen mot sanningssägande bloggar\ref{sanningssaegande bloggar}. Och tur är väl det, när den stolt vajande svenska fanan hotas att halas av shiamuslimska HBTQ-nihilister som törstar efter den utdöende nordiska rasens blod!

\ditem[Uppsala]\label{uppsala}
 är en stad som passande nog ligger på Uppsalaslätten. Här finns Uppsala domkyrka, Gustavianum, Uppsalaslottet, Engelska parken, Karolina Rediviva, Studenternas idrottsplats och mycket annat att besöka och se på. Men det som gör Uppsala riktigt unikt är att ett antal subkulturer som inte finns kvar på annat håll fortfarande lever och frodas i just Uppsala. Där finns det till exempel forfarande gothare och folk som frivilligt lyssnat på Nu Metal. Detta gör att folk som fascineras av det lite aparta dras till staden. Carl von Linné\ref{carl von linné} och Olof Rudbeck är endast två i den månghövdade skara människor som flyttat till Uppsala för att försöka lista ut varför man skulle vilja kombinera rapp\ref{hip-hop} och metal\ref{haardrock}. För att alla ditresta skulle ha någonstans att jobba byggde man tidigt ett universitet i staden, för alla kan faktiskt inte jobba på Slotts senapsfabrik.

\ditem[Uppstoppad uv]\label{uppstoppad uv}
 En uppstoppad uv är en uv\ref{uv} som genom en tragedi, som det alltid är, har dött och därpå påträffats av en människa som stoppat upp den för att bevara den som ett slags prydnad. Uppstoppningen går till så att munstycket till en tryckluftskompressor förs till uvens mun\ref{mun}, det vill säga näbben. Sedan blåses uvens innanmäte ut, och kanske ner i en gryta mustig Strigiformes au Riesling. Därpå pulas mossa och liknande saker ner eller upp i uven. Ståltråd används för att forma uven så att det ser ut som den lever.

\ditem[UR]\label{ur}
 är en förkortning och står för Utbildningsradion, men för samtidigt tankarna till det allmänt förstärkande prefixet ur- som i urbra eller urtråkig. Utbildningsradion är märkligt nog inte en radiokanal utan en del av public services televisionen. UR lär folk att \quotetext{språka} på serbokratiska, hur man sätter samman ett blomarrangemang, hur det är för döva eller finländare att se på TV och mycket, mycket mer. Utan UR hade Sverige fortfarande på 2010-talet sällat sig till världens många u-länder och folkets utbildningsnivå hade varit att jämföra med dumma amerikaners. Kompletera gärna din inlärning via UR med Nissepedia\ref{nissepedia} för att på middagar kunna konversera på ett fritt och avslappnat vis bland de mest pålästa personer, utan att känna dig tillknäppt och efterbliven.

\ditem[Urin]\label{urin}
 är ett lite finare ord för kiss, en guldfärgad vätska som kommer ut ur kroppen efter att man druckit något. Detta är helt normalt och är inget att bli förskräckt över.

\ditem[Utlastarskämt]\label{utlastarskaemt}
 är frekventa skämt i lagermiljö som har som huvudsyfte att underhålla någon, men även påvisa att individen är en så kallad utlastare. Exempel på klassiska utlastarskämt är den upptejpade leksaksgitarren på vilken man kan läsa \quotetext{sex, droger och plasta burar}. Ett annat är att surra fast en cykel (med samma plast som skyddar varorna) fyra meter ovanför marken i en stolpe. Ett mindre lyckat skämt var när Olle plastade fast Ronny i en bur full av ananas och skickade honom till ett daghem i Dorotea.

\ditem[Utrikiska]\label{utrikiska}
 är tungo- eller bokmål som utmärkser sig genom att inte vara svenska. Många svenska musikartister, så som Europe, Stakka Bo och andra har gjort sig ett namn genom att sjunga på just utrikiska.

Utrikiskan har en säregen grammatik som kräver år av övning för att bemästra. Ta till exempel följande exempel;

\begin{itemize}
\item Svenska: Grisen är förvånad.
\item Franska: Liberté, egalitet, gerard depar dieu, et fretârneté.
\end{itemize}

\ditem[Utrotningshotade djur]\label{utrotningshotade djur}
 är djurarter som det inte finns så många individer kvar av. Genomgående för sådana arter är att de har ganska konstiga och lustiga namn. Tyvärr har ingen tittat närmare på detta samband och inga empiriska studier har gjorts av vad som händer om man döper om arterna till något vanligare och lite tråkigare. Djurarter som är utrotningshotade är bland annat: Hawaiis munksäl, Eskimåspov\ref{eskimaaspov}, Dvärgpungsovare\ref{dvaergpungsovare}, Filipinsk apörn\ref{filipinsk apoorn}, Amazonskrake, Rodriguesflyghund, Snefotad ultrapelikan\ref{snefotad ultrapelikan}, Gulsvansad ullapa, Havsmunk\ref{havsmunk}, Flodkanin\ref{flodkanin}.

\ditem[Uv]\label{uv}
 Uvar (latin: \textit{bubo}) är ett släkte fåglar som vid en första anblick kan likna en uggla men egentligen är något helt annat. Uvar kännetecknas av sin skyhöga intelligens, totala brist på empati och allmänna mäktighet. Varje år i mörkaste december hålls Uvarnas Konferens någonstans i de lappländska fjällen. Här diskuterar uvarna hur de ska ta över världen, något som hittills (tack och lov) har misslyckats. 

\ditem[Uv-ljus]\label{uv-ljus}
 är en elektromagnetisk strålning som kommer från uvar\ref{uv}. Varje gång en uv öppnar sin näbb\ref{naebbmun} för att hoa skjuts ljusstrålarna ut och paralyserar allt som träffas av det. Det bästa sättet att skydda sig mot uv-ljus är att vistas i en säker uv-bil, en så kallad \quotetext{SUV}, och/eller att ta för vana att alltid bära uv-säkra solglasögon.

\ditem[Uv-rugby]\label{uv-rugby}
 är en form av rugby framtagen speciellt för uvar\ref{uv}. Som med vanlig rugby är det oklart vad spelet faktiskt går ut på, men uv-rugbyn innebär i alla fall ett väldans massa a-hootin' and a-hollerin' runt en avlång boll.

\ditem[Uvgodis]\label{uvgodis}
 är godis särskilt avsett för uvar\ref{uv}. Det är ett godtaget ord att lägga i Alfapet.

\ditem[Uvmytologi]\label{uvmytologi}
 I den engelska översättningen av det vediska verket Srimad Bhagavatam, kapitel och vers 1.14.14 står att läsa: \quotetext{The shrieks of the owls and their rival crows make my heart tremble. It appears that they want to make a void of the whole universe.} Detta är en uppenbar referens till uvarnas\ref{uv} flertusenåriga plan.

\ditem[Uvsvane]\label{uvsvane}
 En Uvsvane är en maträtt som påminner om en svanskrove och består av en Uv\ref{uv} inbakad i en Svan\ref{svan}. Serveras oftast vid disputationer och andra festliga tillställningar i Västerbotten\ref{vaesterbotten}.

\ditem[Uvtårar]\label{uvtaarar}
 Potent blandning i vätskeform, där alkohol\ref{alkohol} är den instabila\ref{bil} ingrediensen. Gör folk, enligt utsago, lömska och hämndlystna. Blandningens sammansättning presenteras ej här av säkerhetsskäl, helt enkelt.


%%%%%%%%%%%%%%
\newpage
\null
\\
\null
\\
\Huge
V
\normalsize
\\
\null
\\
\null
%%%%%%%%%%%%%%



\ditem[Valentina Vladimirovna Teresjkova]\label{valentina vladimirovna teresjkova}
 (1937 - ) var (och i någon mening är) en Sovjetisk kosmonaut och den första kvinnan i rymden och blev det 1963. Det skulle dröja ytterligare 20 år innan jänkarna skickat upp en kvinna, så tänk på det!

\ditem[Valfrihet]\label{valfrihet}
 Är något som liberaler\ref{tokliberal} ofta hänvisar till i pressade situationer och när orden inte räcker till. T.ex:

att skjuta jonk och sälja röv till radhuspappor samt att kalla en låda från IKEA ett hem. Ja det är ju du som valt den livsstilen so stick with it. Det är helt enkelt lösningen på allt. Du valde själv.

\ditem[Valsvärk]\label{valsvaerk}
 Vanlig åkomma bland dansare. Symptom är bland annat bristningar i dragspelsmuskeln\ref{dragspelsmuskeln} och ömma tår.

\ditem[Valsång]\label{valsaang}
 är en typ av degenererande ljud som drabbar människor i batikkläder, vindsvåningsägare på Södermalm och andra riskgrupper. Efter bara några minuters exponering har valsången intagit lyssnarens hela hjärna och löst upp den sista gnuttan sans. Ett klassiskt exempel är när rockgruppen U2 spårade ur på sin Joshua Tree-turné och skrev upp en 100 kubiks vattentank på ridern där man tryckte ner en strandad tumlare. Fy fan vad det lät.

\ditem[Valuta]\label{valuta}
 är något man betalar med. Valutor har olika värde och man måste därför använda olika mängder valuta när man betalar. Att försöka betala med en valuta som inte är direkt giltig i sammanhanget räknas som socialt extremt.

\uline{Exempel på valutor:}

\begin{itemize}
\item Frimärken
\item Oskrapade skraplotter
\item Skrapade skraplotter med vinst\ref{skrapade skraplotter med vinst}
\item Pantburkar
\item Olåsta cyklar
\item Rikskuponger\ref{rikskuponger}
\item Presentkort
\item Lettisk smuggelcigg
\item Koppar-, zink- och mässingsskrot.
\item Skräp\ref{skraep}
\end{itemize}

\ditem[Vanliga pantade knegare]\label{vanliga pantade knegare}
 är inte anslutna till ett Marxist-leninistiskt parti utan går till jobbet varje dag helt ovetandes om att revolution ligger i deras valkiga händer. Till skillnad från kälkborgaren\ref{kaelkborgare} så har dessa arbetare i alla fall vett att klaga något så jävulskt.

\ditem[Vansinnets historia]\label{vansinnets historia}
 är undertiteln på Håkan Skyttes självbiografi om åren som träpinneslagare i Hoola Bandoola Band. Boken början med att Håkan sitter i en sackosäck hemma i sitt kollektiv och virkar makramé när hans bror Göran\ref{gooran} plötsligt älgar in och skriker att det skett en massa nya orättvisor som dom måste protestera mot. På protestmötet träffar dom Görans dåvarande bandkompis Mikael, som hänförs av Håkans bländande progguppsyn med milt ansiktsuttryck, Gustav Vasa-frisyr\ref{gustav vasa} och helskägg. Mikael inser att en som Håkan är just vad \quotetext{Hoola} (till vardags brukar han kalla bandet så) behöver för att öka sin trovärdighet. För själv ser han ut som en korgosse, och som andre frontman har dom en kroniskt grinig halvdansk. Tyvärr är alla instrument redan upptagna i gruppen så Håkan får nöja sig med att slå på två träpinnar. Publiken betraktar en träpinneslagare i gruppen som väldigt proggigt och succén är ett faktum. Tillsammans åker Håkan och Hoola Bandoola Band runt i flera år och protesterar mot saker. Sen är boken slut.

\ditem[Vaskning]\label{vaskning}
 är ett begrepp\ref{ledingreppet} som nästlade sig in i svenska folkets medvetande sommaren 2009. Då handlade det om champange. Sån skit får väl överklasen hålla på med om de vill, vad vanligt folk gör är att vaska tid. Det gör man genom att istället för att göra det man ska, typ städa, arbeta eller studera, göra något helt annat, gärna icke-produktivt. Nissepedia\ref{nissepedia} skulle inte existera utan omfattande tidsvaskning.

En intressant detalj som aldrig belystes i massmedia är att alla typer av champagneköp är vaskning, oavsett om den hälls i slasken eller inte. Allt bubbel förutom Törley gala\ref{toorley gala} är nämligen pengarna i sjön.

\ditem[Vega Video]\label{vega video}
 var en video- och skivbutikskedja som under 1990-talet fullständigt dominerade marknaden i norra Västmanland och södra Dalarna. Tack vare sitt nätverk av butiker i Norberg, Avesta och Sala kunde kedjan alltid snappa upp de senaste trenderna och vara först med dem på övriga orter där man bedrev kommers. Detta noterades självklart av de dominerande film- och skivbolagen och både Birdnest och Ägg Tapes skickade ofta dit representanter för att försöka sälja in sina senaste stjärnskott såsom Stukas, Rövsvett, Bäddsoffan Brinner, Vrävarna, etc. till butikens personal. Dennis som jobbade i Norberg var tyvärr svårflörtad då han uteslutande lyssnade på Venom och Motörhead.

\ditem[Verklighetens folk]\label{verklighetens folk}
 är en cover med försvenskad text på låten \textit{Common people} av den brittiska popgruppen Pulp. Covern spelades in av den svenske komikern Göran Hägglund, som dock inte uppnådde några större listframgångar med låten.

\ditem[Vernissage]\label{vernissage}
 är något man får gå på om man har vänner som är eller utbildar sig till att bli konstnärer. Det är ett tillfälle för de aspirerande van Gogharna att visa upp sitt ofta hutlöst pretentiösa trams och ett tillfälle för dig att dricka sanslösa mängder rötjut ur bib och svulla OLW-hjärtan. Om man kommer till evenemanget tidigt på dagen är en vernissage ett utmärkt tillfälle att skaffa sig en dagsfylla\ref{dagsfylla}.

\ditem[Veva med kängnäven]\label{veva med kaengnaeven}
 är ett klassiskt tecken för att visa belåtenhet. Lirar Mob 47-Åke ett grisigt solo? Veva med kängnäven! Ser du en träskpunkare\ref{traeskpunkare} som badar i sin egen avföring? Veva med kängnäven! Lyckas du sparka av backspegeln på en sportbil och komma undan med det? Veva med kängnäven!

Att veva med kängnäven ska inte förväxlas med att hytta med näven\ref{hytta med naeven}.

\ditem[Vevlira]\label{vevlira}
 kan ha flera betydelser

\uline{Gammal betydelse: Musikinstrument}

Vevlira brukade vara namnet på ett stränginstrument som spelades flitigt under medeltiden. Det speciella med instrumentet var att du vevade på en vev som var lite som en cirkelformad stråke som gneds mot instrumentets strängar för att skapa ljud. Tonerna på ljuden påverkades genom att trycka på knappar, lite som på en nyckelharpa. Instrumentet användes flitigt av drone-skalder, men efter medeltiden så svalnade intresset för instrumentet, då folk börjat tröttna på drone för att de skotska\ref{skottar} drone-pionjärerna ännu inte skickats till Australien\ref{australien}.

\uline{Ny betydelse: Gitarrteknik}

Efter många diskussioner bland kulturhistoriker så har man nu bestämt att vevlira ska vara namnet på den spelteknik som Pete Townshend, gitarrist, sångare tillika materialförvaltare i rockgruppen The Who, använde. Han slog an strängarna på sina gitarrer (märk plural; han var känd för att byta gitarr lika ofta som han bytte underkläder, orsaken till detta är omtvistad, men ibland slog han an gitarren så hårt att den gick sönder) genom att veva med armen likt en propeller\ref{propeller}. Det gav honom precis det twang i ljudet han var ute efter.

\ditem[Videotex]\label{videotex}
 (i Sverige även kännt som Datavision och Teledata) var från ca 1980 Televerkets stora satsning på ny informationsteknik. Användarna (både företag och privatpersoner) kunde nå centralt lagrad text och enkel grafik via telenätet, och ta del av informationen genom en bildskärmsterminal liknande en enkel dator. Tekniken var enkelriktad så att användaren bara kunde hämta hem information men inte sända någon egen.

Informationen presenterades i en trädliknande sidstruktur och hade en grafisk form som påminner om Text-TV. Svenska massmedieföretag, bland dem TT och flera landsortstidningar, engagerade sig tidigt i den nya tekniken, bl a med nyhetstjänster. Fagersta-Posten\ref{fagersta-posten} engagerade sig dock aldrig. Även Posten byggde under namnet Postel upp tjänster som gjordes tillgängliga genom publika terminaler. Tekniken begränsades av låga överföringshastigheter och att den enkla grafiken inte tillät några bilder. Bland de tjänster som överlevde längst fanns en koppling till bilregistrets databas.

Videotex blev aldrig någon succé i Sverige och lades ner i början av 1990-talet. I Frankrike, under namnet Minitel, hade man med viss framgång använt systemet, med bl.a. gratis terminaler till hela befolkningen. Det mest använda området där var dock enklare former av så kallade \quotetext{heta linjer} där användaren tog emot korta erotiskt orienterade meddelanden.

Videotex skrala genomslag berodde antagligen på att tekniken var relativt dyr att köpa in jämfört med en vanlig telefonkatalog, som i princip kunde uppfylla samma tjänster.

\ditem[Viktiga papper]\label{viktiga papper}
 ska, om de inte förvaras i bakfickan\ref{bakficka}, alltid sättas in i en pärm. Detta går till så här:

\begin{enumerate}
\item Avlägsna eventuella kuvert och liknande.
\item Placera försiktigt pappret i en hålslagare (som kan köpas från välsorterade varuhus eller stjälas från privata arbetsgivare).
\item Kontrollera att pappret ligger rätt.
\item Stansa hål i pappret.
\item Öppna pärmen och öppna pärmens metallöglor.
\item Lägg pappret så att två av öglorna penetrerar två av hålen i arket.
\item Stäng metallöglorna och pärmen.
\end{enumerate}

\ditem[Viktoria (namn)]\label{viktoria (namn)}
 Viktoria hette typ varannan tjej under 90-talet. Således kommer namnet alltid vara förknippat med buffalodojor, jazzbyxor, smygröka bakom knuten, äcklig r and b och misslyckade discon (misslyckade eftersom Viktoria sket i dig om du var kille och frös ut dig om du var brud. Hon hade antagligen askul på alla discon).

I dag heter inte alls lika många Viktoria, vilket beror på att många människor byter namn hela tiden. Ett annat möjligt scenario är att folk som heter Victoria har haft en onormalt hög dödlighet men att kopplingen mellan namnet och dödstalet inte gjorts av vetenskapen.

\ditem[Vild-Hasse]\label{vild-hasse}
 är en korvförsäljare från Malung, Dalarna och livnär sig på att åka runt på marknader och sälja korv och annat som görs av vilt. Vild-Hasses försäljningstaktik bygger till 90\% på att skrika, eller som han själv kallar det, munvighet. Detta har gett honom epitetet \quotetext{Korvens främste skald} och även \quotetext{Korvarnas konung}. När Vild-Hasse inte åker omkring som en annan gårdfarihandlare så kan man hälsa på honom i hans hemmarn Bengtgården där man kan besöka hans lada Vindarnas Tempel och verkligen känna korvatmosfären kring sig.

\uline{Nissequotes}

\begin{itemize}
\item \quotetext{I Guds skafferi finns det bara korv}
\item 
\item "Om faror förfära
\item och nöden tränger på.
\item Om hungern står nära
\item och vännerna gå.
\item Om vänner Dig sviker
\item kan hända dom gör.
\item Ät lite korv.
\item Inget ska då rubba
\item Ditt goda humör."
\end{itemize}

\ditem[Vilskita]\label{vilskita}
 Träcka i gemytligt tempo. Närapå sävligt. Med fördel läsandes en \textit{Kalle Anka och Co} eller någon form av faktabok. En äkta vilskitare lämnar inte avträdet förrän benen somnat minst en gång.

\ditem[Vimpel på pinne]\label{vimpel paa pinne}
 är det tydligaste beviset på att någon är glad eller har roligt.

\ditem[Vin]\label{vin}
 är en blandning av kroppsvätskor och druvsafter, odlade på olika exklusiva håll i världen. Dess främsta egenskap är att det orsakar vinfylla\ref{vinfylla}. Vin framställs genom en komplicerad process som inbegriper både det ena och det andra. Ofta är det någon utomeuropeisk människa som får dra det tyngsta lasset i själva framställningsprocessen, men å andra sidan får en vit människa dricka själva slutprodukten, så på det viset utjämnar det sig i den nyliberala värld vi onekligen lever i. Vill en lära sig mer om vin är det enklast att fråga en vinkännare\ref{vinkaennare}.

\ditem[Vinfylla]\label{vinfylla}
 Liggandes på divan. Huvud\ref{huvud} böjt lätt bakåt. Vindruvor i klasar ovan din öppna mun\ref{mun}. Känner du begär? Tillfredsställ dem. Ljummen vind blåser in under vit toga. Mjuk lust. Du driver runt. Ett tillstånd av långsam extas. Röda löften om ömhet droppar fram mellan våta läppar.

\ditem[Vingmutter]\label{vingmutter}
 En vingmutter är en person med mycket utstående öron. Ett exempel på en sådan är Prins Charles\ref{prins charles} av England.

\ditem[Vinkännare]\label{vinkaennare}
 En vinkännare är en person som inte bara kan mycket om vin\ref{vin}, utan till och med känner och är på \quotetext{du-}-basis med denna kontinentala dryck. Vinkännaren är så van med att diskutera om och fundera på vin att hen inte längre orkar med att uttala förledet vin- i ordet vindruva utan säger \quotetext{en druva}. Medan vi andra väljer vin efter en kombination av lågt pris och rolig etikett vet vinkännaren precis vad tillfället kräver. Medan vi i bästa fall kan känna skillnad på spanskt vin och spanskt lättvin\ref{spanskt laettvin} kan vinkännaren, om så behövs, skapa komplicerade, kodade budskap genom att rada upp vinslattar av olika slag inför en annan vinkännare. Vinkännaren \quotetext{luftar} inte sällan sitt vin och låter det \quotetext{stå och dra till sig} medan vi andra rycker den lilla foliepluggen från kranen på Foot of Africa-lådan redan på farstukvisten och innan vi fått av oss lovikavantarna. Om vi inte bara köper en kasse Kung och Sofiero som brukligt, vill säga.

\ditem[Viskositet]\label{viskositet}
 är ett mått för att beskriva vätskors \quotetext{tröghet}. Desto tjockare en vätska är, desto högre viskositet har den. Eftersom viskositetsbegreppet uppfanns av en tysk (Viskosimund Fritzl år 1847) är det extremt detaljerat, och ämnet beck har exempelvis fyrtiotusen miljarder\ref{fyrtiotusen miljarder} gånger högre viskositet än vatten. Måttet har kritiserats för att vara svårt att använda praktiskt och belackare tar vanligtvis upp det faktum att man inte ens kan fastställa viskositeten i en vanlig människas snor eftersom ämnets tröghet varierar mellan inner- och yttervägg i näsan\ref{naesa}.

\ditem[Vit månad]\label{vit maanad}
 Den finska motsvarigheten till Ramadan.

\ditem[Vladimir Krutov]\label{vladimir krutov}
 är en rysk hockeyspelare.

Under en period bodde Krutov i Östersund och har då rapporterats köra en Ford Fiesta. Därifrån sloganen \quotetext{Ford Fiesta - om det duger åt Krutov så duger det åt dig!}

\ditem[Vodka]\label{vodka}
 Ett finare ord för brännvin\ref{braennvin}. Ordet används främst av folk som vill hålla sig förmer och distansera sig från den obildade pöbeln. Dessa människor köper Smirnoff i tron att det är \quotetext{rysslands bästa vodka} men klarar inte av att uttyda \quotetext{made in UK} på etiketten ens innan man satt i sig innehållet.

Vanligt hederligt folk dricker Renat. Det är gott till all mat.

\ditem[Volvo 240-serien]\label{volvo 240-serien}
 Sammanlagt någonstans kring 4,5 miljarder kilo ren körglädje, gjutna i så gott som hela stycken av världens mest skötsamma stam arbetare.

\ditem[Volvo 740]\label{volvo 740}
 Bil\ref{bil} som oftast går \quotetext{likt en traktor} - driftsäkert och utan krångel. Volvo 740 är även bra på så sätt att den inte behöver omvårdnad likt dess syskon 360 och 480. En annan fördel med just 740 i jämnförelse med andra svenska bilar, som saab, är dess yppeliga bakhjulsdrift som tillåter föraren att köra med \quotetext{ställ} genom kurvor. Med en 740 i ägarregistret kan du även stoltsera som volvoraggare och om inte det är ett plus i kanten så vet jag inte vad som är det.

Volvo 740 tillverkades mellan våren 1984 och hösten 1992.

\ditem[Vädret]\label{vaedret}
 Molnen dominerar i större delen av landet, men lokalt är det klart och kallt. Nu till natten kan det friska i ordentligt på kalfjället, i alla fall i Jämtland och Lappland, men ett snöfall kommer in västerifrån. Det tar sig vidare österut fram till imorgon och en del av snöfallet drar sig också söderut över landet, men det blir förmodligen bara en eller ett par centimeter snö. Mildare luft följer i norr men lokalt kan det forfarande vara kallt, så stora temperaturskillnader i norr imorgon.

\ditem[Vältagravstensfull]\label{vaeltagravstensfull}
 När en person är sexuellt frustrerad och berusad på trettio-fyrtio starkpilsner och går lös på en kyrkogård.

\ditem[Vänort]\label{vaenort}
 En vänort är en kommun i ett främmande land som din kommun har bestämt ska vara kompis med dig. Hela din klass ritar teckningar och skriver brev till en annan klass som sitter i en lika dan skola fast i en del av världen långt, långt bort. Kanske i Danmark\ref{danmark} eller varför inte Estland, ett spännande land på andra sidan Östersjön. Syftet med vänorter är framförallt att bevara freden i världen. Säg väl den människa som skulle vara så hjärtlös att hon sköt en person som hon tvingats skicka slarvigt färglagda huvudfotingar till under hela sin uppväxt?

Vänorten har ofta någon typ av likhet med din egen kommun för att banden ska kännas starkare. Det kan vara att det bor ungefär lika många människor där, att det är lika långt avstånd mellan dem från båda hållen, eller att båda kommunerna ligger i sovjetiska satelitstater. Vänorterna Jämsä och Fagersta\ref{fagersta} har till exempel båda ungefär 20.000 invånare och är finskspråkiga. Stockholm nöjer sig naturligtvis inte med ett grannland utan låter sina barn brevväxla med Mordor.

\ditem[Världens näst ondaste band]\label{vaerldens naest ondaste band}
 är ett begrepp som tillskrivs det band som för tillfället inte räknas som världens ondaste band, utan kvalar in på en andra plats.

Rankningen av band beror på subkultur, då olika subkulturer rankar band efter olika epitet. Fans av grindcore brukar prata om världens snabbaste band (Napalm Death), fans av kängpunk brukar prata om världens råaste band (Disclose), fans av stoner och doom talar om världens tyngsta band (Sleep), och fans av experimentell musik brukar prata om världens konstigaste band (Locust, numera utmanade av Thrones).

Då det kommer till Black Metal så är denna debatt ovanligt, tja, debatterad. Vid en första anblick kan det te sig helt oproblematiskt att utse denna titel. Någon kanske säger \quotetext{Mayhem är världens ondaste band} varpå någon annan säger \quotetext{Nej, Burzum är världens ondaste band}. Då skulle alltså Mayhem vara världens näst ondaste band och Burzum det ondaste. Tredje part kommer då in och påpekar att Burzum inte är ett band utan ett soloprojekt, och diskvalificerar således Burzum från platsen. Då borde alltså Mayhem vara världens ondaste band och kanske Emperor det näst ondaste. Problemet är då att Mayhem har bytt medlemmar jätteofta och det då måste fastställas vilken konstellation som var ondast. Därmed diskvalificeras Mayhem, och sådär håller det på.

För att försöka komma fram till en slutsats här så verkar koncensus på Flashback vara Burzum, medans Close-Ups chefredaktör Robban Becirovic säger Watain\ref{watain}. Således blir världens näst ondaste band: Burzum, eller Watain\ref{watain} då det är allmänt känt att Nifelheim är världens ondaste band (även om vissa menar att det i själva verket är Royal Downfalls republikanska dödspop som tar hem bucklan för det ondaste i musikväg).

\ditem[Världens näst största byggnad]\label{vaerldens naest stoorsta byggnad}
 Folkets palats är med sina 12 våningar (86 meter högt), minst åtta källarplan (alla inte färdigkonstruerade) och 1100 rum världens näst största byggnad belägen mitt i Bukarest. Endast Pentagon är till ytan större. Palatset är ett uttryck för Nicolae Ceauşescus megalomani som gick upp i varv efter ett statsbesök till asien och Nordkorea 1971. För att finansiera bygget togs på 80-talet stora lån från västvärlden på folkets bekostnad; ransonering av livets alla förnödenheter infördes och inhemska varor sattes på export för att bekosta avbetalningarna. För att bygget med tillhörande boulevarden \quotetext{Socialismens seger} skulle få plats beslöt arkitekterna att riva 7000 byggnader, en femtedel av dåvarande Bukarest centrum. 1984 startade byggandet och det sägs att 200 000 arbetare knegade i skift om tre för att hinna med deadline som var satt till 89. 

Ironiskt nog stod merparten av byggnaden klar, undantaget de många våningarna under jord, mitt i revolten dagarna innan Ceauşescu med fru serverades en kärve bly till lunch den 25e december -89. 

Förste person att använda huvudbalkongen Ceauşescu tänkt använda för att tala till folket blev Michael Jackson. Han hälsade ett hav av fans med att skrika \quotetext{I LOVE BUDAPEST!}, varefter han utbuad fick åka hem till USA med sin helikopter från palatsets tak.

Under turistvisningar får man se ca 5\% av byggnadens innanmäte, som idag efter revolutionen 1989 bla rymmer Rumäniens parlament. Herrelösa hundar sägs springa i korridorerna till många turisters förvåning.

\ditem[Världens tråkigaste skämt]\label{vaerldens traakigaste skaemt}
 är, efter omfattande empiriska studier, frasen \quotetext{Working hard or hardly working?}.

\ditem[Världsmusik]\label{vaerldsmusik}
 är musik skapad av folk som inte är vita och som inte kommer från USA eller Västeuropa. Säg att ett gäng Senegaleser sjunger och spelar gitarr-\ref{gitarr} då har du just hört världsmusik. Säg till exempel att du hör en indisk man nynna på en truddelutt...världsmusik! Så många som nittio procent av alla världsmusikskivor har en strand med palmer på framsidan och på baksidan en bild på ett glatt gäng sorglösa svarta musiker som umgås förbehållslöst.

\ditem[Västerbotten]\label{vaesterbotten}
 är en plats där man arbetar. På söndagar\ref{soondag} skäms och ber man, resten av veckan arbetar\ref{goora raett foor sig} man. Här tågar marxistiska skogsarbetare, uppgödda på vegansk husmanskost, in i den täta granskogen för att ideellt bidraga till rikets exportnäring. I landskapets enda tätort, Umeå, regerar socialdemokratins utsände missionär, Lennart Holmgång, i samråd med entreprenörskråets representant, Krister Olsson, över bortskämda kräk med akademisk examen. Här tillåts ingen bakåtsträvande och tillväxtfientlig folkvilja stävja bygg-entusiasmen. Kommunens budget för att godtyckligt flytta omkring olika kommunala institutioner säkras bland annat genom att man årligen låter riva upp och asfaltera om stadens samtliga gator, vilket skapar arbetstillfällen och därmed skatteinkomster. Landskapates många karga bergssluttningar har av dess rättmätiga ägare, det Stockholmska\ref{stockholm} näringslivet, upplåtits till den tålamodsprövande men exotiska urbefolkningen som där vallar sina renar till trolltrummans sataniska rytm.

Västerbotten är, som den uppmärksamme redan slutit sig till, en kulturell smältdegel där samiska snöskotermekaniker lever sida vid sida med laestadianska punklegender och akademiska stjärnskott som lämnat prestigefulla utländska läroverk, lockade av universitetets\ref{universitet} rektors oförtröttliga nigande inför det annorstädes förfördelade storkapitalet (jmfr. Balticgruppen)\ref{balticgruppen}. I inlandet har allianspartierna i Sveriges\ref{sverige} regering låtit den sociala ingenjörskonstens ivriga små bävrar\ref{ivriga smaa baevrar} bygga ett uppsamlingsläger för varghatande Volvoägare och arbetslösa thailändska ungmöer. Det för bygden karaktäristiska svårmodet upplättas av den frihetsälskande lokalpatriotism som tar sig uttryck i skändandet av lika fridlysta som ihjälplågade lodjur.

\ditem[Växjö]\label{vaexjoo}
 är en svensk stad som ligger i Småland eller ungefär där. Trots att Växjö är en ganska gammal stad finns det faktiskt inte så mycket att berätta om den. När Växjö omtalas i populärkulturen är det oftast för att Electric Wizard konstigt nog gjorde sin första sverigespelning där, och för att staden är en av få som har en växande population av stonerskins\ref{stonerskin}.

Om man tar bort X:et i namnet och istället lägger till två R så blir Växjö ett anagram för \quotetext{rävröj}. X kan vara förkortning för \quotetext{straight edge} och RR dess raka motsats \quotetext{rock n roll}. Om man tänker så blir Växjö också ungefär som en palindrom. Kanske är det sådana spaceade resonemang som gör att stonerskinsen söker sig dit.

\ditem[Våld]\label{vaald}
 Att skada andra, fysiskt eller mentalt. Skillnaden mot det närbesläktade ordet jävelskap\ref{jaevelskap} är att det på inget sätt är socialt accepterat att skratta åt någon som blir utsatt för våld.

Vad som däremot är jävligt socialt accepterat är att ogilla våld. Det faktumet lyckades Anton Abele\ref{anton abele} nyttja för att komma in i riksdagen trots sina blott tolv levnadsår. Vi på Nissepedia\ref{nissepedia} betackar oss oftast våld. Oftast, men inte alltid.

\ditem[Warcollapse]\label{warcollapse}
 är ett gäng gubbar som lirar crust. De kommer från Kalmar och flera av dem gillar att röka på jättemycket. Detta faktum avspeglas i flera av deras texter där THC, Stonerpunk och Divine Intoxication är några riktiga pärlor. De skriver ibland om seriösa saker också som att polisen är svin (Nightstick Raids) och att det är skit med krig (Mass Genocide), vilket illustreras särskilt väl på deras skiva Crust as fuck existence.

\ditem[Watain]\label{watain}
 Enligt vissa Världens näst ondaste band\ref{vaerldens naest ondaste band}.

Watain vann 2011 en grammis för albumet \textit{Lawless Darkness}. Två av medlemmarna blev sedan utkastade från den efterföljande grammismiddagen på Café Opera, en innan och en efter huvudrätten.

\ditem[Wctbyxa]\label{wctbyxa}
 Norsk högtidsdräkt.

\ditem[We are the world]\label{we are the world}
 Supergruppen USA for Africas mest kända singel. Otaliga är de gånger barnkörer har framfört sången på skolavslutningar och \quotetext{alternativa} luciafiranden\ref{dansk advent}. Bäst är den på slutet när Bob Dylan tjuter som en sån där get på youtube.

Bono blev så avundsjuk på gruppens framgångar att han bestämde sig för att dra ihop det egna projektet Ireland for Africa där Irlands musikelit också skulle göra gemensam sak. Shane McGowan tackade ja till att medverka på fyllan men hade så klart glömt bort allt på inspelningsdagen. Enya visade också intresse till en början men hoppade av när hennes slagruta visade att Bonos hem låg rakt på ett currykors. Phil Lynott hade just dött så honom var inge idé att fråga och Bob Geldof ville självklart vara med men blockerades av The Edge som vägrade ha med fler gitarrister. Så i slutändan blev det som vanligt bara Bono, Edge och The Coors som än en gång sjöng något smäktade om kärlek. Låten saknar ännu distribution

\ditem[Weiron Holmberg]\label{weiron holmberg}
 Till skillnad från dagens bortklemade scenskoleprodukter hade Holmberg en rad hederliga arbeten samtidigt som han var djupt engagerad i revyscenen. Så småningom fick han spela nyckelroller i 80-talets kioskvältare såsom Jönssonligan, Sällskapsresan\ref{saellskapsresan}, med mera.

Som alla äldre skådespelare från Göteborg\ref{gooteborg} är Holmberg självklart kommunist\ref{kommunist}.

\ditem[Wham, bam, thank you ma'am]\label{wham, bam, thank you maam}
 är den fetaste raden i David Bowies låt Suffragette City.

\ditem[William Banting]\label{william banting}
 var en gravt överviktig engelsk dögrävare som levde på 1800-talet. Under den viktorianska eran var det annars riktigt inne att se ut som en välsmord isterbuk men just i Bantings fall tyckte hans husläkare att det blivit lite för mycket fredagslyx\ref{fredagslyx} och uppmanade honom att göra något åt saken. Banting tänkte att ett smart sätt att tappa i vikt borde vara att äta mindre, och mycket riktigt! Han skrev om sina forskningsrön i en bok som sålde i så feta upplagor att hans eget namn blev synonymt med viktminskning. Sveriges\ref{sverige} bästa motsvarighet är Bantar-Björn.

\ditem[World Wide Web]\label{world wide web}
 är en låt som Nick Borgen, Norges svar på Tom Jones, hittade på. Han tävlade med den i melodifestivalen 1997 och slutade på nionde plats. På melodifestivalen körade it-pinuppan Helen Wellton.

\uline{Text}

\textit{World wide web}
\textit{World wide web}
\textit{Här lever jag lycklig, här finns ingen stress\ref{stress}}
\textit{Här är min nya hemadress}

\textit{Jag väntat så länge så jag tog min chans}
\textit{Jag ville flytta nån annanstans}
\textit{Ikväll är det party, jag flyttar in}
\textit{Så följ med mig hem på ostron och vin\ref{vin}}

\textit{Oh uh oh uh ohh, jag har party ikväll}
\textit{Oh uh oh uh ohh, om du vill får du gärna följa med}
\textit{När staden krymper och du känner dig less}
\textit{Sök upp mig på min hemadress}

\textit{World wide web}
\textit{World wide web}
\textit{Här lever jag lycklig, här finns ingen stress\ref{stress}}
\textit{Här är min nya hemadress}

\textit{Ta med dig vänner till ett häftigt place}
\textit{Surfa hem till mig i cyberspace}
\textit{De vildaste drömmar vi hade förr}
\textit{Kanske finns här bakom min öppna dörr}

\textit{Oh uh oh uh ohh, jag har party ikväll}
\textit{Oh uh oh uh ohh, om du vill får du gärna följa med}
\textit{När staden krymper och du känner dig less}
\textit{Sök upp mig på min hemadress}

\textit{World wide web}
\textit{World wide web}
\textit{Här lever jag lycklig, här finns ingen stress\ref{stress}}
\textit{Här är min nya hemadress}

\ditem[Wunderbaum]\label{wunderbaum}
 eller Magic tree som den kallas på Brittiska öarna, är en doftprodukt som tyvärr inte alls är tysk utan tvärt om amerikansk som hängs upp i inre backspegeln\ref{inre backspegel} på respektabla bilar för att sprida välbefinnande. 1952 rullade den första granen ut från fabriken i Watertown, New York, och blev snabbt en braksuccé för upphovsmannen Julius Sämann som till vardags arbetade som mjölkbud och kemist. Originaldoften var den ännu populära tallbarr, som dominerade marknaden tills Sämann introducerade vanilj på 80-talet. I Sverige har granarna funnits sedan 1962 och är enligt företagets hemsida en symbol för \quotetext{friskhet och kvalitet}. Och det finns väl inte mycket att orda om där.

%%%%%%%%%%%%%%
\newpage
\null
\\
\null
\\
\Huge
X
\normalsize
\\
\null
\\
\null
%%%%%%%%%%%%%%

\ditem[X3m sports]\label{x3m sports}
 (uttalas /extri:m spårtsh/) är inte alls sporter utan sådana fritidsaktiviteter som unga män som inte spelade fotboll höll på med på 90-talet, dvs snowboard (åka på en skida), klättervägg (ung. \quotetext{inte-nudda-mark}) och forsränning (gummibåt). x3m sports skapade, trots dess utövares bergfasta övertygelse om att det var en anti-social subkultur för obotliga ensamrävar, en mycket stor marknad som omsatte enorma summor pengar, speciellt genom att entreprenörer hällde ihop lite olika vätskor i en burk, tillförde karamellfärg och sålde det till x3m sportutövarna som gladeligen betalade i tron att det var ett slags magisk dryck (energidryck) som gav dem energi och kraft att klättra på sina väggar, åka på sin lilla skida och fan och hans moster. Utövarna lyssnade ofta och gärna på x3ma band så som Millencollin, Bad Religion och NOFX och hade sina byxor halvt nerdragna, samt mössa inomhus. Nuförtiden för fenomenet en tynande tillvaro, till nästan allas glädje, och utövas mest av australiensare\ref{australien} som talar med för hög röst.


%%%%%%%%%%%%%%
\newpage
\null
\\
\null
\\
\Huge
Y
\normalsize
\\
\null
\\
\null
%%%%%%%%%%%%%%

\ditem[Yngwiefiering]\label{yngwiefiering}
 är när en person börjar skrika och skräna till synes utan anledning. Därefter häller personen i sig stora mängder sprit och börjar bete sig ännu värre. Yngwiefieringen har börjat. Det enda sättet att kurera en person som börjat yngwiefieras är att ge den en spruta bon joviaccin. Snacket kommer då fortsätta gå men aldrig någonsin leda till handling.


%%%%%%%%%%%%%%
\newpage
\null
\\
\null
\\
\Huge
Z
\normalsize
\\
\null
\\
\null
%%%%%%%%%%%%%%

\ditem[ZiL-fil]\label{zil-fil}
 är smeknamnet på de vägfiler i Moskva som är särskilt avsedda för toppolitiker och höga tjänstemän\ref{storfraesare}, till exempel Karelin\ref{aleksandr karelin}. Filerna anlades i mitten av 1960-talet på initiativ av politbyråns dåvarande ordförande Leonid Brezhnev, som tyckte att det var för jävla tråkigt att sitta i bilkö. Namnet syftar på det ryska limousinemärket ZiL som tillhandahöll nästan alla de pansarbilar som sovjetiska ledare föredrog att resa i. Filerna löper längst ut till vänster på flera av centrala Moskvas huvudleder och vållar ibland tyst irritation hos medtrafikanter som får vänta framför rödljuset en halvtimme för att kunna korsta en ZiL-fil (det finns nämligen en särskild trafikcentral på Kreml som kontrollerar stoppljusen runt omkring så att ingen viktig oljegark\ref{oljegark} blir stående). I perestrojkans efterdyningar lättade Gorbatjov en aning på restriktionerna så att även utryckningsfordon får nyttja filerna, men för övriga är det än idag fortfarande stopp. Vad påföljden blir för den som olovligen kör i en ZiL-fil har Nissepedia\ref{nissepedia} tyvärr inte lyckats utröna men vi tror definitivt inte att det är värt det.

\ditem[Zoom]\label{zoom}
 är ett mestadels optiskt fenomen som förvränger uppfattningen av rumsliga avstånd. Det är exempelvis zoomen som gör att en myra blir stor som en jättemyrslok\ref{jaettemyrslok} om du tittar på den i en kikare\ref{kikare}. Det låter lite skrämmande, för om myror var så stora skulle ju jättemyrsloken dö ut. Men det är ingen fara, för zoom är som sagt bara ett optiskt fenomen och inte på riktigt.

För den som inte har vit rock på sig och jobbar i ett laboratorium med att läsa tjocka böcker hela dagarna, kan det vara lite svårt att förstå hur detta egentligen fungerar. Men tänk då att du ligger hemma i soffan med en Tuborg\ref{tuborg} i handen och lyssnar på \textit{Dark side of the moon}. Du stänger ögonen och lagom till \textit{Time} har du omsluts så pass av ljuden att du känner hur stjärnorna lyser klarare där ute och himlen kryper närmare. Din kropp blir lättare och när så \textit{Eclipse} äntligen kickar igång är det som att du och Gubben i månen är bästa polare. Det är för att du med hjälp av skivan zoomat in rymden.


%%%%%%%%%%%%%%
\newpage
\null
\\
\null
\\
\Huge
Å
\normalsize
\\
\null
\\
\null
%%%%%%%%%%%%%%

\ditem[Åka på safari]\label{aaka paa safari}
 Slang för att dricka öl med olika sorters djur på burken. För att det ska räknas som ett riktigt safari måste bolagskassen (även känd som Djungelkasse) innehålla minst fem olika djuröl. Nissepedia tipsar naturligtvis om dom bästa alternativen samt fallgroparna.

\uline{Illeröl}

Formellt kallad \textit{5,2:an}. Billigt pris, neutral smak och ingår i standardsortimentet. Helst vill du så klart bara köpa denna, men så jobbar inte en riktig äventyrare

\uline{Krokodilöl}

Guldmedalj i \textit{World beer cup} 1991. Dessutom Benny Bus\ref{benny bus} favoritöl. Ett givet val.

\uline{Elefantöl}

Farfars favorit. Finns bara i liten burk så glider snabbt ner. Föredömlig APK då det är en dansk\ref{danmark} produkt.

\uline{Bjørnebryg}

Har tappat lite av den glans som omgärdade burken för ett par år sedan. Håller dock fortfarande vad den lovar och blir avslagen efter att du druckit ungefär en tredjedel.

\uline{Old speckled hen}

Gammal späckad höna, ska det va nåt? Nja, ur smaksynpunkt är den ungefär lika rolig som att suga på en gammal disktrasa. Men har du börjat med de fyra ovanstående bör smaklökarna vara bortdomnade vid det här laget så det är ingen fara.

\uline{Red seal ale}

Rostig smak och ganska dyr. Köpes endast om expeditionen tar plats på en löne-/bidragshelg.

\uline{Sleepy Bulldog}

Lika mäktig som en Guinness. Den här rackaren bör du beta av tidigt. Utmärkt om du inte hunnit klämma en burgare innan festen.

\uline{Old ox\ref{old ox}}

Samma för- och nackdelar som bulldogen.

\uline{Girafföl}

Mesigare kopia av elefanten. Köpes bara för att få en större blandning.

\uline{Älgöl}

Något som man ska lägga i botten på kassen men är utmärkt avslutning med sin spritiga smak. Perfekt när man börjat tappa tempo.

\uline{Karhu}

En klassiker från vår östra landshalva. Bra standardbärs. Har varit \quotetext{veckans öl} på Carmen i Stockholm i tio år.

\uline{Cobra}

En öl man egentligen bara dricker när man äter indisk mat men saluförs som \quotetext{Internationell märke} hos gröna skylten. Finns även i beställningssortimentet som brittiska \quotetext{Kobra King} om man vill hålla sig inom imperiet.

\uline{Falcon bayersk}

Så självklar att den lätt glöms bort. Du vet redan hur den här smakar.

\uline{Saxon}

Trots sitt balla namn och feta asgam på etiketten är detta en riktigt fånig öl. För allergiker\ref{allergi} då den är glutenfri. Mellanöl från Finland.

\ditem[Åka vikingaskepp]\label{aaka vikingaskepp}
 Slang för att dricka vodka av märket Explorer.

\ditem[Åkarbrasa]\label{aakarbrasa}
 kommer att vara det enda sättet för folk att hålla sig varma på sen miljöpartisterna förbjudit oljepanna och kärnkraftverk. Det går ut på att man kramar sig själv typ, så det är lite som onani.

\ditem[Åke Cato]\label{aake cato}
 Svensk nöjesprofil i klass med Lennart Hyland och Adde Malmberg\ref{adde malmberg}. Återuppstår varje år kring jul med sin monsterhit \textit{Vår julskinka har rymt}. Men Åke är egentligen mycket djupare än så. Så här skrev han till exempel på sin blogg (bara det ett bevis på att han ständigt är aktuell) den 27 november år 2012:

"\textit{Det är oundvikligt att damer och herrar i min ålder då och då kommer att tänka på döden.}
\textit{Sådana tankar bör man omedelbart slå bort då de kan leda till svårartade depressioner.}
\textit{Mitt råd till mina jämnåriga, och alla andra med för den delen, är därför att sluta tänka på döden och i stället tänka på något annat, till exempel kallops, matematik eller lippizanerhästar.}

\textit{Ett ganska visset råd, men jag har inget annat."}

\ditem[Åke Ohlmarks]\label{aake ohlmarks}
 Många svenskar känner Åke Ohlmarks i första hand som översättare av J.R.R Tolkiens\ref{j.r.r tolkien} \textit{Sagan om ringen}-triologi. Men Ohlmarks var så mycket mer än bara översättare. Han var grosshandlarson, granne, far, make, vän och kollega. Den som vill veta mer om den Åke Ohlmarks som hans nära och kära kände, Åke utanför kändisskapets ljuskägla, kan läsa någon eller samtliga av hans sex memoarer.

\ditem[Åkerdisco]\label{aakerdisco}
 Ett åkerdisco påminner om ett skogsrave\ref{skogsrave}, men skiljer sig ändå på vissa sätt från detta, bland annat genom att åkerdiskot äger rum på en åker någonstans i sydsverige\ref{sverige}. Populära musikartister på åkerdiscoteket är Cat Stevens, ABBA och The Temptations.

\ditem[Ålands demitaliseringsdag]\label{aalands demitaliseringsdag}
 är en åländsk högtid som firas varje år den 30:e mars med pompa, ståt och allmän helgdag. Mer specifikt handlar dagen för ålänningarna om att hylla det fredsavtal som slöts efter Krimkriget 1856. Nu ligger ju inte Krimhalvön så jättenära Åland, men Frankrike passade faktiskt på att mula en fästning som ryssarna ställt upp där när man ändå hade vägarna förbi. Nödvändigheten i att fira att man fått en gammal fornborg pajad kan man ju fundera över. Har det verkligen aldrig hänt något mer spektakulärt på Åland, liksom? Vore det inte roligare att fira typ att ålandskrisen\ref{aalandskrisen} fick en lyckligt slut? Men är man ett litet självstyrande landskap får man ibland ta ut svängarna för att klara sig, och ålänningarna svarar bara att \quotetext{-Haters always gonna hate} när man försöker reda ut detta närmare.

\ditem[Ålandskrisen]\label{aalandskrisen}
 När amerikanska spionplan upptäckte ryska raketbaser på Kuba blev det inledningen på det som kom att kallas Kubakrisen. Den svenska motsvarigheten är inte lika dramatisk eller långdragen, men ändå mycket, mycket allvarlig. Åland är som alla vet en ö mellan Sverige\ref{sverige} och Finland\ref{finland} som officiellt hör till Finland, fast ändå inte. Vad som däremot är säkert är att Åland är en demilitariserad zon\ref{aalands demitaliseringsdag}, vilket innebär att inget av länderna tillåts ha militära enheter på ön. Ålandskrisen kom ur att två svenska officerare söp och svinade på en finlandsfärja på en konferensresa till Åbo för att prata om hur man bäst krigar. Dessa två i tjänst och därför uniformerade kronans män klev av för att spy på Åland. Det här i sig är såklart inget problem, hade det inte varit för att de båda var uniformerade. Detta var alltså enligt rådande förordning en krigshandling och Sverige\ref{sverige} och Finland\ref{finland} låg åter i krig för första gången sen träffningen vid Ratan. Det blev dock inte så mycket till krig, utan de svenska officerarna torkade sig runt munnen och bad Tarja Halonen om ursäkt varpå freden åter rådde i Skandinavien.

\ditem[Ålidhem]\label{aalidhem}
 Bostadsområde i Umeå\ref{umeaa} som enligt säkra källor från Flashback befolkas av studenter och hårdrockare\ref{haardrock} med vänsteråsikter.

\ditem[Ånäset]\label{aanaeset}
 är en liten bygd vid västerbottens\ref{vaesterbotten} kustland, mest känd för att världens största osthyvel ståndaktigt hälsar alla som åker förbi på E4an välkomna (alternativt välkomna åter) till ostriket. Maud Olofsson\ref{maud olofsson} ska vid ett besök ha sagt att osthyveln fyllde samma funktion för Ånäset som kolossen gjorde för Rhodos och att bygden aldrig skulle bli bortglömd.

\uline{Andra saker Ånäset gjort sig känt för:}

\begin{itemize}
\item Den utmärkt välsorterade pappershandeln
\item Skoaffären som alltid har rea
\item Ett brutalt familjemord i slutet på 90-talet
\item Att det är en av de få orter i Sverige\ref{sverige} där man fortfarande kan bli jagad av raggare för att man ser ut som en kommunist\ref{kommunist}.
\end{itemize}

\ditem[Årets göteborgare]\label{aarets gooteborgare}
 är en hederstitel som varje år sedan 1993 föräras en person som särskilt utmärkt sig och gjort något berömvärt. Bland pristagarna finns exempelvis: Leif \quotetext{Loket} Olsson, Thomas Ravelli, Lotta Engberg och Thomas von Brömssen.

För att erhålla utmärkelsen ska man bland annat uppfylla följande kriterier:

\begin{itemize}
\item Gjort något gott.
\item Vara född i Göteborg - eller någon annanstans.
\item Vara en god ambassadör för Göteborg.
\end{itemize}

Som ni märker på tidigare pristagare är detta i princip bara kriterier som finns där för syns skull. Det viktigaste verkar i själva verket vara att man supit med Glenn Strömberg.

\ditem[Åtta]\label{aatta}
 är en seglarknop som förhindrar att skot löper genom öglor. Det kan verka konstigt men är sant. Till utseendet liknar åttan Konsums\ref{konsumbutik} gamla logotyp ställd på högkant.



%%%%%%%%%%%%%%
\newpage
\null
\\
\null
\\
\Huge
Ä
\normalsize
\\
\null
\\
\null
%%%%%%%%%%%%%%



\ditem[Äckligt godis]\label{aeckligt godis}
 är ett av de bästa bevisen på att valfrihet inte är den universella lösningen på alla världens problem. I motsats till vad den danske\ref{danmark} filosofen Lars von Trier säger får man med äckligt godis inte ta det goda med det onda utan det äckliga med det onda. Det räcker inte med att Karius och Baktius invarderar din munhåla och bildar fetor ex ore\ref{fetor ex ore}, du ska dessutom lida romrussinets alla kval på vägen. Precis som Lenin sade behövs här en stark kader som tar kontroll över frågan och leder folket i rätt riktning. Aldrig mera Plopp Lakrits!

\ditem[Ädelost]\label{aedelost}
 är en så kallad oxymoron (från grekiskans \quotetext{dum som en dum tjur}), alltså en sammansättning av två element som är varandras antites. Det finns inget ädelt med ost, särskilt inte med vad som kallas ädelost, då den är möglig och luktar konstigt. Akten att äta ost kan dock vara ädel om den utförs i rätt kontext (typ när man hjälper en kompis äta upp sista biten pizza på dennes talrik då resten av gänget är borta, så att vännen ska slippa skämmas, trots att man är bautamätt och håller på att sprängas. Alternativt då man, när ens respektive eller någons idiotiska kusin\ref{kusin} välter ut fonduegrytan, kastar sig fram för att skydda de man älskar från den kokande osten genom att äta den i luften). Men det finns inget ädelt med att sitta och äta en ostmacka\ref{ostmacka} i köket, inte ens om osten är möglig och luktar skit.

\ditem[Äganderätt]\label{aeganderaett}
 kan vara lite vad som helst, beroende på perspektiv\ref{perspektiv}.
Den vidast spridda definitionen går dock ut på att dom rika ska kunna få vad dom vill från alla som inte är rika.

\ditem[Ägg]\label{aegg}
 Att kalla någon för ett ägg betyder att man anser personen vara något sinnesslö.

\ditem[Ägmästare]\label{aegmaestare}
 En ägmästare är någon som, medvetet eller omedvetet, får det att äga för andra. En typisk ägmästare sprider äg omkring sig och behöver bara kliva in i ett rum eller kränga på sig en gitarr\ref{gitarr} för att andra ska känna att allt plötsligt blev jävligt sjysst.

\ditem[Ärtpåse]\label{aertpaase}
 En ärtpåse är ett idrottsredskap och består vanligtvis av två ihopsydda tygbitar. Mellan dessa finns en mängd torkade ärtor som förlänar påsen en viss tyngd. Ärtorna gör att påsen är samtidigt mjuk och hård på ett ganska konstigt vis.

Ärtpåsens huvudsakliga användningsområde är olika övningar inom skolgymnastiken. I många av dessa ska två eller flera lag hämta och lämna ärtpåsar, eller \quotetext{skatter} som de då kallas, på olika utmarkerade ytor i gymnastiksalen, emedan konkurrerande lag ämnar hämta tillbaka dem. Ett ickeobligatoriskt försvårande moment är att en av deltagarna fungerar som \quotetext{von Ribbentrop} och tillåts stjäla ärtpåsar från alla lag. De ärtpåsar \quotetext{Ribbentrop} stulit räknas som kasserade och kommer ej tillbaka in i spelet.

Mer stillsamma övningar går ut på att påsen balanseras på hjässan medan gymnastikdeltageran agilt balanserar på en bom. Ärtpåsen bildar tillsammans med det färgade tygbandet, ett viktigt element i trollkull\ref{trollkull}, basen i gymnastikämnets redskapspark.

\uline{Obeskrivligt förtryck}

Ärtpåsen har ännu inte upptagits på olympiekommitténs officiella lista över godkända idrottsredskap, trots ihärdiga påtryckningar. Denna nonchalans kan med all rätta uppfattas som provocerande och ovärdig ett organ av nämnda kommittés dignitet.

\ditem[Ättestupa]\label{aettestupa}
 En ättestupa är en plats, ofta på ett berg, där gamla och sjuka togs av daga för att inte vara familjen eller samhället till last. Åldringarna sätts i en pulka och skickas utför stupet till efterlivet, sen går de efterlevande vidare med sina sysslor. Inget mer med det.

\uline{Ättestupans historia}

Den moderna vetenskapen menade länge att det här inte var något som hörde den moderna världen till och att det inte praktiserades någonstans annat än i historiens annaler, men ack så fel förståsigpåarna\ref{foorstaasigpaaare} hade. Det hela började 2011 i Malå\ref{malaa} när den senmoderna kapitalismens oförmåga att skapa värde lett till att landstingspolitikerna inte hade något annat val än att dra in ambulansen där. Byborna protesterade högljutt till ingen nytta och fick på nytt börja släpa sina till åren komna släktingar till Erik Sjulsa-stenen vid Tjamstans\ref{tjamstan} södra bergsida. Gubbarna och tanterna sattes i ackjor, fick höra något i stil med \quotetext{Förlåt farfar, men vi orkar faktiskt inte ta hand om dig längre. Ha det bra!} och sen med en lätt men bestämd knuff skickas utför stupet. Denna braksuccé spred sig till resten av landet i takt med att åldringsvården i Sverige\ref{sverige} börjat gå ut på att väga kissblöjor istället för att faktiskt ta hand om gamlingar och typ ge dom mat och liknande av Carema bortprioriterade arbetsmoment.

\ditem[Äventyrsbad]\label{aeventyrsbad}
 Ett säkert tecken på att en kommun är på väg ner i fördärvet är när man anlägger ett äventyrsbad.



%%%%%%%%%%%%%%
\newpage
\null
\\
\null
\\
\Huge
Ö
\normalsize
\\
\null
\\
\null
%%%%%%%%%%%%%%



\ditem[Ödla på pinne]\label{oodla paa pinne}
 Kinas nationalrätt. En vanlig kinesisk risbonde äter i regel den mindre lyxiga sortens ödla på pinne, kopparorm\ref{kopparorm} på pinne, samtidigt som en välbärgad grisfarmare eller kalligrafimästare har råd med ödla med vingar på pinne.

\ditem[Öra]\label{oora}
 är en kroppsdel som man lyssnar med. Det är som en tratt av kött som sitter på sidan av huvudet. Känn efter i höjd med ögonen ungefär så ska du se att du hittar rätt.

\ditem[Örnäset]\label{oornaeset}
 ligger för er som inte vet,i Luleå. Invånarna i detta trivsamma område gillar sprit, knark och sport på tv.

Har man tur, så kan man en fredagskväll stöta på en generös gubbe alternativt kvinna med finskt förnamn och tatueringar på händerna utanför kvarten; som bjuder på en slant eller två.

Örnäsets folkdräkt består av wctbyxor\ref{wctbyxa} och huvtröja och bäres av innevånarna till både vardags och fest.

Stora horder med glopar\ref{glop} finns vanligen utanför matvarubutiken Coop Forum under varierande tider på dygnet.

Det bästa med Örnäset är Centrum där man förutom Coop finne två friseringar, en tatueringsstudio, ett suspekt gym, en blomsteraffär, två banker, ett turklivs, två pizzerior, en spelbutik, en dataaffär och två krogar. Innan Coop byggdes fanns även Systembolag, detta medför att man än idag kan träffa på en del professionella drickare som liksom inte släppt taget än.

Utöver Centrum finns den majestätiska Edeforsgatan, pensionärsghettot Kronogårdsringen samt en del villaområden i utkanterna där det uppenbarligen bor folk men ingen vet vem de är.

\uline{Kuriosa}

Torbjörn Säfve står staty utanför spelbutiken.

Den ena frisersalongen var tidigare hundfrisering, oklart om det är samma innehavare.

\quotetext{Colamannen} träffas varje dag i Örnäset centrum.

\ditem[Övre magmunnen]\label{oovre magmunnen}
 Den övre magmunnen är en ringmuskel som sitter mellan matstrupen och magsäcken. Dess funktion är att förhindra att mat tränger tillbaka upp i matstrupen. Det händer dock ibland att den förblir öppen och orsakar då olika olägenheter, däribland fetor ex ore\ref{fetor ex ore}, vilket i sin tur ofta leder till skilsmässa.



%%%%%%%%%%%%%%
\null
\\
\null
\\
\Huge
Symboler
\normalsize
\\
\null
\\
\null
%%%%%%%%%%%%%%



\ditem[13]\label{13}
 är benämningen på den fasta nyckel\ref{fasta nycklar} som används för att justera muttrar på alla maskiner med lite självrespekt. Kräver din maskin mindre nycklar än så har du med all säkerhet blivit lurad av en storfräsare\ref{storfraesare} som skrattar gott när hen tömmer sin hästhandlarplånbok\ref{haesthandlarplaanbok} framför en dräglande bankir. Ställ in skräpet\ref{trivselskrot} bland dom andra trasiga plastleksakerna i garaget och skaffa något rejält istället!

\uline{Trivia}

Oddsen att just denna fasta nyckel\ref{fasta nycklar} är nednött till obrukbarhet är 100\% högre än för de andra nycklarna i ditt set.

\ditem[50 million piece of shit]\label{50 million piece of shit}
 Brittiska tidningars smeknamn på fotbollsspelaren Fernando Torres efter att denne sålts till Chelsea FC och bara gjorde två betydelselösa mål under första säsongen.

\end{description}
\end{multicols}

\newpage

\setlength\parindent{0pt}
\clearpage
\markboth{}{}
~\newline
{\textbf{\large Ac/dc}}
\begin{multicols}{2}
AC/DC-gitarr \mydots \indexref{acdc-gitarr}\newline
Australien \mydots \indexref{australien}\newline
Bonfire \mydots \indexref{bonfire}\newline
Den nya sångaren \mydots \indexref{den nya saangaren}\newline
Je ne sais quoi \mydots \indexref{je ne sais quoi}\newline
\end{multicols}
~\newline
{\textbf{\large Antiken}}
\begin{multicols}{2}
Abdera \mydots \indexref{abdera}\newline
De gamla grekerna \mydots \indexref{de gamla grekerna}\newline
Epikurism \mydots \indexref{epikurism}\newline
Hefaistos \mydots \indexref{hefaistos}\newline
Johannes Brost \mydots \indexref{johannes brost}\newline
Oidipuskomplex \mydots \indexref{oidipuskomplex}\newline
\end{multicols}
~\newline
{\textbf{\large Arkitektur}}
\begin{multicols}{2}
Dojo \mydots \indexref{dojo}\newline
Fördelar med att bo i gryt \mydots \indexref{foordelar med att bo i gryt}\newline
Globen \mydots \indexref{globen}\newline
Kaknästornet \mydots \indexref{kaknaestornet}\newline
Nedlagda industribyggnader \mydots \indexref{nedlagda industribyggnader}\newline
Stia \mydots \indexref{stia}\newline
Världens näst största byggnad \mydots \indexref{vaerldens naest stoorsta byggnad}\newline
\end{multicols}
~\newline
{\textbf{\large Bandyspelare}}
\begin{multicols}{2}
Buss \mydots \indexref{buss}\newline
Gösta Snoddas Nordgren \mydots \indexref{goosta snoddas nordgren}\newline
Pelle Fosshaug \mydots \indexref{pelle fosshaug}\newline
Shockrockare \mydots \indexref{shockrockare}\newline
\end{multicols}
~\newline
{\textbf{\large Belgien}}
\begin{multicols}{2}
Bailando \mydots \indexref{bailando}\newline
Belgien \mydots \indexref{belgien}\newline
Belgisk jättekanin \mydots \indexref{belgisk jaettekanin}\newline
Belgisk öl \mydots \indexref{belgisk ool}\newline
Háfrónska \mydots \indexref{háfrónska}\newline
Place Jourdan \mydots \indexref{place jourdan}\newline
Slagg \mydots \indexref{slagg}\newline
\end{multicols}
~\newline
{\textbf{\large Bergsbestigning}}
\begin{multicols}{2}
Edmund Hillary \mydots \indexref{edmund hillary}\newline
George Everest \mydots \indexref{george everest}\newline
Mount Everest \mydots \indexref{mount everest}\newline
Tenzing Norgay \mydots \indexref{tenzing norgay}\newline
Snöskor \mydots \indexref{snooskor}\newline
\end{multicols}
~\newline
{\textbf{\large Bizarra fenomen}}
\begin{multicols}{2}
Äckligt godis \mydots \indexref{aeckligt godis}\newline
Allting \mydots \indexref{allting}\newline
Ambigram \mydots \indexref{ambigram}\newline
Arselhaka \mydots \indexref{arselhaka}\newline
Australien \mydots \indexref{australien}\newline
Bade \mydots \indexref{bade}\newline
Blandsvulst \mydots \indexref{blandsvulst}\newline
Brakflopp \mydots \indexref{brakflopp}\newline
Chemtrails \mydots \indexref{chemtrails}\newline
Dokumentärhora \mydots \indexref{dokumentaerhora}\newline
Dressmann \mydots \indexref{dressmann}\newline
Fagersta-Posten \mydots \indexref{fagersta-posten}\newline
Flanera \mydots \indexref{flanera}\newline
Freikörperkultur \mydots \indexref{freikoorperkultur}\newline
Gurkvatten \mydots \indexref{gurkvatten}\newline
Hippie \mydots \indexref{hippie}\newline
Innebandy \mydots \indexref{innebandy}\newline
Kalaskula \mydots \indexref{kalaskula}\newline
Kommunslogan \mydots \indexref{kommunslogan}\newline
Ledingreppet \mydots \indexref{ledingreppet}\newline
Lillgammal \mydots \indexref{lillgammal}\newline
Märkliga sammanträffanden \mydots \indexref{maerkliga sammantraeffanden}\newline
Manslyssna \mydots \indexref{manslyssna}\newline
Norsjöblicken \mydots \indexref{norsjooblicken}\newline
Parisare \mydots \indexref{parisare}\newline
Pelle Svensson \mydots \indexref{pelle svensson}\newline
Personligt varumärke \mydots \indexref{personligt varumaerke}\newline
Repet \mydots \indexref{repet}\newline
Sextant \mydots \indexref{sextant}\newline
Skogsrave \mydots \indexref{skogsrave}\newline
Sommar \mydots \indexref{sommar}\newline
Sorbet \mydots \indexref{sorbet}\newline
Stenlapp \mydots \indexref{stenlapp}\newline
Stöcksjö sunny resorts \mydots \indexref{stoocksjoo sunny resorts}\newline
Svåra saker \mydots \indexref{svaara saker}\newline
Svarta tavlan \mydots \indexref{svarta tavlan}\newline
Tjena Roger! \mydots \indexref{tjena roger!}\newline
Tysk toalett \mydots \indexref{tysk toalett}\newline
Utlastarskämt \mydots \indexref{utlastarskaemt}\newline
Uv-ljus \mydots \indexref{uv-ljus}\newline
Valfrihet \mydots \indexref{valfrihet}\newline
Vaskning \mydots \indexref{vaskning}\newline
\end{multicols}
~\newline
{\textbf{\large Brott och förseelser}}
\begin{multicols}{2}
Balticgruppen \mydots \indexref{balticgruppen}\newline
Den nordiska smuggeltriangeln \mydots \indexref{den nordiska smuggeltriangeln}\newline
Dubbelsovla \mydots \indexref{dubbelsovla}\newline
Flaka \mydots \indexref{flaka}\newline
Gylfa \mydots \indexref{gylfa}\newline
Privatspanare \mydots \indexref{privatspanare}\newline
Rygga \mydots \indexref{rygga}\newline
Shockrockare \mydots \indexref{shockrockare}\newline
Självmordsspåret \mydots \indexref{sjaelvmordsspaaret}\newline
Stråtrövare \mydots \indexref{straatroovare}\newline
\end{multicols}
~\newline
{\textbf{\large Bubologi}}
\begin{multicols}{2}
Berguv \mydots \indexref{berguv}\newline
Klo \mydots \indexref{klo}\newline
Krypa upp i soffan som en uv \mydots \indexref{krypa upp i soffan som en uv}\newline
Uppstoppad uv \mydots \indexref{uppstoppad uv}\newline
Uv \mydots \indexref{uv}\newline
Uv-ljus \mydots \indexref{uv-ljus}\newline
Uv-rugby \mydots \indexref{uv-rugby}\newline
Uvgodis \mydots \indexref{uvgodis}\newline
Uvsvane \mydots \indexref{uvsvane}\newline
Uvtårar \mydots \indexref{uvtaarar}\newline
\end{multicols}
~\newline
{\textbf{\large Danmark}}
\begin{multicols}{2}
Ankfot \mydots \indexref{ankfot}\newline
Boetius de Dacia \mydots \indexref{boetius de dacia}\newline
Christianiacykel \mydots \indexref{christianiacykel}\newline
Danmark \mydots \indexref{danmark}\newline
Dannebrogen \mydots \indexref{dannebrogen}\newline
Dansk advent \mydots \indexref{dansk advent}\newline
Dansk forskning \mydots \indexref{dansk forskning}\newline
Dansk jul \mydots \indexref{dansk jul}\newline
Dansk kanot \mydots \indexref{dansk kanot}\newline
Dansk kortlek \mydots \indexref{dansk kortlek}\newline
Dansk kostcirkel \mydots \indexref{dansk kostcirkel}\newline
Dansk lättöl \mydots \indexref{dansk laettool}\newline
Dansk midsommar \mydots \indexref{dansk midsommar}\newline
Dansk onsdag \mydots \indexref{dansk onsdag}\newline
Dansk påsk \mydots \indexref{dansk paask}\newline
Dansk sax \mydots \indexref{dansk sax}\newline
Dansk skalle \mydots \indexref{dansk skalle}\newline
Dansk skrock \mydots \indexref{dansk skrock}\newline
Dansk tubkikare \mydots \indexref{dansk tubkikare}\newline
Danska hedersbetygelser \mydots \indexref{danska hedersbetygelser}\newline
Danskt penicillin \mydots \indexref{danskt penicillin}\newline
Den nordiska smuggeltriangeln \mydots \indexref{den nordiska smuggeltriangeln}\newline
Der er et yndigt land \mydots \indexref{der er et yndigt land}\newline
Fläsksvålar \mydots \indexref{flaesksvaalar}\newline
Folketinget \mydots \indexref{folketinget}\newline
Fyn \mydots \indexref{fyn}\newline
Gamle Ole \mydots \indexref{gamle ole}\newline
Gruk \mydots \indexref{gruk}\newline
Lars Krogh \mydots \indexref{lars krogh}\newline
Lego \mydots \indexref{lego}\newline
Mads Mikkelsen \mydots \indexref{mads mikkelsen}\newline
Nordisk kombination \mydots \indexref{nordisk kombination}\newline
Propellerkeps \mydots \indexref{propellerkeps}\newline
Rød pølse \mydots \indexref{roood pooolse}\newline
Rövgitarr \mydots \indexref{roovgitarr}\newline
Sengekantsfilm \mydots \indexref{sengekantsfilm}\newline
Skagen \mydots \indexref{skagen}\newline
Tivoli \mydots \indexref{tivoli}\newline
Tuborg \mydots \indexref{tuborg}\newline
\end{multicols}
~\newline
{\textbf{\large Dialekter}}
\begin{multicols}{2}
Dialektik \mydots \indexref{dialektik}\newline
He \mydots \indexref{he}\newline
Hemmansvärde \mydots \indexref{hemmansvaerde}\newline
Háfrónska \mydots \indexref{háfrónska}\newline
Klo \mydots \indexref{klo}\newline
\end{multicols}
~\newline
{\textbf{\large Djurriket}}
\begin{multicols}{2}
Albert II \mydots \indexref{albert ii}\newline
Alkisschäfer \mydots \indexref{alkisschaefer}\newline
Antilop \mydots \indexref{antilop}\newline
Apor vi minns \mydots \indexref{apor vi minns}\newline
Barn \mydots \indexref{barn}\newline
Belgisk jättekanin \mydots \indexref{belgisk jaettekanin}\newline
Belka \mydots \indexref{belka}\newline
Berguv \mydots \indexref{berguv}\newline
Björn (djur) \mydots \indexref{bjoorn (djur)}\newline
Blåval \mydots \indexref{blaaval}\newline
Brandklipparen \mydots \indexref{brandklipparen}\newline
Brismonstret \mydots \indexref{brismonstret}\newline
Brugd \mydots \indexref{brugd}\newline
Bruksgök \mydots \indexref{bruksgook}\newline
Bruvd \mydots \indexref{bruvd}\newline
Charlotte \mydots \indexref{charlotte}\newline
Crustpippi \mydots \indexref{crustpippi}\newline
Delfin \mydots \indexref{delfin}\newline
Delfinapa \mydots \indexref{delfinapa}\newline
Djur \mydots \indexref{djur}\newline
Doktorand \mydots \indexref{doktorand}\newline
Dvärgpungsovare \mydots \indexref{dvaergpungsovare}\newline
Enkelbeckasin \mydots \indexref{enkelbeckasin}\newline
Eskimåspov \mydots \indexref{eskimaaspov}\newline
Fåglar som går \mydots \indexref{faaglar som gaar}\newline
Filipinsk apörn \mydots \indexref{filipinsk apoorn}\newline
Flodkanin \mydots \indexref{flodkanin}\newline
Flundra \mydots \indexref{flundra}\newline
Gnagare \mydots \indexref{gnagare}\newline
Häst \mydots \indexref{haest}\newline
Ham \mydots \indexref{ham}\newline
Havsmunk \mydots \indexref{havsmunk}\newline
Holmsunds tropikhus \mydots \indexref{holmsunds tropikhus}\newline
Hönsgård \mydots \indexref{hoonsgaard}\newline
Hyena \mydots \indexref{hyena}\newline
Ivriga små bävrar \mydots \indexref{ivriga smaa baevrar}\newline
Jättemyrslok \mydots \indexref{jaettemyrslok}\newline
Katt \mydots \indexref{katt}\newline
Kiwi \mydots \indexref{kiwi}\newline
Kloakdjur \mydots \indexref{kloakdjur}\newline
Körp \mydots \indexref{koorp}\newline
Kopparorm \mydots \indexref{kopparorm}\newline
Kräftor \mydots \indexref{kraeftor}\newline
Krypa ihop i soffan som en katt \mydots \indexref{krypa ihop i soffan som en katt}\newline
Krypa upp i soffan som en brugd \mydots \indexref{krypa upp i soffan som en brugd}\newline
Krypa upp i soffan som en uv \mydots \indexref{krypa upp i soffan som en uv}\newline
Lothar \mydots \indexref{lothar}\newline
Manet \mydots \indexref{manet}\newline
Muno \mydots \indexref{muno}\newline
Neontetra \mydots \indexref{neontetra}\newline
Noshörningen Nelson \mydots \indexref{noshoorningen nelson}\newline
PATSY award \mydots \indexref{patsy award}\newline
Pelikan \mydots \indexref{pelikan}\newline
Råååål \mydots \indexref{raaaaaaaal}\newline
Sälar \mydots \indexref{saelar}\newline
Sidensvans \mydots \indexref{sidensvans}\newline
Simhud \mydots \indexref{simhud}\newline
Skedstork \mydots \indexref{skedstork}\newline
Snefotad ultrapelikan \mydots \indexref{snefotad ultrapelikan}\newline
Steglits \mydots \indexref{steglits}\newline
Stöcksjö sunny resorts \mydots \indexref{stoocksjoo sunny resorts}\newline
Storspov \mydots \indexref{storspov}\newline
Storspren \mydots \indexref{storspren}\newline
Streptokocker \mydots \indexref{streptokocker}\newline
Surfin' bird \mydots \indexref{surfin bird}\newline
Svan \mydots \indexref{svan}\newline
Talgoxe \mydots \indexref{talgoxe}\newline
Tsygan \mydots \indexref{tsygan}\newline
Utrotningshotade djur \mydots \indexref{utrotningshotade djur}\newline
Uv \mydots \indexref{uv}\newline
\end{multicols}
~\newline
{\textbf{\large Ekonomi}}
\begin{multicols}{2}
Antikvärde \mydots \indexref{antikvaerde}\newline
Botte \mydots \indexref{botte}\newline
Corporate social responsibility \mydots \indexref{corporate social responsibility}\newline
Egendom \mydots \indexref{egendom}\newline
Ekfors Kraft \mydots \indexref{ekfors kraft}\newline
Fejkspons \mydots \indexref{fejkspons}\newline
Grekiska statsobligationer \mydots \indexref{grekiska statsobligationer}\newline
Hästhandlarplånbok \mydots \indexref{haesthandlarplaanbok}\newline
Laissez-faire \mydots \indexref{laissez-faire}\newline
Löneförmån \mydots \indexref{loonefoormaan}\newline
Metalmynt \mydots \indexref{metalmynt}\newline
Nigeria \mydots \indexref{nigeria}\newline
Pizzarulle \mydots \indexref{pizzarulle}\newline
Rikskuponger \mydots \indexref{rikskuponger}\newline
Skrapade skraplotter med vinst \mydots \indexref{skrapade skraplotter med vinst}\newline
Snåltarmen \mydots \indexref{snaaltarmen}\newline
Tia \mydots \indexref{tia}\newline
Valuta \mydots \indexref{valuta}\newline
\end{multicols}
~\newline
{\textbf{\large Fantastiska levnadsöden}}
\begin{multicols}{2}
50 million piece of shit \mydots \indexref{50 million piece of shit}\newline
Åke Cato \mydots \indexref{aake cato}\newline
Åke Ohlmarks \mydots \indexref{aake ohlmarks}\newline
Adde Malmberg \mydots \indexref{adde malmberg}\newline
Aleksandr Karelin \mydots \indexref{aleksandr karelin}\newline
Anton Abele \mydots \indexref{anton abele}\newline
Aron Jonason \mydots \indexref{aron jonason}\newline
Artur Hazelius \mydots \indexref{artur hazelius}\newline
Ayn Rand \mydots \indexref{ayn rand}\newline
Bartolomaios \mydots \indexref{bartolomaios}\newline
Belka \mydots \indexref{belka}\newline
Benny Bus \mydots \indexref{benny bus}\newline
Brandklipparen \mydots \indexref{brandklipparen}\newline
Brian Epstein \mydots \indexref{brian epstein}\newline
Brokiga Blad \mydots \indexref{brokiga blad}\newline
Christer Sandelin \mydots \indexref{christer sandelin}\newline
Den nya sångaren \mydots \indexref{den nya saangaren}\newline
Edmund Hillary \mydots \indexref{edmund hillary}\newline
Erik Hamrén \mydots \indexref{erik hamrén}\newline
Erik Homburger Erikson \mydots \indexref{erik homburger erikson}\newline
Ernst Haeckel \mydots \indexref{ernst haeckel}\newline
Eva Ekeblad \mydots \indexref{eva ekeblad}\newline
Fakta \mydots \indexref{fakta}\newline
Feliks Dzerzjinskij \mydots \indexref{feliks dzerzjinskij}\newline
Folke Pudas \mydots \indexref{folke pudas}\newline
Fredrik Reinfeldt \mydots \indexref{fredrik reinfeldt}\newline
Friedrich Hegel \mydots \indexref{friedrich hegel}\newline
Geezer Butler \mydots \indexref{geezer butler}\newline
George Everest \mydots \indexref{george everest}\newline
Georgij Zjukov \mydots \indexref{georgij zjukov}\newline
Håkan Juholt \mydots \indexref{haakan juholt}\newline
Haile Selassie \mydots \indexref{haile selassie}\newline
Ham \mydots \indexref{ham}\newline
Hertigen av Rothesay \mydots \indexref{hertigen av rothesay}\newline
Homi K. Bhabhas son \mydots \indexref{homi k. bhabhas son}\newline
Hugo Alfvén \mydots \indexref{hugo alfvén}\newline
Incitatus \mydots \indexref{incitatus}\newline
Ingvar Carlsson \mydots \indexref{ingvar carlsson}\newline
Ingvar Kamprad \mydots \indexref{ingvar kamprad}\newline
Isaac Johannes Lamotius \mydots \indexref{isaac johannes lamotius}\newline
Jackie Howe \mydots \indexref{jackie howe}\newline
Jackson Pollock \mydots \indexref{jackson pollock}\newline
Jan Björklund \mydots \indexref{jan bjoorklund}\newline
Jan Wilsgaard \mydots \indexref{jan wilsgaard}\newline
Jerry Williams \mydots \indexref{jerry williams}\newline
Johan Skytte \mydots \indexref{johan skytte}\newline
Johann Neumann \mydots \indexref{johann neumann}\newline
Johannes Brost \mydots \indexref{johannes brost}\newline
John Kellogg \mydots \indexref{john kellogg}\newline
Johnny Takter \mydots \indexref{johnny takter}\newline
Jonathan Guy \mydots \indexref{jonathan guy}\newline
Joseph Lucas \mydots \indexref{joseph lucas}\newline
Kelsey Grammer \mydots \indexref{kelsey grammer}\newline
Kerry King \mydots \indexref{kerry king}\newline
Kladdi Mittänän \mydots \indexref{kladdi mittaenaen}\newline
Klasse Möllberg \mydots \indexref{klasse moollberg}\newline
Läppar som prinskorv \mydots \indexref{laeppar som prinskorv}\newline
Lars Krogh \mydots \indexref{lars krogh}\newline
Lars Levi Laestadius \mydots \indexref{lars levi laestadius}\newline
Lars-Åke Lagrell \mydots \indexref{lars-aake lagrell}\newline
Lenin-Churchhill aka Mysgubbe \mydots \indexref{lenin-churchhill aka mysgubbe}\newline
Lennart Holmlund \mydots \indexref{lennart holmlund}\newline
Lennart Holmlund \mydots \indexref{lennart holmlund}\newline
Linda Norrman Skugge \mydots \indexref{linda norrman skugge}\newline
Lothar \mydots \indexref{lothar}\newline
Lundgren \mydots \indexref{lundgren}\newline
Luís Figo \mydots \indexref{luís figo}\newline
Mads Mikkelsen \mydots \indexref{mads mikkelsen}\newline
Manuel \mydots \indexref{manuel}\newline
Margaret Thatcher \mydots \indexref{margaret thatcher}\newline
Martine \mydots \indexref{martine}\newline
Maski Hallonen \mydots \indexref{maski hallonen}\newline
Mats Lundgren \mydots \indexref{mats lundgren}\newline
Maud Olofsson \mydots \indexref{maud olofsson}\newline
Max Weber \mydots \indexref{max weber}\newline
Micke Alonzo \mydots \indexref{micke alonzo}\newline
Mikis Theodorakis \mydots \indexref{mikis theodorakis}\newline
Muno \mydots \indexref{muno}\newline
Nikolaj Valujev \mydots \indexref{nikolaj valujev}\newline
Tenzing Norgay \mydots \indexref{tenzing norgay}\newline
Noshörningen Nelson \mydots \indexref{noshoorningen nelson}\newline
Olof Palme \mydots \indexref{olof palme}\newline
Oscar Dronjak \mydots \indexref{oscar dronjak}\newline
Paj \mydots \indexref{paj}\newline
Paul du Chaillu \mydots \indexref{paul du chaillu}\newline
Pelle Svensson \mydots \indexref{pelle svensson}\newline
Petrus de Dacia \mydots \indexref{petrus de dacia}\newline
Philibert Humla \mydots \indexref{philibert humla}\newline
Prins Charles \mydots \indexref{prins charles}\newline
Queequeg \mydots \indexref{queequeg}\newline
Randall Finefield \mydots \indexref{randall finefield}\newline
Ranta Runtiringen \mydots \indexref{ranta runtiringen}\newline
Ray Jones IV \mydots \indexref{ray jones iv}\newline
Richard Dybeck \mydots \indexref{richard dybeck}\newline
Shizo Kanaguri \mydots \indexref{shizo kanaguri}\newline
Sigvard Thurneman \mydots \indexref{sigvard thurneman}\newline
Sober-Jimmy \mydots \indexref{sober-jimmy}\newline
Sonny Listons son \mydots \indexref{sonny listons son}\newline
Stor-Anders \mydots \indexref{stor-anders}\newline
Streiff \mydots \indexref{streiff}\newline
The fat Spanish waiter \mydots \indexref{the fat spanish waiter}\newline
The Fog \mydots \indexref{the fog}\newline
Thomas Wassberg \mydots \indexref{thomas wassberg}\newline
Torgny Mogren \mydots \indexref{torgny mogren}\newline
Torsten Bengtsson \mydots \indexref{torsten bengtsson}\newline
Tsygan \mydots \indexref{tsygan}\newline
Valentina Vladimirovna Teresjkova \mydots \indexref{valentina vladimirovna teresjkova}\newline
Vild-Hasse \mydots \indexref{vild-hasse}\newline
Vladimir Krutov \mydots \indexref{vladimir krutov}\newline
Weiron Holmberg \mydots \indexref{weiron holmberg}\newline
William Banting \mydots \indexref{william banting}\newline
Émile Durkheim \mydots \indexref{émile durkheim}\newline
\end{multicols}
~\newline
{\textbf{\large Fascism}}
\begin{multicols}{2}
Äganderätt \mydots \indexref{aeganderaett}\newline
Antiutilitarism \mydots \indexref{antiutilitarism}\newline
Ayn Rand \mydots \indexref{ayn rand}\newline
Balticgruppen \mydots \indexref{balticgruppen}\newline
Blåval \mydots \indexref{blaaval}\newline
Centerpartiet \mydots \indexref{centerpartiet}\newline
Chef \mydots \indexref{chef}\newline
Civilpolis \mydots \indexref{civilpolis}\newline
Corporate social responsibility \mydots \indexref{corporate social responsibility}\newline
Egendom \mydots \indexref{egendom}\newline
Entreprenör \mydots \indexref{entreprenoor}\newline
Folkpartiet \mydots \indexref{folkpartiet}\newline
Fredrik Reinfeldt \mydots \indexref{fredrik reinfeldt}\newline
Fri rörlighet \mydots \indexref{fri roorlighet}\newline
Hårdrockism \mydots \indexref{haardrockism}\newline
Hästkista \mydots \indexref{haestkista}\newline
Hegemoni \mydots \indexref{hegemoni}\newline
Incitament \mydots \indexref{incitament}\newline
Individ \mydots \indexref{individ}\newline
Ingvar Kamprad \mydots \indexref{ingvar kamprad}\newline
Innebandy \mydots \indexref{innebandy}\newline
Institut och tankesmedjor \mydots \indexref{institut och tankesmedjor}\newline
Jan Björkblund \mydots \indexref{jan bjoorkblund}\newline
Kälkborgare \mydots \indexref{kaelkborgare}\newline
Kalaskula \mydots \indexref{kalaskula}\newline
Kristdemokraterna \mydots \indexref{kristdemokraterna}\newline
Lätt misshandel \mydots \indexref{laett misshandel}\newline
Laissez-faire \mydots \indexref{laissez-faire}\newline
Lennart Holmlund \mydots \indexref{lennart holmlund}\newline
Lennart Holmlund \mydots \indexref{lennart holmlund}\newline
Margaret Thatcher \mydots \indexref{margaret thatcher}\newline
Maud Olofsson \mydots \indexref{maud olofsson}\newline
Moderat \mydots \indexref{moderat}\newline
Musselini \mydots \indexref{musselini}\newline
Nyliberalism \mydots \indexref{nyliberalism}\newline
Privatisering \mydots \indexref{privatisering}\newline
Rasism \mydots \indexref{rasism}\newline
Realister \mydots \indexref{realister}\newline
Saltsjöbadsavtalet \mydots \indexref{saltsjoobadsavtalet}\newline
Sjundedagsadventistisk skola \mydots \indexref{sjundedagsadventistisk skola}\newline
Smygfascist \mydots \indexref{smygfascist}\newline
Snutröv \mydots \indexref{snutroov}\newline
Sommarplågsmusiker \mydots \indexref{sommarplaagsmusiker}\newline
Sverigedemokraterna \mydots \indexref{sverigedemokraterna}\newline
Tokliberal \mydots \indexref{tokliberal}\newline
Torsten Bengtsson \mydots \indexref{torsten bengtsson}\newline
Valfrihet \mydots \indexref{valfrihet}\newline
Viktiga papper \mydots \indexref{viktiga papper}\newline
Vinkännare \mydots \indexref{vinkaennare}\newline
\end{multicols}
~\newline
{\textbf{\large Fatastiska levnadsöden}}
\begin{multicols}{2}
Boetius de Dacia \mydots \indexref{boetius de dacia}\newline
\end{multicols}
~\newline
{\textbf{\large Film}}
\begin{multicols}{2}
Colin Nutley \mydots \indexref{colin nutley}\newline
Hardware \mydots \indexref{hardware}\newline
Homi K. Bhabhas son \mydots \indexref{homi k. bhabhas son}\newline
Jacques Touillaud \mydots \indexref{jacques touillaud}\newline
Johann Neumann \mydots \indexref{johann neumann}\newline
Johannes Brost \mydots \indexref{johannes brost}\newline
Mads Mikkelsen \mydots \indexref{mads mikkelsen}\newline
Nicholas Cage-film \mydots \indexref{nicholas cage-film}\newline
Pörr \mydots \indexref{poorr}\newline
Ray Jones IV \mydots \indexref{ray jones iv}\newline
Sällskapsresan \mydots \indexref{saellskapsresan}\newline
Schwarzwald Larsson \mydots \indexref{schwarzwald larsson}\newline
Sengekantsfilm \mydots \indexref{sengekantsfilm}\newline
\end{multicols}
~\newline
{\textbf{\large Filosofi}}
\begin{multicols}{2}
Antiutilitarism \mydots \indexref{antiutilitarism}\newline
Boetius de Dacia \mydots \indexref{boetius de dacia}\newline
De gamla grekerna \mydots \indexref{de gamla grekerna}\newline
Den tyska mustigheten \mydots \indexref{den tyska mustigheten}\newline
Dialektik \mydots \indexref{dialektik}\newline
Epikurism \mydots \indexref{epikurism}\newline
Färskost \mydots \indexref{faerskost}\newline
Foucaultfingret \mydots \indexref{foucaultfingret}\newline
Framtiden \mydots \indexref{framtiden}\newline
Friedrich Hegel \mydots \indexref{friedrich hegel}\newline
Gurkmajonnäs \mydots \indexref{gurkmajonnaes}\newline
Hegemoni \mydots \indexref{hegemoni}\newline
Kikare \mydots \indexref{kikare}\newline
RAC \mydots \indexref{rac}\newline
Räksallad \mydots \indexref{raeksallad}\newline
Raison d'être \mydots \indexref{raison dêtre}\newline
Romantik \mydots \indexref{romantik}\newline
Tjänstemannateoretisk \mydots \indexref{tjaenstemannateoretisk}\newline
\end{multicols}
~\newline
{\textbf{\large Folk och personlighetstyper}}
\begin{multicols}{2}
Årets göteborgare \mydots \indexref{aarets gooteborgare}\newline
Barn \mydots \indexref{barn}\newline
Boris Jeltsin \mydots \indexref{boris jeltsin}\newline
Brinner för att sälja \mydots \indexref{brinner foor att saelja}\newline
Carlshöjdare \mydots \indexref{carlshoojdare}\newline
Crustare \mydots \indexref{crustare}\newline
Deadhead \mydots \indexref{deadhead}\newline
Den ambitiösa studenten \mydots \indexref{den ambitioosa studenten}\newline
Den oambitiösa studenten \mydots \indexref{den oambitioosa studenten}\newline
Duvgubbar \mydots \indexref{duvgubbar}\newline
Entreprenör \mydots \indexref{entreprenoor}\newline
Folk födda före 1970 \mydots \indexref{folk foodda foore 1970}\newline
Fulsnygg \mydots \indexref{fulsnygg}\newline
Gammpojkar \mydots \indexref{gammpojkar}\newline
Gammstinta \mydots \indexref{gammstinta}\newline
Genuint snåla människor \mydots \indexref{genuint snaala maenniskor}\newline
Gitarrkille \mydots \indexref{gitarrkille}\newline
Glesbygdsball \mydots \indexref{glesbygdsball}\newline
Globetrotter \mydots \indexref{globetrotter}\newline
Glop \mydots \indexref{glop}\newline
Gubbsäker \mydots \indexref{gubbsaeker}\newline
Hårdrockare med gomspalt \mydots \indexref{haardrockare med gomspalt}\newline
Hårdrockare och vitaminer \mydots \indexref{haardrockare och vitaminer}\newline
Hippie \mydots \indexref{hippie}\newline
Kommunanställd småpåve \mydots \indexref{kommunanstaelld smaapaave}\newline
Kukenkillar \mydots \indexref{kukenkillar}\newline
Kulaker \mydots \indexref{kulaker}\newline
Latinsk facebook-rocker \mydots \indexref{latinsk facebook-rocker}\newline
Lurkuk \mydots \indexref{lurkuk}\newline
Medelklassvänner \mydots \indexref{medelklassvaenner}\newline
Mjukis \mydots \indexref{mjukis}\newline
Mungo Jerryhatare \mydots \indexref{mungo jerryhatare}\newline
Nollpresterare \mydots \indexref{nollpresterare}\newline
Perversa elektriker \mydots \indexref{perversa elektriker}\newline
Praktarsle (negativ) \mydots \indexref{praktarsle (negativ)}\newline
Psykedelisk morförälder \mydots \indexref{psykedelisk morfooraelder}\newline
Realister \mydots \indexref{realister}\newline
Skottar \mydots \indexref{skottar}\newline
Skotte \mydots \indexref{skotte}\newline
Småstadsalternativ \mydots \indexref{smaastadsalternativ}\newline
Smuldegspappor \mydots \indexref{smuldegspappor}\newline
Smygsexist \mydots \indexref{smygsexist}\newline
Snälla killar som aldrig får ligga \mydots \indexref{snaella killar som aldrig faar ligga}\newline
Sopletare \mydots \indexref{sopletare}\newline
Stockholmare \mydots \indexref{stockholmare}\newline
Storbossnörd \mydots \indexref{storbossnoord}\newline
Storswänsk \mydots \indexref{storswaensk}\newline
Strulputte \mydots \indexref{strulputte}\newline
Taxichaufför \mydots \indexref{taxichauffoor}\newline
Tomten \mydots \indexref{tomten}\newline
Tysk tårttant \mydots \indexref{tysk taarttant}\newline
Vanliga pantade knegare \mydots \indexref{vanliga pantade knegare}\newline
Vingmutter \mydots \indexref{vingmutter}\newline
\end{multicols}
~\newline
{\textbf{\large Föreningsliv och folkrörelser}}
\begin{multicols}{2}
Alice Tegnér \mydots \indexref{alice tegnér}\newline
Berghem HC \mydots \indexref{berghem hc}\newline
FFSSB \mydots \indexref{ffssb}\newline
Förebyggande skallgångskedja \mydots \indexref{foorebyggande skallgaangskedja}\newline
Inititativ Anusmark \mydots \indexref{inititativ anusmark}\newline
Lundin Petroleum \mydots \indexref{lundin petroleum}\newline
Nissepedia \mydots \indexref{nissepedia}\newline
Pangsionärerna \mydots \indexref{pangsionaererna}\newline
Svenska jägareförbundet \mydots \indexref{svenska jaegarefoorbundet}\newline
Svenska Kennelklubben \mydots \indexref{svenska kennelklubben}\newline
Svenska Kennetklubben \mydots \indexref{svenska kennetklubben}\newline
Svenskt näringsliv \mydots \indexref{svenskt naeringsliv}\newline
\end{multicols}
~\newline
{\textbf{\large Grekland}}
\begin{multicols}{2}
De gamla grekerna \mydots \indexref{de gamla grekerna}\newline
\end{multicols}
~\newline
{\textbf{\large Handel}}
\begin{multicols}{2}
Abu Garcia \mydots \indexref{abu garcia}\newline
Ansiktsmålning \mydots \indexref{ansiktsmaalning}\newline
Axe \mydots \indexref{axe}\newline
Blåvitt \mydots \indexref{blaavitt}\newline
Britts mode \mydots \indexref{britts mode}\newline
Den nordiska smuggeltriangeln \mydots \indexref{den nordiska smuggeltriangeln}\newline
Femma \mydots \indexref{femma}\newline
Fiskeredskapsaffär \mydots \indexref{fiskeredskapsaffaer}\newline
Flaka \mydots \indexref{flaka}\newline
Gylfa \mydots \indexref{gylfa}\newline
Hänga på låset \mydots \indexref{haenga paa laaset}\newline
Indiska \mydots \indexref{indiska}\newline
Kombinationsaffär \mydots \indexref{kombinationsaffaer}\newline
Konsumbutik \mydots \indexref{konsumbutik}\newline
Lena \mydots \indexref{lena}\newline
Mackshopping \mydots \indexref{mackshopping}\newline
Merchband \mydots \indexref{merchband}\newline
Orrkammens isolering och gokart \mydots \indexref{orrkammens isolering och gokart}\newline
Rött \mydots \indexref{roott}\newline
Stenlapp \mydots \indexref{stenlapp}\newline
Stentrollsaffär \mydots \indexref{stentrollsaffaer}\newline
\end{multicols}
~\newline
{\textbf{\large Historia}}
\begin{multicols}{2}
Abu Garcia \mydots \indexref{abu garcia}\newline
Ales Stenar \mydots \indexref{ales stenar}\newline
Artur Hazelius \mydots \indexref{artur hazelius}\newline
Dackefejden \mydots \indexref{dackefejden}\newline
Det stora fosterländska kriget \mydots \indexref{det stora fosterlaendska kriget}\newline
Grand Funk Railroad \mydots \indexref{grand funk railroad}\newline
Hakkors vi minns \mydots \indexref{hakkors vi minns}\newline
Historiska händelser i badrum \mydots \indexref{historiska haendelser i badrum}\newline
I'm so tired I could sleep on a clothesline \mydots \indexref{im so tired i could sleep on a clothesline}\newline
Korv med bröd \mydots \indexref{korv med brood}\newline
Maginotlinjen \mydots \indexref{maginotlinjen}\newline
Medeltiden \mydots \indexref{medeltiden}\newline
Naturhistoriska museet \mydots \indexref{naturhistoriska museet}\newline
Ornässtugans dass \mydots \indexref{ornaesstugans dass}\newline
Påsk \mydots \indexref{paask}\newline
Paxa \mydots \indexref{paxa}\newline
Riksregalier \mydots \indexref{riksregalier}\newline
Skräckväldet \mydots \indexref{skraeckvaeldet}\newline
Streiff \mydots \indexref{streiff}\newline
\end{multicols}
~\newline
{\textbf{\large Hårdrock}}
\begin{multicols}{2}
AC/DC-gitarr \mydots \indexref{acdc-gitarr}\newline
Angel of death \mydots \indexref{angel of death}\newline
Backpatch \mydots \indexref{backpatch}\newline
Bockskäggsmetal \mydots \indexref{bockskaeggsmetal}\newline
Den nya sångaren \mydots \indexref{den nya saangaren}\newline
False metal \mydots \indexref{false metal}\newline
Front row banger \mydots \indexref{front row banger}\newline
Geezer Butler \mydots \indexref{geezer butler}\newline
Hårdrock \mydots \indexref{haardrock}\newline
Hårdrockare med gomspalt \mydots \indexref{haardrockare med gomspalt}\newline
Hårdrockare och vitaminer \mydots \indexref{haardrockare och vitaminer}\newline
Hårdrockism \mydots \indexref{haardrockism}\newline
Kerry King \mydots \indexref{kerry king}\newline
Lemmy-bas \mydots \indexref{lemmy-bas}\newline
Manowar \mydots \indexref{manowar}\newline
Math metal \mydots \indexref{math metal}\newline
Norge \mydots \indexref{norge}\newline
Sign of the hammer \mydots \indexref{sign of the hammer}\newline
Slayerklass \mydots \indexref{slayerklass}\newline
Thrashzan \mydots \indexref{thrashzan}\newline
\end{multicols}
~\newline
{\textbf{\large Högtider}}
\begin{multicols}{2}
Ålands demitaliseringsdag \mydots \indexref{aalands demitaliseringsdag}\newline
Dansk advent \mydots \indexref{dansk advent}\newline
Dansk jul \mydots \indexref{dansk jul}\newline
Dansk midsommar \mydots \indexref{dansk midsommar}\newline
Dansk onsdag \mydots \indexref{dansk onsdag}\newline
Dansk påsk \mydots \indexref{dansk paask}\newline
Djurens nobelpris \mydots \indexref{djurens nobelpris}\newline
Fredagsmys \mydots \indexref{fredagsmys}\newline
Gå och köpa tidningen \mydots \indexref{gaa och koopa tidningen}\newline
Långfredagen \mydots \indexref{laangfredagen}\newline
Lucia \mydots \indexref{lucia}\newline
Luciavaka \mydots \indexref{luciavaka}\newline
Påsk \mydots \indexref{paask}\newline
Polska helgdagar \mydots \indexref{polska helgdagar}\newline
Roliga timmen \mydots \indexref{roliga timmen}\newline
Trasmattans dag \mydots \indexref{trasmattans dag}\newline
\end{multicols}
~\newline
{\textbf{\large I språkets periferi}}
\begin{multicols}{2}
Åka vikingaskepp \mydots \indexref{aaka vikingaskepp}\newline
Abdera \mydots \indexref{abdera}\newline
Ädelost \mydots \indexref{aedelost}\newline
Ägg \mydots \indexref{aegg}\newline
Ägmästare \mydots \indexref{aegmaestare}\newline
Alternativa namn på bakverk \mydots \indexref{alternativa namn paa bakverk}\newline
Ana uvar i mossen \mydots \indexref{ana uvar i mossen}\newline
Botte \mydots \indexref{botte}\newline
Bröstarkt \mydots \indexref{broostarkt}\newline
Brumma \mydots \indexref{brumma}\newline
Bull \mydots \indexref{bull}\newline
CQD \mydots \indexref{cqd}\newline
Dank \mydots \indexref{dank}\newline
Epikurism \mydots \indexref{epikurism}\newline
Ernst Haeckel \mydots \indexref{ernst haeckel}\newline
Faktoid \mydots \indexref{faktoid}\newline
Finljuga \mydots \indexref{finljuga}\newline
Försåvitt \mydots \indexref{foorsaavitt}\newline
Franska svordomar \mydots \indexref{franska svordomar}\newline
Ftw \mydots \indexref{ftw}\newline
Fyrtiotusen miljarder \mydots \indexref{fyrtiotusen miljarder}\newline
Habbadixen! \mydots \indexref{habbadixen!}\newline
Handvass \mydots \indexref{handvass}\newline
He \mydots \indexref{he}\newline
Hemliga koder \mydots \indexref{hemliga koder}\newline
Háfrónska \mydots \indexref{háfrónska}\newline
I'm so tired I could sleep on a clothesline \mydots \indexref{im so tired i could sleep on a clothesline}\newline
Inga lejon \mydots \indexref{inga lejon}\newline
Italienska svordomar \mydots \indexref{italienska svordomar}\newline
Je ne sais quoi \mydots \indexref{je ne sais quoi}\newline
Kineseri \mydots \indexref{kineseri}\newline
Klo \mydots \indexref{klo}\newline
Konjektural \mydots \indexref{konjektural}\newline
Közösülés \mydots \indexref{koozoosülés}\newline
Kulturstökigt \mydots \indexref{kulturstookigt}\newline
Lillnöjd \mydots \indexref{lillnoojd}\newline
Mäklarsvenska \mydots \indexref{maeklarsvenska}\newline
Mango safe \mydots \indexref{mango safe}\newline
Nu går slakten på Bomans vind! \mydots \indexref{nu gaar slakten paa bomans vind!}\newline
Ohemul \mydots \indexref{ohemul}\newline
Paddan i pannrummet \mydots \indexref{paddan i pannrummet}\newline
Pizzarulle \mydots \indexref{pizzarulle}\newline
Pjäxfett \mydots \indexref{pjaexfett}\newline
Pluta \mydots \indexref{pluta}\newline
Problematiskt \mydots \indexref{problematiskt}\newline
Prunka \mydots \indexref{prunka}\newline
Råååål \mydots \indexref{raaaaaaaal}\newline
Rugga \mydots \indexref{rugga}\newline
Semikolon \mydots \indexref{semikolon}\newline
Sinkadus \mydots \indexref{sinkadus}\newline
Skita i den korvbröda kökssoffan \mydots \indexref{skita i den korvbrooda kookssoffan}\newline
Skita i det blå skåpet \mydots \indexref{skita i det blaa skaapet}\newline
Skrymmande \mydots \indexref{skrymmande}\newline
Slan \mydots \indexref{slan}\newline
Sportmössa \mydots \indexref{sportmoossa}\newline
Stenlapp \mydots \indexref{stenlapp}\newline
Sugmästare \mydots \indexref{sugmaestare}\newline
Svinpäls \mydots \indexref{svinpaels}\newline
Tjena Roger! \mydots \indexref{tjena roger!}\newline
Tomte \mydots \indexref{tomte}\newline
Trepipsproblem \mydots \indexref{trepipsproblem}\newline
Trevlig \mydots \indexref{trevlig}\newline
Trollprutt \mydots \indexref{trollprutt}\newline
Tuborg \mydots \indexref{tuborg}\newline
Wham, bam, thank you ma'am \mydots \indexref{wham, bam, thank you maam}\newline
William Banting \mydots \indexref{william banting}\newline
\end{multicols}
~\newline
{\textbf{\large Informationsteknologi}}
\begin{multicols}{2}
Bailando \mydots \indexref{bailando}\newline
Bildekal \mydots \indexref{bildekal}\newline
Brevlåda \mydots \indexref{brevlaada}\newline
Dagens Nyheter \mydots \indexref{dagens nyheter}\newline
Den lilla boken \mydots \indexref{den lilla boken}\newline
Diskett \mydots \indexref{diskett}\newline
Facebook \mydots \indexref{facebook}\newline
Fagersta-Posten \mydots \indexref{fagersta-posten}\newline
Fax \mydots \indexref{fax}\newline
Hålkort \mydots \indexref{haalkort}\newline
Internet \mydots \indexref{internet}\newline
Kaknästornet \mydots \indexref{kaknaestornet}\newline
Matematikmaskinnämnden \mydots \indexref{matematikmaskinnaemnden}\newline
Modem \mydots \indexref{modem}\newline
Nissepedia \mydots \indexref{nissepedia}\newline
Orolighetskeps \mydots \indexref{orolighetskeps}\newline
Pneumatiska rör \mydots \indexref{pneumatiska roor}\newline
Postångare \mydots \indexref{postaangare}\newline
Posten \mydots \indexref{posten}\newline
Postlåda \mydots \indexref{postlaada}\newline
Rikssamtal \mydots \indexref{rikssamtal}\newline
Röksignaler \mydots \indexref{rooksignaler}\newline
Sanningssägande bloggar \mydots \indexref{sanningssaegande bloggar}\newline
Strömavbrott \mydots \indexref{stroomavbrott}\newline
Svarta tavlan \mydots \indexref{svarta tavlan}\newline
Televerket \mydots \indexref{televerket}\newline
Television \mydots \indexref{television}\newline
Tjock-TV \mydots \indexref{tjock-tv}\newline
Videotex \mydots \indexref{videotex}\newline
World Wide Web \mydots \indexref{world wide web}\newline
\end{multicols}
~\newline
{\textbf{\large Inredning}}
\begin{multicols}{2}
Furu \mydots \indexref{furu}\newline
Kombinationsaffär \mydots \indexref{kombinationsaffaer}\newline
Trivselskrot \mydots \indexref{trivselskrot}\newline
Wunderbaum \mydots \indexref{wunderbaum}\newline
\end{multicols}
~\newline
{\textbf{\large Kategori}}
\begin{multicols}{2}
Stigma \mydots \indexref{stigma}\newline
\end{multicols}
~\newline
{\textbf{\large Konst och kultur}}
\begin{multicols}{2}
Ansiktsmålning \mydots \indexref{ansiktsmaalning}\newline
Arbetslinjen \mydots \indexref{arbetslinjen}\newline
ASEA-grönt \mydots \indexref{asea-groont}\newline
Berätta \mydots \indexref{beraetta}\newline
Bintje \mydots \indexref{bintje}\newline
Bonad \mydots \indexref{bonad}\newline
Brokiga Blad \mydots \indexref{brokiga blad}\newline
Ernst Billgren \mydots \indexref{ernst billgren}\newline
Evert Taubes värld \mydots \indexref{evert taubes vaerld}\newline
Finsk sommarsoppa \mydots \indexref{finsk sommarsoppa}\newline
Fiska kräfta med ficklampa \mydots \indexref{fiska kraefta med ficklampa}\newline
Fotografering \mydots \indexref{fotografering}\newline
Glasse \mydots \indexref{glasse}\newline
Hamlet \mydots \indexref{hamlet}\newline
Högtalartips \mydots \indexref{hoogtalartips}\newline
Hur man ritar ett snyggt lodjurshuvud \mydots \indexref{hur man ritar ett snyggt lodjurshuvud}\newline
International cloud atlas \mydots \indexref{international cloud atlas}\newline
Jackson Pollock \mydots \indexref{jackson pollock}\newline
Kattbrosch \mydots \indexref{kattbrosch}\newline
Kinneviksliljan \mydots \indexref{kinneviksliljan}\newline
Kir \mydots \indexref{kir}\newline
Kladdi Mittänän \mydots \indexref{kladdi mittaenaen}\newline
Klasse Möllberg \mydots \indexref{klasse moollberg}\newline
Korvbröd \mydots \indexref{korvbrood}\newline
Krigsgrisen \mydots \indexref{krigsgrisen}\newline
Kulturstökigt \mydots \indexref{kulturstookigt}\newline
Ligga med kulturstockholm \mydots \indexref{ligga med kulturstockholm}\newline
Luftgitarr \mydots \indexref{luftgitarr}\newline
Masonitemuséet i Rundvik \mydots \indexref{masonitemuséet i rundvik}\newline
Min kära gamla soppeskål \mydots \indexref{min kaera gamla soppeskaal}\newline
Musselini \mydots \indexref{musselini}\newline
Nudist \mydots \indexref{nudist}\newline
Ornässtugans dass \mydots \indexref{ornaesstugans dass}\newline
PATSY award \mydots \indexref{patsy award}\newline
Pörr \mydots \indexref{poorr}\newline
Potatistryck \mydots \indexref{potatistryck}\newline
Richard Dybeck \mydots \indexref{richard dybeck}\newline
Schwarzwald Larsson \mydots \indexref{schwarzwald larsson}\newline
Skinhead \mydots \indexref{skinhead}\newline
Sonett (engelsk) \mydots \indexref{sonett (engelsk)}\newline
Sputnik \mydots \indexref{sputnik}\newline
Teenage Mutant Ninja Turtles \mydots \indexref{teenage mutant ninja turtles}\newline
Telverksorange \mydots \indexref{telverksorange}\newline
Träbjörn \mydots \indexref{traebjoorn}\newline
Ugglekonst \mydots \indexref{ugglekonst}\newline
Uppstoppad uv \mydots \indexref{uppstoppad uv}\newline
Vansinnets historia \mydots \indexref{vansinnets historia}\newline
Vega Video \mydots \indexref{vega video}\newline
Vernissage \mydots \indexref{vernissage}\newline
Vit månad \mydots \indexref{vit maanad}\newline
\end{multicols}
~\newline
{\textbf{\large Korsningar mellan frukt och fisk}}
\begin{multicols}{2}
Turtlestestet \mydots \indexref{turtlestestet}\newline
\end{multicols}
~\newline
{\textbf{\large Krig och elände}}
\begin{multicols}{2}
Åka på safari \mydots \indexref{aaka paa safari}\newline
Ålandskrisen \mydots \indexref{aalandskrisen}\newline
Crustpippi \mydots \indexref{crustpippi}\newline
Det stora vadet 2013 \mydots \indexref{det stora vadet 2013}\newline
Geezer Butler \mydots \indexref{geezer butler}\newline
Georgij Zjukov \mydots \indexref{georgij zjukov}\newline
Gruk \mydots \indexref{gruk}\newline
Hudiksvall \mydots \indexref{hudiksvall}\newline
Krigsgrisen \mydots \indexref{krigsgrisen}\newline
Maginotlinjen \mydots \indexref{maginotlinjen}\newline
Manowar \mydots \indexref{manowar}\newline
Oscar Dronjak \mydots \indexref{oscar dronjak}\newline
Pangsionärerna \mydots \indexref{pangsionaererna}\newline
Polis \mydots \indexref{polis}\newline
Sarin \mydots \indexref{sarin}\newline
Spärrballong \mydots \indexref{spaerrballong}\newline
Spritvev \mydots \indexref{spritvev}\newline
Stjärtlapp \mydots \indexref{stjaertlapp}\newline
Utrotningshotade djur \mydots \indexref{utrotningshotade djur}\newline
Vänort \mydots \indexref{vaenort}\newline
\end{multicols}
~\newline
{\textbf{\large Kvinnonamn}}
\begin{multicols}{2}
Blaze Baylika \mydots \indexref{blaze baylika}\newline
Charlotte \mydots \indexref{charlotte}\newline
Erna \mydots \indexref{erna}\newline
Gertrud \mydots \indexref{gertrud}\newline
Gittan \mydots \indexref{gittan}\newline
Gunborg \mydots \indexref{gunborg}\newline
Hillevi \mydots \indexref{hillevi}\newline
Lena \mydots \indexref{lena}\newline
Siv-Berit \mydots \indexref{siv-berit}\newline
Snutnamn \mydots \indexref{snutnamn}\newline
Viktoria (namn) \mydots \indexref{viktoria (namn)}\newline
\end{multicols}
~\newline
{\textbf{\large Känsloliv}}
\begin{multicols}{2}
Besvikelse \mydots \indexref{besvikelse}\newline
Bitterhet \mydots \indexref{bitterhet}\newline
Förvirring \mydots \indexref{foorvirring}\newline
Lättnad \mydots \indexref{laettnad}\newline
Påsmygande själv-alienation \mydots \indexref{paasmygande sjaelv-alienation}\newline
Panik \mydots \indexref{panik}\newline
Romantik \mydots \indexref{romantik}\newline
Skam \mydots \indexref{skam}\newline
Stress \mydots \indexref{stress}\newline
\end{multicols}
~\newline
{\textbf{\large Lagar och förordningar}}
\begin{multicols}{2}
Ankeborgslagstiftning \mydots \indexref{ankeborgslagstiftning}\newline
Backa om \mydots \indexref{backa om}\newline
Drakonisk lag \mydots \indexref{drakonisk lag}\newline
Femtusen invånare-regeln \mydots \indexref{femtusen invaanare-regeln}\newline
Hytta med näven \mydots \indexref{hytta med naeven}\newline
ISO 216 \mydots \indexref{iso 216}\newline
Juridisk rådgivare \mydots \indexref{juridisk raadgivare}\newline
Konventikelplakatet \mydots \indexref{konventikelplakatet}\newline
Lätt misshandel \mydots \indexref{laett misshandel}\newline
LAS \mydots \indexref{las}\newline
Polis \mydots \indexref{polis}\newline
\end{multicols}
~\newline
{\textbf{\large Litteratur}}
\begin{multicols}{2}
Åke Ohlmarks \mydots \indexref{aake ohlmarks}\newline
Aforismer \mydots \indexref{aforismer}\newline
Annika \mydots \indexref{annika}\newline
Atlantica \mydots \indexref{atlantica}\newline
Att psykedelisera sin vardag \mydots \indexref{att psykedelisera sin vardag}\newline
Barnagans förträffliga pedagogik \mydots \indexref{barnagans foortraeffliga pedagogik}\newline
Beat \mydots \indexref{beat}\newline
Berätta \mydots \indexref{beraetta}\newline
Bibeln \mydots \indexref{bibeln}\newline
Burre \mydots \indexref{burre}\newline
Dävertspotting \mydots \indexref{daevertspotting}\newline
De förlösande thinneråren \mydots \indexref{de foorloosande thinneraaren}\newline
Den lilla boken \mydots \indexref{den lilla boken}\newline
Den vedervärdige mannen från Säffle \mydots \indexref{den vedervaerdige mannen fraan saeffle}\newline
Det politiska i C.C.Rs texter \mydots \indexref{det politiska i c.c.rs texter}\newline
Det susar i Säfve \mydots \indexref{det susar i saefve}\newline
Din avkomma och du \mydots \indexref{din avkomma och du}\newline
En bärs, en bärs, min järndanksamling för en bärs \mydots \indexref{en baers, en baers, min jaerndanksamling foor en baers}\newline
Enkido \mydots \indexref{enkido}\newline
Gengas \mydots \indexref{gengas}\newline
Glida under radarn \mydots \indexref{glida under radarn}\newline
Gruk \mydots \indexref{gruk}\newline
Haiku \mydots \indexref{haiku}\newline
Hundra sätt att få ligga \mydots \indexref{hundra saett att faa ligga}\newline
Ivar Lo Johansson \mydots \indexref{ivar lo johansson}\newline
J.R.R Tolkien \mydots \indexref{j.r.r tolkien}\newline
Jan Björkblund \mydots \indexref{jan bjoorkblund}\newline
Kalle Ankas pocket \mydots \indexref{kalle ankas pocket}\newline
Kriminalroman \mydots \indexref{kriminalroman}\newline
Kvinnlig författare-knepet \mydots \indexref{kvinnlig foorfattare-knepet}\newline
Lättlagade festrätter från Anderssons skafferi \mydots \indexref{laettlagade festraetter fraan anderssons skafferi}\newline
Limerick \mydots \indexref{limerick}\newline
Margit Sandemo \mydots \indexref{margit sandemo}\newline
När livet blir alldeles för mycket - Prof. Etiennes bästa gömställen, i urval \mydots \indexref{naer livet blir alldeles foor mycket - prof. etiennes baesta goomstaellen, i urval}\newline
Oscar Wilde \mydots \indexref{oscar wilde}\newline
Philibert Humla \mydots \indexref{philibert humla}\newline
Porträtt av det postmoderna renässansgeniet som ung \mydots \indexref{portraett av det postmoderna renaessansgeniet som ung}\newline
Rasmus Klump \mydots \indexref{rasmus klump}\newline
Skrattfnatt \mydots \indexref{skrattfnatt}\newline
Snälla mamma, mata mig som vore du en fågel \mydots \indexref{snaella mamma, mata mig som vore du en faagel}\newline
Sonett (engelsk) \mydots \indexref{sonett (engelsk)}\newline
Tolkien och den svarta magin \mydots \indexref{tolkien och den svarta magin}\newline
Uvmytologi \mydots \indexref{uvmytologi}\newline
Vansinnets historia \mydots \indexref{vansinnets historia}\newline
\end{multicols}
~\newline
{\textbf{\large Malå}}
\begin{multicols}{2}
Adak \mydots \indexref{adak}\newline
Flisbil \mydots \indexref{flisbil}\newline
Malå \mydots \indexref{malaa}\newline
Malålistan \mydots \indexref{malaalistan}\newline
Malåparkering \mydots \indexref{malaaparkering}\newline
Malårca \mydots \indexref{malaarca}\newline
Rökå \mydots \indexref{rookaa}\newline
Sågverk \mydots \indexref{saagverk}\newline
Skräckväldet \mydots \indexref{skraeckvaeldet}\newline
Tjamstan \mydots \indexref{tjamstan}\newline
\end{multicols}
~\newline
{\textbf{\large Mansnamn}}
\begin{multicols}{2}
Albin \mydots \indexref{albin}\newline
Bintje \mydots \indexref{bintje}\newline
Botte \mydots \indexref{botte}\newline
Brandklipparen \mydots \indexref{brandklipparen}\newline
Carl von Linné \mydots \indexref{carl von linné}\newline
Conny \mydots \indexref{conny}\newline
Dennis \mydots \indexref{dennis}\newline
Ebbe \mydots \indexref{ebbe}\newline
Edmund \mydots \indexref{edmund}\newline
Edvin \mydots \indexref{edvin}\newline
Glenn \mydots \indexref{glenn}\newline
Göran \mydots \indexref{gooran}\newline
Gösta \mydots \indexref{goosta}\newline
Jens \mydots \indexref{jens}\newline
Jubal \mydots \indexref{jubal}\newline
Kent \mydots \indexref{kent}\newline
King Edward \mydots \indexref{king edward}\newline
Lothar \mydots \indexref{lothar}\newline
Petter \mydots \indexref{petter}\newline
Rolf \mydots \indexref{rolf}\newline
Snutnamn \mydots \indexref{snutnamn}\newline
Svotto \mydots \indexref{svotto}\newline
Torbjörn \mydots \indexref{torbjoorn}\newline
\end{multicols}
~\newline
{\textbf{\large Mat}}
\begin{multicols}{2}
Åka på safari \mydots \indexref{aaka paa safari}\newline
Äckligt godis \mydots \indexref{aeckligt godis}\newline
Ädelost \mydots \indexref{aedelost}\newline
Äganderätt \mydots \indexref{aeganderaett}\newline
Alternativa namn på bakverk \mydots \indexref{alternativa namn paa bakverk}\newline
Aurora \mydots \indexref{aurora}\newline
Backa om \mydots \indexref{backa om}\newline
Bacon \mydots \indexref{bacon}\newline
Balutägg \mydots \indexref{balutaegg}\newline
Banan \mydots \indexref{banan}\newline
Bananas \mydots \indexref{bananas}\newline
Bintje \mydots \indexref{bintje}\newline
Brännvin \mydots \indexref{braennvin}\newline
Brugdguldet \mydots \indexref{brugdguldet}\newline
Bukfylla \mydots \indexref{bukfylla}\newline
Buss \mydots \indexref{buss}\newline
Cacao creme \mydots \indexref{cacao creme}\newline
Calskrove \mydots \indexref{calskrove}\newline
Champis \mydots \indexref{champis}\newline
Chipslåda \mydots \indexref{chipslaada}\newline
Dansk kostcirkel \mydots \indexref{dansk kostcirkel}\newline
Dansk lättöl \mydots \indexref{dansk laettool}\newline
Delfin \mydots \indexref{delfin}\newline
Den arga groggen \mydots \indexref{den arga groggen}\newline
Dubbelsovla \mydots \indexref{dubbelsovla}\newline
Eva Ekeblad \mydots \indexref{eva ekeblad}\newline
Färskost \mydots \indexref{faerskost}\newline
Falafel \mydots \indexref{falafel}\newline
Fet och grisig mat döpt efter lyxiga ställen/personer \mydots \indexref{fet och grisig mat doopt efter lyxiga staellenpersoner}\newline
Finsk sommarsoppa \mydots \indexref{finsk sommarsoppa}\newline
Fläsksvålar \mydots \indexref{flaesksvaalar}\newline
Folkkök \mydots \indexref{folkkook}\newline
Frasses \mydots \indexref{frasses}\newline
Fredagslyx \mydots \indexref{fredagslyx}\newline
Frukt \mydots \indexref{frukt}\newline
Fruktsallad \mydots \indexref{fruktsallad}\newline
Fudge \mydots \indexref{fudge}\newline
Gamle Ole \mydots \indexref{gamle ole}\newline
Glassbåt \mydots \indexref{glassbaat}\newline
Grissini \mydots \indexref{grissini}\newline
Gurka \mydots \indexref{gurka}\newline
Gurkmajonnäs \mydots \indexref{gurkmajonnaes}\newline
Gurkvatten \mydots \indexref{gurkvatten}\newline
Hasch \mydots \indexref{hasch}\newline
Hasselbackspotatis \mydots \indexref{hasselbackspotatis}\newline
Hawaii-pizza \mydots \indexref{hawaii-pizza}\newline
Hockeypulver \mydots \indexref{hockeypulver}\newline
Hundkäx \mydots \indexref{hundkaex}\newline
Ica prästost \mydots \indexref{ica praestost}\newline
Kaffekask \mydots \indexref{kaffekask}\newline
King Edward \mydots \indexref{king edward}\newline
Kokt \mydots \indexref{kokt}\newline
Kolonialdricka \mydots \indexref{kolonialdricka}\newline
Kopi Luwak \mydots \indexref{kopi luwak}\newline
Korv i smörpapper \mydots \indexref{korv i smoorpapper}\newline
Korv med bröd \mydots \indexref{korv med brood}\newline
Kräftbete \mydots \indexref{kraeftbete}\newline
Lättlagade festrätter från Anderssons skafferi \mydots \indexref{laettlagade festraetter fraan anderssons skafferi}\newline
Lättöl \mydots \indexref{laettool}\newline
Lappskojs \mydots \indexref{lappskojs}\newline
Lenin-Churchhill aka Mysgubbe \mydots \indexref{lenin-churchhill aka mysgubbe}\newline
Mäklarbricka \mydots \indexref{maeklarbricka}\newline
Minusmat \mydots \indexref{minusmat}\newline
Myspyssockerkaka \mydots \indexref{myspyssockerkaka}\newline
Odon \mydots \indexref{odon}\newline
Old ox \mydots \indexref{old ox}\newline
Ödla på pinne \mydots \indexref{oodla paa pinne}\newline
Ostmacka \mydots \indexref{ostmacka}\newline
På fat \mydots \indexref{paa fat}\newline
Päronhalva \mydots \indexref{paeronhalva}\newline
Paj \mydots \indexref{paj}\newline
Palle Kuling \mydots \indexref{palle kuling}\newline
Parisare \mydots \indexref{parisare}\newline
Picknickbog \mydots \indexref{picknickbog}\newline
Pölsa \mydots \indexref{poolsa}\newline
Potatisbar \mydots \indexref{potatisbar}\newline
Prinskorv \mydots \indexref{prinskorv}\newline
Proletära bär \mydots \indexref{proletaera baer}\newline
Punkgryta \mydots \indexref{punkgryta}\newline
Räkmacka \mydots \indexref{raekmacka}\newline
Räksallad \mydots \indexref{raeksallad}\newline
Räksmörgås \mydots \indexref{raeksmoorgaas}\newline
Rikemanssidan \mydots \indexref{rikemanssidan}\newline
Rögad ål \mydots \indexref{roogad aal}\newline
Rød pølse \mydots \indexref{roood pooolse}\newline
Sinkadus \mydots \indexref{sinkadus}\newline
Skolbespisningsmat \mydots \indexref{skolbespisningsmat}\newline
Skotte \mydots \indexref{skotte}\newline
Skruvkapsylöl \mydots \indexref{skruvkapsylool}\newline
Snus \mydots \indexref{snus}\newline
Snutkaffe \mydots \indexref{snutkaffe}\newline
Sorbet \mydots \indexref{sorbet}\newline
Spansk haloumi \mydots \indexref{spansk haloumi}\newline
Spanskt lättvin \mydots \indexref{spanskt laettvin}\newline
Stark mat \mydots \indexref{stark mat}\newline
Stödkorv \mydots \indexref{stoodkorv}\newline
Svag mat \mydots \indexref{svag mat}\newline
Sviskonpaj \mydots \indexref{sviskonpaj}\newline
Te \mydots \indexref{te}\newline
Tegare \mydots \indexref{tegare}\newline
Tjack \mydots \indexref{tjack}\newline
Tofu \mydots \indexref{tofu}\newline
Törley gala \mydots \indexref{toorley gala}\newline
Trocadero \mydots \indexref{trocadero}\newline
Turtlestestet \mydots \indexref{turtlestestet}\newline
Uvgodis \mydots \indexref{uvgodis}\newline
Uvsvane \mydots \indexref{uvsvane}\newline
Uvtårar \mydots \indexref{uvtaarar}\newline
Vin \mydots \indexref{vin}\newline
Vodka \mydots \indexref{vodka}\newline
\end{multicols}
~\newline
{\textbf{\large Matematik}}
\begin{multicols}{2}
Åtta \mydots \indexref{aatta}\newline
E=mc2 \mydots \indexref{e=mc2}\newline
Elva \mydots \indexref{elva}\newline
Femma \mydots \indexref{femma}\newline
Fyra \mydots \indexref{fyra}\newline
Fyrtiotusen miljarder \mydots \indexref{fyrtiotusen miljarder}\newline
Matematikmaskinnämnden \mydots \indexref{matematikmaskinnaemnden}\newline
Math metal \mydots \indexref{math metal}\newline
Nia \mydots \indexref{nia}\newline
Postiljon \mydots \indexref{postiljon}\newline
Sexa \mydots \indexref{sexa}\newline
Sjua \mydots \indexref{sjua}\newline
Snöskor \mydots \indexref{snooskor}\newline
Sportmössa \mydots \indexref{sportmoossa}\newline
Tia \mydots \indexref{tia}\newline
Tolva \mydots \indexref{tolva}\newline
Trea \mydots \indexref{trea}\newline
\end{multicols}
~\newline
{\textbf{\large Material}}
\begin{multicols}{2}
Beskinnad \mydots \indexref{beskinnad}\newline
\end{multicols}
~\newline
{\textbf{\large Media}}
\begin{multicols}{2}
Axess tv \mydots \indexref{axess tv}\newline
Mattias Alkberg \mydots \indexref{mattias alkberg}\newline
Radioreklam \mydots \indexref{radioreklam}\newline
\end{multicols}
~\newline
{\textbf{\large Meteorologi}}
\begin{multicols}{2}
Dimma \mydots \indexref{dimma}\newline
Inga lejon \mydots \indexref{inga lejon}\newline
International cloud atlas \mydots \indexref{international cloud atlas}\newline
Moln \mydots \indexref{moln}\newline
Roy Andersson-väder \mydots \indexref{roy andersson-vaeder}\newline
Skillnaden mellan ånga och dimma \mydots \indexref{skillnaden mellan aanga och dimma}\newline
Vädret \mydots \indexref{vaedret}\newline
\end{multicols}
~\newline
{\textbf{\large Mode}}
\begin{multicols}{2}
Allväderstövlar \mydots \indexref{allvaederstoovlar}\newline
Ankfot \mydots \indexref{ankfot}\newline
Backpatch \mydots \indexref{backpatch}\newline
Bakficka \mydots \indexref{bakficka}\newline
Bar överkropp \mydots \indexref{bar ooverkropp}\newline
Britts mode \mydots \indexref{britts mode}\newline
Chapeau de paysan \mydots \indexref{chapeau de paysan}\newline
Cirkuspung \mydots \indexref{cirkuspung}\newline
Crustknytning \mydots \indexref{crustknytning}\newline
Cykelhjälm \mydots \indexref{cykelhjaelm}\newline
Dragsko \mydots \indexref{dragsko}\newline
Dressmann \mydots \indexref{dressmann}\newline
Fejkspons \mydots \indexref{fejkspons}\newline
Finskt pannband \mydots \indexref{finskt pannband}\newline
Gore-Tex \mydots \indexref{gore-tex}\newline
Gråmelerad T-shirt \mydots \indexref{graamelerad t-shirt}\newline
Gustav Vasa \mydots \indexref{gustav vasa}\newline
Hällefors \mydots \indexref{haellefors}\newline
Hästhandlarplånbok \mydots \indexref{haesthandlarplaanbok}\newline
Herrkläder \mydots \indexref{herrklaeder}\newline
Hockey \mydots \indexref{hockey}\newline
Huvudduk \mydots \indexref{huvudduk}\newline
Jeansröv \mydots \indexref{jeansroov}\newline
Kalle anka \mydots \indexref{kalle anka}\newline
Kanadensisk frack \mydots \indexref{kanadensisk frack}\newline
Kattbrosch \mydots \indexref{kattbrosch}\newline
Kepsar med olika företagslogotyper \mydots \indexref{kepsar med olika fooretagslogotyper}\newline
Klädsamt ful \mydots \indexref{klaedsamt ful}\newline
Kommunistglasögon \mydots \indexref{kommunistglasoogon}\newline
Kortbyxor \mydots \indexref{kortbyxor}\newline
Kroppshydda \mydots \indexref{kroppshydda}\newline
Kvinnokläder \mydots \indexref{kvinnoklaeder}\newline
Leggings \mydots \indexref{leggings}\newline
Likgömmarmössa \mydots \indexref{likgoommarmoossa}\newline
Luktagott \mydots \indexref{luktagott}\newline
Mimmi Pigg \mydots \indexref{mimmi pigg}\newline
Näverslips \mydots \indexref{naeverslips}\newline
Orolighetskeps \mydots \indexref{orolighetskeps}\newline
Paj \mydots \indexref{paj}\newline
Propellerkeps \mydots \indexref{propellerkeps}\newline
Pysselbyxa \mydots \indexref{pysselbyxa}\newline
Ryggtryck \mydots \indexref{ryggtryck}\newline
Sämskskinn \mydots \indexref{saemskskinn}\newline
Särk \mydots \indexref{saerk}\newline
Sans pants \mydots \indexref{sans pants}\newline
Självförtroendeplagg \mydots \indexref{sjaelvfoortroendeplagg}\newline
Skäpparkrans \mydots \indexref{skaepparkrans}\newline
Skinnslips \mydots \indexref{skinnslips}\newline
Skjortponcho \mydots \indexref{skjortponcho}\newline
Småbyxa \mydots \indexref{smaabyxa}\newline
Småstadsalternativ \mydots \indexref{smaastadsalternativ}\newline
Sneakers \mydots \indexref{sneakers}\newline
Snowjoggers \mydots \indexref{snowjoggers}\newline
Speedos \mydots \indexref{speedos}\newline
Stonerskin \mydots \indexref{stonerskin}\newline
Strumpor \mydots \indexref{strumpor}\newline
Tantkläder \mydots \indexref{tantklaeder}\newline
Tuff-frysa \mydots \indexref{tuff-frysa}\newline
Wctbyxa \mydots \indexref{wctbyxa}\newline
\end{multicols}
~\newline
{\textbf{\large Musik}}
\begin{multicols}{2}
Åkerdisco \mydots \indexref{aakerdisco}\newline
AC/DC-gitarr \mydots \indexref{acdc-gitarr}\newline
Alice Tegnér \mydots \indexref{alice tegnér}\newline
Amebix \mydots \indexref{amebix}\newline
Angel of death \mydots \indexref{angel of death}\newline
Arbetarklassrock \mydots \indexref{arbetarklassrock}\newline
Bailando \mydots \indexref{bailando}\newline
Beat \mydots \indexref{beat}\newline
Bellman \mydots \indexref{bellman}\newline
Bockskäggsmetal \mydots \indexref{bockskaeggsmetal}\newline
Bonfire \mydots \indexref{bonfire}\newline
Brian Epstein \mydots \indexref{brian epstein}\newline
Bruksortskäng \mydots \indexref{bruksortskaeng}\newline
CCM \mydots \indexref{ccm}\newline
Charles Manson \mydots \indexref{charles manson}\newline
Christer Sandelin \mydots \indexref{christer sandelin}\newline
Crass \mydots \indexref{crass}\newline
Crustare \mydots \indexref{crustare}\newline
Crustpippi \mydots \indexref{crustpippi}\newline
Deadhead \mydots \indexref{deadhead}\newline
Den nya sångaren \mydots \indexref{den nya saangaren}\newline
Deportees-trevlig \mydots \indexref{deportees-trevlig}\newline
Der er et yndigt land \mydots \indexref{der er et yndigt land}\newline
Det politiska i C.C.Rs texter \mydots \indexref{det politiska i c.c.rs texter}\newline
Dimma \mydots \indexref{dimma}\newline
Dokumentärhora \mydots \indexref{dokumentaerhora}\newline
Dopesmoker \mydots \indexref{dopesmoker}\newline
Dr. Alban \mydots \indexref{dr. alban}\newline
Dragbasun \mydots \indexref{dragbasun}\newline
Evert Taube \mydots \indexref{evert taube}\newline
Fakta \mydots \indexref{fakta}\newline
False metal \mydots \indexref{false metal}\newline
Finsk inställning till rock \mydots \indexref{finsk instaellning till rock}\newline
Finsk pappersbruksarbetarkraut \mydots \indexref{finsk pappersbruksarbetarkraut}\newline
Fixed gear metal \mydots \indexref{fixed gear metal}\newline
Flöjt \mydots \indexref{floojt}\newline
Första sjuan \mydots \indexref{foorsta sjuan}\newline
Första skivanalibi \mydots \indexref{foorsta skivanalibi}\newline
Fransk gubbstoner \mydots \indexref{fransk gubbstoner}\newline
Fri rörlighet \mydots \indexref{fri roorlighet}\newline
Front row banger \mydots \indexref{front row banger}\newline
Frukt är gott \mydots \indexref{frukt aer gott}\newline
Geezer Butler \mydots \indexref{geezer butler}\newline
Gitarr \mydots \indexref{gitarr}\newline
Gitarrmys \mydots \indexref{gitarrmys}\newline
Grand Funk Railroad \mydots \indexref{grand funk railroad}\newline
Grind \mydots \indexref{grind}\newline
Gubbrock \mydots \indexref{gubbrock}\newline
Hårdrock \mydots \indexref{haardrock}\newline
Hårdrockare med gomspalt \mydots \indexref{haardrockare med gomspalt}\newline
Hårdrockare och vitaminer \mydots \indexref{haardrockare och vitaminer}\newline
Headbanga \mydots \indexref{headbanga}\newline
Helgvolym \mydots \indexref{helgvolym}\newline
Hemmets Härold \mydots \indexref{hemmets haerold}\newline
Hip-hop \mydots \indexref{hip-hop}\newline
Horgalåten \mydots \indexref{horgalaaten}\newline
Hugo Alfvén \mydots \indexref{hugo alfvén}\newline
International harvester \mydots \indexref{international harvester}\newline
Ivar Lo Johansson \mydots \indexref{ivar lo johansson}\newline
Jeansröv \mydots \indexref{jeansroov}\newline
Jerry Williams \mydots \indexref{jerry williams}\newline
Känslo-Oi! \mydots \indexref{kaenslo-oi!}\newline
Karel Gott \mydots \indexref{karel gott}\newline
Kerry King \mydots \indexref{kerry king}\newline
Keytar \mydots \indexref{keytar}\newline
Kinesiska muren \mydots \indexref{kinesiska muren}\newline
Klasse Möllberg \mydots \indexref{klasse moollberg}\newline
Knocking on heavens door \mydots \indexref{knocking on heavens door}\newline
Könsrock \mydots \indexref{koonsrock}\newline
Lars Krogh \mydots \indexref{lars krogh}\newline
Lemmy-bas \mydots \indexref{lemmy-bas}\newline
Lista över dis-namn \mydots \indexref{lista oover dis-namn}\newline
Luftgitarr \mydots \indexref{luftgitarr}\newline
Måndag \mydots \indexref{maandag}\newline
Manowar \mydots \indexref{manowar}\newline
Math metal \mydots \indexref{math metal}\newline
Mattias Alkberg \mydots \indexref{mattias alkberg}\newline
Merchband \mydots \indexref{merchband}\newline
Micke Alonzo \mydots \indexref{micke alonzo}\newline
Mikis Theodorakis \mydots \indexref{mikis theodorakis}\newline
Min kära gamla soppeskål \mydots \indexref{min kaera gamla soppeskaal}\newline
Mob 47 \mydots \indexref{mob 47}\newline
Musikhögskolemusik \mydots \indexref{musikhoogskolemusik}\newline
Mystiska band \mydots \indexref{mystiska band}\newline
Näs-flås \mydots \indexref{naes-flaas}\newline
Nedsatt sikt \mydots \indexref{nedsatt sikt}\newline
Old Black \mydots \indexref{old black}\newline
Oliver/Dawson Saxon \mydots \indexref{oliverdawson saxon}\newline
Ououou \mydots \indexref{ououou}\newline
Panflöjt \mydots \indexref{panfloojt}\newline
Pelle Karlsson \mydots \indexref{pelle karlsson}\newline
Petter \mydots \indexref{petter}\newline
Plocka päron \mydots \indexref{plocka paeron}\newline
Pop-rock \mydots \indexref{pop-rock}\newline
Postpostrock \mydots \indexref{postpostrock}\newline
Punk \mydots \indexref{punk}\newline
RAC \mydots \indexref{rac}\newline
Raw justice \mydots \indexref{raw justice}\newline
Rövgitarr \mydots \indexref{roovgitarr}\newline
Saltbas \mydots \indexref{saltbas}\newline
Samtida nordisk undergroundmusik \mydots \indexref{samtida nordisk undergroundmusik}\newline
Shockrockare \mydots \indexref{shockrockare}\newline
Sjua \mydots \indexref{sjua}\newline
Sjungande trummis \mydots \indexref{sjungande trummis}\newline
Skogsrave \mydots \indexref{skogsrave}\newline
Slayerklass \mydots \indexref{slayerklass}\newline
Smygfascist \mydots \indexref{smygfascist}\newline
Sober-Jimmy \mydots \indexref{sober-jimmy}\newline
Sommarplågsmusiker \mydots \indexref{sommarplaagsmusiker}\newline
Stonerskin \mydots \indexref{stonerskin}\newline
Surfin' bird \mydots \indexref{surfin bird}\newline
Tantsång \mydots \indexref{tantsaang}\newline
The Fog \mydots \indexref{the fog}\newline
Thrashzan \mydots \indexref{thrashzan}\newline
Träd, Gräs och Stenar \mydots \indexref{traed, graes och stenar}\newline
Träskpunkare \mydots \indexref{traeskpunkare}\newline
Trans \mydots \indexref{trans}\newline
Trollpunk \mydots \indexref{trollpunk}\newline
Världens näst ondaste band \mydots \indexref{vaerldens naest ondaste band}\newline
Världsmusik \mydots \indexref{vaerldsmusik}\newline
Valsång \mydots \indexref{valsaang}\newline
Verklighetens folk \mydots \indexref{verklighetens folk}\newline
Vevlira \mydots \indexref{vevlira}\newline
Warcollapse \mydots \indexref{warcollapse}\newline
Watain \mydots \indexref{watain}\newline
We are the world \mydots \indexref{we are the world}\newline
Wham, bam, thank you ma'am \mydots \indexref{wham, bam, thank you maam}\newline
World Wide Web \mydots \indexref{world wide web}\newline
\end{multicols}
~\newline
{\textbf{\large Människokroppen}}
\begin{multicols}{2}
Allergi \mydots \indexref{allergi}\newline
Arbetarblåsa \mydots \indexref{arbetarblaasa}\newline
Arselhaka \mydots \indexref{arselhaka}\newline
Bar överkropp \mydots \indexref{bar ooverkropp}\newline
Barbados \mydots \indexref{barbados}\newline
Beskinnad \mydots \indexref{beskinnad}\newline
Blomkålsöra \mydots \indexref{blomkaalsoora}\newline
Bonnseg \mydots \indexref{bonnseg}\newline
Dragspelsmuskeln \mydots \indexref{dragspelsmuskeln}\newline
Fetma \mydots \indexref{fetma}\newline
Fetor ex ore \mydots \indexref{fetor ex ore}\newline
Flatologi \mydots \indexref{flatologi}\newline
Fnysning \mydots \indexref{fnysning}\newline
Foucaultfingret \mydots \indexref{foucaultfingret}\newline
Fulsnygg \mydots \indexref{fulsnygg}\newline
Glimröv \mydots \indexref{glimroov}\newline
Hårdrockare med gomspalt \mydots \indexref{haardrockare med gomspalt}\newline
Hårdrockare och vitaminer \mydots \indexref{haardrockare och vitaminer}\newline
Haka (vanlig) \mydots \indexref{haka (vanlig)}\newline
Hockeyröv \mydots \indexref{hockeyroov}\newline
Holmund \mydots \indexref{holmund}\newline
Huvud \mydots \indexref{huvud}\newline
Hyperhidros \mydots \indexref{hyperhidros}\newline
Hytta med näven \mydots \indexref{hytta med naeven}\newline
Indianmuskler \mydots \indexref{indianmuskler}\newline
Jeansröv \mydots \indexref{jeansroov}\newline
Kissemiss \mydots \indexref{kissemiss}\newline
Knixa \mydots \indexref{knixa}\newline
Kräftor \mydots \indexref{kraeftor}\newline
Mun \mydots \indexref{mun}\newline
Näbbmun \mydots \indexref{naebbmun}\newline
Näs-flås \mydots \indexref{naes-flaas}\newline
Näsa \mydots \indexref{naesa}\newline
Öra \mydots \indexref{oora}\newline
Övre magmunnen \mydots \indexref{oovre magmunnen}\newline
Praktarsle (positiv) \mydots \indexref{praktarsle (positiv)}\newline
Rygg \mydots \indexref{rygg}\newline
Snåltarmen \mydots \indexref{snaaltarmen}\newline
Snutröv \mydots \indexref{snutroov}\newline
T-rexarmar \mydots \indexref{t-rexarmar}\newline
Tunntarmen \mydots \indexref{tunntarmen}\newline
Tvåa \mydots \indexref{tvaaa}\newline
Urin \mydots \indexref{urin}\newline
\end{multicols}
~\newline
{\textbf{\large Måttenheter}}
\begin{multicols}{2}
Armlängd \mydots \indexref{armlaengd}\newline
Botte \mydots \indexref{botte}\newline
Femma \mydots \indexref{femma}\newline
Fyrtiotusen miljarder \mydots \indexref{fyrtiotusen miljarder}\newline
Hemmansvärde \mydots \indexref{hemmansvaerde}\newline
Mangel \mydots \indexref{mangel}\newline
Snärt \mydots \indexref{snaert}\newline
Zoom \mydots \indexref{zoom}\newline
\end{multicols}
~\newline
{\textbf{\large Nissepedia}}
\begin{multicols}{2}
Nissepedia \mydots \indexref{nissepedia}\newline
\end{multicols}
~\newline
{\textbf{\large Norge}}
\begin{multicols}{2}
Den nordiska smuggeltriangeln \mydots \indexref{den nordiska smuggeltriangeln}\newline
\end{multicols}
~\newline
{\textbf{\large Nöje}}
\begin{multicols}{2}
Åkerdisco \mydots \indexref{aakerdisco}\newline
Adde Malmberg \mydots \indexref{adde malmberg}\newline
Alkoläskfylla \mydots \indexref{alkolaeskfylla}\newline
Alvparty \mydots \indexref{alvparty}\newline
Bade \mydots \indexref{bade}\newline
Bärsfylla \mydots \indexref{baersfylla}\newline
Boris Jeltsin \mydots \indexref{boris jeltsin}\newline
Eva Ekeblad \mydots \indexref{eva ekeblad}\newline
Finljuga \mydots \indexref{finljuga}\newline
Folkölsfylla \mydots \indexref{folkoolsfylla}\newline
Fredagslyx \mydots \indexref{fredagslyx}\newline
Fredagsmys \mydots \indexref{fredagsmys}\newline
Freikörperkultur \mydots \indexref{freikoorperkultur}\newline
Glädjevetenskaper \mydots \indexref{glaedjevetenskaper}\newline
Göras till åtlöje inför hela svenska folket \mydots \indexref{gooras till aatlooje infoor hela svenska folket}\newline
Hästkista \mydots \indexref{haestkista}\newline
Hasch \mydots \indexref{hasch}\newline
Hemmets Härold \mydots \indexref{hemmets haerold}\newline
Jävelskap \mydots \indexref{jaevelskap}\newline
Kelsey Grammer \mydots \indexref{kelsey grammer}\newline
Kicki Danielsson \mydots \indexref{kicki danielsson}\newline
Malårca \mydots \indexref{malaarca}\newline
Myntsamleri \mydots \indexref{myntsamleri}\newline
Punkscenshumor \mydots \indexref{punkscenshumor}\newline
Roliga timmen \mydots \indexref{roliga timmen}\newline
Rulla hatt \mydots \indexref{rulla hatt}\newline
Spritfylla \mydots \indexref{spritfylla}\newline
Stenad \mydots \indexref{stenad}\newline
Svensk bilsemester \mydots \indexref{svensk bilsemester}\newline
Tjack \mydots \indexref{tjack}\newline
Vältagravstensfull \mydots \indexref{vaeltagravstensfull}\newline
Världens tråkigaste skämt \mydots \indexref{vaerldens traakigaste skaemt}\newline
Vinfylla \mydots \indexref{vinfylla}\newline
\end{multicols}
~\newline
{\textbf{\large Paternalism}}
\begin{multicols}{2}
Balticgruppen \mydots \indexref{balticgruppen}\newline
Bilprovningen \mydots \indexref{bilprovningen}\newline
Civilpolis \mydots \indexref{civilpolis}\newline
Drakonisk lag \mydots \indexref{drakonisk lag}\newline
Institut och tankesmedjor \mydots \indexref{institut och tankesmedjor}\newline
Torsten Bengtsson \mydots \indexref{torsten bengtsson}\newline
\end{multicols}
~\newline
{\textbf{\large Pizza}}
\begin{multicols}{2}
Calskrove \mydots \indexref{calskrove}\newline
Calzona \mydots \indexref{calzona}\newline
Hawaii-pizza \mydots \indexref{hawaii-pizza}\newline
Pizzaracer \mydots \indexref{pizzaracer}\newline
Pizzarulle \mydots \indexref{pizzarulle}\newline
Turtlestestet \mydots \indexref{turtlestestet}\newline
\end{multicols}
~\newline
{\textbf{\large Polen}}
\begin{multicols}{2}
Bjudsprit \mydots \indexref{bjudsprit}\newline
Polska helgdagar \mydots \indexref{polska helgdagar}\newline
\end{multicols}
~\newline
{\textbf{\large Politik och debatt}}
\begin{multicols}{2}
Äventyrsbad \mydots \indexref{aeventyrsbad}\newline
Annie Lööf \mydots \indexref{annie loooof}\newline
Anti-speciesism \mydots \indexref{anti-speciesism}\newline
Anton Abele \mydots \indexref{anton abele}\newline
Arbetsplatskamp \mydots \indexref{arbetsplatskamp}\newline
Avbolagisering \mydots \indexref{avbolagisering}\newline
Centerpartiet \mydots \indexref{centerpartiet}\newline
Ett kille \mydots \indexref{ett kille}\newline
Feminism \mydots \indexref{feminism}\newline
Feministknepet \mydots \indexref{feministknepet}\newline
Fnysning \mydots \indexref{fnysning}\newline
Folketinget \mydots \indexref{folketinget}\newline
Folkpartiet \mydots \indexref{folkpartiet}\newline
Fri rörlighet \mydots \indexref{fri roorlighet}\newline
Frihet \mydots \indexref{frihet}\newline
Grå eminens \mydots \indexref{graa eminens}\newline
Håkan Juholt \mydots \indexref{haakan juholt}\newline
Hårdrockism \mydots \indexref{haardrockism}\newline
Hegemoni \mydots \indexref{hegemoni}\newline
Holmund \mydots \indexref{holmund}\newline
Ingvar Carlsson \mydots \indexref{ingvar carlsson}\newline
Insändarsignaturer \mydots \indexref{insaendarsignaturer}\newline
Intellektuell regression \mydots \indexref{intellektuell regression}\newline
Italienska svordomar \mydots \indexref{italienska svordomar}\newline
Jan Björklund \mydots \indexref{jan bjoorklund}\newline
Kommunist \mydots \indexref{kommunist}\newline
Kristdemokraterna \mydots \indexref{kristdemokraterna}\newline
Kvinnligt alibi \mydots \indexref{kvinnligt alibi}\newline
Lennart Holmlund \mydots \indexref{lennart holmlund}\newline
Lennart Holmlund \mydots \indexref{lennart holmlund}\newline
Malålistan \mydots \indexref{malaalistan}\newline
Mats Lundgren \mydots \indexref{mats lundgren}\newline
Miljöpartiet \mydots \indexref{miljoopartiet}\newline
Moderat \mydots \indexref{moderat}\newline
Moderator \mydots \indexref{moderator}\newline
Nyliberalism \mydots \indexref{nyliberalism}\newline
Oberoende olympiska deltagare \mydots \indexref{oberoende olympiska deltagare}\newline
Privatisering \mydots \indexref{privatisering}\newline
Processa mot länsstyrelsen \mydots \indexref{processa mot laensstyrelsen}\newline
Pudaslåda \mydots \indexref{pudaslaada}\newline
Randall Finefield \mydots \indexref{randall finefield}\newline
Sko \mydots \indexref{sko}\newline
Socialdemokrati \mydots \indexref{socialdemokrati}\newline
Solidaritet \mydots \indexref{solidaritet}\newline
Storswänsk \mydots \indexref{storswaensk}\newline
Svart alibi \mydots \indexref{svart alibi}\newline
Syndikalism \mydots \indexref{syndikalism}\newline
Tokliberal \mydots \indexref{tokliberal}\newline
Trotta \mydots \indexref{trotta}\newline
Våld \mydots \indexref{vaald}\newline
Valfrihet \mydots \indexref{valfrihet}\newline
\end{multicols}
~\newline
{\textbf{\large Psykologi och beteenden}}
\begin{multicols}{2}
Ägg \mydots \indexref{aegg}\newline
Arbetslinjen \mydots \indexref{arbetslinjen}\newline
Arggissa \mydots \indexref{arggissa}\newline
Bakisångest \mydots \indexref{bakisaangest}\newline
Bananas \mydots \indexref{bananas}\newline
Blottare \mydots \indexref{blottare}\newline
Busskruven \mydots \indexref{busskruven}\newline
Den tyska mustigheten \mydots \indexref{den tyska mustigheten}\newline
Deportees-trevlig \mydots \indexref{deportees-trevlig}\newline
Det omedvetna \mydots \indexref{det omedvetna}\newline
Erik Homburger Erikson \mydots \indexref{erik homburger erikson}\newline
Finsk inställning till rock \mydots \indexref{finsk instaellning till rock}\newline
Förvirring \mydots \indexref{foorvirring}\newline
Frikyrkligt lycklig \mydots \indexref{frikyrkligt lycklig}\newline
Front row banger \mydots \indexref{front row banger}\newline
Fryntlig \mydots \indexref{fryntlig}\newline
Fulhybris \mydots \indexref{fulhybris}\newline
Fyllevolontära \mydots \indexref{fyllevolontaera}\newline
Gå och köpa tidningen \mydots \indexref{gaa och koopa tidningen}\newline
Grisfull \mydots \indexref{grisfull}\newline
Gubbsäker \mydots \indexref{gubbsaeker}\newline
Hänga på låset \mydots \indexref{haenga paa laaset}\newline
Intellektuell regression \mydots \indexref{intellektuell regression}\newline
International harvester \mydots \indexref{international harvester}\newline
Japan \mydots \indexref{japan}\newline
Källkritik \mydots \indexref{kaellkritik}\newline
Kärlek \mydots \indexref{kaerlek}\newline
Kränkt \mydots \indexref{kraenkt}\newline
Kvicktänkt \mydots \indexref{kvicktaenkt}\newline
Lättnad \mydots \indexref{laettnad}\newline
Lättöl \mydots \indexref{laettool}\newline
Mani \mydots \indexref{mani}\newline
Norsjöblicken \mydots \indexref{norsjooblicken}\newline
Oidipuskomplex \mydots \indexref{oidipuskomplex}\newline
Påsmygande själv-alienation \mydots \indexref{paasmygande sjaelv-alienation}\newline
Panflöjt \mydots \indexref{panfloojt}\newline
Passa tider \mydots \indexref{passa tider}\newline
Paxa \mydots \indexref{paxa}\newline
Post-coitus \mydots \indexref{post-coitus}\newline
Proggig inre frid \mydots \indexref{proggig inre frid}\newline
Rasism \mydots \indexref{rasism}\newline
Särske \mydots \indexref{saerske}\newline
Samurajernas hederskodex \mydots \indexref{samurajernas hederskodex}\newline
Självförtroendeplagg \mydots \indexref{sjaelvfoortroendeplagg}\newline
Självmordsspåret \mydots \indexref{sjaelvmordsspaaret}\newline
Skuggan \mydots \indexref{skuggan}\newline
Släktträffsberusning \mydots \indexref{slaekttraeffsberusning}\newline
Storhetsvansinne \mydots \indexref{storhetsvansinne}\newline
Strulputte \mydots \indexref{strulputte}\newline
Ta för sig \mydots \indexref{ta foor sig}\newline
Tantnöjd \mydots \indexref{tantnoojd}\newline
Thorstenkram \mydots \indexref{thorstenkram}\newline
Turtlestestet \mydots \indexref{turtlestestet}\newline
Vältagravstensfull \mydots \indexref{vaeltagravstensfull}\newline
Yngwiefiering \mydots \indexref{yngwiefiering}\newline
\end{multicols}
~\newline
{\textbf{\large Redskap}}
\begin{multicols}{2}
13 \mydots \indexref{13}\newline
Ärtpåse \mydots \indexref{aertpaase}\newline
Anglosax \mydots \indexref{anglosax}\newline
Bilbatteri \mydots \indexref{bilbatteri}\newline
Bricka \mydots \indexref{bricka}\newline
Bruksgök \mydots \indexref{bruksgook}\newline
Cigg \mydots \indexref{cigg}\newline
Cykelhjälm \mydots \indexref{cykelhjaelm}\newline
Dävert \mydots \indexref{daevert}\newline
Dank \mydots \indexref{dank}\newline
Dansk sax \mydots \indexref{dansk sax}\newline
Dansk tubkikare \mydots \indexref{dansk tubkikare}\newline
Diagram \mydots \indexref{diagram}\newline
Diskett \mydots \indexref{diskett}\newline
Dragspel \mydots \indexref{dragspel}\newline
Dyckert \mydots \indexref{dyckert}\newline
Fasta nycklar \mydots \indexref{fasta nycklar}\newline
Flöjtfodral \mydots \indexref{floojtfodral}\newline
Gitarr \mydots \indexref{gitarr}\newline
Glasögon \mydots \indexref{glasoogon}\newline
Glassbutt \mydots \indexref{glassbutt}\newline
Godisautomat \mydots \indexref{godisautomat}\newline
Grind \mydots \indexref{grind}\newline
Hacka \mydots \indexref{hacka}\newline
Handjagare \mydots \indexref{handjagare}\newline
International harvester \mydots \indexref{international harvester}\newline
Järnspett \mydots \indexref{jaernspett}\newline
Jättemyrslokssele \mydots \indexref{jaettemyrslokssele}\newline
Karbinhake \mydots \indexref{karbinhake}\newline
Kattstrypare \mydots \indexref{kattstrypare}\newline
Keytar \mydots \indexref{keytar}\newline
Kikare \mydots \indexref{kikare}\newline
Kökssoffan \mydots \indexref{kookssoffan}\newline
Korp \mydots \indexref{korp}\newline
Krus \mydots \indexref{krus}\newline
Lego \mydots \indexref{lego}\newline
Lemmy-bas \mydots \indexref{lemmy-bas}\newline
Looppedal \mydots \indexref{looppedal}\newline
Luffarskål \mydots \indexref{luffarskaal}\newline
Mjukpack cigg \mydots \indexref{mjukpack cigg}\newline
Molotov cocktail \mydots \indexref{molotov cocktail}\newline
Muttersvarvare \mydots \indexref{muttersvarvare}\newline
Olja \mydots \indexref{olja}\newline
Oscar Wilde \mydots \indexref{oscar wilde}\newline
Påsförslutare \mydots \indexref{paasfoorslutare}\newline
Palltruck \mydots \indexref{palltruck}\newline
Propeller \mydots \indexref{propeller}\newline
Pudaslåda \mydots \indexref{pudaslaada}\newline
Riksregalier \mydots \indexref{riksregalier}\newline
Sandpapper \mydots \indexref{sandpapper}\newline
Senilsnöre \mydots \indexref{senilsnoore}\newline
Singelsnurra \mydots \indexref{singelsnurra}\newline
Skiftnyckel \mydots \indexref{skiftnyckel}\newline
Skräp \mydots \indexref{skraep}\newline
Smidesstäd \mydots \indexref{smidesstaed}\newline
Spärrballong \mydots \indexref{spaerrballong}\newline
Stjärtlapp \mydots \indexref{stjaertlapp}\newline
Stryknin \mydots \indexref{stryknin}\newline
Toalettpapper \mydots \indexref{toalettpapper}\newline
Tornedalslåset \mydots \indexref{tornedalslaaset}\newline
Tråg \mydots \indexref{traag}\newline
\end{multicols}
~\newline
{\textbf{\large Religion}}
\begin{multicols}{2}
Anton Lavey \mydots \indexref{anton lavey}\newline
Bartolomaios \mydots \indexref{bartolomaios}\newline
CCM \mydots \indexref{ccm}\newline
Dansk skrock \mydots \indexref{dansk skrock}\newline
Fiacre \mydots \indexref{fiacre}\newline
Gud \mydots \indexref{gud}\newline
Helgarderat efterliv \mydots \indexref{helgarderat efterliv}\newline
Helvetet \mydots \indexref{helvetet}\newline
Jesaja 36:12 \mydots \indexref{jesaja 36:12}\newline
Jesus \mydots \indexref{jesus}\newline
Jubal \mydots \indexref{jubal}\newline
Katolik \mydots \indexref{katolik}\newline
Motpåven i Gränna \mydots \indexref{motpaaven i graenna}\newline
Storspren \mydots \indexref{storspren}\newline
Strövtåg \mydots \indexref{stroovtaag}\newline
Uvmytologi \mydots \indexref{uvmytologi}\newline
\end{multicols}
~\newline
{\textbf{\large Rymdfart}}
\begin{multicols}{2}
Albert II \mydots \indexref{albert ii}\newline
Belka \mydots \indexref{belka}\newline
Jurij Gagarin \mydots \indexref{jurij gagarin}\newline
Martine \mydots \indexref{martine}\newline
Matematikmaskinnämnden \mydots \indexref{matematikmaskinnaemnden}\newline
Sputnik \mydots \indexref{sputnik}\newline
Strelka \mydots \indexref{strelka}\newline
Tsygan \mydots \indexref{tsygan}\newline
Valentina Vladimirovna Teresjkova \mydots \indexref{valentina vladimirovna teresjkova}\newline
\end{multicols}
~\newline
{\textbf{\large Samlevnad}}
\begin{multicols}{2}
Barndom \mydots \indexref{barndom}\newline
Bjuddosa \mydots \indexref{bjuddosa}\newline
Bjudsprit \mydots \indexref{bjudsprit}\newline
Bjudtermos \mydots \indexref{bjudtermos}\newline
Blandfylla \mydots \indexref{blandfylla}\newline
Bortamatch \mydots \indexref{bortamatch}\newline
Brandklipparen \mydots \indexref{brandklipparen}\newline
Brobo \mydots \indexref{brobo}\newline
Currykondom \mydots \indexref{currykondom}\newline
Din avkomma och du \mydots \indexref{din avkomma och du}\newline
Feministknepet \mydots \indexref{feministknepet}\newline
Filateli \mydots \indexref{filateli}\newline
Förlovad \mydots \indexref{foorlovad}\newline
Gå och köpa tidningen \mydots \indexref{gaa och koopa tidningen}\newline
Gammpojkar \mydots \indexref{gammpojkar}\newline
Gammstinta \mydots \indexref{gammstinta}\newline
Gitarrmys \mydots \indexref{gitarrmys}\newline
Gnussa \mydots \indexref{gnussa}\newline
Kattsläkt \mydots \indexref{kattslaekt}\newline
Kusin \mydots \indexref{kusin}\newline
Kvinnlig författare-knepet \mydots \indexref{kvinnlig foorfattare-knepet}\newline
Läsesalsflört \mydots \indexref{laesesalsfloort}\newline
Ledingreppet \mydots \indexref{ledingreppet}\newline
Ligga med kulturstockholm \mydots \indexref{ligga med kulturstockholm}\newline
Lurkuk \mydots \indexref{lurkuk}\newline
Måg \mydots \indexref{maag}\newline
Pörr \mydots \indexref{poorr}\newline
Post-coitus \mydots \indexref{post-coitus}\newline
Postmodern morförälder \mydots \indexref{postmodern morfooraelder}\newline
Prunka \mydots \indexref{prunka}\newline
Romantik \mydots \indexref{romantik}\newline
Skensnygg \mydots \indexref{skensnygg}\newline
Snälla killar som aldrig får ligga \mydots \indexref{snaella killar som aldrig faar ligga}\newline
\end{multicols}
~\newline
{\textbf{\large Sexualitet}}
\begin{multicols}{2}
Grunka \mydots \indexref{grunka}\newline
Jeansröv \mydots \indexref{jeansroov}\newline
Luktagott \mydots \indexref{luktagott}\newline
Odon \mydots \indexref{odon}\newline
Produktionsknull \mydots \indexref{produktionsknull}\newline
\end{multicols}
~\newline
{\textbf{\large Sjukdomar och åkommor}}
\begin{multicols}{2}
Allergi \mydots \indexref{allergi}\newline
Ankfot \mydots \indexref{ankfot}\newline
Dagvill \mydots \indexref{dagvill}\newline
Dansk semester \mydots \indexref{dansk semester}\newline
Den tyska mustigheten \mydots \indexref{den tyska mustigheten}\newline
Fetor ex ore \mydots \indexref{fetor ex ore}\newline
Fotbollspundare \mydots \indexref{fotbollspundare}\newline
Galna ko-sjukan \mydots \indexref{galna ko-sjukan}\newline
Gråmelerad T-shirt \mydots \indexref{graamelerad t-shirt}\newline
Hyperhidros \mydots \indexref{hyperhidros}\newline
Kinneviksliljan \mydots \indexref{kinneviksliljan}\newline
Psoriasis \mydots \indexref{psoriasis}\newline
Snusbrist \mydots \indexref{snusbrist}\newline
Storhetsvansinne \mydots \indexref{storhetsvansinne}\newline
Surrande ljud \mydots \indexref{surrande ljud}\newline
Syfilis \mydots \indexref{syfilis}\newline
Traditionell finsk medicin \mydots \indexref{traditionell finsk medicin}\newline
Traditionell kinesisk medicin \mydots \indexref{traditionell kinesisk medicin}\newline
Valsvärk \mydots \indexref{valsvaerk}\newline
\end{multicols}
~\newline
{\textbf{\large Sjöfart och navigation}}
\begin{multicols}{2}
Åka vikingaskepp \mydots \indexref{aaka vikingaskepp}\newline
Åtta \mydots \indexref{aatta}\newline
Aktersegla \mydots \indexref{aktersegla}\newline
Batyskaf \mydots \indexref{batyskaf}\newline
Bosporen \mydots \indexref{bosporen}\newline
CQD \mydots \indexref{cqd}\newline
Däcka \mydots \indexref{daecka}\newline
Dävert \mydots \indexref{daevert}\newline
Dansk kanot \mydots \indexref{dansk kanot}\newline
Flytväst \mydots \indexref{flytvaest}\newline
Glassbåt \mydots \indexref{glassbaat}\newline
Jolle \mydots \indexref{jolle}\newline
Postångare \mydots \indexref{postaangare}\newline
Propeller \mydots \indexref{propeller}\newline
Sextant \mydots \indexref{sextant}\newline
Skäpparkrans \mydots \indexref{skaepparkrans}\newline
Snesegla \mydots \indexref{snesegla}\newline
Tåga \mydots \indexref{taaga}\newline
Tura \mydots \indexref{tura}\newline
\end{multicols}
~\newline
{\textbf{\large Sport och spel}}
\begin{multicols}{2}
50 million piece of shit \mydots \indexref{50 million piece of shit}\newline
Ärtpåse \mydots \indexref{aertpaase}\newline
Aleksandr Karelin \mydots \indexref{aleksandr karelin}\newline
Andörjan \mydots \indexref{andoorjan}\newline
Backa med släp \mydots \indexref{backa med slaep}\newline
Berghem HC \mydots \indexref{berghem hc}\newline
Bortamatch \mydots \indexref{bortamatch}\newline
Dansk kortlek \mydots \indexref{dansk kortlek}\newline
Den lilla boken \mydots \indexref{den lilla boken}\newline
Det stora vadet 2013 \mydots \indexref{det stora vadet 2013}\newline
Duvgubbar \mydots \indexref{duvgubbar}\newline
Elva \mydots \indexref{elva}\newline
Erik Hamrén \mydots \indexref{erik hamrén}\newline
Förstklassig cricket \mydots \indexref{foorstklassig cricket}\newline
Fotboll \mydots \indexref{fotboll}\newline
Glima \mydots \indexref{glima}\newline
Gnagare \mydots \indexref{gnagare}\newline
Golf \mydots \indexref{golf}\newline
Gubbafint \mydots \indexref{gubbafint}\newline
Habbadixen! \mydots \indexref{habbadixen!}\newline
Hockey \mydots \indexref{hockey}\newline
Innebandy \mydots \indexref{innebandy}\newline
Johnny Takter \mydots \indexref{johnny takter}\newline
Kalixare \mydots \indexref{kalixare}\newline
Landslagsuppehåll \mydots \indexref{landslagsuppehaall}\newline
Lars-Åke Lagrell \mydots \indexref{lars-aake lagrell}\newline
Luís Figo \mydots \indexref{luís figo}\newline
Målvakt \mydots \indexref{maalvakt}\newline
Nikolaj Valujev \mydots \indexref{nikolaj valujev}\newline
Nitlott \mydots \indexref{nitlott}\newline
Nordisk kombination \mydots \indexref{nordisk kombination}\newline
Oberoende olympiska deltagare \mydots \indexref{oberoende olympiska deltagare}\newline
Pelle Svensson \mydots \indexref{pelle svensson}\newline
Säcklöpning \mydots \indexref{saeckloopning}\newline
Sexa \mydots \indexref{sexa}\newline
Shizo Kanaguri \mydots \indexref{shizo kanaguri}\newline
Sitta \mydots \indexref{sitta}\newline
Skottfint \mydots \indexref{skottfint}\newline
Sportmössa \mydots \indexref{sportmoossa}\newline
Stjärtlapp \mydots \indexref{stjaertlapp}\newline
Stora Grabbars och Tjejers Märke \mydots \indexref{stora grabbars och tjejers maerke}\newline
Svälta räv \mydots \indexref{svaelta raev}\newline
The fat Spanish waiter \mydots \indexref{the fat spanish waiter}\newline
Thomas Wassberg \mydots \indexref{thomas wassberg}\newline
Torgny Mogren \mydots \indexref{torgny mogren}\newline
Trilobit \mydots \indexref{trilobit}\newline
Trollkull \mydots \indexref{trollkull}\newline
Uv-rugby \mydots \indexref{uv-rugby}\newline
Vladimir Krutov \mydots \indexref{vladimir krutov}\newline
X3m sports \mydots \indexref{x3m sports}\newline
\end{multicols}
~\newline
{\textbf{\large Svea rike}}
\begin{multicols}{2}
Blekinge \mydots \indexref{blekinge}\newline
Boden \mydots \indexref{boden}\newline
Bohuslän \mydots \indexref{bohuslaen}\newline
Den nordiska smuggeltriangeln \mydots \indexref{den nordiska smuggeltriangeln}\newline
För nit och redlighet i rikets tjänst \mydots \indexref{foor nit och redlighet i rikets tjaenst}\newline
Inititativ Anusmark \mydots \indexref{inititativ anusmark}\newline
Matematikmaskinnämnden \mydots \indexref{matematikmaskinnaemnden}\newline
Missbrukare \mydots \indexref{missbrukare}\newline
Norrbotten \mydots \indexref{norrbotten}\newline
Norrtälje \mydots \indexref{norrtaelje}\newline
Skåne \mydots \indexref{skaane}\newline
Skellefteå \mydots \indexref{skellefteaa}\newline
Umeå \mydots \indexref{umeaa}\newline
Uppland \mydots \indexref{uppland}\newline
Uppsala \mydots \indexref{uppsala}\newline
Västerbotten \mydots \indexref{vaesterbotten}\newline
\end{multicols}
~\newline
{\textbf{\large Sörland}}
\begin{multicols}{2}
Hällefors \mydots \indexref{haellefors}\newline
Rimbo \mydots \indexref{rimbo}\newline
\end{multicols}
~\newline
{\textbf{\large Television}}
\begin{multicols}{2}
Klasse Möllberg \mydots \indexref{klasse moollberg}\newline
Taggen \mydots \indexref{taggen}\newline
Television \mydots \indexref{television}\newline
Tjock-TV \mydots \indexref{tjock-tv}\newline
UR \mydots \indexref{ur}\newline
\end{multicols}
~\newline
{\textbf{\large Temafylla}}
\begin{multicols}{2}
Åka på safari \mydots \indexref{aaka paa safari}\newline
Åka vikingaskepp \mydots \indexref{aaka vikingaskepp}\newline
Alkoläskfylla \mydots \indexref{alkolaeskfylla}\newline
Bärsfylla \mydots \indexref{baersfylla}\newline
Belgisk öl \mydots \indexref{belgisk ool}\newline
Bilsupa \mydots \indexref{bilsupa}\newline
Blandfylla \mydots \indexref{blandfylla}\newline
Dagsfylla \mydots \indexref{dagsfylla}\newline
Folkölsfylla \mydots \indexref{folkoolsfylla}\newline
Ha bärs \mydots \indexref{ha baers}\newline
Luciavaka \mydots \indexref{luciavaka}\newline
Släktträffsberusning \mydots \indexref{slaekttraeffsberusning}\newline
SPAP \mydots \indexref{spap}\newline
Spritfylla \mydots \indexref{spritfylla}\newline
Supa ensam \mydots \indexref{supa ensam}\newline
\end{multicols}
~\newline
{\textbf{\large Tidlösa skämt}}
\begin{multicols}{2}
Driva med överheten \mydots \indexref{driva med ooverheten}\newline
Fisljud \mydots \indexref{fisljud}\newline
Klä ut sig till ett djur \mydots \indexref{klae ut sig till ett djur}\newline
Moona röv \mydots \indexref{moona roov}\newline
Prutta högljutt \mydots \indexref{prutta hoogljutt}\newline
Spegelmöte \mydots \indexref{spegelmoote}\newline
\end{multicols}
~\newline
{\textbf{\large Trafik}}
\begin{multicols}{2}
A-traktor \mydots \indexref{a-traktor}\newline
Angel of death \mydots \indexref{angel of death}\newline
Backa med släp \mydots \indexref{backa med slaep}\newline
Bil \mydots \indexref{bil}\newline
Blankt protokoll \mydots \indexref{blankt protokoll}\newline
Buss \mydots \indexref{buss}\newline
Buss 85 \mydots \indexref{buss 85}\newline
Busskruven \mydots \indexref{busskruven}\newline
Christianiacykel \mydots \indexref{christianiacykel}\newline
Dieselbil med lastgaller \mydots \indexref{dieselbil med lastgaller}\newline
E=mc2 \mydots \indexref{e=mc2}\newline
Eulalia II \mydots \indexref{eulalia ii}\newline
Fiacre \mydots \indexref{fiacre}\newline
Flerväxlad cykel \mydots \indexref{flervaexlad cykel}\newline
Flisbil \mydots \indexref{flisbil}\newline
Fyrhjulsdrift \mydots \indexref{fyrhjulsdrift}\newline
Gengas \mydots \indexref{gengas}\newline
Globetrotter \mydots \indexref{globetrotter}\newline
Handskfack \mydots \indexref{handskfack}\newline
Inre backspegel \mydots \indexref{inre backspegel}\newline
Irländsk parkering \mydots \indexref{irlaendsk parkering}\newline
Joseph Lucas \mydots \indexref{joseph lucas}\newline
Lobotomobil \mydots \indexref{lobotomobil}\newline
Malåparkering \mydots \indexref{malaaparkering}\newline
Miljöbil \mydots \indexref{miljoobil}\newline
Opel kadett \mydots \indexref{opel kadett}\newline
Orrkammens isolering och gokart \mydots \indexref{orrkammens isolering och gokart}\newline
Pizzaracer \mydots \indexref{pizzaracer}\newline
Putsbilar \mydots \indexref{putsbilar}\newline
Stig \mydots \indexref{stig}\newline
Svensk bilsemester \mydots \indexref{svensk bilsemester}\newline
Tegsnäsare \mydots \indexref{tegsnaesare}\newline
Volvo 740 \mydots \indexref{volvo 740}\newline
ZiL-fil \mydots \indexref{zil-fil}\newline
\end{multicols}
~\newline
{\textbf{\large Tyskland}}
\begin{multicols}{2}
Den tyska mustigheten \mydots \indexref{den tyska mustigheten}\newline
Friedrich Hegel \mydots \indexref{friedrich hegel}\newline
Läppar som prinskorv \mydots \indexref{laeppar som prinskorv}\newline
Tysk toalett \mydots \indexref{tysk toalett}\newline
Tyskland \mydots \indexref{tyskland}\newline
Wunderbaum \mydots \indexref{wunderbaum}\newline
\end{multicols}
~\newline
{\textbf{\large Umeå universitet}}
\begin{multicols}{2}
Arkivskita \mydots \indexref{arkivskita}\newline
\end{multicols}
~\newline
{\textbf{\large Underverk}}
\begin{multicols}{2}
Berghem HC \mydots \indexref{berghem hc}\newline
Brugdguldet \mydots \indexref{brugdguldet}\newline
Globen \mydots \indexref{globen}\newline
Kaknästornet \mydots \indexref{kaknaestornet}\newline
Sveriges sju underverk \mydots \indexref{sveriges sju underverk}\newline
Världens näst största byggnad \mydots \indexref{vaerldens naest stoorsta byggnad}\newline
\end{multicols}
~\newline
{\textbf{\large Veckodagar}}
\begin{multicols}{2}
Fredag \mydots \indexref{fredag}\newline
Lördag \mydots \indexref{loordag}\newline
Måndag \mydots \indexref{maandag}\newline
Söndag \mydots \indexref{soondag}\newline
Tisdag \mydots \indexref{tisdag}\newline
Torsdagar \mydots \indexref{torsdagar}\newline
\end{multicols}
~\newline
{\textbf{\large Vetenskaper}}
\begin{multicols}{2}
Akademiker \mydots \indexref{akademiker}\newline
Armlängd \mydots \indexref{armlaengd}\newline
Astronomi \mydots \indexref{astronomi}\newline
Atlantica \mydots \indexref{atlantica}\newline
Dansk forskning \mydots \indexref{dansk forskning}\newline
Den ambitiösa studenten \mydots \indexref{den ambitioosa studenten}\newline
Dialektik \mydots \indexref{dialektik}\newline
Djurens nobelpris \mydots \indexref{djurens nobelpris}\newline
Flatologi \mydots \indexref{flatologi}\newline
Folkölsförädling \mydots \indexref{folkoolsfooraedling}\newline
Forskningsinstitut i Schweiz \mydots \indexref{forskningsinstitut i schweiz}\newline
Glädjevetenskaper \mydots \indexref{glaedjevetenskaper}\newline
Grafologi \mydots \indexref{grafologi}\newline
Källkritik \mydots \indexref{kaellkritik}\newline
Kattguld \mydots \indexref{kattguld}\newline
Lobotomobil \mydots \indexref{lobotomobil}\newline
Naturens dialektik \mydots \indexref{naturens dialektik}\newline
Naturhistoriska museet \mydots \indexref{naturhistoriska museet}\newline
Nissepedia \mydots \indexref{nissepedia}\newline
Paleontologi \mydots \indexref{paleontologi}\newline
Patentkork \mydots \indexref{patentkork}\newline
Perspektiv \mydots \indexref{perspektiv}\newline
Postkolonialism \mydots \indexref{postkolonialism}\newline
Postseminarium \mydots \indexref{postseminarium}\newline
Professor skytteanus \mydots \indexref{professor skytteanus}\newline
Titta på ord \mydots \indexref{titta paa ord}\newline
Traditionell finsk medicin \mydots \indexref{traditionell finsk medicin}\newline
Traditionell kinesisk medicin \mydots \indexref{traditionell kinesisk medicin}\newline
Turtlestestet \mydots \indexref{turtlestestet}\newline
Viskositet \mydots \indexref{viskositet}\newline
\end{multicols}
~\newline
{\textbf{\large Vägbeläggningar}}
\begin{multicols}{2}
Berguv \mydots \indexref{berguv}\newline
Sand \mydots \indexref{sand}\newline
Slagg \mydots \indexref{slagg}\newline
Träd, Gräs och Stenar \mydots \indexref{traed, graes och stenar}\newline
\end{multicols}
~\newline
{\textbf{\large Växtriket}}
\begin{multicols}{2}
Beskinnad \mydots \indexref{beskinnad}\newline
Carl von Linné \mydots \indexref{carl von linné}\newline
Hundkäx \mydots \indexref{hundkaex}\newline
Slentrian \mydots \indexref{slentrian}\newline
Träd, Gräs och Stenar \mydots \indexref{traed, graes och stenar}\newline
Wunderbaum \mydots \indexref{wunderbaum}\newline
\end{multicols}
~\newline
{\textbf{\large Vår värld}}
\begin{multicols}{2}
Ålidhem \mydots \indexref{aalidhem}\newline
Ånäset \mydots \indexref{aanaeset}\newline
Abdera \mydots \indexref{abdera}\newline
Adak \mydots \indexref{adak}\newline
Asbest \mydots \indexref{asbest}\newline
Australien \mydots \indexref{australien}\newline
Axtorpet \mydots \indexref{axtorpet}\newline
Bajsalåkta \mydots \indexref{bajsalaakta}\newline
Barbados \mydots \indexref{barbados}\newline
Belgien \mydots \indexref{belgien}\newline
Bergslagen \mydots \indexref{bergslagen}\newline
Boden \mydots \indexref{boden}\newline
Bohuslän \mydots \indexref{bohuslaen}\newline
Danmark \mydots \indexref{danmark}\newline
Dannebrogen \mydots \indexref{dannebrogen}\newline
Dansk lättöl \mydots \indexref{dansk laettool}\newline
England \mydots \indexref{england}\newline
Evert Taubes värld \mydots \indexref{evert taubes vaerld}\newline
Fagersta \mydots \indexref{fagersta}\newline
Finland \mydots \indexref{finland}\newline
Fridhem \mydots \indexref{fridhem}\newline
Fyn \mydots \indexref{fyn}\newline
Göteborg \mydots \indexref{gooteborg}\newline
Håbroa \mydots \indexref{haabroa}\newline
Häcklefjäll \mydots \indexref{haecklefjaell}\newline
Halland \mydots \indexref{halland}\newline
Hertsökullar \mydots \indexref{hertsookullar}\newline
Holmsunds tropikhus \mydots \indexref{holmsunds tropikhus}\newline
Hudiksvall \mydots \indexref{hudiksvall}\newline
Incitatus \mydots \indexref{incitatus}\newline
Inititativ Anusmark \mydots \indexref{inititativ anusmark}\newline
Isebrogen \mydots \indexref{isebrogen}\newline
Island \mydots \indexref{island}\newline
Japan \mydots \indexref{japan}\newline
Jordbruksort \mydots \indexref{jordbruksort}\newline
Kaknästornet \mydots \indexref{kaknaestornet}\newline
Köping \mydots \indexref{kooping}\newline
Korv-Ivars \mydots \indexref{korv-ivars}\newline
Libanon \mydots \indexref{libanon}\newline
Lule \mydots \indexref{lule}\newline
Malå \mydots \indexref{malaa}\newline
Malårca \mydots \indexref{malaarca}\newline
Mount Everest \mydots \indexref{mount everest}\newline
Naturhistoriska museet \mydots \indexref{naturhistoriska museet}\newline
Nigeria \mydots \indexref{nigeria}\newline
Norge \mydots \indexref{norge}\newline
Norrbotten \mydots \indexref{norrbotten}\newline
Norrtälje \mydots \indexref{norrtaelje}\newline
Norsjöblicken \mydots \indexref{norsjooblicken}\newline
Nya Zeeland \mydots \indexref{nya zeeland}\newline
Örnäset \mydots \indexref{oornaeset}\newline
Ornässtugans dass \mydots \indexref{ornaesstugans dass}\newline
Peru \mydots \indexref{peru}\newline
Piteå \mydots \indexref{piteaa}\newline
Rökå \mydots \indexref{rookaa}\newline
Rött \mydots \indexref{roott}\newline
Ryckepungvägen \mydots \indexref{ryckepungvaegen}\newline
Sågverk \mydots \indexref{saagverk}\newline
Skåne \mydots \indexref{skaane}\newline
Skagen \mydots \indexref{skagen}\newline
Skellefteå \mydots \indexref{skellefteaa}\newline
Skinnskatteberg \mydots \indexref{skinnskatteberg}\newline
Slätt \mydots \indexref{slaett}\newline
Spanien \mydots \indexref{spanien}\newline
Spanskt lättvin \mydots \indexref{spanskt laettvin}\newline
Stockholm \mydots \indexref{stockholm}\newline
Storsien \mydots \indexref{storsien}\newline
Sverige \mydots \indexref{sverige}\newline
Sylt \mydots \indexref{sylt}\newline
Tengah \mydots \indexref{tengah}\newline
Tjamstan \mydots \indexref{tjamstan}\newline
Tyskland \mydots \indexref{tyskland}\newline
Umeå \mydots \indexref{umeaa}\newline
United States of America \mydots \indexref{united states of america}\newline
Universitet \mydots \indexref{universitet}\newline
Uppland \mydots \indexref{uppland}\newline
Uppsala \mydots \indexref{uppsala}\newline
Vänort \mydots \indexref{vaenort}\newline
Världens näst största byggnad \mydots \indexref{vaerldens naest stoorsta byggnad}\newline
Västerbotten \mydots \indexref{vaesterbotten}\newline
Växjö \mydots \indexref{vaexjoo}\newline
\end{multicols}
~\newline
{\textbf{\large Yrken}}
\begin{multicols}{2}
Akademiker \mydots \indexref{akademiker}\newline
Barnuppfostran \mydots \indexref{barnuppfostran}\newline
Basist i refused \mydots \indexref{basist i refused}\newline
Biologer \mydots \indexref{biologer}\newline
Bönder \mydots \indexref{boonder}\newline
Cirkusdirektör \mydots \indexref{cirkusdirektoor}\newline
Civilpolis \mydots \indexref{civilpolis}\newline
Folkhjälte \mydots \indexref{folkhjaelte}\newline
Förståsigpåare \mydots \indexref{foorstaasigpaaare}\newline
Frilans \mydots \indexref{frilans}\newline
Göra rätt för sig \mydots \indexref{goora raett foor sig}\newline
Juridisk rådgivare \mydots \indexref{juridisk raadgivare}\newline
Krokodiljägare \mydots \indexref{krokodiljaegare}\newline
Kulturarbetare \mydots \indexref{kulturarbetare}\newline
Lärare \mydots \indexref{laerare}\newline
Mångsysslarpensionär \mydots \indexref{maangsysslarpensionaer}\newline
Mördare \mydots \indexref{moordare}\newline
Oljegark \mydots \indexref{oljegark}\newline
Perversa elektriker \mydots \indexref{perversa elektriker}\newline
Privatspanare \mydots \indexref{privatspanare}\newline
Professor skytteanus \mydots \indexref{professor skytteanus}\newline
Riddare \mydots \indexref{riddare}\newline
Småskurk \mydots \indexref{smaaskurk}\newline
Sommarplågsmusiker \mydots \indexref{sommarplaagsmusiker}\newline
Sopletare \mydots \indexref{sopletare}\newline
Storfräsare \mydots \indexref{storfraesare}\newline
Stråtrövare \mydots \indexref{straatroovare}\newline
Syo \mydots \indexref{syo}\newline
\end{multicols}
~\newline
{\textbf{\large Årstider}}
\begin{multicols}{2}
Sommar \mydots \indexref{sommar}\newline
\end{multicols}
~\newline
{\textbf{\large Överlevnadsknep}}
\begin{multicols}{2}
Åkarbrasa \mydots \indexref{aakarbrasa}\newline
Ättestupa \mydots \indexref{aettestupa}\newline
Alkohol \mydots \indexref{alkohol}\newline
Arkivskita \mydots \indexref{arkivskita}\newline
Bärsfylla \mydots \indexref{baersfylla}\newline
Bensträckare \mydots \indexref{benstraeckare}\newline
Bjuddosa \mydots \indexref{bjuddosa}\newline
Blandfylla \mydots \indexref{blandfylla}\newline
Bröka \mydots \indexref{brooka}\newline
Brunka \mydots \indexref{brunka}\newline
Cacao creme \mydots \indexref{cacao creme}\newline
Calzona \mydots \indexref{calzona}\newline
Cheeseburga \mydots \indexref{cheeseburga}\newline
Dagsedel \mydots \indexref{dagsedel}\newline
Dagsfylla \mydots \indexref{dagsfylla}\newline
Dansk lättöl \mydots \indexref{dansk laettool}\newline
Danskt penicillin \mydots \indexref{danskt penicillin}\newline
Det gamla Silvio Berlusconi-knepet \mydots \indexref{det gamla silvio berlusconi-knepet}\newline
Det gamla Thore Skogman-knepet \mydots \indexref{det gamla thore skogman-knepet}\newline
Feministknepet \mydots \indexref{feministknepet}\newline
Fiska kräfta med ficklampa \mydots \indexref{fiska kraefta med ficklampa}\newline
Flaka \mydots \indexref{flaka}\newline
Flytväst \mydots \indexref{flytvaest}\newline
Folkölsfylla \mydots \indexref{folkoolsfylla}\newline
Foucaultfingret \mydots \indexref{foucaultfingret}\newline
Gå och snickra \mydots \indexref{gaa och snickra}\newline
Gore-Tex \mydots \indexref{gore-tex}\newline
Grekiska statsobligationer \mydots \indexref{grekiska statsobligationer}\newline
Grisfull \mydots \indexref{grisfull}\newline
Grunka \mydots \indexref{grunka}\newline
Gubbsova \mydots \indexref{gubbsova}\newline
Gylfa \mydots \indexref{gylfa}\newline
Ha bärs \mydots \indexref{ha baers}\newline
Helgarderat efterliv \mydots \indexref{helgarderat efterliv}\newline
Helgvolym \mydots \indexref{helgvolym}\newline
Jag ska bara bli full först \mydots \indexref{jag ska bara bli full foorst}\newline
Mango safe \mydots \indexref{mango safe}\newline
Mjukpack cigg \mydots \indexref{mjukpack cigg}\newline
När livet blir alldeles för mycket - Prof. Etiennes bästa gömställen, i urval \mydots \indexref{naer livet blir alldeles foor mycket - prof. etiennes baesta goomstaellen, i urval}\newline
På fat \mydots \indexref{paa fat}\newline
Processa mot länsstyrelsen \mydots \indexref{processa mot laensstyrelsen}\newline
Prokrastrinering \mydots \indexref{prokrastrinering}\newline
Prunka \mydots \indexref{prunka}\newline
Pudaslåda \mydots \indexref{pudaslaada}\newline
Rygga \mydots \indexref{rygga}\newline
Sittsova \mydots \indexref{sittsova}\newline
Skrunka \mydots \indexref{skrunka}\newline
Skuggan \mydots \indexref{skuggan}\newline
Sopletare \mydots \indexref{sopletare}\newline
Spärrballong \mydots \indexref{spaerrballong}\newline
Spanskt lättvin \mydots \indexref{spanskt laettvin}\newline
SPAP \mydots \indexref{spap}\newline
Spegel \mydots \indexref{spegel}\newline
Spunka \mydots \indexref{spunka}\newline
Stödkorv \mydots \indexref{stoodkorv}\newline
Svälta räv \mydots \indexref{svaelta raev}\newline
Tornedalslåset \mydots \indexref{tornedalslaaset}\newline
Trotta \mydots \indexref{trotta}\newline
Tuff-frysa \mydots \indexref{tuff-frysa}\newline
Vilskita \mydots \indexref{vilskita}\newline
Vinfylla \mydots \indexref{vinfylla}\newline
\end{multicols}
\clearpage
\begin{multicols}{2}
\end{multicols}
{\huge{\textbf{1}}}
\begin{multicols}{2}
13 \mydots \indexref{13}\newline
\end{multicols}
{\huge{\textbf{5}}}
\begin{multicols}{2}
50 million piece of shit \mydots \indexref{50 million piece of shit}\newline
\end{multicols}
{\huge{\textbf{A}}}
\begin{multicols}{2}
A-traktor \mydots \indexref{a-traktor}\newline
Abdera \mydots \indexref{abdera}\newline
Abu Garcia \mydots \indexref{abu garcia}\newline
AC/DC-gitarr \mydots \indexref{acdc-gitarr}\newline
Adak \mydots \indexref{adak}\newline
Adde Malmberg \mydots \indexref{adde malmberg}\newline
Aforismer \mydots \indexref{aforismer}\newline
Akademiker \mydots \indexref{akademiker}\newline
Aktersegla \mydots \indexref{aktersegla}\newline
Albert II \mydots \indexref{albert ii}\newline
Albin \mydots \indexref{albin}\newline
Aleksandr Karelin \mydots \indexref{aleksandr karelin}\newline
Ales Stenar \mydots \indexref{ales stenar}\newline
Alice Tegnér \mydots \indexref{alice tegnér}\newline
Alkisschäfer \mydots \indexref{alkisschaefer}\newline
Alkohol \mydots \indexref{alkohol}\newline
Alkoläskfylla \mydots \indexref{alkolaeskfylla}\newline
Allergi \mydots \indexref{allergi}\newline
Allting \mydots \indexref{allting}\newline
Allväderstövlar \mydots \indexref{allvaederstoovlar}\newline
Alternativa namn på bakverk \mydots \indexref{alternativa namn paa bakverk}\newline
Alvparty \mydots \indexref{alvparty}\newline
Ambigram \mydots \indexref{ambigram}\newline
Amebix \mydots \indexref{amebix}\newline
Ana uvar i mossen \mydots \indexref{ana uvar i mossen}\newline
Andörjan \mydots \indexref{andoorjan}\newline
Angel of death \mydots \indexref{angel of death}\newline
Anglosax \mydots \indexref{anglosax}\newline
Ankeborgslagstiftning \mydots \indexref{ankeborgslagstiftning}\newline
Ankfot \mydots \indexref{ankfot}\newline
Annie Lööf \mydots \indexref{annie loooof}\newline
Annika \mydots \indexref{annika}\newline
Ansiktsmålning \mydots \indexref{ansiktsmaalning}\newline
Anti-speciesism \mydots \indexref{anti-speciesism}\newline
Antikvärde \mydots \indexref{antikvaerde}\newline
Antilop \mydots \indexref{antilop}\newline
Antiutilitarism \mydots \indexref{antiutilitarism}\newline
Anton Abele \mydots \indexref{anton abele}\newline
Anton Lavey \mydots \indexref{anton lavey}\newline
Apor vi minns \mydots \indexref{apor vi minns}\newline
Arbetarblåsa \mydots \indexref{arbetarblaasa}\newline
Arbetarklassrock \mydots \indexref{arbetarklassrock}\newline
Arbetslinjen \mydots \indexref{arbetslinjen}\newline
Arbetsplatskamp \mydots \indexref{arbetsplatskamp}\newline
Arggissa \mydots \indexref{arggissa}\newline
Arkivskita \mydots \indexref{arkivskita}\newline
Armlängd \mydots \indexref{armlaengd}\newline
Aron Jonason \mydots \indexref{aron jonason}\newline
Arselhaka \mydots \indexref{arselhaka}\newline
Artur Hazelius \mydots \indexref{artur hazelius}\newline
Asbest \mydots \indexref{asbest}\newline
ASEA-grönt \mydots \indexref{asea-groont}\newline
Astronomi \mydots \indexref{astronomi}\newline
Atlantica \mydots \indexref{atlantica}\newline
Att psykedelisera sin vardag \mydots \indexref{att psykedelisera sin vardag}\newline
Aurora \mydots \indexref{aurora}\newline
Australien \mydots \indexref{australien}\newline
Avbolagisering \mydots \indexref{avbolagisering}\newline
Axe \mydots \indexref{axe}\newline
Axess tv \mydots \indexref{axess tv}\newline
Axtorpet \mydots \indexref{axtorpet}\newline
Ayn Rand \mydots \indexref{ayn rand}\newline
\end{multicols}
{\huge{\textbf{B}}}
\begin{multicols}{2}
Backa med släp \mydots \indexref{backa med slaep}\newline
Backa om \mydots \indexref{backa om}\newline
Backpatch \mydots \indexref{backpatch}\newline
Bacon \mydots \indexref{bacon}\newline
Bade \mydots \indexref{bade}\newline
Bailando \mydots \indexref{bailando}\newline
Bajsalåkta \mydots \indexref{bajsalaakta}\newline
Bakficka \mydots \indexref{bakficka}\newline
Bakisångest \mydots \indexref{bakisaangest}\newline
Balticgruppen \mydots \indexref{balticgruppen}\newline
Balutägg \mydots \indexref{balutaegg}\newline
Banan \mydots \indexref{banan}\newline
Bananas \mydots \indexref{bananas}\newline
Bar överkropp \mydots \indexref{bar ooverkropp}\newline
Barbados \mydots \indexref{barbados}\newline
Barn \mydots \indexref{barn}\newline
Barnagans förträffliga pedagogik \mydots \indexref{barnagans foortraeffliga pedagogik}\newline
Barndom \mydots \indexref{barndom}\newline
Barnuppfostran \mydots \indexref{barnuppfostran}\newline
Bartolomaios \mydots \indexref{bartolomaios}\newline
Basist i refused \mydots \indexref{basist i refused}\newline
Batyskaf \mydots \indexref{batyskaf}\newline
Beat \mydots \indexref{beat}\newline
Belgien \mydots \indexref{belgien}\newline
Belgisk jättekanin \mydots \indexref{belgisk jaettekanin}\newline
Belgisk öl \mydots \indexref{belgisk ool}\newline
Belka \mydots \indexref{belka}\newline
Bellman \mydots \indexref{bellman}\newline
Benny Bus \mydots \indexref{benny bus}\newline
Bensträckare \mydots \indexref{benstraeckare}\newline
Berghem HC \mydots \indexref{berghem hc}\newline
Bergslagen \mydots \indexref{bergslagen}\newline
Berguv \mydots \indexref{berguv}\newline
Berätta \mydots \indexref{beraetta}\newline
Beskinnad \mydots \indexref{beskinnad}\newline
Besvikelse \mydots \indexref{besvikelse}\newline
Bibeln \mydots \indexref{bibeln}\newline
Bil \mydots \indexref{bil}\newline
Bilbatteri \mydots \indexref{bilbatteri}\newline
Bildekal \mydots \indexref{bildekal}\newline
Bilprovningen \mydots \indexref{bilprovningen}\newline
Bilsupa \mydots \indexref{bilsupa}\newline
Bintje \mydots \indexref{bintje}\newline
Biologer \mydots \indexref{biologer}\newline
Bitterhet \mydots \indexref{bitterhet}\newline
Bjuddosa \mydots \indexref{bjuddosa}\newline
Bjudsprit \mydots \indexref{bjudsprit}\newline
Bjudtermos \mydots \indexref{bjudtermos}\newline
Björn (djur) \mydots \indexref{bjoorn (djur)}\newline
Blandfylla \mydots \indexref{blandfylla}\newline
Blandsvulst \mydots \indexref{blandsvulst}\newline
Blankt protokoll \mydots \indexref{blankt protokoll}\newline
Blaze Baylika \mydots \indexref{blaze baylika}\newline
Blekinge \mydots \indexref{blekinge}\newline
Blomkålsöra \mydots \indexref{blomkaalsoora}\newline
Blottare \mydots \indexref{blottare}\newline
Blåval \mydots \indexref{blaaval}\newline
Blåvitt \mydots \indexref{blaavitt}\newline
Bockskäggsmetal \mydots \indexref{bockskaeggsmetal}\newline
Boden \mydots \indexref{boden}\newline
Boetius de Dacia \mydots \indexref{boetius de dacia}\newline
Bohuslän \mydots \indexref{bohuslaen}\newline
Bonad \mydots \indexref{bonad}\newline
Bonfire \mydots \indexref{bonfire}\newline
Bonnseg \mydots \indexref{bonnseg}\newline
Boris Jeltsin \mydots \indexref{boris jeltsin}\newline
Bortamatch \mydots \indexref{bortamatch}\newline
Bosporen \mydots \indexref{bosporen}\newline
Botte \mydots \indexref{botte}\newline
Brakflopp \mydots \indexref{brakflopp}\newline
Brandklipparen \mydots \indexref{brandklipparen}\newline
Brevlåda \mydots \indexref{brevlaada}\newline
Brian Epstein \mydots \indexref{brian epstein}\newline
Bricka \mydots \indexref{bricka}\newline
Brinner för att sälja \mydots \indexref{brinner foor att saelja}\newline
Brismonstret \mydots \indexref{brismonstret}\newline
Britts mode \mydots \indexref{britts mode}\newline
Brobo \mydots \indexref{brobo}\newline
Brokiga Blad \mydots \indexref{brokiga blad}\newline
Brugd \mydots \indexref{brugd}\newline
Brugdguldet \mydots \indexref{brugdguldet}\newline
Bruksgök \mydots \indexref{bruksgook}\newline
Bruksortskäng \mydots \indexref{bruksortskaeng}\newline
Brumma \mydots \indexref{brumma}\newline
Brunka \mydots \indexref{brunka}\newline
Bruvd \mydots \indexref{bruvd}\newline
Brännvin \mydots \indexref{braennvin}\newline
Bröka \mydots \indexref{brooka}\newline
Bröstarkt \mydots \indexref{broostarkt}\newline
Bukfylla \mydots \indexref{bukfylla}\newline
Bull \mydots \indexref{bull}\newline
Burre \mydots \indexref{burre}\newline
Buss \mydots \indexref{buss}\newline
Buss 85 \mydots \indexref{buss 85}\newline
Busskruven \mydots \indexref{busskruven}\newline
Bärsfylla \mydots \indexref{baersfylla}\newline
Bönder \mydots \indexref{boonder}\newline
\end{multicols}
{\huge{\textbf{C}}}
\begin{multicols}{2}
Cacao creme \mydots \indexref{cacao creme}\newline
Calskrove \mydots \indexref{calskrove}\newline
Calzona \mydots \indexref{calzona}\newline
Carl von Linné \mydots \indexref{carl von linné}\newline
Carlshöjdare \mydots \indexref{carlshoojdare}\newline
Carpe diem \mydots \indexref{carpe diem}\newline
CCM \mydots \indexref{ccm}\newline
Centerpartiet \mydots \indexref{centerpartiet}\newline
Champis \mydots \indexref{champis}\newline
Chapeau de paysan \mydots \indexref{chapeau de paysan}\newline
Charles Manson \mydots \indexref{charles manson}\newline
Charlotte \mydots \indexref{charlotte}\newline
Cheeseburga \mydots \indexref{cheeseburga}\newline
Chef \mydots \indexref{chef}\newline
Chemtrails \mydots \indexref{chemtrails}\newline
Chipslåda \mydots \indexref{chipslaada}\newline
Christer Sandelin \mydots \indexref{christer sandelin}\newline
Christianiacykel \mydots \indexref{christianiacykel}\newline
Cigg \mydots \indexref{cigg}\newline
Cirkusdirektör \mydots \indexref{cirkusdirektoor}\newline
Cirkuspung \mydots \indexref{cirkuspung}\newline
Civilpolis \mydots \indexref{civilpolis}\newline
Colin Nutley \mydots \indexref{colin nutley}\newline
Conny \mydots \indexref{conny}\newline
Corporate social responsibility \mydots \indexref{corporate social responsibility}\newline
CQD \mydots \indexref{cqd}\newline
Crass \mydots \indexref{crass}\newline
Crustare \mydots \indexref{crustare}\newline
Crustknytning \mydots \indexref{crustknytning}\newline
Crustpippi \mydots \indexref{crustpippi}\newline
Currykondom \mydots \indexref{currykondom}\newline
Cykelhjälm \mydots \indexref{cykelhjaelm}\newline
\end{multicols}
{\huge{\textbf{D}}}
\begin{multicols}{2}
Dackefejden \mydots \indexref{dackefejden}\newline
Dagens Nyheter \mydots \indexref{dagens nyheter}\newline
Dagsedel \mydots \indexref{dagsedel}\newline
Dagsfylla \mydots \indexref{dagsfylla}\newline
Dagvill \mydots \indexref{dagvill}\newline
Dank \mydots \indexref{dank}\newline
Danmark \mydots \indexref{danmark}\newline
Dannebrogen \mydots \indexref{dannebrogen}\newline
Dansk advent \mydots \indexref{dansk advent}\newline
Dansk forskning \mydots \indexref{dansk forskning}\newline
Dansk jul \mydots \indexref{dansk jul}\newline
Dansk kanot \mydots \indexref{dansk kanot}\newline
Dansk kortlek \mydots \indexref{dansk kortlek}\newline
Dansk kostcirkel \mydots \indexref{dansk kostcirkel}\newline
Dansk lättöl \mydots \indexref{dansk laettool}\newline
Dansk midsommar \mydots \indexref{dansk midsommar}\newline
Dansk onsdag \mydots \indexref{dansk onsdag}\newline
Dansk påsk \mydots \indexref{dansk paask}\newline
Dansk sax \mydots \indexref{dansk sax}\newline
Dansk semester \mydots \indexref{dansk semester}\newline
Dansk skalle \mydots \indexref{dansk skalle}\newline
Dansk skrock \mydots \indexref{dansk skrock}\newline
Dansk tubkikare \mydots \indexref{dansk tubkikare}\newline
Danska hedersbetygelser \mydots \indexref{danska hedersbetygelser}\newline
Danskt penicillin \mydots \indexref{danskt penicillin}\newline
De förlösande thinneråren \mydots \indexref{de foorloosande thinneraaren}\newline
De gamla grekerna \mydots \indexref{de gamla grekerna}\newline
Deadhead \mydots \indexref{deadhead}\newline
Delfin \mydots \indexref{delfin}\newline
Delfinapa \mydots \indexref{delfinapa}\newline
Den ambitiösa studenten \mydots \indexref{den ambitioosa studenten}\newline
Den arga groggen \mydots \indexref{den arga groggen}\newline
Den lilla boken \mydots \indexref{den lilla boken}\newline
Den nordiska smuggeltriangeln \mydots \indexref{den nordiska smuggeltriangeln}\newline
Den nya sångaren \mydots \indexref{den nya saangaren}\newline
Den oambitiösa studenten \mydots \indexref{den oambitioosa studenten}\newline
Den tyska mustigheten \mydots \indexref{den tyska mustigheten}\newline
Den vedervärdige mannen från Säffle \mydots \indexref{den vedervaerdige mannen fraan saeffle}\newline
Dennis \mydots \indexref{dennis}\newline
Deportees-trevlig \mydots \indexref{deportees-trevlig}\newline
Der er et yndigt land \mydots \indexref{der er et yndigt land}\newline
Det gamla Silvio Berlusconi-knepet \mydots \indexref{det gamla silvio berlusconi-knepet}\newline
Det gamla Thore Skogman-knepet \mydots \indexref{det gamla thore skogman-knepet}\newline
Det omedvetna \mydots \indexref{det omedvetna}\newline
Det politiska i C.C.Rs texter \mydots \indexref{det politiska i c.c.rs texter}\newline
Det stora fosterländska kriget \mydots \indexref{det stora fosterlaendska kriget}\newline
Det stora vadet 2013 \mydots \indexref{det stora vadet 2013}\newline
Det susar i Säfve \mydots \indexref{det susar i saefve}\newline
Diagram \mydots \indexref{diagram}\newline
Dialektik \mydots \indexref{dialektik}\newline
Dieselbil med lastgaller \mydots \indexref{dieselbil med lastgaller}\newline
Dimma \mydots \indexref{dimma}\newline
Din avkomma och du \mydots \indexref{din avkomma och du}\newline
Diskett \mydots \indexref{diskett}\newline
Djur \mydots \indexref{djur}\newline
Djurens nobelpris \mydots \indexref{djurens nobelpris}\newline
Dojo \mydots \indexref{dojo}\newline
Doktorand \mydots \indexref{doktorand}\newline
Dokumentärhora \mydots \indexref{dokumentaerhora}\newline
Dopesmoker \mydots \indexref{dopesmoker}\newline
Dr. Alban \mydots \indexref{dr. alban}\newline
Dragbasun \mydots \indexref{dragbasun}\newline
Dragsko \mydots \indexref{dragsko}\newline
Dragspel \mydots \indexref{dragspel}\newline
Dragspelsmuskeln \mydots \indexref{dragspelsmuskeln}\newline
Drakonisk lag \mydots \indexref{drakonisk lag}\newline
Dressmann \mydots \indexref{dressmann}\newline
Driva med överheten \mydots \indexref{driva med ooverheten}\newline
Dubbelsovla \mydots \indexref{dubbelsovla}\newline
Duvgubbar \mydots \indexref{duvgubbar}\newline
Dvärgpungsovare \mydots \indexref{dvaergpungsovare}\newline
Dyckert \mydots \indexref{dyckert}\newline
Däcka \mydots \indexref{daecka}\newline
Dävert \mydots \indexref{daevert}\newline
Dävertspotting \mydots \indexref{daevertspotting}\newline
\end{multicols}
{\huge{\textbf{E}}}
\begin{multicols}{2}
E=mc2 \mydots \indexref{e=mc2}\newline
Ebbe \mydots \indexref{ebbe}\newline
Edmund \mydots \indexref{edmund}\newline
Edmund Hillary \mydots \indexref{edmund hillary}\newline
Edvin \mydots \indexref{edvin}\newline
Egendom \mydots \indexref{egendom}\newline
En bärs, en bärs, min järndanksamling för en bärs \mydots \indexref{en baers, en baers, min jaerndanksamling foor en baers}\newline
Ekfors Kraft \mydots \indexref{ekfors kraft}\newline
Elva \mydots \indexref{elva}\newline
\end{multicols}
{\huge{\textbf{É}}}
\begin{multicols}{2}
Émile Durkheim \mydots \indexref{émile durkheim}\newline
\end{multicols}
{\huge{\textbf{E}}}
\begin{multicols}{2}
England \mydots \indexref{england}\newline
Enkelbeckasin \mydots \indexref{enkelbeckasin}\newline
Enkido \mydots \indexref{enkido}\newline
Entreprenör \mydots \indexref{entreprenoor}\newline
Epikurism \mydots \indexref{epikurism}\newline
Erik Hamrén \mydots \indexref{erik hamrén}\newline
Erik Homburger Erikson \mydots \indexref{erik homburger erikson}\newline
Erna \mydots \indexref{erna}\newline
Ernst Billgren \mydots \indexref{ernst billgren}\newline
Ernst Haeckel \mydots \indexref{ernst haeckel}\newline
Eskimåspov \mydots \indexref{eskimaaspov}\newline
Ett kille \mydots \indexref{ett kille}\newline
Etta \mydots \indexref{etta}\newline
Eulalia II \mydots \indexref{eulalia ii}\newline
Eva Ekeblad \mydots \indexref{eva ekeblad}\newline
Evert Taube \mydots \indexref{evert taube}\newline
Evert Taubes värld \mydots \indexref{evert taubes vaerld}\newline
\end{multicols}
{\huge{\textbf{F}}}
\begin{multicols}{2}
Facebook \mydots \indexref{facebook}\newline
Fagersta \mydots \indexref{fagersta}\newline
Fagersta-Posten \mydots \indexref{fagersta-posten}\newline
Fakta \mydots \indexref{fakta}\newline
Faktoid \mydots \indexref{faktoid}\newline
Falafel \mydots \indexref{falafel}\newline
False metal \mydots \indexref{false metal}\newline
Fasta nycklar \mydots \indexref{fasta nycklar}\newline
Fax \mydots \indexref{fax}\newline
Fejkspons \mydots \indexref{fejkspons}\newline
Feliks Dzerzjinskij \mydots \indexref{feliks dzerzjinskij}\newline
Feminism \mydots \indexref{feminism}\newline
Feministknepet \mydots \indexref{feministknepet}\newline
Femma \mydots \indexref{femma}\newline
Femtusen invånare-regeln \mydots \indexref{femtusen invaanare-regeln}\newline
Fet och grisig mat döpt efter lyxiga ställen/personer \mydots \indexref{fet och grisig mat doopt efter lyxiga staellenpersoner}\newline
Fetma \mydots \indexref{fetma}\newline
Fetor ex ore \mydots \indexref{fetor ex ore}\newline
FFSSB \mydots \indexref{ffssb}\newline
Fiacre \mydots \indexref{fiacre}\newline
Filateli \mydots \indexref{filateli}\newline
Filipinsk apörn \mydots \indexref{filipinsk apoorn}\newline
Finland \mydots \indexref{finland}\newline
Finljuga \mydots \indexref{finljuga}\newline
Finsk inställning till rock \mydots \indexref{finsk instaellning till rock}\newline
Finsk pappersbruksarbetarkraut \mydots \indexref{finsk pappersbruksarbetarkraut}\newline
Finsk sommarsoppa \mydots \indexref{finsk sommarsoppa}\newline
Finskt pannband \mydots \indexref{finskt pannband}\newline
Fiska kräfta med ficklampa \mydots \indexref{fiska kraefta med ficklampa}\newline
Fiskeredskapsaffär \mydots \indexref{fiskeredskapsaffaer}\newline
Fisljud \mydots \indexref{fisljud}\newline
Fixed gear metal \mydots \indexref{fixed gear metal}\newline
Flaka \mydots \indexref{flaka}\newline
Flanera \mydots \indexref{flanera}\newline
Flatologi \mydots \indexref{flatologi}\newline
Flerväxlad cykel \mydots \indexref{flervaexlad cykel}\newline
Flisbil \mydots \indexref{flisbil}\newline
Flodkanin \mydots \indexref{flodkanin}\newline
Flundra \mydots \indexref{flundra}\newline
Flytväst \mydots \indexref{flytvaest}\newline
Fläsksvålar \mydots \indexref{flaesksvaalar}\newline
Flöjt \mydots \indexref{floojt}\newline
Flöjtfodral \mydots \indexref{floojtfodral}\newline
Fnysning \mydots \indexref{fnysning}\newline
Folk födda före 1970 \mydots \indexref{folk foodda foore 1970}\newline
Folke Pudas \mydots \indexref{folke pudas}\newline
Folketinget \mydots \indexref{folketinget}\newline
Folkhjälte \mydots \indexref{folkhjaelte}\newline
Folkkök \mydots \indexref{folkkook}\newline
Folkpartiet \mydots \indexref{folkpartiet}\newline
Folkölsfylla \mydots \indexref{folkoolsfylla}\newline
Folkölsförädling \mydots \indexref{folkoolsfooraedling}\newline
Forskningsinstitut i Schweiz \mydots \indexref{forskningsinstitut i schweiz}\newline
Fotboll \mydots \indexref{fotboll}\newline
Fotbollspundare \mydots \indexref{fotbollspundare}\newline
Fotografering \mydots \indexref{fotografering}\newline
Foucaultfingret \mydots \indexref{foucaultfingret}\newline
Framtiden \mydots \indexref{framtiden}\newline
Fransk gubbstoner \mydots \indexref{fransk gubbstoner}\newline
Franska svordomar \mydots \indexref{franska svordomar}\newline
Frasses \mydots \indexref{frasses}\newline
Fredag \mydots \indexref{fredag}\newline
Fredagslyx \mydots \indexref{fredagslyx}\newline
Fredagsmys \mydots \indexref{fredagsmys}\newline
Fredrik Reinfeldt \mydots \indexref{fredrik reinfeldt}\newline
Freikörperkultur \mydots \indexref{freikoorperkultur}\newline
Fri rörlighet \mydots \indexref{fri roorlighet}\newline
Fridhem \mydots \indexref{fridhem}\newline
Friedrich Hegel \mydots \indexref{friedrich hegel}\newline
Frihet \mydots \indexref{frihet}\newline
Frikyrkligt lycklig \mydots \indexref{frikyrkligt lycklig}\newline
Frilans \mydots \indexref{frilans}\newline
Front row banger \mydots \indexref{front row banger}\newline
Frukt \mydots \indexref{frukt}\newline
Frukt är gott \mydots \indexref{frukt aer gott}\newline
Frukters sociopsykologiska dimension \mydots \indexref{frukters sociopsykologiska dimension}\newline
Fruktsallad \mydots \indexref{fruktsallad}\newline
Fryntlig \mydots \indexref{fryntlig}\newline
Ftw \mydots \indexref{ftw}\newline
Fudge \mydots \indexref{fudge}\newline
Fulhybris \mydots \indexref{fulhybris}\newline
Fulsnygg \mydots \indexref{fulsnygg}\newline
Furu \mydots \indexref{furu}\newline
Fyllevolontära \mydots \indexref{fyllevolontaera}\newline
Fyn \mydots \indexref{fyn}\newline
Fyra \mydots \indexref{fyra}\newline
Fyrhjulsdrift \mydots \indexref{fyrhjulsdrift}\newline
Fyrtiotusen miljarder \mydots \indexref{fyrtiotusen miljarder}\newline
Färskost \mydots \indexref{faerskost}\newline
Fåglar som går \mydots \indexref{faaglar som gaar}\newline
För nit och redlighet i rikets tjänst \mydots \indexref{foor nit och redlighet i rikets tjaenst}\newline
Fördelar med att bo i gryt \mydots \indexref{foordelar med att bo i gryt}\newline
Förebyggande skallgångskedja \mydots \indexref{foorebyggande skallgaangskedja}\newline
Förlovad \mydots \indexref{foorlovad}\newline
Första sjuan \mydots \indexref{foorsta sjuan}\newline
Första skivanalibi \mydots \indexref{foorsta skivanalibi}\newline
Förstklassig cricket \mydots \indexref{foorstklassig cricket}\newline
Förståsigpåare \mydots \indexref{foorstaasigpaaare}\newline
Försåvitt \mydots \indexref{foorsaavitt}\newline
Förvirring \mydots \indexref{foorvirring}\newline
\end{multicols}
{\huge{\textbf{G}}}
\begin{multicols}{2}
Galna ko-sjukan \mydots \indexref{galna ko-sjukan}\newline
Gamle Ole \mydots \indexref{gamle ole}\newline
Gammpojkar \mydots \indexref{gammpojkar}\newline
Gammstinta \mydots \indexref{gammstinta}\newline
Geezer Butler \mydots \indexref{geezer butler}\newline
Gengas \mydots \indexref{gengas}\newline
Genuint snåla människor \mydots \indexref{genuint snaala maenniskor}\newline
George Everest \mydots \indexref{george everest}\newline
Georgij Zjukov \mydots \indexref{georgij zjukov}\newline
Gertrud \mydots \indexref{gertrud}\newline
Gitarr \mydots \indexref{gitarr}\newline
Gitarrkille \mydots \indexref{gitarrkille}\newline
Gitarrmys \mydots \indexref{gitarrmys}\newline
Gittan \mydots \indexref{gittan}\newline
Glassbutt \mydots \indexref{glassbutt}\newline
Glassbåt \mydots \indexref{glassbaat}\newline
Glasse \mydots \indexref{glasse}\newline
Glasögon \mydots \indexref{glasoogon}\newline
Glenn \mydots \indexref{glenn}\newline
Glesbygdsball \mydots \indexref{glesbygdsball}\newline
Glida under radarn \mydots \indexref{glida under radarn}\newline
Glima \mydots \indexref{glima}\newline
Glimröv \mydots \indexref{glimroov}\newline
Globen \mydots \indexref{globen}\newline
Globetrotter \mydots \indexref{globetrotter}\newline
Glop \mydots \indexref{glop}\newline
Glädjevetenskaper \mydots \indexref{glaedjevetenskaper}\newline
Gnagare \mydots \indexref{gnagare}\newline
Gnussa \mydots \indexref{gnussa}\newline
Godisautomat \mydots \indexref{godisautomat}\newline
Golf \mydots \indexref{golf}\newline
Gore-Tex \mydots \indexref{gore-tex}\newline
Grafologi \mydots \indexref{grafologi}\newline
Grand Funk Railroad \mydots \indexref{grand funk railroad}\newline
Grekiska statsobligationer \mydots \indexref{grekiska statsobligationer}\newline
Grind \mydots \indexref{grind}\newline
Grisfull \mydots \indexref{grisfull}\newline
Grissini \mydots \indexref{grissini}\newline
Gruk \mydots \indexref{gruk}\newline
Grunka \mydots \indexref{grunka}\newline
Grå eminens \mydots \indexref{graa eminens}\newline
Gråmelerad T-shirt \mydots \indexref{graamelerad t-shirt}\newline
Gubbafint \mydots \indexref{gubbafint}\newline
Gubbrock \mydots \indexref{gubbrock}\newline
Gubbsova \mydots \indexref{gubbsova}\newline
Gubbsäker \mydots \indexref{gubbsaeker}\newline
Gud \mydots \indexref{gud}\newline
Gunborg \mydots \indexref{gunborg}\newline
Gurka \mydots \indexref{gurka}\newline
Gurkmajonnäs \mydots \indexref{gurkmajonnaes}\newline
Gurkvatten \mydots \indexref{gurkvatten}\newline
Gustav Vasa \mydots \indexref{gustav vasa}\newline
Gylfa \mydots \indexref{gylfa}\newline
Gå och köpa tidningen \mydots \indexref{gaa och koopa tidningen}\newline
Gå och snickra \mydots \indexref{gaa och snickra}\newline
Göra rätt för sig \mydots \indexref{goora raett foor sig}\newline
Göran \mydots \indexref{gooran}\newline
Göras till åtlöje inför hela svenska folket \mydots \indexref{gooras till aatlooje infoor hela svenska folket}\newline
Gösta \mydots \indexref{goosta}\newline
Gösta Snoddas Nordgren \mydots \indexref{goosta snoddas nordgren}\newline
Göteborg \mydots \indexref{gooteborg}\newline
\end{multicols}
{\huge{\textbf{H}}}
\begin{multicols}{2}
Ha bärs \mydots \indexref{ha baers}\newline
Habbadixen! \mydots \indexref{habbadixen!}\newline
Hacka \mydots \indexref{hacka}\newline
Haiku \mydots \indexref{haiku}\newline
Haile Selassie \mydots \indexref{haile selassie}\newline
Haka (vanlig) \mydots \indexref{haka (vanlig)}\newline
Hakkors vi minns \mydots \indexref{hakkors vi minns}\newline
Halland \mydots \indexref{halland}\newline
Ham \mydots \indexref{ham}\newline
Hamlet \mydots \indexref{hamlet}\newline
Handjagare \mydots \indexref{handjagare}\newline
Handskfack \mydots \indexref{handskfack}\newline
Handvass \mydots \indexref{handvass}\newline
Hardware \mydots \indexref{hardware}\newline
Hasch \mydots \indexref{hasch}\newline
Hasselbackspotatis \mydots \indexref{hasselbackspotatis}\newline
Havsmunk \mydots \indexref{havsmunk}\newline
Hawaii-pizza \mydots \indexref{hawaii-pizza}\newline
He \mydots \indexref{he}\newline
Headbanga \mydots \indexref{headbanga}\newline
Hefaistos \mydots \indexref{hefaistos}\newline
Hegemoni \mydots \indexref{hegemoni}\newline
Helgarderat efterliv \mydots \indexref{helgarderat efterliv}\newline
Helgvolym \mydots \indexref{helgvolym}\newline
Helvetet \mydots \indexref{helvetet}\newline
Hemliga koder \mydots \indexref{hemliga koder}\newline
Hemmansvärde \mydots \indexref{hemmansvaerde}\newline
Hemmets Härold \mydots \indexref{hemmets haerold}\newline
Herrkläder \mydots \indexref{herrklaeder}\newline
Hertigen av Rothesay \mydots \indexref{hertigen av rothesay}\newline
Hertsökullar \mydots \indexref{hertsookullar}\newline
Hillevi \mydots \indexref{hillevi}\newline
Hip-hop \mydots \indexref{hip-hop}\newline
Hippie \mydots \indexref{hippie}\newline
Historiska händelser i badrum \mydots \indexref{historiska haendelser i badrum}\newline
Hobofobi \mydots \indexref{hobofobi}\newline
Hockey \mydots \indexref{hockey}\newline
Hockeypulver \mydots \indexref{hockeypulver}\newline
Hockeyröv \mydots \indexref{hockeyroov}\newline
Holland \mydots \indexref{holland}\newline
Holmsunds tropikhus \mydots \indexref{holmsunds tropikhus}\newline
Holmund \mydots \indexref{holmund}\newline
Homi K. Bhabhas son \mydots \indexref{homi k. bhabhas son}\newline
Horgalåten \mydots \indexref{horgalaaten}\newline
Hudiksvall \mydots \indexref{hudiksvall}\newline
Hugo Alfvén \mydots \indexref{hugo alfvén}\newline
Hundkäx \mydots \indexref{hundkaex}\newline
Hundra sätt att få ligga \mydots \indexref{hundra saett att faa ligga}\newline
Hur man ritar ett snyggt lodjurshuvud \mydots \indexref{hur man ritar ett snyggt lodjurshuvud}\newline
Huvud \mydots \indexref{huvud}\newline
Huvudduk \mydots \indexref{huvudduk}\newline
Hyena \mydots \indexref{hyena}\newline
Hyperhidros \mydots \indexref{hyperhidros}\newline
Hytta med näven \mydots \indexref{hytta med naeven}\newline
Háfrónska \mydots \indexref{háfrónska}\newline
Häcklefjäll \mydots \indexref{haecklefjaell}\newline
Hällefors \mydots \indexref{haellefors}\newline
Hänga på låset \mydots \indexref{haenga paa laaset}\newline
Häst \mydots \indexref{haest}\newline
Hästhandlarplånbok \mydots \indexref{haesthandlarplaanbok}\newline
Hästkista \mydots \indexref{haestkista}\newline
Håbroa \mydots \indexref{haabroa}\newline
Håkan Juholt \mydots \indexref{haakan juholt}\newline
Hålkort \mydots \indexref{haalkort}\newline
Hårdrock \mydots \indexref{haardrock}\newline
Hårdrockare med gomspalt \mydots \indexref{haardrockare med gomspalt}\newline
Hårdrockare och vitaminer \mydots \indexref{haardrockare och vitaminer}\newline
Hårdrockism \mydots \indexref{haardrockism}\newline
Högtalartips \mydots \indexref{hoogtalartips}\newline
Hönsgård \mydots \indexref{hoonsgaard}\newline
\end{multicols}
{\huge{\textbf{I}}}
\begin{multicols}{2}
I'm so tired I could sleep on a clothesline \mydots \indexref{im so tired i could sleep on a clothesline}\newline
Ica prästost \mydots \indexref{ica praestost}\newline
Incitament \mydots \indexref{incitament}\newline
Incitatus \mydots \indexref{incitatus}\newline
Indianmuskler \mydots \indexref{indianmuskler}\newline
Indiska \mydots \indexref{indiska}\newline
Individ \mydots \indexref{individ}\newline
Inga lejon \mydots \indexref{inga lejon}\newline
Ingvar Carlsson \mydots \indexref{ingvar carlsson}\newline
Ingvar Kamprad \mydots \indexref{ingvar kamprad}\newline
Inititativ Anusmark \mydots \indexref{inititativ anusmark}\newline
Innebandy \mydots \indexref{innebandy}\newline
Inre backspegel \mydots \indexref{inre backspegel}\newline
Institut och tankesmedjor \mydots \indexref{institut och tankesmedjor}\newline
Insändarsignaturer \mydots \indexref{insaendarsignaturer}\newline
Intellektuell regression \mydots \indexref{intellektuell regression}\newline
International cloud atlas \mydots \indexref{international cloud atlas}\newline
International harvester \mydots \indexref{international harvester}\newline
Internet \mydots \indexref{internet}\newline
Irländsk parkering \mydots \indexref{irlaendsk parkering}\newline
Isaac Johannes Lamotius \mydots \indexref{isaac johannes lamotius}\newline
Isebrogen \mydots \indexref{isebrogen}\newline
Island \mydots \indexref{island}\newline
ISO 216 \mydots \indexref{iso 216}\newline
Italienska svordomar \mydots \indexref{italienska svordomar}\newline
Ivar Lo Johansson \mydots \indexref{ivar lo johansson}\newline
Ivriga små bävrar \mydots \indexref{ivriga smaa baevrar}\newline
\end{multicols}
{\huge{\textbf{J}}}
\begin{multicols}{2}
J.R.R Tolkien \mydots \indexref{j.r.r tolkien}\newline
Jackie Howe \mydots \indexref{jackie howe}\newline
Jackson Pollock \mydots \indexref{jackson pollock}\newline
Jacques Touillaud \mydots \indexref{jacques touillaud}\newline
Jag ska bara bli full först \mydots \indexref{jag ska bara bli full foorst}\newline
Jan Björkblund \mydots \indexref{jan bjoorkblund}\newline
Jan Björklund \mydots \indexref{jan bjoorklund}\newline
Jan Wilsgaard \mydots \indexref{jan wilsgaard}\newline
Japan \mydots \indexref{japan}\newline
Je ne sais quoi \mydots \indexref{je ne sais quoi}\newline
Jeansröv \mydots \indexref{jeansroov}\newline
Jens \mydots \indexref{jens}\newline
Jerry Williams \mydots \indexref{jerry williams}\newline
Jesaja 36:12 \mydots \indexref{jesaja 36:12}\newline
Jesus \mydots \indexref{jesus}\newline
Johan Skytte \mydots \indexref{johan skytte}\newline
Johann Neumann \mydots \indexref{johann neumann}\newline
Johannes Brost \mydots \indexref{johannes brost}\newline
John Kellogg \mydots \indexref{john kellogg}\newline
Johnny Takter \mydots \indexref{johnny takter}\newline
Jolle \mydots \indexref{jolle}\newline
Jonathan Guy \mydots \indexref{jonathan guy}\newline
Jordbruksort \mydots \indexref{jordbruksort}\newline
Joseph Lucas \mydots \indexref{joseph lucas}\newline
Jubal \mydots \indexref{jubal}\newline
Juridisk rådgivare \mydots \indexref{juridisk raadgivare}\newline
Jurij Gagarin \mydots \indexref{jurij gagarin}\newline
Järnspett \mydots \indexref{jaernspett}\newline
Järvhägn \mydots \indexref{jaervhaegn}\newline
Jättemyrslok \mydots \indexref{jaettemyrslok}\newline
Jättemyrslokssele \mydots \indexref{jaettemyrslokssele}\newline
Jävelskap \mydots \indexref{jaevelskap}\newline
\end{multicols}
{\huge{\textbf{K}}}
\begin{multicols}{2}
Kaffekask \mydots \indexref{kaffekask}\newline
Kaknästornet \mydots \indexref{kaknaestornet}\newline
Kalaskula \mydots \indexref{kalaskula}\newline
Kalixare \mydots \indexref{kalixare}\newline
Kalle anka \mydots \indexref{kalle anka}\newline
Kalle Ankas pocket \mydots \indexref{kalle ankas pocket}\newline
Kanadensisk frack \mydots \indexref{kanadensisk frack}\newline
Karbinhake \mydots \indexref{karbinhake}\newline
Karel Gott \mydots \indexref{karel gott}\newline
Katolik \mydots \indexref{katolik}\newline
Katt \mydots \indexref{katt}\newline
Kattbrosch \mydots \indexref{kattbrosch}\newline
Kattguld \mydots \indexref{kattguld}\newline
Kattsläkt \mydots \indexref{kattslaekt}\newline
Kattstrypare \mydots \indexref{kattstrypare}\newline
Kelsey Grammer \mydots \indexref{kelsey grammer}\newline
Kent \mydots \indexref{kent}\newline
Kepsar med olika företagslogotyper \mydots \indexref{kepsar med olika fooretagslogotyper}\newline
Kerry King \mydots \indexref{kerry king}\newline
Keytar \mydots \indexref{keytar}\newline
Kicki Danielsson \mydots \indexref{kicki danielsson}\newline
Kikare \mydots \indexref{kikare}\newline
Kineseri \mydots \indexref{kineseri}\newline
Kinesiska muren \mydots \indexref{kinesiska muren}\newline
King Edward \mydots \indexref{king edward}\newline
Kinneviksliljan \mydots \indexref{kinneviksliljan}\newline
Kir \mydots \indexref{kir}\newline
Kissemiss \mydots \indexref{kissemiss}\newline
Kiwi \mydots \indexref{kiwi}\newline
Kladdi Mittänän \mydots \indexref{kladdi mittaenaen}\newline
Klasse Möllberg \mydots \indexref{klasse moollberg}\newline
Klassikerbilssjälvmord \mydots \indexref{klassikerbilssjaelvmord}\newline
Klo \mydots \indexref{klo}\newline
Kloakdjur \mydots \indexref{kloakdjur}\newline
Klä ut sig till ett djur \mydots \indexref{klae ut sig till ett djur}\newline
Klädsamt ful \mydots \indexref{klaedsamt ful}\newline
Knixa \mydots \indexref{knixa}\newline
Knocking on heavens door \mydots \indexref{knocking on heavens door}\newline
Kodnacke \mydots \indexref{kodnacke}\newline
Kokt \mydots \indexref{kokt}\newline
Kolonialdricka \mydots \indexref{kolonialdricka}\newline
Kombinationsaffär \mydots \indexref{kombinationsaffaer}\newline
Kommunanställd småpåve \mydots \indexref{kommunanstaelld smaapaave}\newline
Kommunist \mydots \indexref{kommunist}\newline
Kommunistglasögon \mydots \indexref{kommunistglasoogon}\newline
Kommunslogan \mydots \indexref{kommunslogan}\newline
Konjektural \mydots \indexref{konjektural}\newline
Konsum på Haga \mydots \indexref{konsum paa haga}\newline
Konsumbutik \mydots \indexref{konsumbutik}\newline
Konventikelplakatet \mydots \indexref{konventikelplakatet}\newline
Kopi Luwak \mydots \indexref{kopi luwak}\newline
Kopparorm \mydots \indexref{kopparorm}\newline
Korp \mydots \indexref{korp}\newline
Kortbyxor \mydots \indexref{kortbyxor}\newline
Korv i smörpapper \mydots \indexref{korv i smoorpapper}\newline
Korv med bröd \mydots \indexref{korv med brood}\newline
Korv-Ivars \mydots \indexref{korv-ivars}\newline
Korvbröd \mydots \indexref{korvbrood}\newline
Krigsgrisen \mydots \indexref{krigsgrisen}\newline
Kriminalroman \mydots \indexref{kriminalroman}\newline
Kristdemokraterna \mydots \indexref{kristdemokraterna}\newline
Krokodiljägare \mydots \indexref{krokodiljaegare}\newline
Kroppshydda \mydots \indexref{kroppshydda}\newline
Krus \mydots \indexref{krus}\newline
Krypa ihop i soffan som en katt \mydots \indexref{krypa ihop i soffan som en katt}\newline
Krypa upp i soffan som en brugd \mydots \indexref{krypa upp i soffan som en brugd}\newline
Krypa upp i soffan som en uv \mydots \indexref{krypa upp i soffan som en uv}\newline
Kräftbete \mydots \indexref{kraeftbete}\newline
Kräftor \mydots \indexref{kraeftor}\newline
Kränkt \mydots \indexref{kraenkt}\newline
Kukenkillar \mydots \indexref{kukenkillar}\newline
Kulaker \mydots \indexref{kulaker}\newline
Kulturarbetare \mydots \indexref{kulturarbetare}\newline
Kulturstökigt \mydots \indexref{kulturstookigt}\newline
Kusin \mydots \indexref{kusin}\newline
Kvicktänkt \mydots \indexref{kvicktaenkt}\newline
Kvinnlig författare-knepet \mydots \indexref{kvinnlig foorfattare-knepet}\newline
Kvinnligt alibi \mydots \indexref{kvinnligt alibi}\newline
Kvinnokläder \mydots \indexref{kvinnoklaeder}\newline
Kälkborgare \mydots \indexref{kaelkborgare}\newline
Källkritik \mydots \indexref{kaellkritik}\newline
Känslo-Oi! \mydots \indexref{kaenslo-oi!}\newline
Kärlek \mydots \indexref{kaerlek}\newline
Kökssoffan \mydots \indexref{kookssoffan}\newline
Könsrock \mydots \indexref{koonsrock}\newline
Köping \mydots \indexref{kooping}\newline
Körp \mydots \indexref{koorp}\newline
Közösülés \mydots \indexref{koozoosülés}\newline
\end{multicols}
{\huge{\textbf{L}}}
\begin{multicols}{2}
Laissez-faire \mydots \indexref{laissez-faire}\newline
Landslagsuppehåll \mydots \indexref{landslagsuppehaall}\newline
Lappskojs \mydots \indexref{lappskojs}\newline
Lars Krogh \mydots \indexref{lars krogh}\newline
Lars Levi Laestadius \mydots \indexref{lars levi laestadius}\newline
Lars-Åke Lagrell \mydots \indexref{lars-aake lagrell}\newline
LAS \mydots \indexref{las}\newline
Latinsk facebook-rocker \mydots \indexref{latinsk facebook-rocker}\newline
Ledingreppet \mydots \indexref{ledingreppet}\newline
Leggings \mydots \indexref{leggings}\newline
Lego \mydots \indexref{lego}\newline
Lemmy-bas \mydots \indexref{lemmy-bas}\newline
Lena \mydots \indexref{lena}\newline
Lennart Holmlund \mydots \indexref{lennart holmlund}\newline
Lenin-Churchhill aka Mysgubbe \mydots \indexref{lenin-churchhill aka mysgubbe}\newline
Libanon \mydots \indexref{libanon}\newline
Ligga med kulturstockholm \mydots \indexref{ligga med kulturstockholm}\newline
Likgömmarmössa \mydots \indexref{likgoommarmoossa}\newline
Lillgammal \mydots \indexref{lillgammal}\newline
Lillnöjd \mydots \indexref{lillnoojd}\newline
Limerick \mydots \indexref{limerick}\newline
Linda Norrman Skugge \mydots \indexref{linda norrman skugge}\newline
Lista över dis-namn \mydots \indexref{lista oover dis-namn}\newline
Ljudtekniker \mydots \indexref{ljudtekniker}\newline
Lobotomobil \mydots \indexref{lobotomobil}\newline
Looppedal \mydots \indexref{looppedal}\newline
Lothar \mydots \indexref{lothar}\newline
Lucia \mydots \indexref{lucia}\newline
Luciavaka \mydots \indexref{luciavaka}\newline
Luffarskål \mydots \indexref{luffarskaal}\newline
Luftgitarr \mydots \indexref{luftgitarr}\newline
Luktagott \mydots \indexref{luktagott}\newline
Lule \mydots \indexref{lule}\newline
Lundgren \mydots \indexref{lundgren}\newline
Lundin Petroleum \mydots \indexref{lundin petroleum}\newline
Lurkuk \mydots \indexref{lurkuk}\newline
Luís Figo \mydots \indexref{luís figo}\newline
Lycksele \mydots \indexref{lycksele}\newline
Läppar som prinskorv \mydots \indexref{laeppar som prinskorv}\newline
Lärare \mydots \indexref{laerare}\newline
Läsesalsflört \mydots \indexref{laesesalsfloort}\newline
Lätt misshandel \mydots \indexref{laett misshandel}\newline
Lättlagade festrätter från Anderssons skafferi \mydots \indexref{laettlagade festraetter fraan anderssons skafferi}\newline
Lättnad \mydots \indexref{laettnad}\newline
Lättöl \mydots \indexref{laettool}\newline
Lådaktivism \mydots \indexref{laadaktivism}\newline
Långfredagen \mydots \indexref{laangfredagen}\newline
Löneförmån \mydots \indexref{loonefoormaan}\newline
Lördag \mydots \indexref{loordag}\newline
\end{multicols}
{\huge{\textbf{M}}}
\begin{multicols}{2}
Mackshopping \mydots \indexref{mackshopping}\newline
Mads Mikkelsen \mydots \indexref{mads mikkelsen}\newline
Maginotlinjen \mydots \indexref{maginotlinjen}\newline
Malå \mydots \indexref{malaa}\newline
Malålistan \mydots \indexref{malaalistan}\newline
Malåparkering \mydots \indexref{malaaparkering}\newline
Malårca \mydots \indexref{malaarca}\newline
Manet \mydots \indexref{manet}\newline
Mangel \mydots \indexref{mangel}\newline
Mango safe \mydots \indexref{mango safe}\newline
Mani \mydots \indexref{mani}\newline
Manowar \mydots \indexref{manowar}\newline
Manslyssna \mydots \indexref{manslyssna}\newline
Manuel \mydots \indexref{manuel}\newline
Martine \mydots \indexref{martine}\newline
Margaret Thatcher \mydots \indexref{margaret thatcher}\newline
Margit Sandemo \mydots \indexref{margit sandemo}\newline
Maski Hallonen \mydots \indexref{maski hallonen}\newline
Masonitemuséet i Rundvik \mydots \indexref{masonitemuséet i rundvik}\newline
Matematikmaskinnämnden \mydots \indexref{matematikmaskinnaemnden}\newline
Math metal \mydots \indexref{math metal}\newline
Mats Lundgren \mydots \indexref{mats lundgren}\newline
Mattias Alkberg \mydots \indexref{mattias alkberg}\newline
Maud Olofsson \mydots \indexref{maud olofsson}\newline
Max Weber \mydots \indexref{max weber}\newline
Medelklassvänner \mydots \indexref{medelklassvaenner}\newline
Medeltiden \mydots \indexref{medeltiden}\newline
Merchband \mydots \indexref{merchband}\newline
Metalmynt \mydots \indexref{metalmynt}\newline
Micke Alonzo \mydots \indexref{micke alonzo}\newline
Mikis Theodorakis \mydots \indexref{mikis theodorakis}\newline
Miljöbil \mydots \indexref{miljoobil}\newline
Miljöpartiet \mydots \indexref{miljoopartiet}\newline
Mimmi Pigg \mydots \indexref{mimmi pigg}\newline
Min kära gamla soppeskål \mydots \indexref{min kaera gamla soppeskaal}\newline
Minusmat \mydots \indexref{minusmat}\newline
Missbrukare \mydots \indexref{missbrukare}\newline
Mjukis \mydots \indexref{mjukis}\newline
Mjukpack cigg \mydots \indexref{mjukpack cigg}\newline
Mob 47 \mydots \indexref{mob 47}\newline
Modem \mydots \indexref{modem}\newline
Moderat \mydots \indexref{moderat}\newline
Moderator \mydots \indexref{moderator}\newline
Moln \mydots \indexref{moln}\newline
Molotov cocktail \mydots \indexref{molotov cocktail}\newline
Moona röv \mydots \indexref{moona roov}\newline
Motpåven i Gränna \mydots \indexref{motpaaven i graenna}\newline
Mount Everest \mydots \indexref{mount everest}\newline
Mun \mydots \indexref{mun}\newline
Mungo Jerryhatare \mydots \indexref{mungo jerryhatare}\newline
Muno \mydots \indexref{muno}\newline
Musikhögskolemusik \mydots \indexref{musikhoogskolemusik}\newline
Musselini \mydots \indexref{musselini}\newline
Muttersvarvare \mydots \indexref{muttersvarvare}\newline
Myntsamleri \mydots \indexref{myntsamleri}\newline
Myspyssockerkaka \mydots \indexref{myspyssockerkaka}\newline
Mystiska band \mydots \indexref{mystiska band}\newline
Myt \mydots \indexref{myt}\newline
Mäklarbricka \mydots \indexref{maeklarbricka}\newline
Mäklarsvenska \mydots \indexref{maeklarsvenska}\newline
Märkliga sammanträffanden \mydots \indexref{maerkliga sammantraeffanden}\newline
Måg \mydots \indexref{maag}\newline
Målvakt \mydots \indexref{maalvakt}\newline
Måndag \mydots \indexref{maandag}\newline
Mångsysslarpensionär \mydots \indexref{maangsysslarpensionaer}\newline
Mördare \mydots \indexref{moordare}\newline
\end{multicols}
{\huge{\textbf{N}}}
\begin{multicols}{2}
Naturens dialektik \mydots \indexref{naturens dialektik}\newline
Naturhistoriska museet \mydots \indexref{naturhistoriska museet}\newline
Nedlagda industribyggnader \mydots \indexref{nedlagda industribyggnader}\newline
Nedsatt sikt \mydots \indexref{nedsatt sikt}\newline
Neontetra \mydots \indexref{neontetra}\newline
Nia \mydots \indexref{nia}\newline
Nicholas Cage-film \mydots \indexref{nicholas cage-film}\newline
Nigeria \mydots \indexref{nigeria}\newline
Nikolaj Valujev \mydots \indexref{nikolaj valujev}\newline
Nissepedia \mydots \indexref{nissepedia}\newline
Nitlott \mydots \indexref{nitlott}\newline
Nollpresterare \mydots \indexref{nollpresterare}\newline
Norberg \mydots \indexref{norberg}\newline
Nordisk kombination \mydots \indexref{nordisk kombination}\newline
Tenzing Norgay \mydots \indexref{tenzing norgay}\newline
Norge \mydots \indexref{norge}\newline
Norrbotten \mydots \indexref{norrbotten}\newline
Norrtälje \mydots \indexref{norrtaelje}\newline
Norsjöblicken \mydots \indexref{norsjooblicken}\newline
Noshörningen Nelson \mydots \indexref{noshoorningen nelson}\newline
Nu går slakten på Bomans vind! \mydots \indexref{nu gaar slakten paa bomans vind!}\newline
Nudist \mydots \indexref{nudist}\newline
Nya Zeeland \mydots \indexref{nya zeeland}\newline
Nyliberalism \mydots \indexref{nyliberalism}\newline
Näbbmun \mydots \indexref{naebbmun}\newline
När livet blir alldeles för mycket - Prof. Etiennes bästa gömställen, i urval \mydots \indexref{naer livet blir alldeles foor mycket - prof. etiennes baesta goomstaellen, i urval}\newline
Nära vän till familjen \mydots \indexref{naera vaen till familjen}\newline
Näs-flås \mydots \indexref{naes-flaas}\newline
Näsa \mydots \indexref{naesa}\newline
Näverslips \mydots \indexref{naeverslips}\newline
\end{multicols}
{\huge{\textbf{O}}}
\begin{multicols}{2}
Oberoende olympiska deltagare \mydots \indexref{oberoende olympiska deltagare}\newline
Odon \mydots \indexref{odon}\newline
Ohemul \mydots \indexref{ohemul}\newline
Oidipuskomplex \mydots \indexref{oidipuskomplex}\newline
Old Black \mydots \indexref{old black}\newline
Old ox \mydots \indexref{old ox}\newline
Oliver/Dawson Saxon \mydots \indexref{oliverdawson saxon}\newline
Olja \mydots \indexref{olja}\newline
Oljegark \mydots \indexref{oljegark}\newline
Olof Palme \mydots \indexref{olof palme}\newline
Opel kadett \mydots \indexref{opel kadett}\newline
Ornässtugans dass \mydots \indexref{ornaesstugans dass}\newline
Orolighetskeps \mydots \indexref{orolighetskeps}\newline
Orrkammens isolering och gokart \mydots \indexref{orrkammens isolering och gokart}\newline
Oscar Dronjak \mydots \indexref{oscar dronjak}\newline
Oscar Wilde \mydots \indexref{oscar wilde}\newline
Ostmacka \mydots \indexref{ostmacka}\newline
Ououou \mydots \indexref{ououou}\newline
\end{multicols}
{\huge{\textbf{P}}}
\begin{multicols}{2}
Paddan i pannrummet \mydots \indexref{paddan i pannrummet}\newline
Paj \mydots \indexref{paj}\newline
Paleontologi \mydots \indexref{paleontologi}\newline
Palle Kuling \mydots \indexref{palle kuling}\newline
Palltruck \mydots \indexref{palltruck}\newline
Panflöjt \mydots \indexref{panfloojt}\newline
Pangsionärerna \mydots \indexref{pangsionaererna}\newline
Panik \mydots \indexref{panik}\newline
Parisare \mydots \indexref{parisare}\newline
Passa tider \mydots \indexref{passa tider}\newline
Patentkork \mydots \indexref{patentkork}\newline
PATSY award \mydots \indexref{patsy award}\newline
Paul du Chaillu \mydots \indexref{paul du chaillu}\newline
Paxa \mydots \indexref{paxa}\newline
Pelikan \mydots \indexref{pelikan}\newline
Pelle Fosshaug \mydots \indexref{pelle fosshaug}\newline
Pelle Karlsson \mydots \indexref{pelle karlsson}\newline
Pelle Svensson \mydots \indexref{pelle svensson}\newline
Personlig assistans \mydots \indexref{personlig assistans}\newline
Personligt varumärke \mydots \indexref{personligt varumaerke}\newline
Perspektiv \mydots \indexref{perspektiv}\newline
Peru \mydots \indexref{peru}\newline
Perversa elektriker \mydots \indexref{perversa elektriker}\newline
Petrus de Dacia \mydots \indexref{petrus de dacia}\newline
Petter \mydots \indexref{petter}\newline
Philibert Humla \mydots \indexref{philibert humla}\newline
Picknickbog \mydots \indexref{picknickbog}\newline
Piteå \mydots \indexref{piteaa}\newline
Pizzaracer \mydots \indexref{pizzaracer}\newline
Pizzarulle \mydots \indexref{pizzarulle}\newline
Pjäxfett \mydots \indexref{pjaexfett}\newline
PK \mydots \indexref{pk}\newline
Place Jourdan \mydots \indexref{place jourdan}\newline
Plocka päron \mydots \indexref{plocka paeron}\newline
Pluta \mydots \indexref{pluta}\newline
Pneumatiska rör \mydots \indexref{pneumatiska roor}\newline
Poetisk rättvisa \mydots \indexref{poetisk raettvisa}\newline
Polis \mydots \indexref{polis}\newline
Polska helgdagar \mydots \indexref{polska helgdagar}\newline
Pop-rock \mydots \indexref{pop-rock}\newline
Porträtt av det postmoderna renässansgeniet som ung \mydots \indexref{portraett av det postmoderna renaessansgeniet som ung}\newline
Post-coitus \mydots \indexref{post-coitus}\newline
Posten \mydots \indexref{posten}\newline
Postiljon \mydots \indexref{postiljon}\newline
Postkolonialism \mydots \indexref{postkolonialism}\newline
Postlåda \mydots \indexref{postlaada}\newline
Postmodern morförälder \mydots \indexref{postmodern morfooraelder}\newline
Postpostrock \mydots \indexref{postpostrock}\newline
Postseminarium \mydots \indexref{postseminarium}\newline
Postångare \mydots \indexref{postaangare}\newline
Potatisbar \mydots \indexref{potatisbar}\newline
Potatistryck \mydots \indexref{potatistryck}\newline
Praktarsle (negativ) \mydots \indexref{praktarsle (negativ)}\newline
Praktarsle (positiv) \mydots \indexref{praktarsle (positiv)}\newline
Prins Charles \mydots \indexref{prins charles}\newline
Prinskorv \mydots \indexref{prinskorv}\newline
Privatisering \mydots \indexref{privatisering}\newline
Privatspanare \mydots \indexref{privatspanare}\newline
Problematiskt \mydots \indexref{problematiskt}\newline
Processa mot länsstyrelsen \mydots \indexref{processa mot laensstyrelsen}\newline
Produktionsknull \mydots \indexref{produktionsknull}\newline
Prof. Etienne \mydots \indexref{prof. etienne}\newline
Professor skytteanus \mydots \indexref{professor skytteanus}\newline
Proggig inre frid \mydots \indexref{proggig inre frid}\newline
Prokrastrinering \mydots \indexref{prokrastrinering}\newline
Proletära bär \mydots \indexref{proletaera baer}\newline
Propeller \mydots \indexref{propeller}\newline
Propellerkeps \mydots \indexref{propellerkeps}\newline
Prunka \mydots \indexref{prunka}\newline
Prutta högljutt \mydots \indexref{prutta hoogljutt}\newline
Psoriasis \mydots \indexref{psoriasis}\newline
Psykedelisk morförälder \mydots \indexref{psykedelisk morfooraelder}\newline
Pudaslåda \mydots \indexref{pudaslaada}\newline
Punk \mydots \indexref{punk}\newline
Punkgryta \mydots \indexref{punkgryta}\newline
Punkscenshumor \mydots \indexref{punkscenshumor}\newline
Putsbilar \mydots \indexref{putsbilar}\newline
Pysselbyxa \mydots \indexref{pysselbyxa}\newline
På fat \mydots \indexref{paa fat}\newline
Påsförslutare \mydots \indexref{paasfoorslutare}\newline
Påsk \mydots \indexref{paask}\newline
Påsmygande själv-alienation \mydots \indexref{paasmygande sjaelv-alienation}\newline
Päronhalva \mydots \indexref{paeronhalva}\newline
Pölsa \mydots \indexref{poolsa}\newline
Pörr \mydots \indexref{poorr}\newline
\end{multicols}
{\huge{\textbf{Q}}}
\begin{multicols}{2}
Queequeg \mydots \indexref{queequeg}\newline
\end{multicols}
{\huge{\textbf{R}}}
\begin{multicols}{2}
RAC \mydots \indexref{rac}\newline
Radioreklam \mydots \indexref{radioreklam}\newline
Rage-quita \mydots \indexref{rage-quita}\newline
Raison d'être \mydots \indexref{raison dêtre}\newline
Randall Finefield \mydots \indexref{randall finefield}\newline
Ranta Runtiringen \mydots \indexref{ranta runtiringen}\newline
Rasism \mydots \indexref{rasism}\newline
Rasmus Klump \mydots \indexref{rasmus klump}\newline
Raw justice \mydots \indexref{raw justice}\newline
Ray Jones IV \mydots \indexref{ray jones iv}\newline
Realister \mydots \indexref{realister}\newline
Repet \mydots \indexref{repet}\newline
Richard Dybeck \mydots \indexref{richard dybeck}\newline
Riddare \mydots \indexref{riddare}\newline
Rikemanssidan \mydots \indexref{rikemanssidan}\newline
Rikskuponger \mydots \indexref{rikskuponger}\newline
Riksregalier \mydots \indexref{riksregalier}\newline
Rikssamtal \mydots \indexref{rikssamtal}\newline
Rimbo \mydots \indexref{rimbo}\newline
Rolf \mydots \indexref{rolf}\newline
Roliga timmen \mydots \indexref{roliga timmen}\newline
Romantik \mydots \indexref{romantik}\newline
Roy Andersson-väder \mydots \indexref{roy andersson-vaeder}\newline
Rugga \mydots \indexref{rugga}\newline
Rulla hatt \mydots \indexref{rulla hatt}\newline
Ryckepungvägen \mydots \indexref{ryckepungvaegen}\newline
Rygg \mydots \indexref{rygg}\newline
Rygga \mydots \indexref{rygga}\newline
Ryggtryck \mydots \indexref{ryggtryck}\newline
Räkmacka \mydots \indexref{raekmacka}\newline
Räksallad \mydots \indexref{raeksallad}\newline
Räksmörgås \mydots \indexref{raeksmoorgaas}\newline
Råååål \mydots \indexref{raaaaaaaal}\newline
Rögad ål \mydots \indexref{roogad aal}\newline
Röksignaler \mydots \indexref{rooksignaler}\newline
Rökå \mydots \indexref{rookaa}\newline
Rött \mydots \indexref{roott}\newline
Rövgitarr \mydots \indexref{roovgitarr}\newline
Rød pølse \mydots \indexref{roood pooolse}\newline
\end{multicols}
{\huge{\textbf{S}}}
\begin{multicols}{2}
Saltbas \mydots \indexref{saltbas}\newline
Saltsjöbadsavtalet \mydots \indexref{saltsjoobadsavtalet}\newline
Samtida nordisk undergroundmusik \mydots \indexref{samtida nordisk undergroundmusik}\newline
Samurajernas hederskodex \mydots \indexref{samurajernas hederskodex}\newline
Sand \mydots \indexref{sand}\newline
Sandpapper \mydots \indexref{sandpapper}\newline
Sanningssägande bloggar \mydots \indexref{sanningssaegande bloggar}\newline
Sans pants \mydots \indexref{sans pants}\newline
Sarin \mydots \indexref{sarin}\newline
Semikolon \mydots \indexref{semikolon}\newline
Sengekantsfilm \mydots \indexref{sengekantsfilm}\newline
Senilsnöre \mydots \indexref{senilsnoore}\newline
Sexa \mydots \indexref{sexa}\newline
Sextant \mydots \indexref{sextant}\newline
Shizo Kanaguri \mydots \indexref{shizo kanaguri}\newline
Shockrockare \mydots \indexref{shockrockare}\newline
Sidensvans \mydots \indexref{sidensvans}\newline
Sign of the hammer \mydots \indexref{sign of the hammer}\newline
Sigvard Thurneman \mydots \indexref{sigvard thurneman}\newline
Simhud \mydots \indexref{simhud}\newline
Singelsnurra \mydots \indexref{singelsnurra}\newline
Sinkadus \mydots \indexref{sinkadus}\newline
Sitta \mydots \indexref{sitta}\newline
Sittsova \mydots \indexref{sittsova}\newline
Siv-Berit \mydots \indexref{siv-berit}\newline
Sjua \mydots \indexref{sjua}\newline
Sjundedagsadventistisk skola \mydots \indexref{sjundedagsadventistisk skola}\newline
Sjungande trummis \mydots \indexref{sjungande trummis}\newline
Självförtroendeplagg \mydots \indexref{sjaelvfoortroendeplagg}\newline
Självmordsspåret \mydots \indexref{sjaelvmordsspaaret}\newline
Skagen \mydots \indexref{skagen}\newline
Skam \mydots \indexref{skam}\newline
Skedstork \mydots \indexref{skedstork}\newline
Skellefteå \mydots \indexref{skellefteaa}\newline
Skensnygg \mydots \indexref{skensnygg}\newline
Skiftnyckel \mydots \indexref{skiftnyckel}\newline
Skillnaden mellan ånga och dimma \mydots \indexref{skillnaden mellan aanga och dimma}\newline
Skinhead \mydots \indexref{skinhead}\newline
Skinnskatteberg \mydots \indexref{skinnskatteberg}\newline
Skinnslips \mydots \indexref{skinnslips}\newline
Skita i den korvbröda kökssoffan \mydots \indexref{skita i den korvbrooda kookssoffan}\newline
Skita i det blå skåpet \mydots \indexref{skita i det blaa skaapet}\newline
Skjortponcho \mydots \indexref{skjortponcho}\newline
Sko \mydots \indexref{sko}\newline
Skogsrave \mydots \indexref{skogsrave}\newline
Skolbespisningsmat \mydots \indexref{skolbespisningsmat}\newline
Skottar \mydots \indexref{skottar}\newline
Skotte \mydots \indexref{skotte}\newline
Skottfint \mydots \indexref{skottfint}\newline
Skrapade skraplotter med vinst \mydots \indexref{skrapade skraplotter med vinst}\newline
Skrattfnatt \mydots \indexref{skrattfnatt}\newline
Skrunka \mydots \indexref{skrunka}\newline
Skruvkapsylöl \mydots \indexref{skruvkapsylool}\newline
Skrymmande \mydots \indexref{skrymmande}\newline
Skräckväldet \mydots \indexref{skraeckvaeldet}\newline
Skräp \mydots \indexref{skraep}\newline
Skröna \mydots \indexref{skroona}\newline
Skuggan \mydots \indexref{skuggan}\newline
Skäpparkrans \mydots \indexref{skaepparkrans}\newline
Skåne \mydots \indexref{skaane}\newline
Slagg \mydots \indexref{slagg}\newline
Slan \mydots \indexref{slan}\newline
Slayerklass \mydots \indexref{slayerklass}\newline
Slentrian \mydots \indexref{slentrian}\newline
Släktträffsberusning \mydots \indexref{slaekttraeffsberusning}\newline
Slätt \mydots \indexref{slaett}\newline
Smidesstäd \mydots \indexref{smidesstaed}\newline
Smuldegspappor \mydots \indexref{smuldegspappor}\newline
Smygfascist \mydots \indexref{smygfascist}\newline
Smygsexist \mydots \indexref{smygsexist}\newline
Småbyxa \mydots \indexref{smaabyxa}\newline
Småskurk \mydots \indexref{smaaskurk}\newline
Småstadsalternativ \mydots \indexref{smaastadsalternativ}\newline
Sneakers \mydots \indexref{sneakers}\newline
Snefotad ultrapelikan \mydots \indexref{snefotad ultrapelikan}\newline
Snesegla \mydots \indexref{snesegla}\newline
Snowjoggers \mydots \indexref{snowjoggers}\newline
Snus \mydots \indexref{snus}\newline
Snusbrist \mydots \indexref{snusbrist}\newline
Snutkaffe \mydots \indexref{snutkaffe}\newline
Snutnamn \mydots \indexref{snutnamn}\newline
Snutröv \mydots \indexref{snutroov}\newline
Snälla killar som aldrig får ligga \mydots \indexref{snaella killar som aldrig faar ligga}\newline
Snälla mamma, mata mig som vore du en fågel \mydots \indexref{snaella mamma, mata mig som vore du en faagel}\newline
Snärt \mydots \indexref{snaert}\newline
Snåltarmen \mydots \indexref{snaaltarmen}\newline
Snöskor \mydots \indexref{snooskor}\newline
Sober-Jimmy \mydots \indexref{sober-jimmy}\newline
Socialdemokrati \mydots \indexref{socialdemokrati}\newline
Solidaritet \mydots \indexref{solidaritet}\newline
Sommar \mydots \indexref{sommar}\newline
Sommarplågsmusiker \mydots \indexref{sommarplaagsmusiker}\newline
Sonett (engelsk) \mydots \indexref{sonett (engelsk)}\newline
Sonny Listons son \mydots \indexref{sonny listons son}\newline
Sopletare \mydots \indexref{sopletare}\newline
Sorbet \mydots \indexref{sorbet}\newline
Spanien \mydots \indexref{spanien}\newline
Spansk haloumi \mydots \indexref{spansk haloumi}\newline
Spanskt lättvin \mydots \indexref{spanskt laettvin}\newline
SPAP \mydots \indexref{spap}\newline
Speedos \mydots \indexref{speedos}\newline
Spegel \mydots \indexref{spegel}\newline
Spegelmöte \mydots \indexref{spegelmoote}\newline
Sportmössa \mydots \indexref{sportmoossa}\newline
Spritfylla \mydots \indexref{spritfylla}\newline
Spritvev \mydots \indexref{spritvev}\newline
Spunka \mydots \indexref{spunka}\newline
Sputnik \mydots \indexref{sputnik}\newline
Spärrballong \mydots \indexref{spaerrballong}\newline
Stark mat \mydots \indexref{stark mat}\newline
Steglits \mydots \indexref{steglits}\newline
Stenad \mydots \indexref{stenad}\newline
Stenlapp \mydots \indexref{stenlapp}\newline
Stentrollsaffär \mydots \indexref{stentrollsaffaer}\newline
Stia \mydots \indexref{stia}\newline
Stig \mydots \indexref{stig}\newline
Stigma \mydots \indexref{stigma}\newline
Stjärtlapp \mydots \indexref{stjaertlapp}\newline
Stockholm \mydots \indexref{stockholm}\newline
Stockholmare \mydots \indexref{stockholmare}\newline
Stonerskin \mydots \indexref{stonerskin}\newline
Stor-Anders \mydots \indexref{stor-anders}\newline
Stora Grabbars och Tjejers Märke \mydots \indexref{stora grabbars och tjejers maerke}\newline
Storbossnörd \mydots \indexref{storbossnoord}\newline
Storfräsare \mydots \indexref{storfraesare}\newline
Storhetsvansinne \mydots \indexref{storhetsvansinne}\newline
Storsien \mydots \indexref{storsien}\newline
Storspov \mydots \indexref{storspov}\newline
Storspren \mydots \indexref{storspren}\newline
Storswänsk \mydots \indexref{storswaensk}\newline
Streiff \mydots \indexref{streiff}\newline
Strelka \mydots \indexref{strelka}\newline
Streptokocker \mydots \indexref{streptokocker}\newline
Stress \mydots \indexref{stress}\newline
Strulputte \mydots \indexref{strulputte}\newline
Strumpor \mydots \indexref{strumpor}\newline
Stryknin \mydots \indexref{stryknin}\newline
Stråtrövare \mydots \indexref{straatroovare}\newline
Strömavbrott \mydots \indexref{stroomavbrott}\newline
Strövtåg \mydots \indexref{stroovtaag}\newline
Stöcksjö sunny resorts \mydots \indexref{stoocksjoo sunny resorts}\newline
Stödkorv \mydots \indexref{stoodkorv}\newline
Sugmästare \mydots \indexref{sugmaestare}\newline
Supa ensam \mydots \indexref{supa ensam}\newline
Surfin' bird \mydots \indexref{surfin bird}\newline
Surrande ljud \mydots \indexref{surrande ljud}\newline
Svag mat \mydots \indexref{svag mat}\newline
Svan \mydots \indexref{svan}\newline
Svart alibi \mydots \indexref{svart alibi}\newline
Svarta tavlan \mydots \indexref{svarta tavlan}\newline
Svensk bilsemester \mydots \indexref{svensk bilsemester}\newline
Svenska jägareförbundet \mydots \indexref{svenska jaegarefoorbundet}\newline
Svenska Kennelklubben \mydots \indexref{svenska kennelklubben}\newline
Svenska Kennetklubben \mydots \indexref{svenska kennetklubben}\newline
Svenskt näringsliv \mydots \indexref{svenskt naeringsliv}\newline
Sverige \mydots \indexref{sverige}\newline
Sverigedemokraterna \mydots \indexref{sverigedemokraterna}\newline
Sveriges sju underverk \mydots \indexref{sveriges sju underverk}\newline
Svinpäls \mydots \indexref{svinpaels}\newline
Sviskonpaj \mydots \indexref{sviskonpaj}\newline
Svotto \mydots \indexref{svotto}\newline
Svälta räv \mydots \indexref{svaelta raev}\newline
Svåra saker \mydots \indexref{svaara saker}\newline
Schwarzwald Larsson \mydots \indexref{schwarzwald larsson}\newline
Syfilis \mydots \indexref{syfilis}\newline
Sylt \mydots \indexref{sylt}\newline
Syndikalism \mydots \indexref{syndikalism}\newline
Synonymer för anus \mydots \indexref{synonymer foor anus}\newline
Syo \mydots \indexref{syo}\newline
Säcklöpning \mydots \indexref{saeckloopning}\newline
Sälar \mydots \indexref{saelar}\newline
Sällskapsresan \mydots \indexref{saellskapsresan}\newline
Sämskskinn \mydots \indexref{saemskskinn}\newline
Särk \mydots \indexref{saerk}\newline
Särske \mydots \indexref{saerske}\newline
Sågverk \mydots \indexref{saagverk}\newline
Sårrengöringsvätska \mydots \indexref{saarrengooringsvaetska}\newline
Söndag \mydots \indexref{soondag}\newline
\end{multicols}
{\huge{\textbf{T}}}
\begin{multicols}{2}
T-rexarmar \mydots \indexref{t-rexarmar}\newline
Ta för sig \mydots \indexref{ta foor sig}\newline
Taggen \mydots \indexref{taggen}\newline
Talgoxe \mydots \indexref{talgoxe}\newline
Tantkläder \mydots \indexref{tantklaeder}\newline
Tantnöjd \mydots \indexref{tantnoojd}\newline
Tantsång \mydots \indexref{tantsaang}\newline
Taxichaufför \mydots \indexref{taxichauffoor}\newline
Te \mydots \indexref{te}\newline
Teenage Mutant Ninja Turtles \mydots \indexref{teenage mutant ninja turtles}\newline
Tegare \mydots \indexref{tegare}\newline
Tegsnäsare \mydots \indexref{tegsnaesare}\newline
Televerket \mydots \indexref{televerket}\newline
Television \mydots \indexref{television}\newline
Telverksorange \mydots \indexref{telverksorange}\newline
Tengah \mydots \indexref{tengah}\newline
The fat Spanish waiter \mydots \indexref{the fat spanish waiter}\newline
The Fog \mydots \indexref{the fog}\newline
Thomas Wassberg \mydots \indexref{thomas wassberg}\newline
Thorstenkram \mydots \indexref{thorstenkram}\newline
Thrashzan \mydots \indexref{thrashzan}\newline
Tia \mydots \indexref{tia}\newline
Tisdag \mydots \indexref{tisdag}\newline
Titta på ord \mydots \indexref{titta paa ord}\newline
Tivoli \mydots \indexref{tivoli}\newline
Tjack \mydots \indexref{tjack}\newline
Tjamstan \mydots \indexref{tjamstan}\newline
Tjena Roger! \mydots \indexref{tjena roger!}\newline
Tjock-TV \mydots \indexref{tjock-tv}\newline
Tjänstemannateoretisk \mydots \indexref{tjaenstemannateoretisk}\newline
Toalettpapper \mydots \indexref{toalettpapper}\newline
Tofu \mydots \indexref{tofu}\newline
Tokliberal \mydots \indexref{tokliberal}\newline
Tolkien och den svarta magin \mydots \indexref{tolkien och den svarta magin}\newline
Tolva \mydots \indexref{tolva}\newline
Tomte \mydots \indexref{tomte}\newline
Tomten \mydots \indexref{tomten}\newline
Torbjörn \mydots \indexref{torbjoorn}\newline
Torgny Mogren \mydots \indexref{torgny mogren}\newline
Tornedalslåset \mydots \indexref{tornedalslaaset}\newline
Torsdagar \mydots \indexref{torsdagar}\newline
Torsten Bengtsson \mydots \indexref{torsten bengtsson}\newline
Traditionell finsk medicin \mydots \indexref{traditionell finsk medicin}\newline
Traditionell kinesisk medicin \mydots \indexref{traditionell kinesisk medicin}\newline
Trans \mydots \indexref{trans}\newline
Trasmattans dag \mydots \indexref{trasmattans dag}\newline
Trea \mydots \indexref{trea}\newline
Trepipsproblem \mydots \indexref{trepipsproblem}\newline
Trevlig \mydots \indexref{trevlig}\newline
Trilobit \mydots \indexref{trilobit}\newline
Trivselskrot \mydots \indexref{trivselskrot}\newline
Trocadero \mydots \indexref{trocadero}\newline
Trollkull \mydots \indexref{trollkull}\newline
Trollprutt \mydots \indexref{trollprutt}\newline
Trollpunk \mydots \indexref{trollpunk}\newline
Trotta \mydots \indexref{trotta}\newline
Träbjörn \mydots \indexref{traebjoorn}\newline
Träd, Gräs och Stenar \mydots \indexref{traed, graes och stenar}\newline
Träskpunkare \mydots \indexref{traeskpunkare}\newline
Tråg \mydots \indexref{traag}\newline
Tsygan \mydots \indexref{tsygan}\newline
Tuborg \mydots \indexref{tuborg}\newline
Tuff-frysa \mydots \indexref{tuff-frysa}\newline
Tunntarmen \mydots \indexref{tunntarmen}\newline
Tura \mydots \indexref{tura}\newline
Turtlestestet \mydots \indexref{turtlestestet}\newline
Tvåa \mydots \indexref{tvaaa}\newline
Tysk toalett \mydots \indexref{tysk toalett}\newline
Tysk tårttant \mydots \indexref{tysk taarttant}\newline
Tyskland \mydots \indexref{tyskland}\newline
Tåga \mydots \indexref{taaga}\newline
Törley gala \mydots \indexref{toorley gala}\newline
\end{multicols}
{\huge{\textbf{U}}}
\begin{multicols}{2}
Ugglekonst \mydots \indexref{ugglekonst}\newline
Umeå \mydots \indexref{umeaa}\newline
United States of America \mydots \indexref{united states of america}\newline
Universitet \mydots \indexref{universitet}\newline
Uppland \mydots \indexref{uppland}\newline
Uppsala \mydots \indexref{uppsala}\newline
Uppstoppad uv \mydots \indexref{uppstoppad uv}\newline
UR \mydots \indexref{ur}\newline
Urin \mydots \indexref{urin}\newline
Utlastarskämt \mydots \indexref{utlastarskaemt}\newline
Utrikiska \mydots \indexref{utrikiska}\newline
Utrotningshotade djur \mydots \indexref{utrotningshotade djur}\newline
Uv \mydots \indexref{uv}\newline
Uv-ljus \mydots \indexref{uv-ljus}\newline
Uv-rugby \mydots \indexref{uv-rugby}\newline
Uvgodis \mydots \indexref{uvgodis}\newline
Uvmytologi \mydots \indexref{uvmytologi}\newline
Uvsvane \mydots \indexref{uvsvane}\newline
Uvtårar \mydots \indexref{uvtaarar}\newline
\end{multicols}
{\huge{\textbf{V}}}
\begin{multicols}{2}
Valentina Vladimirovna Teresjkova \mydots \indexref{valentina vladimirovna teresjkova}\newline
Valfrihet \mydots \indexref{valfrihet}\newline
Valsvärk \mydots \indexref{valsvaerk}\newline
Valsång \mydots \indexref{valsaang}\newline
Valuta \mydots \indexref{valuta}\newline
Vanliga pantade knegare \mydots \indexref{vanliga pantade knegare}\newline
Vansinnets historia \mydots \indexref{vansinnets historia}\newline
Vaskning \mydots \indexref{vaskning}\newline
Vega Video \mydots \indexref{vega video}\newline
Verklighetens folk \mydots \indexref{verklighetens folk}\newline
Vernissage \mydots \indexref{vernissage}\newline
Veva med kängnäven \mydots \indexref{veva med kaengnaeven}\newline
Vevlira \mydots \indexref{vevlira}\newline
Videotex \mydots \indexref{videotex}\newline
Viktiga papper \mydots \indexref{viktiga papper}\newline
Viktoria (namn) \mydots \indexref{viktoria (namn)}\newline
Vild-Hasse \mydots \indexref{vild-hasse}\newline
Vilskita \mydots \indexref{vilskita}\newline
Vimpel på pinne \mydots \indexref{vimpel paa pinne}\newline
Vin \mydots \indexref{vin}\newline
Vinfylla \mydots \indexref{vinfylla}\newline
Vingmutter \mydots \indexref{vingmutter}\newline
Vinkännare \mydots \indexref{vinkaennare}\newline
Viskositet \mydots \indexref{viskositet}\newline
Vit månad \mydots \indexref{vit maanad}\newline
Vladimir Krutov \mydots \indexref{vladimir krutov}\newline
Vodka \mydots \indexref{vodka}\newline
Volvo 240-serien \mydots \indexref{volvo 240-serien}\newline
Volvo 740 \mydots \indexref{volvo 740}\newline
Vädret \mydots \indexref{vaedret}\newline
Vältagravstensfull \mydots \indexref{vaeltagravstensfull}\newline
Vänort \mydots \indexref{vaenort}\newline
Världens näst ondaste band \mydots \indexref{vaerldens naest ondaste band}\newline
Världens näst största byggnad \mydots \indexref{vaerldens naest stoorsta byggnad}\newline
Världens tråkigaste skämt \mydots \indexref{vaerldens traakigaste skaemt}\newline
Världsmusik \mydots \indexref{vaerldsmusik}\newline
Västerbotten \mydots \indexref{vaesterbotten}\newline
Växjö \mydots \indexref{vaexjoo}\newline
Våld \mydots \indexref{vaald}\newline
\end{multicols}
{\huge{\textbf{W}}}
\begin{multicols}{2}
Warcollapse \mydots \indexref{warcollapse}\newline
Watain \mydots \indexref{watain}\newline
Wctbyxa \mydots \indexref{wctbyxa}\newline
We are the world \mydots \indexref{we are the world}\newline
Weiron Holmberg \mydots \indexref{weiron holmberg}\newline
Wham, bam, thank you ma'am \mydots \indexref{wham, bam, thank you maam}\newline
William Banting \mydots \indexref{william banting}\newline
World Wide Web \mydots \indexref{world wide web}\newline
Wunderbaum \mydots \indexref{wunderbaum}\newline
\end{multicols}
{\huge{\textbf{X}}}
\begin{multicols}{2}
X3m sports \mydots \indexref{x3m sports}\newline
\end{multicols}
{\huge{\textbf{Y}}}
\begin{multicols}{2}
Yngwiefiering \mydots \indexref{yngwiefiering}\newline
\end{multicols}
{\huge{\textbf{Z}}}
\begin{multicols}{2}
ZiL-fil \mydots \indexref{zil-fil}\newline
Zoom \mydots \indexref{zoom}\newline
\end{multicols}
{\huge{\textbf{A}}}
\begin{multicols}{2}
Åka på safari \mydots \indexref{aaka paa safari}\newline
Åka vikingaskepp \mydots \indexref{aaka vikingaskepp}\newline
Åkarbrasa \mydots \indexref{aakarbrasa}\newline
Åke Cato \mydots \indexref{aake cato}\newline
Åke Ohlmarks \mydots \indexref{aake ohlmarks}\newline
Åkerdisco \mydots \indexref{aakerdisco}\newline
Ålands demitaliseringsdag \mydots \indexref{aalands demitaliseringsdag}\newline
Ålandskrisen \mydots \indexref{aalandskrisen}\newline
Ålidhem \mydots \indexref{aalidhem}\newline
Ånäset \mydots \indexref{aanaeset}\newline
Årets göteborgare \mydots \indexref{aarets gooteborgare}\newline
Åtta \mydots \indexref{aatta}\newline
Äckligt godis \mydots \indexref{aeckligt godis}\newline
Ädelost \mydots \indexref{aedelost}\newline
Äganderätt \mydots \indexref{aeganderaett}\newline
Ägg \mydots \indexref{aegg}\newline
Ägmästare \mydots \indexref{aegmaestare}\newline
Ärtpåse \mydots \indexref{aertpaase}\newline
Ättestupa \mydots \indexref{aettestupa}\newline
Äventyrsbad \mydots \indexref{aeventyrsbad}\newline
\end{multicols}
{\huge{\textbf{O}}}
\begin{multicols}{2}
Ödla på pinne \mydots \indexref{oodla paa pinne}\newline
Öra \mydots \indexref{oora}\newline
Örnäset \mydots \indexref{oornaeset}\newline
Övre magmunnen \mydots \indexref{oovre magmunnen}\newline
\end{multicols}










\end{document}